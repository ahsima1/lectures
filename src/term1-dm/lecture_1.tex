\Section{Введение}

\noindent
$ (a, b, c, ) $ - упорядоченный набор \\
$ \{ a, b, c \} $ - неупорядоченный набор \\
$ i = 1 .. k = \overline{1, k} $ \\
$ \lceil x \rceil $ - наименьшее целое $ \geq x $ \\
$ \lfloor \rfloor $ - наибольшее целое $ \leq x $ \\
$ \N $ - натуральные числа, без 0 \\
$ \Z $ - целые числа \\
$ \R $ - вещественные числа \\
$ \N + 0 = \N \cup \{ 0 \}  $ \\
$ \forall $ - "для любого" \\
$ \exists $ - "существует" \\
$ \nexists $ - "не существует" \\
$ \exists ! $ - "существует и единственный" \\
$ \wedge $ - И \\
$ \vee $ - ИЛИ \\
$ \neg $ - НЕ \\
$ \{ x | ... \} $ - множество таких x, что ... \\

\begin{definition}
	Множество - любая определённая совокупность объектов.
\end{definition}
$ x \in M $ - принадлежит\\
$ x \notin M $ - не принадлежит \\
$ \emptyset $ - пустое множество \\
$ U $ - универсальное множество, универсум \\
\begin{enumerate}
	\item Перечислить элементы \\
	$ M := \{ a, b,c, ..., z \} $ 
	\item Характеристический предикат \\
	$ M = \{ x | P(x) \} \ \ \ M = \{ n | n \in N \& n < 10 \} $
	\item Порождающая процедура \\
	$ M := \{x| x := f \} $ \\
	$ \{a_i\}_{i=1}^{k} \rightarrow \{a_1, a_2, ..., a_k \} $ \\
	$ \mathcal{M} = \{M_{\alpha}\}_{\alpha \in A} $ \\
\end{enumerate}

$ A \cup \{x\} = A + x \stackrel{\triangle}{=} \{ y | y \in A \vee y = x \} $ \\
$ A \setminus \{x\} \stackrel{\triangle}{=} A - x =  \{ y | y \in A \wedge y \neq x \} $ \\
Противоречие Рассела \\
$ Y = \{ X | X \neq X \} $ \\
$ \letus \exists Y \rightarrow Y \in Y $ или $ Y \notin Y $ \\
Если $ Y \in Y $ то $ Y \in Y \Rightarrow Y \centernot\subset Y $ \\
Если $ Y \notin Y $ то $ Y \notin Y $ \\
\begin{definition}
	Мультимножества \\
	$ \letus X = \{  \} $
	$ \stackrel{\triangle}{X} = $ 
\end{definition}

$ \forall i \in \overline{1, n} $ \\
$ a_i : ( a_i = 0, a_i = 1 ) $ - мультимн-во явл индикатором \\
$ \stackrel{\wedge}{X} = < a_1(x_1), ... > $ над $ X = \{x_1, ...\} $ \\
Конечные последовательности \\
$ \letus X = \{x_1, x_2, ..., x_n \} $ \\
$ \stackrel{\wedge}{X} = $\\
$ X = \{a, b\}, \ \ \multiset{X} = <2(a), 2(b) > $\\
$ aabb, abab, baba, bbaa, abba, baab $ - вектор \\

$ A \subset B \defeq x \in A \Rightarrow x \in B \forall x \in A $ \\
A - подмн-во, B - надмн-во \\
$ \forall M, \emptyset \subset M $ \\
$ \letus X = \{ x_1, ..., x_n \} Y \subset X, \exists! $ индикатор $ \multiset{Y} = < a_1(x_1), ..., a_n(x_n) > $ над $ X, \forall i \in \overline{1, n} (a_i = 1 \Leftrightarrow x_i \in Y)  $ \\
$ \forall A,B (A \subset B, B \subset A \Leftrightarrow A = B) $ \\
$ \forall A,B,C ( A \subset B, B \subset C, A \subset C) $ \\
$ (A \varsubsetneq B \subseteq C \Rightarrow A \varsubsetneq C)  $\\
\begin{definition}
	Равномощные множества \\
	A и B взаимно однозначное соответствие (биекция) \\
	A и B изоморфны, $ A \sim B $ \\
	$ a \in A, b \in B, a \mapsto b $ \\
	$ \N, 2\N $ \\
	$ N \sim 2\N $ \\
	$ |A| =|B| \defeq A \sim B $\\
	$ \forall A, |A| = |A| $ \\
	$ \forall A,B, |A| = |B| \Rightarrow |B| = |A| $ \\
	$\forall A,B,C, |A|=|B|, |B|=|C| \Rightarrow |A| = |C| $ 
\end{definition}
\Subsection{Конечные и бесконечные множества}
Часть меньше целого - не работает для бесконечных \\
A - конечное, если у него нет равномощного собственного подмножества \\
$ \forall B, (B \subset A \& |B|=|A|) \Rightarrow B = A  \ \ |B| < \infty$ \\
$ \exists B, (B \subset A \& |B| = |A| \& B \neq A) $ - бесконечное множество \\
$ |\N| = \infty $ \\
$ 2 \N = \N $ \\
\begin{theorem}
	Множество,  имеющее бесконечные подмножества само является бесконечным \\
	$ (B \subset A \& B = \infty \Rightarrow |A| = \infty) $ 	
\end{theorem}
\Subsection{Счётные и несчётные множества} 

Счётное - бесконечное, равномощное $\N$ \\
$ \Z \ \ n = 0, n \rightarrow 1 $ \\
$ n > 0, n \rightarrow 2n $ \\
$ n < 0, n \rightarrow 2n + 1 $ \\
$ T(1) = 1, T(2) = 3, T(3) = 6 $ \\
$ T(k) = \sum_{i=1}^{k} i = \dfrac{(k+1)k}{2} = T(k-1) + k $ \\
\begin{theorem}
	Множество всех подмножеств натурального ряда несчётно \\
	\begin{proof}
	Пусть множество счётно $ |X| < \infty, X \sim \N $ \\
	$ \letus N(X) $ номер мн-ва X \\
	$ Y $ содержит в себе номера множеств, не содержащих в себе этот номер \\
	$ Y := \{ x \in N | \exists X \subset N (x = N(x), x \notin X) \} $ \\
	$ y = N(Y) $ Если $y \notin Y, y \in Y $ \\
	Если $ y \in Y \Rightarrow y \notin Y $ \\
	Противоречие 
	\end{proof}
\end{theorem}
Счётным является бесконечное мн-во всех конечных подмножеств $\N$ \\
\begin{theorem}
	Любое непустое конечное множество равномощно некоторому отрезку натурального ряда \\
	$ \forall A ( |A| \neq 0, \& |A| < \infty, \rightarrow  \exists k \subset \N, (|A| = |1, ..., k|)) $\\
\end{theorem}
\begin{theorem}
	Любой отрезок натурального ряда конечен \\
	Пусть есть бесконечный отрезок.
	Наименьшее n, $ 1 ... n | = \infty $ \\
	$ \exists A : |1...N| 1..n \sim A $ \\
	$ n \rightarrow i $ \\
	
\end{theorem}
 
\subsection{Операции над множествами} 
\begin{enumerate}
	\item Объединение $ A \cup B \defeq \{ x | x\in A \vee x \in B \} $ 
	\item Пересечение $ A \cap B \defeq \{ x | x \in A \& x \in B \}$
	\item Разность $ A \setminus B \defeq \{x| x \in A \& x \notin B \} $
	\item Симметрическая разность $ A \triangle B \defeq (A \cup B) \setminus (A \cap B) = \{ x | (x \in A \& x \notin B) \vee (x \notin A \& x \in B)  \}$ 
	\item $ \overline{A} \defeq =  \{ x | x \notin A \} $ \\
	$ \overline{A} = U \setminus A $
\end{enumerate}
$
\begin{array}{|c|c|c|c|c|c|c|}
	\hline 
	& \cup \cap  & \cup \setminus  & \cup \triangle  & \cap \setminus & \cap \triangle & \setminus \triangle  \\ 
	\hline 
	A \cup B & A \cup B & A \cup B  & A \cup B  &  &  &  \\ 
	\hline 
	A \cap B& A \cap B &  &  & A \cap B & A \cap B &  \\ 
	\hline 
	A \setminus B &  & A \setminus B &  & A \setminus B &  &  A \setminus B\\ 
	\hline 
	A \triangle B &  &  & A \triangle B  &  & A \triangle B  & A \triangle B   \\ 
	\hline 

\end{array} 
$


\Subsection{Разбиение и покрытие}

$ \letus \eps = \{ E_i \}_{i\in I} - $ семейство подмножеств \\
$ \eps $ - покрытие M $ \forall x \in M, (\exists i \in I, x = E_i) $ \\
$ \eps $ - дизъюнктным если $ \forall i, j \in I, i \neq j, E_i \cap E_j = \emptyset $
$ \eps $ - рабиение - дизъюнктное покрытие \\

\begin{theorem}
	$ \eps = \{ E_i \}_{i\in I} $ - дизъюнктное семейство подмн-ва M то существует разбиение $ B = \{ B_i \}_{i\in I} $ Каждый элемент  $ \eps $ подмножество блока разбиения B 
	\begin{proof}
		$ i_0 \in I, B = \{ B_i \}_{i\in I} B_{i_0} = M \setminus \cup_{i\in I - i_0} E_i  \forall i \in I - i_0 ( B_i = E_c) $ 
	\end{proof}
\end{theorem}
$ M = \{1,2,3\}, \eps = \{\{ 1 \}, \{2\}, \} $ \\
$ B = \{ \{1\}, \{ 2,3\} \} $ \\

\Subsection{Булеан} 
Множество всех подмножеств мн-ва M - булеан мн-ва M и обозначается $ 2^M $\\
$ 2^M \defeq \{ A | A \subset M \} $\\
\begin{theorem}
	M - конечно $ \Leftrightarrow |2^M| = 2^{|M|} $
	\begin{proof}
		$ |M| = 0, 2^M = \{ \emptyset \}, |2^M| = 0 $ \\
		$ M (M < k, |2^{M}| = 2^{|M|}  ) $ \\
		$ M = \{ a_1, ..., a_k \}, |M| = k $
		$ M_1 = \{ x \in 2^M | a_k \notin x \} $ \\
		$ M_2 = \{ x \in 2^M | a_k \in x \}  $
	\end{proof}
\end{theorem}
В булеане можно выделить семейство подмн-в \\
$ C^k(M) = \{ S \subset M | |S| = k \} $ \\
$ C^0 (M) = \{\emptyset\} $ \\
$ C^1 (M)  $ - разбиение \\
$ |M| = n, 2^M = \cup_{x=0}^{n} C^k(M), |C^k (M)| = C_k^n $ \\
Булеан множества мощнее самого множества \\
\Subsection{Свойства операций над мн-вами}
Идемпотентность $ A \cup A = A, A \cap A = A $\\
Коммутативность $ A \cup B = B \cup A $ \\
Ассоциативность $ A \cup B \cup C  = A \cup (A \cup B )$\\
Дистрибутивность $ A \cup (B \cap C) = (A \cup B) \cap (A \cup C) $ \\
Поглощение $ (A \cap B) \cup A = A, (A \cup B) \cap B = A $ \\
Св-ва нуля $ A \cup \emptyset = A, A \cap \emptyset = \emptyset $ \\
Св-ва единицы $ A \cup U = U, A \cap U = A $ \\
Инволютивность $ \overline{\overline{A}} = A $ \\
Законы Де-Моргана $ \overline{A \cap B} = \overline{A} \cup \overline{B} $\\
$ \overline{A \cup B} = s $
Свойства дополнения $ A \cup \overline{A} U, A \cap \overline{A} = \emptyset $\\
Знак для разности $ A \setminus B = A \cap \overline{B} $ 











