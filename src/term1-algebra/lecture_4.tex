\Subsection{Гомоморфизмы и изоморфизмы колец}


\begin{definition}
	$ R_1 $ и $R_2 $ - кольца \\
	$ f : R_1 \rightarrow R_2 - $ гомоморфизм \\
	$  \forall a, b \in R f(a+b) = f(a) + f(b) $
	$ f(ab) = f(a)f(b)$
\end{definition}

\begin{definition}
	Биективный гомоморфизм назывется изоморфизм  \\
	$ f : R_1 \rightarrow R_2 - $ изоморфизм \\
	$ R_1, R_2 - $ изоморфные кольца
	$ R_1 \cong R_2$
\end{definition}
\begin{properties}
	\begin{enumerate}
		\item $ f(0) = f(0+0) = f(0) + f(0) $ \\
		$ f(0_{R_1}) = 0_{R_2} $
		\item Если $f(1_{R_1}) $ обратим в $R_2$, то \\
		$ f(1_{R_1}) = 1_{R_2} $ \\
		$ f(1) = f(1 \cdot 1) = f(1) \cdot f(1) $
		
	\end{enumerate}
\end{properties}
В некоторых книгах для колец с 1 рассматривают более узкое понятие гомоморфизма, требуют, чтобы $ f(1_{R_1}) = 1_{R_2} $\\
\begin{example}
	$ R_1 \subseteq R_2 $ \\
	$ f: R_1 \rightarrow R_2 $ \\
	$ r \mapsto r $ \\
	$ \Z \rightarrow \mathbb{Q} $
\end{example}
\begin{example}
	чётные $ \mapsto 0 $ \\
	нечётные $ \mapsto 1 $ \\
	$ f: \Z \rightarrow \mathbb{F}_2 $
\end{example}
\begin{properties}
	$ f: R_1 \rightarrow R_2 $ - изоморфизм \\
	$ f^{-1} : R_2 \rightarrow R_1 $ изоморфизм \\
	$ c, d \in R_2, \exists a,b \in R_1, c = f(a), d = f(b) $ \\
	$ a = f^{-1}(c), b = f^{-1} (d) $ \\
	$ f^{-1} (c+d) = f^{-1} (f(a) + f(b)) = f^{-1} (f(a+b)) = a+b = f^{-1} (c) + f^{-1} (d) $\\
	Упражн. $ f^{-1} (c+d) = f^{-1}(c) + f^{-1}(d) $ 
\end{properties}
% pic1
Упр Композиция гомоморфизмов - гомоморфизм\\
Композиция изоморфизмов - изоморфизм\\
$ R_1 \cong R_2 \cong R_3 \Rightarrow R_1 \cong R_3 $

\Section{Поле комплексных чисел} 

\subsection{Поле комплексных чисел} 

$ \mathbb{C} $ (complex) \\
$ \R \times \R = \mathbb{C}  $\\
\begin{definition}
	$ (a, b) + (c, d) = (a+c, b+d) $\\
	$(a,b) (c, d) = (ac - bd, bc + ad) $\\
\end{definition}
\begin{theorem}
	$ \mathbb{C}, +, \cdot $ - поле 
	\begin{proof}
		\begin{enumerate}
			\item $ (a, b) + (c, d) + (e, f) = (a + c, b+ d) + (e, f) = ((a+c) + e, (b+d) + f) = (a + (c+e), b + (d+f)) = (a, b) + (c+e, d+f) = (a, b) + ((c, d) + (e, f))$
			\item обратный - $ (0, 0) $
			\item $-(a, b) = (-a, -b) $
			\item $ (a, b) + (c, d) = (c, d) + (a, b) $ 
			\item $ (a, b) ((c,d) (e, f)) = (a, b)(ce-df, cf+de) = (ace - adf - bcf - bde , acf+ade+bce-bdf)  $ \\
			$ ((a, b)(c, d))(e, f) = (ac - bd, ad, bc) (e,f) = (ace - bde - cdf - bcf,  acf - bdf + ade + bce)  $
			\item $(1, 0) (a, b) = (1\cdot a - 0 \cdot b, 1 \cdot b - 0 \cdot a) = (a, b)$ \\
			$ (a, b)(1, 0) = (a \cdot 1 - b \cdot 0 , a \cdot 0 + b \cdot 1) $ 
			\item $ (a,b) (c, d) = (ac-bd, bd+bc) $ \\
			$ (c, d)(a, b) = (ca - db, cb + da) = (ac - bd, bc + ad) $ 
			\item упр дистрибутивн
			\item $ (a, b) \neq (0, 0) $ \\
			$ \exists ? (x, y) : (a, b)(x, y) = (1, 0) $ \\
			$ ax - by, ay - bx $ \\
			$ ax - by = 1  \mid \cdot a$\\
			$ bx + ay = 0  \mid  \cdot b$\\
			$ (a^2 + b^2)x = a, a_2 + b_2 > 0 $\\
			$ x = \dfrac{a}{a^2 + b^2} $ \\
			$ (b^2 + a^2) y = -b $ \\
			$ y = \dfrac{-b}{a^2 + b^2} $ % pic2, 3
 		\end{enumerate}
	\end{proof}
\end{theorem}
$ \{ (a, 0) \mid a \in \R   \} $ \\
$ (a, 0) + (b, 0) = (a + b, 0) $ \\
$ (a, 0)(b, 0) = (ab - 0, a\cdot 0 + 0 \cdot b) = (ab, 0) $\\
$ \R \Rightarrow  \{ (a, 0) \mid a \in \R   \} $ \\
$ a \mapsto (a, 0) $ \\
Это - гомоморфизм, биекция, след изоморфизм \\
$ \mathbb{C} $ содержит подполе, изоморфное полю $ \R $\\
Отождествим $ \R $ с этим подполем \\
$ (a, 0) = a $ \\
$ i = (0, 1) $ \\
$ i^2 = (0, 1)(0,1) = (0 \cdot 0 - 1 \cdot 1 , 0 \cdot 1 + 1 \cdot 0) = (-1, 0) = -1 $ \\
$ i^2 = -1 $ \\
$ (a, b) = (a, 0) + (0, b) = (a, 0) + (b, 0) = (a, 0) + (b_0) (0, 1) $ \\
$ (a, b) = a + b \cdot i $ - алгебраическая форма записи комплексного числа \\
$ z = a + bi $ \\
a - вещественная часть $ Re(z) - $ real \\
b - мнимая часть $ Im(z) - $ imaginary \\
Геометрическое изображение комплексных чисел \\
Комплексные числа удобно отождествлять с точками вещественной плоскости. \\
\begin{remark}
	$ (a + bi) (c+di) = ac + adi + bic + bidi = ac - bd + (ad+bc)i $\\
	$ (a_1b)(c_1d) = (ac - bd, ad + bc) $ 
\end{remark}
% pic4
\Subsection{Комплексное сопряжение}
$ \mathbb{C}, z = a + bi, a, b \in \R $ \\
\begin{definition}
	$\overline{z} = a - bi  $ называется комплексно сопряжённым с $ z $\\
	Геометрический смысл- отражение относительно вещественной оси. % pic5
\end{definition}
\begin{properties}
	\begin{enumerate}
		\item $ \overline{z_1 + z_2} = \overline{z_1} + \overline{z_2} $ 
		\item $ \overline{z_1 + ... + z_k} =   \overline{z_1} + ... + \overline{z_k}$ 
		\item $ \overline{z_1z_2} = \overline{z_1}\overline{z_2} $ 
		\item $ \overline{z_1 \cdot ... \cdot z_k} =   \overline{z_1} \cdot ... \cdot \overline{z_k}$ 
		\item $ \overline{\overline{z}} = z $ 
		\item $ z = \overline{z} \Rightarrow z \in \R $ 
		\item $ z + \overline{z} \in \R $ 
		\item $ z \cdot \overline{z} \in \R $  $z \cdot \overline{z} = 0 \Leftrightarrow z = 0 $
		\item $ a \in \R \ \ \overline{az} = a \overline{z} $
	\end{enumerate}
	\begin{proof}
		\begin{enumerate}
			\item $ z_1 + z_2 = a + c + (b+d) i $ \\
			$ (a, b) + (c, d) = (a+c + b+ d) = a + c + (b+d) i $  \\
			$ \overline{z_1} = a- bi \ \ \ \overline{z_2} = c - di $ \\
			$ \overline{z_1} + \overline{z_2} = (a+c) + ((-b) + (-d))i = (a+c) - (b+d) i = \overline{z_1 + z_2} $
			\item Индукция по числу слагаемых 
			\item $ z_1z_2 = (a + bi)(c+di) = ac+bid+adi + bic = ac - bd + (ad-bc)i $ \\
			$ \overline{z_1}\overline{z_2} = (a-bi)(c-di) = ac - adi - bic + bid = ac - bd - (ad + bc) i = \overline{z_1z_2} $
			\item Индукция по числу сомножителей 
			\item Очевидно
			\item $ z = \overline{z} \Leftrightarrow a + bi = a - bi \Leftrightarrow b = -b \Leftrightarrow 2b = 0 \Leftrightarrow b = 0 $
			\item $ z + \overline{z} = a + bi + a - bi = 2a = 2 Re(z) \in \R $
			\item  $ z \cdot \overline{z} = (a + bi) ( a- bi ) = a^2 - b^2 $ \\
			$ z \cdot \overline{z} = Re^2(z) + Im^2(z) \geq 0 $\\
			$ z \cdot \overline{z} \Leftrightarrow Re(z) = 0 = Im(z) \Leftrightarrow z = 0 $
			\item $ a \in \R \overline{az} = \overline{a} \overline{z} = a \overline{z}$
		\end{enumerate}
	\end{proof}
\end{properties}
Замечание - сопряжение - изоморфизм поля комплексных чисел на самого себя\\
\begin{definition}
	$ \sqrt{z \cdot \overline{z} } = \sqrt{Re^2(z) + Im^2 (z) } $ \\
	Называется модулем z и обозначается $ |z| $ \\
	Замечание $ z \cdot \overline{z} = |z|^2$ если $ z \neq 0 $\\
	% pic6
\end{definition}

\Subsection{Тригонометрическая форма записи комплексного числа}

%pic6
Угол $ \phi $ определён не однозначно, а с точностью до слагаемого $ 2 \pi k, k \in \Z $\\
$ z = a + bi $\\
$ r^2 = a^2 + b^2 $ \\
$ r = |z| $ \\
$ \phi $ - аргумент $z$\\
$ \phi = arg z, Arg(z) = \{ \phi + 2 \pi k | k \in \Z \} $\\
$ x \sim y $ если $ \exists k \in \Z , x-y=2\pi k$ \\
Упраж
$ \phi = arg z, Arg z = [ \phi ] = [arg z] $\\

Если полярные коорд $ z \in \mathbb{C} $ равны $ (r, \phi) $ \\
$ z = r \cos(\phi) + r \sin (\phi ) i = r (\cos(\phi) + \sin(\phi) i ) $ - тригонометрическая форма записи\\
$ r \in \R_{\geq 0} $ \\
Если $ z = r(\cos(\phi) + i \sin(\phi) ) $ и $ r \geq 0, r \in \R $ , то тогда $\phi$ - одно из значений $arg(z)$ \\

$ z = a + bi \ \ \ \ \ \ r = |z| $\\
$ a = r \cdot \cos(\phi) $ \\
$ b = r \cdot \sin(\phi) $ \\
$ a \neq 0 $ \\
\begin{enumerate}
	\item $ a > 0, \phi = \arctan \dfrac{b}{a} + 2\pi k$
	\item $ a < 0, \phi = \arctan \dfrac{b}{a} + \pi + 2\pi k $ 
	\item $ a = 0, b > 0, \phi = \dfrac{\pi}{2} + 2 \pi k $ 
	\item $ a =0, b < 0, \phi = -\dfrac{\pi}{2} + 2 \pi k $
\end{enumerate}
$ z = r(\cos \phi + i \sin \phi)  \ \ \ r = |z|, \phi = arg z $ \\
$ |z|^2 = r^2\cos^2\phi + i^2 \sin^2\phi=r^2 $\\
$ r = |z| $ \\
$ z_1 = r_1(\cos \phi_1 + i \sin\phi_1) $ \\
$ z_2 = r_2(\cos \phi_2 + i \sin\phi_2) $ \\
$ z_1z_2 = r_1r_2(\cos\phi_1 + i \sin\phi_1) (\cos\phi_2 + i \sin\phi_2) =
r_1r_2 (cos(\phi_1 + \phi2) + i \sin(\phi_1 + \phi_2)) $ - тригонометрическая форма $ z_1z_2 $ \\
При перемножении комплексных чисел модули перемножаются, а аргументы складываются.
\begin{consequence}
	$ |z_1z_2| = |z_1||z_2| \ \ arg(z_1) + arg(z_2) = arg(z_1z_2)$
\end{consequence} 
\begin{consequence}
	$ |z_1z_2|^2 = |z_1|^2 \cdot|z_2|^2 $\\
	$ z_1 = a + bi, z_2 = c + di $ \\
	$ (a^2+b^2)(c^2 + d^2) = (ac-bd)^2 + (ad + bc)^2 $ - Тождество Фибоначчи \\
	\begin{remark}
		$ (a_1^2 + a_2^2 + a_3^2 + a_4^2)(b_1^2 + b_2^2 + b_3^2 + b_4^2) = c_1^2+ ... + c_4^2, c_i = c_i(\vec{a}, \vec{b})$ \\
		$ (a_1^2 + ... + a_8^2)(b_1^2 + ... + b_8^2) = c_1^2+ ... + c_8^2, c_i = c_i(\vec{a}, \vec{b}) $
	\end{remark}
\end{consequence} 
\begin{consequence}
	$ z_1 ... z_k $ \\
	$ z_j =  $ %pic7
\end{consequence}
\begin{consequence} Формула Муавра
	$ k \in \Z, z \neq 0 , z = r(\cos \phi + i \sin \phi) $ \\
	$  z^k = r^k (\cos k\phi + i \sin k\phi)$ 
	\begin{proof}
		$ k > 0 \  \ $ В сл. 3 $ z_1 = ... = z_k = z $ \\
		$ k = 0 \ \  1 = 1(\cos 0 + i \sin 0) $ \\
		$ k = -1 , z^{-1} = \dfrac{\overline{z}}{|z|^2} $ \\
		$ |z^-1| = \sqrt{ \dfrac{\overline{z}}{|z|^2}  \dfrac{z}{|z|^2}}  = \dfrac{1}{z} $ \\
		$ arg z^{-1} = arg(  \dfrac{\overline{z}}{|z|^2} ) = arg \overline{z} = - arg z $
	\end{proof}
	$ z^{-1} = \dfrac{1}{|z|} (\cos\phi + i \sin (-\phi)) $ \\
	$ z^{-k} = (z^{-1})^k $
\end{consequence}
\Subsection{Извлечение корней из комплексных чисел}

\begin{definition}
	$ z \in \mathbb{C} $ \\
	$ x^n = z $ \\
	$ x \in \mathbb{C} $ называется корнем n-й степени из z если $x^n = z$ \\
\end{definition}
\noindent
Если $ z = 0 $ то единственным корнем n-й степени будет 0 \\
Если $ z \neq 0 $, то существует в точности n корней n-й степени из z\\
(попарно различных)
$ z = 0, x^n=0 \Rightarrow x = 0$ \\
Если $ x \neq 0, \exists x^{-1} \in \mathbb{C} $\\
$ 1 = (x^{-1})^n x^n = (x^{-1})^n \cdot 0 = 0 $ Противоречие \\
$ z = R(\cos \phi + i \sin \phi) $ \\
$ x = r(\cos \phi + i \sin \phi) $ \\
$ x^n = z $ \\
$ r^n(\cos n \psi + i \sin s\psi) = R(\cos \phi + i \sin \phi) $ \\
$ z^n = R $  \\
$ n \psi = \phi + 2 \pi k, k \in \Z $\\
$ r = \sqrt[n]{R} $ корень в веществ. степени.
$ \psi = \dfrac{\phi}{n} + \dfrac{2 \pi k}{n}, k \in \Z $ \\
Подставляя, проверяем \\
$ x_k = \sqrt[n]{R} (\cos( \dfrac{\phi}{n} + \dfrac{2 \pi k}{n}) + i sin( \dfrac{\phi}{n} + \dfrac{2 \pi k}{n} )$ \\
$ x_k^n = z $\\
Все решения $ x^n=z $ имеют вид $x_k, k = 0, ..., n-1$ и для этих k решения попарно различны (ровно n решений) \\
\begin{theorem}
	$ a \in \Z, b \in \Z \setminus \{0\} $ \\
	$ \Rightarrow \exists $ единственные $ q, d \in \Z $ \\
	$ a = bq + d, 0 \leq d < |b| $
	\begin{proof}
		$ x \in \R $\\
		$ [x] - целая часть x $ \\
		Наибольшее целое, не превосходящее x \\
		$ [2] = 2, [\pi] = 3, [-\pi] = -4 $ \\
		$ {x} = x - [x] $ \\
		$ m = [z] \Leftrightarrow m \in \Z, m \leq x < m+1 $ \\
		$ \dfrac{a}{b} = \left[ \dfrac{a}{b} \right] + \left\{ \dfrac{a}{b} \right\} $ \\
		$ \alpha = a - bq $ 
		$ d = \left( \dfrac{a}{b} - q \right) \cdot b = \left\{ \dfrac{a}{b} \right\} \cdot b $\\
		$ 0 \leq \left\{ \dfrac{a}{b} \right\} < 1 $ \\
		$ 0 \leq \left\{ \dfrac{a}{b} \right\} \cdot b < b $ \\
		$ b < 0, -b > 0,  \exists q_1, d $ \\
		$ a = (-b)q + d, 0 \leq d < -b = |b| $ \\
		$ q = -q_1 $ \\
		$ a = b \cdot (-q_1) + d = bq+d, 0 \leq d < |b| $ 
	\end{proof}
\end{theorem}

$ x_l = \sqrt[n]{R} (\cos( \dfrac{\phi}{n} + \dfrac{2 \pi k}{n}) + i sin( \dfrac{\phi}{n} + \dfrac{2 \pi k}{n} ) $ \\
$ \exists k \in \{0, 1, ..., n-1\} $ \\
$ x_k = x_l $ \\
$ l = nq+b, 0 < k < n $ \\
$ \dfrac{2 \pi l}{n} = \dfrac{2 \pi nq}{n} + \dfrac{2 \pi k}{n} = 2\pi q + \dfrac{2 \pi k}{n}  $\\
$ \dfrac{\phi}{n} + \dfrac{2 \pi k}{n} = \dfrac{ \phi }{n } + \dfrac{2 \pi k}{n} + 2 \pi q $ \\
$ x_k = x_l  (arg(x_l) = arg(x_k) + 2 \pi q$ \\
Если $ k, j \in  \{0,1,..., n^{k+1}\}, k \neq j $ \\
То $ x_k \neq x_j $ \\
$ \dfrac{\phi}{n} + \dfrac{2 \pi k}{n} - ( \dfrac{\phi}{n} + \dfrac{2 \pi k}{n}) = \dfrac{2 \pi (k - j)}{n} $ \\
$ 0 \leq k, j < n $\\
$ -n < k - j < n $ \\
$ -2 \pi < \dfrac{2 \pi (k - j)}{n}  <  2 \pi $ \\
Если $ x_k = x_j $ то $ k- j = 0 $ \\
$ x_k = x_j \Leftrightarrow n \divby b - l $ \\

$ r = \sqrt[n]{R} $ \\
Все корни лежат в вершинах правильного n-угольника, вписанного в окружность, радиусом r 
\subsubsection{Корни из 1}
$ 1 = \cos(0) + i \sin (0) $ 
$ R = 1, \phi = 0 $ \\
Корни n-й степени из 1 это $ \cos \dfrac{2 \pi k}{n} + i \sin \dfrac{2 \pi k}{n} $ \\
$ \sqrt[n]{z} $ так можно обозначать множество всех корней из z но не какой-то конкретный. \\
\begin{definition}
	$ \eps^n = 1 $ \\
	$ \eps $ называется первообразным корнем степени n из 1, если $ \eps^m \neq 1, 0 < m < n $ \\
	
	\begin{tabular}{|c|c|c|}
		\hline
		n & все корни & первообразные \\
		\hline
		1 & 1 & 1 \\
		\hline
		2 & 1, -1 & -1 \\
		\hline
		3 & 1, $\dfrac{-1 \pm \sqrt{3} i }{2}$&  $\dfrac{-1 \pm \sqrt{3} i }{2}$\\
		\hline
		4 & 1, i, -1, -i & i, -i  \\
		\hline
		6 & $ \pm 1, \dfrac{\pm1 \pm \sqrt{3} i }{2}$&  $\dfrac{1 \pm \sqrt{3} i }{2}$\\
		\hline
	\end{tabular} 
\end{definition}

$ \eps = \cos \dfrac{2 \pi k}{ n} + i \sin \dfrac{2 \pi k}{n} $ \\
Когда $ \eps $ - первообр? \\
 Когда $ \eps $ - не первообр? \\
$ \eps^m = 1, 0 < m < n $ \\
$ \dfrac{2 \pi km}{n} = 2 \pi l $ \\
$ km = ln $ \\
$ \dfrac{k}{n}$ - сократима \\
$ n \divby k \cdot m \Rightarrow n$ и $k $ имеют общ делитель $ > 1 $\\
$ \eps $ - первообразные, когда $k $ и $ n $ - взаимно просты \\
Первообразных корней степени n из 1 ровно столько, сколько чисел от 0 до n-1, взаимно простых с n. \\

\subsection{Применение комплексных чисел для вычисления пригонометрических сумм} 

$ 1 + \cos \phi + \cos 2\phi + ... + \cos n \phi $ \\
$ S_1 = \phi = 2 \pi k \ \ \ S_1 = n + 1 $ \\
$ \phi \neq 2 \pi k $ \\
$ S_2 = 0 + \sin \phi + \sin 2\phi + ... + \sin n \phi $ \\
$ S = S_1 + i \cdot S_2 $\\
$ S_1 = Re(S) , S(2) = Im(S) $ \\
$ S  = 1 + \cos \phi + i \sin \phi + \cos 2\phi +  i \sin 2\phi + ... \cos n \phi + i \sin n \phi $ \\
$ q = \cos \phi + i \sin \phi $ \\
$ S = 1 + q + q^2 + ... + q^n , q \neq 1 $\\
$ S = 1 + q + ... + q^n $ \\
$ qS = q + ... + q^{n+1} $ \\
$ (1-q) S = 1 - q^{n+1} $ \\
$ S = \dfrac{1 - q^{n+1}}{1 - q} $ \\
$ t = \cos \dfrac{\phi}{2} + i \sin \dfrac{\phi}{2} $ \\
$ q = t^2 $ \\
$ S = \dfrac{1- t^{2(n+1)}}{1-t^2} = \dfrac{t^{2(n+1)} - 1}{t^2 - 1} = \dfrac{t^{n+1} ( t^{n+1} - t^{-n+1} )}{t(t - t^{-1})} $ \\
$ t = \cos \dfrac{\phi}{2} + i \sin \dfrac{\phi}{2} $ \\
$ t^{-1} = \cos -\dfrac{\phi}{2} + i \sin -\dfrac{\phi}{2} $ \\
$ t^{n+1} = \cos \dfrac{(n+1)\phi}{2} + i \sin \dfrac{(n+1)\phi}{2} $ \\
$ S = t^n \left( \dfrac{t^{n+1} - t^{-(n+1)}}{t - t^{-1}} \right) = \left( \cos \dfrac{n\phi}{2} + i \sin \dfrac{n\phi}{2} \right) \dfrac{2i \sin \dfrac{(n+1) \phi }{2}}{2i \sin \dfrac{\phi}{22}} $\\
$ S_1 = ReS = $%pic1 
\subsection{Применение комплексных чисел в планиметрии} 
$ z_0 = a + bi $ \\
Окружность с центром z радиусом r \\
$ |z - z_0 | = r$ \\
$ z =  x + iy  $ \\
$ \sqrt{(x-a)^2 + (y-b)^2} = r$ \\
$ (x_1, y_1), (x_2, y_2) \in \R^2 $ \\
$ x_1x_2 + y_1y_2 $ \\
$ z_1 = x_1 + iy_1 $ \\
$ z_2 = x_2 + i y_2 $\\
$ Re(z_1 \overline{z_2}) = x_1x_2 + y_1y_2 $\\
$ (x_1, y_1) \perp (x_2, y_2) \Leftrightarrow Re(z_1, \overline{z_2}) = 0 $\\

\begin{theorem}
	3 высоты в треугольнике пересекаются в одной точке \\
	\begin{proof}
		Надо доказать, что прямая, проходящая через 0 и z перпендикулярна стороне $ z_1, z_2 $ \\
		$ Re((z - z_2) \overline{z_1} ) = 0 $ \\
		$ Re((z - z_1) \overline{z_2} ) = 0 $ \\
		$ Re(z \overline{z_1} - z_2 \overline{z_1} - z \overline{z_2} + z_1\overline{z_2}) = 0 $ \\
		
		$ A = z_1 \overline{z_2} - z_2 \overline{z_1} $ \\
		$ \overline{A} = \overline{z_1} z_2 - \overline{z_2} z_1 = -A $ \\
		$ \Rightarrow A $ чисто мнимое число, $ Re(A) = 0 $  
	\end{proof}
\end{theorem}






