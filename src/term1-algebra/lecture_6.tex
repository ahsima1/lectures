\Section{Матрицы и операции над ними}

R - кольцо\\
$ \begin{pmatrix}
	a_{11} & ... & a_{1m} \\
	...& ...& ... \\
	a_{n1} & ... & a_{nm}
\end{pmatrix}\\
Mat(R, n, m) \\
A = (a_{ij})$\\
$ R \times Mat(R, n, m) \rightarrow Mat(R, n, m) $\\
$ a \times A =  \begin{pmatrix}
a\cdot a_{11} & ... & a\cdot  a_{1m} \\
...& ...& ... \\
a\cdot  a_{n1} & ... & a\cdot a_{nm}
\end{pmatrix}$ \\
$ A + B = (a_{ij} + b_{ij})  \forall i < n, j < m$ \\
$ + : Mat(R, n, m) \times Mat(R, n, m) \rightarrow Mat(R, n, m) $\\
$ \cdot : Mat(R, n, m) \times Mat(R, m, l) \rightarrow Mat(R, n, l) $ \\
$ A (i, j), B(j, k) $ \\
$ AB = C = (c_{i k}) $ \\
$ c_{ik} = \sum_{j=1}^{m} a_{ij} b_{jk} $ \\
$ \begin{pmatrix}
	a_{i1} & ... & a_{im} \\
\end{pmatrix}  \begin{pmatrix}
	b_{1k}  \\
	...  \\
	b_{mk}
\end{pmatrix} = (c_{ik})\\$
$A - n \times m, B - m \times l $\\
$ x_1, ..., x_e $ \\
$y_1, ..., y_m $ \\
$z_1, ..., z_m $ \\
Линейное преобразование $ f(ax) = a \cdot f(x) $\\ 
y линейно зависит от х \\
z линейно зависит от y\\
$ y_1 = b_{11} x_1 + ... + b_{1l}x_l $ \\
$ \vdots ---- $\\
$ y_m = b_{m1}x_1 + ...+ b_{ml} x_l$

$ z_i = \sum_{r=1}^m a_{ir} y_r = \sum_{r=1}^{m} a_{ir}(\sum_{j=1}^{l} b_r x_j )  = \sum_{r=1}^{m} \sum_{j=1}^{l} a_{ir}b{rj}x_j =\\=  \sum_{i=1}^{l} \sum_{j=1}^{m} a_{ir}b{rj}x_j = \sum_{i=1}^{l}  (\sum_{r=1}^{m}a_{ir}b_{rj})x_j $ \\
Матрицы с одним столбцом называются векторами-столбцами, с 1 строкой - векторами-строками.
 
$ X =\begin{pmatrix}
X_1 \\
\vdots \\
X_l
\end{pmatrix}  Y=\begin{pmatrix}
Y_1 \\
\vdots \\
Y_m
\end{pmatrix}X =\begin{pmatrix}
Z_1 \\
\vdots \\
Z_n
\end{pmatrix}$\\
$ Y = B \cdot X, Z = A \cdot Y, Z = (A\cdot B) \cdot X $\\
$ A \in M(n,m,R)\\
B \in M(m,l,R)\\
C \in M(l,k,R)\\
(AB)C = A(BC) $\\
$ ((AB)C)_{ij} = \sum_{s=1}^{l} (AB)_{ij} C_{sj} $\\
$ \sum_{s=1}^{l} (\sum_{r=1}^{m} (A_{ir} B{rs} )) = \sum_{r=1}^{m} \sum_{s=1}^{l} A_{ir} (B_{rs} C_{sj}) = \sum_{r=1}^{m} A_{ir} (\sum_{s=1}^{l} B_{rs} C_{sj}) = \sum_{r=1}^{m} A_{ir} (BC)_{rj}  = (A(BC))_{ij}$\\
$ A \in M(n,m,R)\\
B,C \in M(m,l,R)$\\
$ D \in M(l,k,R) $ \\
Тогда $ A(B+C) = AB+AC$\\
$ (B+C)D = BC+CD $ - доказать упражнение \\

$ a\in R, B,C  - $ матрицы одного размера \\
1. $ a(B+C) = aB + aC $\\
2. $ a(BC) = (aB)C $\\
3. Если R -коммутативно \\
$ a(BC) = B(aC) $\\

Матричное умножение некоммутативно \\
$ A - n \times m$\\
$ B m \times l $ \\
$ AB, l \neq m \Rightarrow BA $ не определено \\
$ l = m $\\
$ AB - n \times n, BA - m \times m $\\
Если $ n \neq n $ то размеры получающихся матриц не совпадают
$ l = m = n $ \\
$ n = 1, R $- коммутативно, $ M(1,1,R) $ - коммутативно \\
$ n \geq 2 - $ не коммутативно.\\
$ A = \begin{pmatrix}
0 & 1\\
0 & 0 
\end{pmatrix}
B = \begin{pmatrix}
0 & 0 \\
1 & 0
\end{pmatrix} \\
AB = \begin{pmatrix}
1 & 0 \\
0 & 0 
\end{pmatrix}
BA = \begin{pmatrix}
0 & 0\\
0 & 1
\end{pmatrix} AB \neq BA$
\begin{theorem}
	R - кольцо с 1\\
	$ M_n (R) = M(n,n,R) - $ кольцо с 1, если $ n \geq 2 $ то не коммутативн\\
	$ \overline{M_n} (R)  $ - кольцо матриц размера n под кольцом R \\
	Для сложения свойства очевидны. 
	$ \mathds{O} = \begin{pmatrix}
		0 & ... & 0 \\
		\vdots & ... & \vdots \\
		0 & ... & 0
	\end{pmatrix} $\\
	 $ A = (a_{ij}) \\
	 -A = (-a_{ij}) $
	 $ \mathds{1} = \begin{pmatrix}
	 1 & ... & 0 \\
	 \vdots & 1 & \vdots \\
	 0 & ... & 1
	 \end{pmatrix} $\\
	 Если $ n \geq 2, $  в $ M_n(R) $ есть нетривиальные делители нуля - не всякий элемент обратим \\
	 $ A = \begin{pmatrix}
	 	0 & 1\\
	 	0 & 0 
	 \end{pmatrix} \neq \mathds{0} $ \\
	 $ A^2 = \begin{pmatrix}
	 0 & 1\\
	 0 & 0 
	 \end{pmatrix} \begin{pmatrix}
	 0 & 1\\
	 0 & 0 
	 \end{pmatrix} = \begin{pmatrix}
	 0 & 0\\
	 0 & 0 
	 \end{pmatrix} 
	 A^2 = 0 \Rightarrow A - $ необратим
 \end{theorem}

\subsection{Ещё раз о комплексных числах}

\begin{definition}
	R - кольцо \\
	$ \emptyset \neq R_1 \subseteq R $\\
	$ R_1 - $ подкольцо в R если оно является кольцом отностительно тех же операций, что и в R \\
	$ + : R \times R \rightarrow R $ \\
	$ + : R_1 \times R_1 \rightarrow R_1 $ \\
	Говорят, что $ R_1 $ замкнуто относительно сложения (умножения)
\end{definition}
Предложение $ \emptyset \neq R_1 \subseteq R $ является подкольцом если оно замкнуто по сложению, взятию обратного по сложению и умножения. 
$ a, b \in R_1, a+b \in R_1, -a \in R_1, ab \in R_1 $ \\
Рассмотрим $ M_2(\R) $ \\
$\left\{  \begin{pmatrix}
a & -b\\
b & a 
\end{pmatrix} \top a,b \in \R \right\} = C$ \\
$  \begin{pmatrix}
a & -b\\
b & a 
\end{pmatrix} +  \begin{pmatrix}
c & -d\\
d & c 
\end{pmatrix} =  \begin{pmatrix}
a+c & -(b+d)\\
b+d & a+c 
\end{pmatrix} \in C $ \\
$  \begin{pmatrix}
a &- b\\
b & a 
\end{pmatrix}  \begin{pmatrix}
c & -d\\
d & c 
\end{pmatrix} =  \begin{pmatrix}
ac-bd & -(ad+bc)\\
bc+ad & -bd+ac 
\end{pmatrix} \in C $ \\
$ \begin{pmatrix}
c & -d\\
d & c 
\end{pmatrix}  \begin{pmatrix}
a &- b\\
b & a 
\end{pmatrix} = 
 \begin{pmatrix}
ca-db & -(cb+da)\\
da+cb & -db+ca
\end{pmatrix} =  \begin{pmatrix}
a &- b\\
b & a 
\end{pmatrix}  \begin{pmatrix}
c & -d\\
d & c 
\end{pmatrix} $
C - коммутативное кольцо\\
$ (a,b) \neq (0,0) $ \\
$  \begin{pmatrix}
a &- b\\
b & a 
\end{pmatrix}  \cdot \dfrac{1}{a^2+b^2}  \begin{pmatrix}
a & b\\
-b & a 
\end{pmatrix} = \dfrac{1}{a^2+b^2} \begin{pmatrix}
a^2+b^2 & ab-ba\\
ba-ab & b^2+a^2
\end{pmatrix}   = \begin{pmatrix}
1 & 0\\
0 & 1 
\end{pmatrix}   \Rightarrow C - $ поле \\
Изоморфизм $ \mathbb{C} $ и C \\
$ (a, b) \rightarrow 
 \begin{pmatrix}
a & -b\\
b & a 
\end{pmatrix} $\\
$ \phi : \mathbb{C} \rightarrow C $ \\
$ z_1 = a+bi = (a,b) \\
z_2 = c + di = (c,d) $\\
$ z_1 + z_2 = a + c + (b+d)i $ \
$ \phi(z_1+z_2) = \phi(z_1) + \phi (z_2) $\\
$ z_1 z_2 = ac - bd + (ad + bc)i $ \\
$ \phi(z_1 z_2) = \phi(z_1) \phi(z_2) $\\
$ \mathbb{C} \cong C $

\Section{Тело кватернионов}

$ M_2 (\mathbb{C}) $ \\
$ \mathcal{H} =\left\{ \begin{pmatrix}
z_1 & - \overline{z_2} \\
z_2 &  \overline{z_1}
\end{pmatrix}, z_1,z_2 \in \mathbb{C} \right\}$\\
$  -\begin{pmatrix}
z_1 & - \overline{z_2} \\
z_2 &  \overline{z_1}
\end{pmatrix} =  \begin{pmatrix}
-z_1 & -\overline{z_2} \\
-z_2 &  -\overline{z_1}
\end{pmatrix} \in \mathcal{H} $\\
$ -\overline{(z_2)} = \overline{z_2}$ \\
%pic2,3
$ \mathcal{H} $ - тело, но не поле \\





