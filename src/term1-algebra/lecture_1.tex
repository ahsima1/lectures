\paragraph{Учебники} 

1. Д. К. Фаддеев "Лекции по алгебре" \\
2. А. И. Кострыкин "Введение в алгебру" в 3-х частях \\
3. З. И. Боревич "Определители и матрицы" \\
4. И. В. Виноградов "Основы теории чисел" \\
5. К. Айерленд, М. Роузен Классическое введение в современную теорию чисел \\
6. Д. К. Фаддеев, И. С. Солинский "Сборник задач по высшей алгебре" \\
7. И. В. Проскурков "Сборник задач по линейной алгебре"

\Section{Введение}

\Subsection{Высказывания}

$ A, B, C, ... $ -высказывания \\
$ \cap $ - И, конъюнкция \\
$ \cup $ - ИЛИ, дизъюнкция \\
$ \Rightarrow $ - импликация \\
$ \Leftrightarrow $ - эквивалениция \\
$ \neg $ - НЕ, отрицание \\
$ \oplus $ - XOR, сложение по модулю 2 \\

$ M $ - множество \\
$ \forall x \in M: P(x) $ \\
Для всех $x$ из $M$ выполнено $P(x)$ \\
$ \forall $ - квантор общности \\
$ \exists x \in M:  P(x) $ \\
Существует x в M, такой, что P(x) \\
$ \exists $ - квантор существования \\
$ \neg (\forall x \in M: P(x)) \equiv \exists x \in M: \neg P(x) $ \\
$ \neg ( \exists x \in M: P(x)) \equiv \forall x \in M: \neg P(x) $ \\

\subsection{Множества и операции над ними}
M - конечное множество
$ M = \{ x_1, x_2, x_3, ...\} $ \\
$ \{x_1, x_2\} $ - неупорядоченная пара \\
$ \{x_1, x_2 \} = \{x_2, x_1 \} $ \\
$ \emptyset $ - пустое множество \\
$ A $ - множество, $ a \in A $ \\
$ a $ - элемент множества $ A $ \\
$ a \notin A $ - $a$ не принадлежит $A$ \\
$ \forall x \in B: x \in A  \equiv B \subseteq A $ - $B$ подмножество $A$ \\

$ 
\begin{matrix}
B \subseteq A & B \subset A & B \subseteq A \\
B \subset A & B \subsetneqq A & B \subsetneqq A \\
\end{matrix}
$\\
X - множество 
P(X) - множество подмножеств X\\
$ \{X_i\}_{i \in I} $\\
$ X_i $ - множества\\
$\cup$ - объединение\\
$\cap $- пересечение\\

$ \emptyset \neq \{ \emptyset \} $ \\
M - множество \\
P - свойство элементов множества \\
$ \{ x\in M \mid P(x) \} $ \\
Множество элементов M для которых выполнено свойство P \\
A, B - множества
$ A \setminus B = \{ x \in A \mid x \notin B \} $ \\
$ A \cap B = \{ x \in A \cup B | x \in A \wedge x \in B \} $ \\
Если $ B \subseteq A $ то $ A \setminus B $ - называется дополнением B до A\\
Упорядоченная пара
$ ( a_1, a_2 ) $ \\
$ (a_1 , a_2 ) = ( b_1, b_2 ) \Leftrightarrow a_1 = b_1, a_2 = b_2 $\\
$ A \times B = \{ (a, b) \mid a \in A, b \in B \} $ \\
$ A \times A $ - декартов квадрат \\
$ A^n $ - декартова степень \\
1. $A^n = \{ (a_1 .. a_n ) \mid a_i \in A \} $ \\
2. $ A^n = A^{n-1} \times A  $ \\
$ A_1 \times ... \times A_n = \{ (a_1 ... a_n ) \in (\cup_{i=1}^n A_i) \} \mid $
$ \forall i \in \{1, ... n\} a_i \in A_i $ \\

\Section{Отображения}

A, B - множества \\
Мн-во $ \varGamma_f \subseteq A \times B $ (график отбражения f) \\
$ \forall a \in A \exists b \in B (a, b) \in \varGamma_f $\\
$ \forall a \in A \forall b_1, b_2 \in B ((a_1, b_1) \in \varGamma_f \wedge (a_1, b_2) \in \varGamma_f) \Rightarrow b_1 = b_2  $ \\
Говорим, что задано отображение из множества А в множество B, если задана тройка $ A, B, \varGamma_f $ \\
$A$ - область определения \\
$B $- область значений \\
$\varGamma_f$ - график \\
$ (a, b) \in \varGamma_f \equiv b = f(a) $ \\
$ id_A : A \rightarrow A $ \\
$ \varGamma_f = \{ (a, a) \in A^2 \mid a \in A \} $\\
Отображение $ f: A \rightarrow B $ называется сюръективным, если $ \forall b \in B \exists a \in A (a, b) \in \varGamma_f $\\
$ f: A \twoheadrightarrow B $
f - инъективное отображение, если $ \forall a_1, a_2 \in A \forall b \in B, (a_1, b) \in \varGamma_f \wedge (a_2, b) \in \varGamma_f \Rightarrow a_1 = a_2 $ \\
$ f: A \rightarrowtail B $
f - биективное отображение, если f и сюръекция и инъекция. \\

\Subsection{Композиция отображений}

$ f: A \rightarrow B, g: B \rightarrow C $ \\
$ g \circ f : A \rightarrow C $ \\
$ \varGamma_{g \circ f} = \{ (a, c) \in A \times C \mid \exists b \in B (a, b) \in \varGamma_f \wedge (b, c) \in \varGamma_g \} $ \\
$ (g \circ f)(a)= g(f(a)) $ \\
Теорема ассоциативность композиции\\
$ h \circ (g \circ f) = (h \circ g) \circ f $ \\
$ \varGamma_{h \circ (g \circ f)} = \{ (a, d) \mid \exists c \in C (a, c) \in \varGamma_{g \circ f} \wedge (c, d) \in \varGamma_{h} \} $\\
$ \varGamma_{(h \circ g) \circ f} = \{ (a, d) \in A \mid  \} $ \\
Теорема 2: $ f: A \rightarrow B $ \\
$ id_B \circ f = f \circ id_A  = f $ \\
$ f : A \rightarrow B, g: B \rightarrow A $ \\

$ (f \circ id_A)(a) = f(id_A(a)) = f(a) $ \\
$ (id_B \circ f)(a) = id_B(f(a)) = f(a) $ \\
Теорема3 : Если $ g \circ f = id_A$ то f - инъекция, g - сюръекция \\
$ g \circ f = id_A $ \\
$ \forall a \in A (g \circ f )(a) = a $\\
$ \forall a \in A \ g(f(a)) = a $ \\
Сюръекция \\
$ a \in A $ $ g(b) = a $ $ b = f(a)$\\
Инъекция \\
$ f(a_1) = f(a_2) \Rightarrow a_1 = a_2 $
$ f(a_1) = f(a_2) $ $g(f(a_1)) = g(f(a_2)) $ \\
$ a_1 = id_A(a_1) = (g \circ f )(a_1) = (g \circ f)(a_2) = id_A(a_2) =  a_2 $
Определение $ f: A \rightarrow B $ \\
Отображение $ g: B \rightarrow A $ \\
Если $ g \circ f = id_A $ $ f \circ g = id_B $ g называется обратимым к f \\
Теорема 1. f - обратимо $\Leftrightarrow$ f - биекция\\
2. Если f обратимо то обратное отображение единственно\\
3. $ f: A \rightarrow  B \hspace{1mm} h: B \rightarrow C $ обратимо $ \Rightarrow h \circ f $ обратимо\\
4. $ f: A \rightarrow B $ обратимо $ f^{-1} : B \rightarrow A$ обратимо\\
$ f \circ f^{-1} = id_B $ \\
$ f^{-1} \circ f = id_A $ \\
$ f^{-1} $ обратимо и f обратное к $ f^{-1} $ \\

\Subsection{Отношения на множествах}

A - множество
n - местное отношение к А (n-арное)
это подмножество $ R \subseteq A^n $
Бинарное отношение $ R \subseteq A^2 $
Пример на $\N$ \\
$ a_1 \divby a_2$ если $ \exists c \in \N : a_1 = a_2 c$\\
1. R - рефлексивным, если $ \forall a \in A : aRa $ \\
2. R - антирефлексивным, если $ \forall a \in A : \neg (aRa) $\\
3. R - симметричное, если $ \forall a,b \in A : aRb \Rightarrow bRa $\\
4. R - acимметричное, если $ \forall a,b \in A: aRb \Rightarrow \neg (bRa) $ \\
5. R - aнтисимметрично, если $ \forall a, b \in A : aRb \wedge bRa \Rightarrow a = b $ \\
6. R - транзитивно, если $ \forall a, b, c \in A : aRb \wedge bRc \Rightarrow aRc $ \\

\Subsection{Отношения эквивалентности}

Если R рефлексивно, симметрично и транзитивно, то R - отношение эквивалентности. $ R - \sim $
$A, \sim$ \\
$ a \in A $ $ [a] = \{ b \in A \mid a \sim b \} $ \\
Класс эквивалентности $ [a] $\\
Теорема: $ \sim $ - отношение эквивалентности на A, тогда любые 2 класса эквивалентности либо не пересекаются, либо совпадают.\\
Множество классов эквивалентности \\
$ A / \sim $ - фактормножество мн-ва A по отношению $\sim$ \\
$ A = \Z m \in \N$ \\
Пример: $ a \equiv b (mod m) $ если $a \cdot b \divby m$ \\
