\Subsection{Линейные уравнения и линейные сравнения в ОГИ. Линейные уравнения в кольцах вычетов}

\begin{theorem}
	 R - ОГИ $ ax+by=c $ разрешимо, если $  GCD(a, b) | c $ \\
	Все решения имеют вид $ x = x_0 + \dfrac{b}{d} t, y = y_0 - \dfrac{a}{d}t $ \\
\end{theorem}
Сл. 1 R - ОГИ, $a, b, c \in  R $ \\
$ ax \equiv c (\mod b) $ \\
\begin{theorem}
	1. R - ОГИ, Сравнение разрешимо \\
	$ \Leftrightarrow GCD(a, b) | c $ \\
	2. $ R = \Z, d = GCD(a, b) $ \\
	$ d | c $  \\
	$ ax \equiv c \mod b, x_0 $ частное решение \\
	Имеет d решений, $ x \equiv x_0 + \dfrac{b}{d}t (\mod b), t = 0...d-1$\\
	$ ax \equiv c (\mod b) $ \\
	$ \Leftrightarrow b | ax - c $ \\
	$ \Leftrightarrow \exists y \ \ ax - c = by $ \\
	$ \Leftrightarrow \exists y \ \ ax - by = c $ \\
	Это ур-е разрешимо $ \Leftrightarrow $ НОД$(a,b)$ = НОД$(a, -b)  | c$ \\
	$ x_0, y_0 $ - простые решения \\
	Общ решения $ x = x_0 + \dfrac{b}{d}t , y = y_0 + \dfrac{a}{d}t, t \in R, R = \Z, x_t = x_0 + \dfrac{b}{d} t, t = 0, ... ,d-1 $ \\
	a) $ x_t, x_{t'} $ попарно несравнимы по  $ \mod b $ если $ 0 \leq t \neq t' < d $ \\
	б) Любое другое решение $ x_{t'} $ сравнимо с одним из $ x_t, 0 \leq t < d $ \\
	в) $ 0 \leq t, t' < d $ \\
	$ x_t \equiv x_{t'} (\mod b) $ \\
	$ \dfrac{b}{d} (t - t') \equiv 0 (\mod b) $ \\
	$ b | \dfrac{b(t - t')}{d}  \ \ db | b(t-t'), d | (t - t'), t = t' $ \\
	
	$ t' = dq + t, 0 \leq t < d $ \\
	$ x_{t'} = x_0 + \dfrac{b}{d} t' = x_0 + \dfrac{b}{d} (dq+t) $ \\
	$ = x_0 + \dfrac{b}{d} t + bq = x_t + bq \equiv x_t (\mod b) $ \\
 \end{theorem}


$ R - $ ОГИ $ b \neq 0, a,c \in R $ \\
$ R / (b) $ \\
$ [a] \cdot x = [c] $ ур-ние в $ R/(b) $ \\
\begin{theorem}
	1. $ [a]x = c $ разрешимо $ \Leftrightarrow GCD(a,b) | c $ \\
	2. $ R = \Z, d = GCD(a, b), d | c $ \\
	$ x_0 - $ частное решение \\
	Все решения $ x_0 + \left[\dfrac{b}{d}\right] \cdot [t], 0 \leq t < d$\\
	\begin{proof}
		%pic1,2
	\end{proof}
\end{theorem}
Следствие R - ОГИ $ b \neq 0, b \in R $ \\
$ a \in R, a, b $ вз просты \\
В $ R / (b) $ \\
Ур-е $ [a] \cdot x = [c] $ разрешимо при любой правой части единственным образом \\
\begin{proof}
	Ур-е разрешимо \\
	Пусть $ x, x'$ два решения \\
	$ [a] (x - x') = [0] $ \\
	$ x  = [X], x' = [X'], X, X' \in R $ \\
	$ a (X - X') \equiv 0 (\mod b) $ \\
	$ b | a(X - X') $ \\
	$b, a $ вз просты $ \Rightarrow X \equiv X' (\mod b) $ \\
	$ x = [X] = [X'] = x' $
\end{proof} 
Cл. R - ОГИ, $b \neq 0,  [a] \in R / (b) $ обратим в  $ R/(b) \Leftrightarrow a,b $ вз просты \\
$ [a] $ оброатим в $ R/(b) \Leftrightarrow [a]x = [1] $ разрешим $ \Leftrightarrow GCD(a,b) | 1 \Leftrightarrow a, b $ вз просты \\
\begin{theorem}
	R - ОГИ, $ b \neq 0$ \\
	$ R / (b) $ - поле $ \Leftrightarrow $ b - неприводим \\
	\begin{proof}
		1. $ R \setminus \{0\} = \{\text{Обратимые}\} \cup \{\text{Неприводимые}\} \cup \{\text{Составные}\} $ \\
		$ b \in R^*, (b) = R, R / (b) - $ одноэлем - не поле \\
		$ b $ составное $ b = n \cdot m, n, m $ не обратимые, в частности $ b \centernot| n, b \centernot| m $ \\
		$ b $  - неприв, $ [0] \neq [a] \in R / (b), b \centernot| a $ \\
		Делитель b либо обратим, либо ассоцировани с b $ \Rightarrow $ обшие делители b и a только обратимые $ \Rightarrow $ b и a взаимно просты $ \Rightarrow [a] $ обратим в $ R / (b) $ \\
		$ \Rightarrow R/(b) $ поле \\
		2. Была т-ма \\
		R - о.ц., I - идеал в R \\
		$ R / I $ - поле $ \Leftrightarrow $ I - максимальный идеал \\
		$ b \in R^*, (b) = R, $ не подходит \\
		$ b $ - составное $ b = nm, n, m \notin R^* $ n,m не ассоц с b \\
		$ n | b $ \\
		$ (b) \subsetneqq (n) \subsetneqq R \Rightarrow b $ не макс \\
		$ b $ неприводим \\
		$ (b) \subseteq Y \subseteq R $ \\
		$ \exists n, Y = (n) $ \\
		$ n | b \ \ b = n\cdot m $ \\
		т.к. b неприводим то либо $ n \in R^*, $ либо $ m \in R^* $ \\
		$ n \in R^*, Y= (n) = R $ \\
		$ m \in R^*, b \sim n , Y = (n) = (b) $ \\
		$ \Rightarrow (b) $ - максимальный  
	\end{proof}
\end{theorem}
Сл. $ R = \Z , n \in \N $ \\
$ Z / n\Z $ - поле $ \Leftrightarrow $ n - простое число \\
Поля: $ \Z/2\Z , \Z/3\Z, \Z/5\Z $ \\
Пример: $ \Z[i], p \in \N, p $ простое \\
$ p \equiv 3(4) $ \\
Тогда p - неприводимый элемент в $ \Z[i] $ \\
и $ \Z[i] /(p) $ - поле из $p^2$ эл-тов \\
\begin{proof}
	$ p = (a+bi)(c + di)  $ \\
	$ N(z) = |z|^2 $ \\
	$ p^2 = (a^2 + b^2)(c^2 + d^2) $ \\
	$ a^2 + b^2 = 1, c^2 + d^2 = p^2 $ \\
	$ (a, b) \in \{(0, \pm 1), (\pm 1, 0) \} $ \\
	$ a + bi \in \{1, -1, i, -i\} $ \\
	$ = (\Z[i])^* $ \\
	$ a^2 + b^2 = p, c^2 + d^2 = p $ \\
	$ a^2 \equiv 0, 1 (4) $ \\
	$ b^2 \equiv 0, 1 (4) $ \\
	По mod 4: \\
	$ a^2 + b^2 $ может быть 0,1,2 \\
	Но $ p \equiv 3(4) \Rightarrow p \neq a^2 + b^2 $ 
\end{proof}

P неприв $ \Z [i] $ \\
$ \Z [i] / (p) $ - поле \\
$ \{ [u + vi] | u, v \in \Z, 0 \leq u \leq p-1, 0 \leq v \leq p-1 \}$ \\
Мн-во всех попарно различных классов вычетов по mod  \\
$ x \in \Z[i] /(p) \ \ x = [a + bi] $ \\
$ a = pq + u $ \\
$ b = ps + v, 0 \leq u,v < p $ \\
$ [a + bi] = [pq + u + (ps + v) i] $ \\
$ = [u+vi] + [pq + psi] $ \\
$ = [u+vi] + [p] \cdot [q+si] = [u+vi] $ \\
$ 0 \leq u, u', v, v' < p  \ \ [u + vi] = [u' + v'i] $ \\
$ p | u - u' + (v - v')i $ в $ \Z [i] $ \\
$ u - u' + ( v - v') i = p(c+di) $ \\
$ u - u' = pc $ \\
$ v - v' = pd $ \\
$ p | (u - u') $ в $ \Z $ \\
$ p | (v - v') $ в $ \Z $ \\
$ u = u', v = v' $ \\
$ | \Z[i] / (p) | = p^2 $ \\
$ [a + bi] = [a] + [b][i] $ \\
$ p \equiv 3 (4) $  Поле из $p^2$ элементов \\
$ a + bi, a, b \in \Z /p\Z $ 

Замеч $ p \equiv 1 (4), x^2 = -1 $ разрешимо в $ \Z / p\Z $ \\
Поэтому добавить $ \sqrt{-1} $ не можем \\

\Subsection{Китайская теорема об остатках} 

(*) $ \left\{\begin{array}{cc}
x \equiv a_1 (\mod n_1) \\
... \\
x \equiv a_k (\mod n_k) \\
\end{array}  \right. $ \\
Вопрос: когда (*) - разрешимо? Как выглядят решения? \\
\begin{theorem}
	R - ОГИ \\
	Пусть $ n_1, ..., n_k $ - взаимно просты. Тогда при любом наборе a система разрешима \\
	Причём если $ x, x' $ - решения \\
	$ x \equiv x'(\mod n_1...n_k) $ 
	\begin{proof}
		Единств (по $\mod n_1...n_k$ ) \\
		$ \forall i \ \ x \equiv a_i (\mod n_i) $ \\
		$ \ \ \ \ x' \equiv a_i (\mod n_i) $ \\
		$ x \equiv x' (\mod n_i) $ 
		$ \forall i, n_i | x - x' $ \\
		%pic5,6
		Существование решения \\
		%pic 7,8
		$(*) =  \left\{\begin{array}{cc}
		x \equiv a_1 (\mod n_1) \\
		... \\
		x \equiv a_k (\mod n_k) \\
		\end{array}  \right. $ \\
		$(*^i) $ \\
		$ x = \sum_{s = 1}^{k} a_s \cdot x^{(s)} $ - реш (*) \\
		$ n_i \ \ x_{n_i} \equiv \sum_{s = 1}a_s x^{(s)} = a_1 \cdot 0 + .. + a_i \cdot 1 + a_k \cdot 0 \equiv a_i (\mod n_i)$ \\
		$ s = i, x^{(i)}  \ \ x^{(i)} \equiv 1 (n_i) $ \\
		$ s \neq i, x^{(s)} \equiv 0(n_i) $ \\
 	\end{proof}
\end{theorem}

$ m,n \in R $ вз просты \\
$ \left\{ \begin{array}{cc}
x \equiv a (m) \\
x \equiv b (n) 
\end{array} \right. $ \\
$ S_{mn} $ мн-во представителей по одному из каждого класса сравнимости по модулю (мн-во остатков) \\
$ S_n, S_m $ - аналогично \\
$ r \in S_{mn} \begin{array}{cc}
\exists x \in S_n \ r \equiv a (n) \\
\exists b \in S_n \ r \equiv b (m) 
\end{array} $ \\
$ S_{mn} \rightarrow S_n \times S_m $ \\
$ r \mapsto (a, b) $ \\
К.Т.О. говорит, что это обратимо - биекция \\
То же самое на языке факторколец \\
$ R / (mn) \ \ R / (n) \times R / (m) $ \\
$ R / (n) \times R / (m) $ можно превратить в кольцо \\
$ ([a],  [b]) + ([c], [d]) = ([a+c], [b+d]) $ \\
$ ([a],  [b]) * ([c], [d]) = ([a*c], [b*d]) $ \\
Прямое произведение колец \\
$ R / (mn) \rightarrow R / (n) \times R / (m) $ \\
$ [r] \rightarrow ([a], [b]) $\\
К.Т.О. - означает, что это биекция \\
$ a \equiv r(n) $ \\
$ b \equiv r(m) $ \\
$ r_1 \equiv a_1(n) $ \\
$ r_1 \equiv b_1(m) $ \\
$ r + r_1 \equiv a + a_1 (n) $ \\
$ r + r_1 \equiv b + b_1 (m) $ \\
$ r r_1 \equiv aa_1 (n) $ \\
$ r r_1 \equiv bb_1(m) $ \\
Гомоморфизм колец - изоморфизм колец \\
$ R /(mn) \cong R /(n) \times R(m) $ \\
%pic1
\Subsection{Функция Эйлера}
$ \Z, n \in \N $ \\
$ \phi(n) = |\{ x : 1 \leq x \leq n, GCD(x, n) = 1 \} | = |\{ x : 0 \leq x < n, GCD(x, n) = 1 \} | $ \\
$ \phi(n) = | (\Z / n\Z ) | $ \\
$ \phi(1) = 1, \phi(2) = 1, \phi(3) = 2, \phi(4) = 2, \phi(5) = 4, \phi(6) = 2 $ \\
$ \phi $ - Функция Эйлера \\
\begin{theorem}
	$ n = p_1^{a_1} ... p_k^{a_k} $ \\
	$ p_i $ - простые, $ a_i \geq 1 $ \\
	$ \phi(n) = \prod_{i=1}^{k} (p_i^{a_i} - p_i^{a_i-1}) = n \cdot \prod_{i=1}^k (1 - \dfrac{1}{p_i}) $ 
	\begin{lemma}
		n, m взаимно просты $ \Rightarrow \phi(mn) = \phi(m)\phi(n) $ \\
		\begin{definition}
			$ \alpha $ - мультипликативна, если \\
			$ \forall n,m (GCD(n,m) = 1 \Rightarrow \alpha(nm) = \alpha(n)\alpha(m)) $ 
		\end{definition}
		\begin{proof}
			$ GCD(r, nm) = 1 \Rightarrow GCD(r, n) = 1, GCD(r, m) = 1 $ \\ 
			$ r \equiv a(n) \Rightarrow GCD(a, n) = 1 $ \\
			$ r \equiv b(m) \Rightarrow GCD(b,m) = 1 $ \\
			
			$ GCD(r, nm) > 1 \Rightarrow GCD(r, n) > 1, GCD(r, m) > 1 $ \\
			$ \Rightarrow $ хотя бы 1 выполнено \\
			$ a $ не вз просто с n \\
			b не вз просто с m \\
			$ I_{nm} = \{ 0 \leq x \leq nm | GCD(x, nm) = 1 \} $ \\
			$ I_{n} = \{ 0 \leq x \leq n | GCD(x, n) = 1 \} $ \\
			$ I_{m} = \{ 0 \leq x \leq m | GCD(x, m) = 1 \} $ \\
			$ S_{nm} = \{ 0, ..., nm-1 \} $ \\
			$ S_n = \{0, ..., n-1 \} $ \\
			$ S_m = \{ 0, ..., m-1 \}$ \\
			$ S_{nm} \rightarrowtail S_n \times S_m $ \\
			$ I_{nm} \ \ \ \ \ I_n \times  I_m $ \\
			% pic2
		\end{proof}
	\end{lemma}
	Cл. $ n = p_1^{a_1} ... p_k^{a_k} $ \\
	$ \phi(p_1^{a_1} ... p_k^{a_k}) = \phi(p_1^{a_1}) ... \phi(p_k^{a_k}) $ \\
	\begin{lemma}
		$ \phi(p^a) = p^a - p^{a-1} $ \\
		\begin{proof}
			$ 0, 1, ..., p^a - 1 $ \\
			$ 0, ... p ... 2p ... p(p^{a-1} - 1) $ \\
			Не вз простых с $ p^{a-1} $ 
		\end{proof}
	\end{lemma}
	\begin{proof}
		Применить сл-вие из леммы 1 и лемму 2 
	\end{proof}
\end{theorem} 
\subsection{Теорема Эйлера и Малая теорема Ферма} 
$ n \in \N, (\Z / n\Z)^* $ \\
$ [a] \in  (\Z / n\Z)^*, [a]^k = [a]^l $ \\
$ [a], [a]^2, ... [a]^{k-l} = [1] $ \\
\begin{theorem}
	$ [a] \in (\Z / n\Z)^* \Rightarrow [a]^{\phi(n)} = [1] $ \\
	\begin{proof}
		$ (\Z / n\Z)^* = \{ [b_1], [b_2], ..., [b_{\phi(n)}]  \} $ \\
		$  (\Z / n\Z)^* \rightarrow  (\Z / n\Z)^* $ \\
		$ [b] \mapsto [a][b] = [c] $ \\
		$ [a]^{-1} [c] \leftarrow [c] $ \\
		Рассмотрим \\
		$ [a][b_1] , ..., [a][b_{\phi(n)}] $ \\
		Получим перестановку
		$ [b_1] , ..., [b_{\phi(n)}] $ \\
		$ = $ \\
		$ [a]^{\phi(n)} [b_1] , ..., [b_{\phi(n)}] $ \\
		Домножим на  $ [b_1]^{-1} , ..., [b_{\phi(n)}]^{-1} $ \\
		$ [a]^{\phi(n)} = [1] $ \\
		% pic3
	\end{proof}
\end{theorem}
Следствие (Малая теорема Ферма)\\
p - простое, $GCD(a, p) = 1$ \\
в $ \Z / p\Z, [a]^{p-1} = 1  $ 
\begin{proof}
	$ \phi(p) = p -1 $ \\
	Переформулируем \\
	p - простое, $ p \centernot| a $ \\
	$ a^{p-1} \equiv 1 (p) $ \\
	Замечание \\
	Если $ p \centernot| a \ \ a^p \equiv a(p)$ \\
	$ p | a \ \ a^p \equiv a(p)$ \\
	$ \forall a \in \Z, a^p \equiv a (p) $ \\
	
\end{proof} 

\begin{theorem}
	Теорема Вильсона \\
	p - простое $ \Leftrightarrow (p-1)! \equiv -1 (\mod p) $ \\
	\begin{proof}
		$ \Rightarrow  \ \ p = 2, 1! = 1 \equiv - 1(2) $ \\
		$ p > 2 $ \\
		$ x^2 \equiv 1 (\mod p) $ \\
		$ p | (x^2 - 1)  \ \ p | (x+1)(x-1) $ \\
		$ \Rightarrow x \equiv 1 (p) $ или $ x \equiv p-1 (\mod p) $ \\
		$ 1_!, 2, ..., p-2, p-1_! $ \\
		Все остальные разбиваются на пары взаимно обратных по $ (\mod p) $ \\
		$ a \neq b, a\cdot b \equiv 1 (p) $ \\
		$ (p-1)! = 1 \cdot 2 \cdot ... (p-2) (p-1) $ \\
		$ 1, a_1b_1, a_2b_2, ... , a_{\frac{p-3}{2}}  b_{\frac{p-3}{2}} (p-1) = 1 \cdot 1 \cdot ... (p-1) \equiv -1 (\mod p) $ \\
		$ \Leftarrow $ Если p составное, то $ (p-1)! $ не вз просто с p\\
		$ q | p, q -$ простое $ q < p $ \\
		$ q \leq p-1 $ \\
		$ 1...(p-1) $\\
		$ q | (p-1)! $ \\		
	\end{proof}
\end{theorem}









