\noindent
$\{A_i\}_{i\in I}$ \\
$ M $ - множество множеств \\
$ I $ - индексное множество \\
$ f: I \rightarrow M $ \\
$ i \mapsto A $

\Section{Алгебраические структуры. Группы, кольца, поля.}

\Subsection{Группы}

Опр: На множестве А задана n-арная операция, если задано отображение $ f: A^n \rightarrow A $\\
Бинарные операции n=2, унарные n=1 \\
Опр. Непустое мн-во G с одной бинарной операцией $ * : G \times G \rightarrow G $ если выполнено три аксиомы. \\
1. $ \forall a,b,c \in G \ a * (b * c) = (a *b) * c) \\$
2. $ \exists e \in G \forall a \in G \ e*a=a*e=a $ \\
3. $ \forall a \in G \exists a' \in G \ a*a'=a'*a=e$ \\
Опр: Если G - группа и $ \forall a,b \in G \ a*b=b*a $ G называется абелевой группой. \\
Если $*$ обозначается как $ \cdot$ или знак опускается, то говорят о мультипликативной записи, $e = 1, a' = a^{-1} $. Нейтральный элемент называют единицей\\
Если $*$ обозначается через $+$, говорят об аддитивной записи, $ e=0$, называют нулём группы, $ a' = -a$\\
Для неабелевых групп обычно  используется мультипликативная запись, для абелевых и та и другая. \\
Примеры: \\
1. $ \Z, +, 0 $ \\
$ \mathbb{Q}, + $ \\
$ \R, + $ \\
$ \N, + $ - не группа \\
3. $\mathbb{Q} \setminus \{A\}, \cdot $\\
$\R \setminus \{A\}, \cdot $\\
$\R_{> 0}, \cdot $\\
4. $ \Z \setminus 0, \cdot $ не группа. \\
5. $ \{ \pm 1 \}, \cdot $ группа из двух элементов \\
6. $ X$ - какое-то множество $ X \neq \emptyset $\\
$ S(X) -$ множество всех биекций на $X$\\
$ * - \circ $ - композиция \\
$ S(X) \neq \emptyset \ id_X \in S(X) $ \\
$ \circ $ -ассоциативна \\
$ id_X \circ f = f \circ id_X = f \Rightarrow e = id_X $ \\
$ f \in S(X) \Rightarrow \exists $ обратное отображение \\
$ f \circ f^{-1} = f^{-1} \circ f = id_X $ \\
$ |X| = 1, X=\{X_1\} , |S(X)| = 1, x_1 \mapsto x_1 $ \\
$ |X| = 2, X = \{x_1, x_2\}, id: x_1 \mapsto x_1, x_2 \mapsto x_2 $ \\
$ g: x_1 \mapsto x_2, x_2 \mapsto x_1 $ \\
$ g * g = id $ \\
$ |X| \geq 3, S(X) -$ не абелева. \\
$ f: x_1 \mapsto x_2, x_2 \mapsto x_3, x_3 \mapsto x_3 $ \\
% ------pic1
$ g \circ f: x_1 \mapsto x_3, x_2 \mapsto x_1, x_3 \mapsto x_2  $\\
$ f \circ g: x_1 \mapsto x_2, x_2 \mapsto x_3, x_3 \mapsto x_1 $ \\
$ S(X) $ - симметрическая группа на множестве $X$ \\
$ X = \{ x_1, \dots, x_n \} \ X = \{ 1, \dots, n \} $ \\
$ S_n - $ cимметрическая группа степени n. \\
$ X $ - мн-во \\
$ G = P(X) $ - мн-во подмножеств X
$ \triangle : G * G \rightarrow G $ \\
$ A \triangle B = (A \setminus B) \cup (B \setminus A) = (A \cup B) \setminus (A \cap B)$ \\
$ G, \triangle $ - группа. \\
Найдите нейтральный и обратный. Будет ли группа абелевой. \\
\subsubsection{Свойства групп}
1. Единственность нейтрального. \\
Пусть $e_1 $ и $e_2$ - нейтральные. \\
$ e_1 * e_2 = e_1$ \\
$ \forall a \in G, e*a=a*e=a $\\
2. Единственность противоположного \\ 
$ a \in G $ пусть есть два обратных $ a'$ и $a'' $\\
$ a' * a * a'' = a'*(a*a'') = a'*e=a' $\\
$ (a' *a) *a'' = e * a'' = a'' \Rightarrow a' = a'' $
\subsection{Кольца и поля}

$ R, +, \cdot \hspace*{5cm} R \neq \emptyset $ \\
1. $ \forall a,b,c \in R \ a + (b + c) = (a +b) + c) \\$
2. $ \exists 0 \in R \forall a \in R \ 0+a=a+0=a $ \\
3. $ \forall a \in R \exists -a \in R \ a+(-a)=(-a)+a=0$ \\
4. $ \forall a,b \in R \ a+b=b+a $ \\
5. $ \forall a, b, c \in R \ (a \cdot b) \cdot c = a \cdot (b \cdot c) $ \\
6. $ \exists 1 \in R \ \forall a \in R \ 1 \cdot a = a \cdot 1 = a $ \\
7. $ \forall a \in R \setminus \{0\} \exists a^{-1} \ a \cdot a^{-1} = a^{-1} \cdot a = 1 $ \\
8. $ \forall a, b \in R \ ab=ba$ \\
9. $ \forall a,b,c \in R \ a(b+c) = ab+ac $ \\
10 $ 1 \neq 0 $ \\
Кольцо - выполнены 1-5 и 9 \\
Кольцо с 1 - выполнены 1-6, 9 \\
Коммутативное кольцо - выполнены 1-5, 8, 9 \\
Коммутативное кольцо с 1 - выполнены 1-6, 8, 9 \\
Тело, если выполнены 1-7, 9-10 \\
Поле, если выполнены все 10 аксиом \\
Примеры:
$ \Z, +, \cdot $ Кольцо, коммутативное с 1, не поле. \\
$ \mathbb{Q}, +, \cdot $ Поле \\
$ \R, +, \cdot $ Поле \\
$ 2\Z , +, \cdot $ Кольцо, коммутативное без 1 \\
Существуют конечные поля.\\
$ |F| = 2,3,4$ \\
\begin{tabular}{|c|c|c|}
	\hline 
+	& 0 & 1 \\ 
	\hline 
0	& 0 & 1 \\ 
	\hline 
1	& 1 & 0 \\ 
	\hline 
\end{tabular} 
\begin{tabular}{|c|c|c|}
	\hline 
	$\cdot$	& 0 & 1 \\ 
	\hline 
	0	& 0 & 0  \\ 
	\hline 
	1	&  0 & 1 \\ 
	\hline 
\end{tabular} \\
%-----pic2------
$ R $ -кольцо \\
$ (-(0\cdot a)) \mid 0 \cdot a = (0+0) \cdot a = 0 \cdot a + 0 \cdot a $\\
\hspace*{15mm} $ -(0 \cdot a) + 0 \cdot a = -(0 \cdot a) +  0 \cdot a + 0\cdot a$ \\
\hspace*{15mm}$ 0 = 0 + 0 \cdot a = 0 \cdot a $\\
Упражнение: проверить, что заданное $ \mathbb{F}_2 $ - поле \\
\begin{tabular}{|c|c|c|c|}
	\hline 
+	& 0 & 1 & a \\ 
	\hline 
0	& 0 & 1 & a \\ 
	\hline 
1	& 1 & a & 0 \\ 
	\hline 
a	& a & 0 & 1 \\ 
	\hline 
\end{tabular} 
 \begin{tabular}{|c|c|c|c|}
	\hline 
	$\cdot$	& 0 & 1 & a \\ 
	\hline 
	0	& 0 & 0 & 0 \\ 
	\hline 
	1	& 0 & 1 & a \\ 
	\hline 
	a	& 0 & a & 1 \\ 
	\hline 
\end{tabular} \\
%---pic3, 4
F - поле, $ a \in F \setminus \{0\} $ \\
$ F \setminus \{0\} \rightarrow   $\\
$ x \mapsto ax $
Упражнение: проверить, что заданное $ \mathbb{F}_3 $ - поле \\

