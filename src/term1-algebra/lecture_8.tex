$ S = \{a\} $\\
$ (a) - $ идеал, порождённый a, главный идеал\\
$ (a) = \{ ra + ma | r \in R, m \in \Z \} $\\
Если R кольцо с 1, то $ (a) = \{ra\} = Ra $\\
\begin{definition}
	Коммутативная область целостности R - область главных идеалов, если каждый идеал в ней главный. (ОГИ, PID (principal ideal domain))  
\end{definition}
Пример: $ R = \Z $ \\
$ (n) = \Z \cdot n = n\Z $\\
$ n = 0 \ \ (0) $\\
$ n = 1 \ \ \Z $ \\
$ \Z $ - ОГИ \\
$ K[x], K $ - поле - ОГИ \\
Пример: K - поле \\
$ K[x, y] $ \\
$ (x, y) = \{x\cdot f(x, y) + y\cdot g(x, y)  | f, g \in K[x, y] \}  = \{ h \in K[x, y] | h(0, 0) = 0 \}$ \\
$ h = \sum c_{i,j} x^iy^j = \sum_i \sum_j c_{i, j} x^i y^j + \sum_i c_{0, j} y_j $ \\
$ h(0, 0) = 0, h(x, y) = xf(x, y) + yg(x , y) $\\
$ (x, y) $ не главный \\
$ (x, y) = (h) = \{ h \cdot r | r \in K[x, y] \} $\\
$ x \in (h) $\\
$ y \in (h) $ \\
$ h | x \Rightarrow h \in \{ c, cx | c \in K^* \} $\\
$ h | y \Rightarrow h \in \{ c, cy | c \in K^* \} $ \\
$ h = const \neq 0 $\\
$ h(x, y) = xf(x, y) + yg(x, y) $\\
\Subsection{Операции над идеалами} 

R, $ \{ I_{\alpha} \}_{a \in A} \ \ I_{\alpha} -  $ идеалы в R \\
$ \bigcap_{a \in R} I_{\alpha} - $ идеал в R \\
$ \forall \alpha, 0 \in I_{\alpha} \Rightarrow 0 \in  \bigcap_{a \in R} I_{\alpha} $\\
$ a, b \in  \bigcap_{a \in R} I_{\alpha} $ \\
$ \forall \alpha \in A, a, b \in I_{\alpha} $\\
$ \forall a \in A, a \pm b \in I_{\alpha} $\\
$ r \in R, a \in  \bigcap_{a \in R} I_{\alpha} , a\pm b in  \bigcap_{a \in R} I_{\alpha} $\\
Объединение идеалов идеалом быть не обязано \\
$ 2 \Z \cup 3\Z | 3 + (-2) \notin  2 \Z \cup 3\Z$ \\
$ I_1, ..., I_n $ - идеалы в R \\
$ I_1 + ... + I_n = \{ a_1 + a_2 + .. + a_n | a_j in I_j \}$ \\
$ I_1 I_2 = \{a_1b_1 + ... + a_nb_n | a_j \in I_j, b_j \in I_j \} $\\

$ n\Z \cap m\Z  = LCM(n,m)\Z $\\
$ n\Z \cdot m\Z = nm\Z $\\
$ n\Z + m\Z = GCD(n, m) \Z $ \\

\Subsection{Сравнение по модулю идеала} 

R - кольцо с 1\\
I - двусторонний идеал в R \\
$ a \equiv b(I) $ если $ a - b \in I $ \\
$ \equiv $ - область эквивалентности \\
$ a \equiv a(I), a - a = 0 \in I $\\
$ a \equiv b (I) , a - b \in I, b - a \in I, \b \equiv a(I) $ \\
$ a \equiv b(I) , b \equiv c(I) $ \\
$ a - b \in I, b - c \in I, a - c = (a-b) + (b - c) \in I $ \\
$ [a] $ - класс эквивалентности (сравнимости) \\
$ [a] = a + I = \{a_i | i \in I \} $\\
Если $ a \equiv b(I), c \equiv d(I) $ \\
$ a + c \equiv (b + d)(I) $ \\
$ ac \equiv bd(I) $ \\

$ a - b  \in I, c - d \in I, (a+c) - (b+d) \in I \Rightarrow a+c \equiv b+d(i) $\\

$ a - b \in I, c - d \in I, (a-b)c \in I, b(c - d) \in I $\\
$ ac - bc + bc -bd \in I $\\
$ ac \equiv bd (I) $\\

\Subsection{Гомоморфизмы колец} 
$ R_1, R_2 - $ кольца \\
\begin{definition}
	c$ \phi : R_1 \rightarrow R_2 $ называется гомоморфизмом, если \\
	$ \phi(a+b) = \phi(a) + \phi(b)$ \\
	$ \phi(ab) = \phi(a) \phi(b) $
\end{definition}

$ \phi(0) = \phi(0+0) = \phi(0) + \phi(0) \Rightarrow \phi(0_{R_1}) = 0_{R_2} $\\
$ \ker \phi = \{ a \in R | \phi(a) = 0 \} $\\
Ядро гомоморфизма $ \phi $ \\
$  \phi : R_1 \rightarrow R_2 $ - гомоморфизм \\
Тогда $ \ker \phi $ - двусторонний идеал в R \\
\begin{proof}
	$ 0 \in \ker \phi $
	$ a, b \in \ker \phi $ \\
	$ \phi (a \pm b) = \phi(a) \pm \phi(b) \ \ 0+0=0$
	$ a \pm b \in  \ker \phi $\\
	$ a \in \ker \phi $ \\
	$ r \in R$ \\
	$ \phi(ra) : \phi(r) \cdot \phi(a) = \phi(r) \cdot 0 = 0 \Rightarrow ra \in \ker \phi $\\
	$ \phi (ar) = \phi(a) \phi(r) = 0 \cdot \phi(r) = 0 \Rightarrow ar \in \ker \phi $\\ 
\end{proof}

\Subsection{Факторкольцо по двустороннему идеалу}

$ R, I $ - двусторонний идеал в R \\
$ a \equiv b (I) $\\
$ R / \equiv $ - мн-во всех классов сравнимости \\
Введём на $ R/\equiv $ структуру кольца \\
$ [a] + [b] = [a+b] $\\
$ [a][b] = [ab] $\\
Проверка корректности \\
$ [a] = [c], [b] = [d] $ \\
$ [a+b] = [c+d], [ab] = [cd] $\\
$ a+b \equiv c+d(I) $\\
$ ab = cd (I) $\\
%pic1@
$ [0] + [a] = [0+a] = [a] $ \\
$ -[a] = [-a] $ \\
$ [a] + [-a] = [a + (-a)] = [0] $ \\
Если $ R \ni 1 $ \\
$ [1] \cdot [a] = [1 \cdot a ] = [a] $ \\
$ [a] \cdot [1] = [a \cdot 1] = [a] $  
$ R / \equiv, +, \cdot  $ - кольцо \\
Факторкольцо R по идеалу I \\
$ R / I $ \\
$ R = \Z , I = 2\Z $ \\
$ \Z / 2\Z = \{ [0], [1] \} $ \\

$\begin{array}{c|cc}
	+ & [0] & [1] \\
	\hline
	[0] & [0] & [1] \\ \relax
	[1] & [1] & [0] 
\end{array} \ \ \ \ \ 
\begin{array}{c|cc}
* & [0] & [1] \\
\hline
[0] & [0] & [0] \\ \relax
[1] & [0] & [1] 
\end{array}  $\\
Упражнение $ \Z / 3\Z $ \\
Пример $ \Z / 4/Z = \{ [0], [1], [2], [3] \} $ - кольцо с делителями 0 \\
$ [2] \cdot [2] = 0 $ 

$ \phi : R \rightarrow R / I $ \\
$ a \mapsto [a] $ \\
$ \phi(a + b) = [a+b] = [a] + [b] $\\
$ \phi(ab) = [ab] = [a][b] $\\
$ \phi $ - гомоморфизм сюръективный \\
$ \phi(a) = [0]  \Leftrightarrow [a] = [0] \Leftrightarrow a \equiv 0(I) \Leftrightarrow a \in I $\\
Всякий двусторонний идеал в R есть ядро некоторого гомоморфизма \\
\begin{definition}
	$ R = \Z, I = n\Z $ \\
	$ \Z / n\Z $ - кольцо вычетов по модулю n\\
	$ \Z / (n) $
\end{definition}

\Subsection{Максимальные идеалы}

R - коммутативное кольцо \\
$ I \neq R $ называется максимальным идеалом если для любого J т.ч. $ I \subseteq J \subseteq R $ либо $ I = J $ либо $ R = J $ \\
\begin{theorem}
	R - коммутативное кольцо, I - идеал \\
	$ \Leftarrow R / I $ поле $ \Leftrightarrow $ I - максимальный идеал \\
	$ R / I \ni x \neq 0_{R /I} $ \\
	$ x = [a], a \notin I $ \\
	$ I \subsetneqq Ra + I \subseteq R $\\
	$ Ra + I = R $ \\
	$ 1 \in Ra + I $ \
	$ 1 = ra + i, r \in R, i \in I $\\
	$ [1] = [ra + i] = [r] \cdot [a] + [i] = [r][a] = [r]\cdot x $\\
	$ x^{-1}  = [r] \Rightarrow R / I $ - поле \\
	$ \Rightarrow R / I $ - поле \\
	$ I \subsetneqq Y \subseteq R $ \\
	$ a \in Y \setminus I $ \\
	$ [a] \neq [0] $ \\
	$ \Rightarrow \exists r \in R, [r] [a] = [1] $ \\
	$ r - ra \in I $ \\
	$ 1 = ra + i, r \in Y, i \in I \subseteq J \Rightarrow 1 \in Y  \Rightarrow J = R $ 
\end{theorem}
 
 $ 1 \in R - $ комм \\
 $ a | b \Leftrightarrow (a) \supseteq (b) $ 
 $ b \in (b) \subseteq (a) $ \\
 $ b \in (a) $ \\
 $ b = a \cdot c $ 








   