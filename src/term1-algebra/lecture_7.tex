\subsection{title}

$ 1 = \begin{pmatrix}
1 & 0 \\
0 & 1
\end{pmatrix}
I = \begin{pmatrix}
i & 0 \\
0 & -i
\end{pmatrix}
J = \begin{pmatrix}
0 & i \\
i & 0
\end{pmatrix}
K = \begin{pmatrix}
0 & -i \\
1 & 0
\end{pmatrix} $ \\
$ z = c_0 + c_1 I + c_2 J + c_3 K $ \\
$ I^2 = J^2 = K^2 = -1 $ \\
$ IJ = -JI = K $ \\
$ JK = -KJ = I $\\
$ KI = -IK = K $ \\
Вторая конструкция \\
$ \{ ( c_0, c_1, c_2, c_3 ) | c_i \in \R \}$ \\
$ 1 = (1, 0, 0, 0) $ \\
$ i = (0, 1, 0, 0) \\
j = (0, 0, 1, 0) \\
k = (0, 0, 0, 1) $ \\
$ (c_0, c_1, c_ 2, c_3) \cdot (d_0, d_1, d_2, d_3) = \begin{pmatrix}
c_0d_0 - c_1d_1 - c_2d_2 -c_3d_3\\
c_0d_1 + c_1d_0 + c_2d_3 - c_3d_2\\
c_0d_2 - c_1d_3 + c_2d_0 +c_3d_1 \\
c_0d_3 + c_1d_2 - c_2d_1 + c_3d_0 
\end{pmatrix}
$\\
$ \mathcal{H} \subseteq M_2(\mathbb{C}) $ \\
$ \mapsto \mathbb{H} $ \\
$ c_0\cdot 1 + c_1 I + c_2 J + c_3 K \mapsto (c_0, c_1,c_2, c_3) $\\
$ \{ (a_0, a_1, 0, 0) | a_i \in \R \} = \{ a_0 + a_1I \} $ - Изоморфно комплексным числам  $ \backsimeq $\\
$ \{ (a_0, 0, a_2, 0) | a_i \in \R \} = \{ a_0 + a_2О \}  \backsimeq $ \\
$ \{ (a_0, 0, 0, a_3) | a_i \in \R \} = \{ a_0 + a_3Л \} $ \\

$ \det \left( \begin{pmatrix} a & b \\ c & d \end{pmatrix} \cdot \begin{pmatrix} a' & b' \\ c' & d' \end{pmatrix} \right) = \det \begin{pmatrix} a & b \\ c & d \end{pmatrix} \cdot \det \begin{pmatrix} a' & b' \\ c' & d' \end{pmatrix} $ \\
$ \begin{pmatrix} a & b \\ c & d \end{pmatrix} \cdot \begin{pmatrix} a' & b' \\ c' & d' \end{pmatrix} = \begin{pmatrix} aa' + bc' & ab' + bd' \\ ca' + dc' & cb' + dd' \end{pmatrix} $ \\
$ (aa' + bc')(cb' + dd') - (ab' + bd')(ca'+dc') ?=(ad-bc)(a'd' -b'c') $\\ % pic1
$ \mathcal{H} =  \left\{\begin{pmatrix} z_1 & - \overline{z}_2 \\ z^2 &  \overline{z_1} \end{pmatrix} | z_i \in \mathbb{C}  \right\} $ \\
$ z_1 = a_0 + a_1 i, z_2 = a_3 + a_2 i $ \\
$ \det \begin{pmatrix}  z_1 & - \overline{z}_2 \\ z^2 &  \overline{z_1}  \end{pmatrix} = z_1 \overline{z_1} + z_2 \overline{z_2} = a_0^2 +a_1^2+a_2^2+a_3^2  $\\
$\sqrt{ a_0^2 +a_1^2+a_2^2+a_3^2 } - $ модуль кватерниона $ a_0 + a_1I + a_2J + a_3K $\\
$ (c_0 + c_1I + c_2J + c_3K) \cdot (d_0 + d_1I + d_2J + d_3 K)  $ \\
$ (c_0^2 + c_1^2 + c_2^2 + c_3^2) \cdot (d_0^2 + d_1^2 + d_2^2 + d_3^2) = \\
(c_0d_0 - c_1d_1 - c_2d_2 -c_3d_3 )^2\\
(c_0d_1 + c_1d_0 + c_2d_3 - c_3d_2)^2\\
(c_0d_2 - c_1d_3 + c_2d_0 +c_3d_1)^2 \\
(c_0d_3 + c_1d_2 - c_2d_1 + c_3d_0 )^2 $ \\
$ z = (c_0, c_1, c_2, c_3) = c_0 + c_1 i + c_2 j + c_3 k $ \\
$ Re(z) = c_0 $ - вещественная часть\\
$ \vec{V} = c_1i + c_2j + c_3k $ - векторная часть \\
$ z = c_0 + \vec{V} $ \\
$ \overline{z} = c_0 - \vec{V} = c_0 - c_1i -c_2j - c_3k $ \\
$ z \cdot \overline{z} = (c_0 + c_1i + c_2j + c_3k) (c_0 - c_1 -c_2j -c_3k) = c_0^2 + c_1^2 + c_2^2 + c_3^2 = | z | $ \\
2-й способ д-ва тождества Эйлера \\
$ \overline{\overline{z}} = z $ \\
$ \overline{z_1 + z_2 } = z_1 + \overline{z_2} $ \\
$ \overline{z_1z_2} = \overline{z_2} \cdot \overline{z_1} $ \\
$ a \in \R \ \ \ \overline{az} = a \cdot \overline{z} $\\
$ |z_1z_2|^2 = z_1z_2\cdot \overline{ z_1z_2} = z_1 z_2 \overline{ z_2} \overline{z_1} =  = z_1 (z_2 \overline{z_2}) \overline{ z_1} = |z_1|^2 \cdot |z_2|^2$ \\ 

$ (c_0 + \vec{V}) (d_0) + \vec{U} = c_0d_0 - \vec{v} \cdot  \vec{u} + c_0 \cdot \vec{u} + \vec{v} \cdot d_0  $ \\%pic2,3

\Section{Теория делимости в коммутативных кольцах}

$ R $ - комм кольцо \\
$ a | b $ a делит b $ \exists c : b = ac $ \\
$ b \divby a  $ b делится на a \\

\begin{properties}
	\begin{enumerate}
		\item  $ a | 0 $
		\item $ a | b , b | c \Rightarrow a | c $ 
		\item $ R > 1 \Rightarrow a | a, a = a \cdot 1 $ \\
		Замечание $ a | b \& b | a \centernot\Rightarrow a = b $
		\item $ a| b, a|c \Rightarrow a | (b \pm c ) $
	\end{enumerate}
\end{properties}
\begin{definition}
	Если $ a \cdot b = 0 $, но $ a, b \neq 0 $ то a и b называются нетривиальными делителями нуля \\
	R - область целостности если нет нетривиальных делителей нуля 
\end{definition}
Замечание: для некомм R : \\
Если $ b = ac $ ,то a - левый, b - правый делитель \\

$ R > 1 - $ комм \\
$ R^* = \{ a \in R : a | 1 \} = \{ a \in R : \exists b, ab=1 \} $ - мн-во обратимых элементов \\
$ R^*m \cdot $ - группа \\
$ R^* - $ мультипликативная группа кольца \\
$ R^* \times R^* \rightarrow R^* $\\
%pic5
$ a \in R^* \exists b ab=1 \Rightarrow ab = ba = 1 \Rightarrow b \in R^* $\\
Примеры: $ \Z^* = \{\pm 1\} $ \\
K - поле $ K^* = K \setminus \{0\} $\\
К - поле $ (k[x])^* = K^* $ \\
Упр. 1 $ \Z[i] = \{ a + bi | a,b \in \Z \} \subseteq \mathbb{C} $ - Гауссовы числа\\
Д-те что $ \Z[i] $ - кольцо и найдите $(\Z[i])^* $\\
Упр. 2 $ \omega = \dfrac{-1 + \sqrt{3} i}{2}, \omega^3 = 1 $ \\
$ \Z[\omega] = \{ a + b\omega | a,b \in \Z \} $ - Кольцо чисел Эйзенштейна \\
$ d \in \Z $ не явл целым квадратом \\
$ \Z[\sqrt{d}] \subseteq \mathbb{C} $ если $ d < 0 $ \\
$ \hspace*{12mm} \subseteq \R $ если $ d > 0 $ \\
$ \Z[\sqrt{d}] = \{ a + b\sqrt{d} | a, b \in \Z  \} $ \\
Каждый эл-т из $ \Z[\sqrt{d}] $ единственным образом представляется в виде $ a + b\sqrt{d}, a,b \in \Z $ \\
5. Д-те, что $ (\Z[\sqrt{2}])^*, (Z[sqrt{3}])^*, (\Z[\sqrt{5}])^*, (\Z[\sqrt{6}])^* $ - бесконечны \\ % pic6

\Subsection{Ассоцированность} 

R - кольцо коммут, $ 1 \in R $ \\
a ассоцировано с b если $ a | b \& b | a $ \\
$ a \sim b $\\
$ \sim $ - отнош эквивалентности \\
\begin{theorem}
	1. Если $ \varepsilon \in R^* $, то $ a \sim a\eps $\\
	2. Если R - область целостности, $ a \sim b \Rightarrow \exists \eps \in R^*, b = a\eps $
	\begin{proof}
		1. %pic
		2. $ a | b \& b | a $ \\
		$ a = 0 \Rightarrow b = 0, 0 = 0 \cdot 1 \Rightarrow \eps = 1	$ \\
		$ a \neq 0, a | b \ \exists \eps \in R, b = a\eps $ \\
		$ b | a \ \exists r \in R \Rightarrow a = r \cdot b $ \\
		$ a = r \eps a  $ \\
		$ 0 = a (1 - re) \Rightarrow 1 - r \eps = 0 $ \\
		$ \eps | 1, \eps \in R^* $
	\end{proof}
\end{theorem}
Примеры \\
$ \Z, Z^* = \{\pm 1 \}$ \\
$ \{0\}, \{\pm 1\}, \{ \pm 2 \} $ - классы ассоцированности \\
$ K[X], K - $ поле $ \{0\} $ \\
Любой другой класс ассоцированности содержит ровно 1 многочлен со старш коэфф 1(приведённый, унитарный) \\

\Subsection{Идеал в кольце}

R - произвольное кольцо\\ %pic
$I = R$ \\
$ I = \{0\} $ - идеалы \\ 
Для некомм колец различают левые и правые идеалы \\
1. остаётся \\
2(левый). $ \forall a \in I, \forall r \in R, ra \in I, R\cdot I \subseteq I $ \\ 2(правый). $ \forall a \in I, \forall r \in R, ar \in I,  I \cdot R \subseteq  (\sum r_is_i + \sum m_js_j)I $ \\

Замеч. в усл 1 $ \forall a,b \in I, a \pm b \in I $\\
$ 1' : \forall a,b \in I, a-b \in I , \\
1' \Rightarrow 1 $ I - непусто \\
$ a \in I, I \neq 0 $ \\
$ a - a \in I, 0 \in I $\\
$ a \in I, 0 \in  I, 0 - a \in I, -a \in I $\\
$ \forall a,b, -b \in I, a - (-b) \in I $\\

$ R, S \subseteq R $ \\
Какой наименьш идеал содержит S \\
$ s \in S \subseteq I $ \\
$ \forall r \in R, r \cdot s \in I $\\
$ \forall m \in \Z, m \cdot s \in I $ \\
$ \{ \sum_{i=1}^{n} r_is_i + \sum_{j=1}^{k} m_js_j | n, k \in \N \cup \{0\}, r_i \in R, m_j \in \Z \} $ - это мн-во образует идеал \\
а значит это и есть минимальный идеал, содержащий S \\
$ (\sum r_is_i + \sum m_js_j) \pm  (\sum r'_is'_i + \sum m'_js'_j) - $ вновь сумма такого же вида \\
Идеал, порождённый мн-вом S $ (S) $ \\












