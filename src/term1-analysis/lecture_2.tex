
\begin{theorem} Принцип вложенных отрезков\\

Пусть $A_i = [a_i, b_i], i = 1,2,3,\dots$ - отрезок \\
$ a_i, h \in \R, a_i < b_i $ и $ A_i \supset A_{i+1} , i=1,2,\dots $ \\
Тогда $ \bigcap_{i=1}^{\infty} A_i \neq \emptyset $\\
\begin{proof} $ X = \{ a_i | i=1,2\dots \}, Y=\{b_i | i = 1,2,\dots\} $ \\
$ \forall x \in X, \forall y \in Y: x \leq y $ \\
$ x = a_i \in N $\\
$ y = b_j  \in N$ \\
$ \exists i \leq j \ a_i \leq a_j < b_j \ a_i \leq a_{i+1} \leq a_{i+2} $ \\
$ \exists i > j \ a_i < b_i \leq b_j \ b_j \geq b_{j+1} \geq b_{g+2} $\\
Т. е. $ a_i < b_j $ \\
$ \exists c \in \R : \forall x \in X : x \leq c $ \\
$ \hspace*{16mm} \forall y \in Y : y \geq c $ \\
$ a_i \leq c \leq b_i $ т. е. $c \in A_i$ \\ 
$\Rightarrow \bigcap_{i=1}^{\infty} A_i \neq \emptyset$ \\
$ \bigcap_{i=1}^{\infty} A_i $ - точка или отрезок \\
\end{proof}
Аналогичное утверждение не верно для интервалов. $ A_i = (0, \frac{1}{i} ]$ \\
Аналогичное утверждение не верно для $\mathbb{Q}$ - $ \{a_i\} = \{1, 1.4, 1,41, \dots\}   \ \{b_i\} = \{2, 1.5, 1,42, \dots\},   \bigcap_{i=1}^{\infty} A_i = \sqrt{2} = \emptyset$
\end{theorem}

\Subsection{Мощность множества}

\begin{definition}
	Множества называются равномощными если существует биекция $ f: X \rightarrow Y $. Равномощность - отношение эквивалентности.\\
\end{definition}

\begin{definition} $ \sim $ - отношение эквивалентности \\
	\begin{enumerate}
		\item $X \sim X \ id_X$
		\item $ X \sim Y \Rightarrow Y \sim X$ 
		\item $ X \sim Y, Y \sim Z \Rightarrow X \sim Z $ 
		\item $ f: X \rightarrow Y, g: Y \rightarrow Z $ - биекции $\Rightarrow g \circ f $ - биекция 
	\end{enumerate}
\end{definition}
Mощность конечного множества - неотрицательное целое число. \\

\begin{definition}
	Множество X называется счётным, если оно равномощно множеству $ \N $
\end{definition}

\begin{example} $ $ \\
	$ \N \setminus \{1\} $ - счётное \\
	$ \N \setminus \{1\} \mapsto \N $\\
	$ x \mapsto x - 1 $ 
\end{example}

\begin{definition}
	Собственное подмножество - подмножество отличное от всего множества. 
\end{definition}

\begin{theorem}
	Любое подмножество счётного множества конечно или счётно \\
	$ x = \{ x, x_1, x_2, x_3, \dots | x_i \neq x_j, i \neq j \}$ \\
	\begin{proof}
		Пусть $ Y \subset X $ \\
		Y конечнно - очевидно \\
		Y бесконечно $ y_1 = x_{i_1}, i_1 $ - минимально \\
		$ y_2 = x_{i_2} $ т.ч. $ x_{i_2} \in Y \setminus \{y_1\} $ и $ i_2 $ - минимально\\
		$ y_3 = x_{i_3} $ т.ч. $ x_{i_3} \in Y \setminus \{y_1, y_2\} $ и $ i_3 $ - минимально\\
		И т.д. Все $ Y \setminus \{y_1, \dots, y_n \} $ непусты т.к. Y бесконечно. \\
		Проверим $ Y = \{y_1, y_2, \dots, y_n \} $ \\
		Пусть $ y \in Y \setminus \{ y_1, \dots, y_n \} $ \\
		$ \exists j : y = x_j $ \\
		$ y \in Y \setminus \{ y_1, \dots, y_n \} $ при любом n \\
		$ \Rightarrow \forall n \in \N, i_n < j $ В силу процедуры выбора $ y_n $ \\
		$\underbrace{i_1, i_2, \dots, i_j}_{\in \N} < j $ - не бывает \\
		$ \N \rightarrow Y - $ биекция
	\end{proof}
\end{theorem}


Мощность X не превосходит мощности Y если X равномощно подмножеству Y. $ |X| \leq |Y| $ \\
Равносильно инъекции $X \rightarrow Y$ \\
$ |X| \leq |\N \Rightarrow |X| \in \N_0 \mid  |X| = |N| $\\
$ |X| \leq Y, |Y| \leq |Z| \Rightarrow |X| \leq |Z|$ \\
Задача: доказать, если $|X| \leq |Y|, |Y| \leq |X| $, то $ |X| = |Y|$ \\
\begin{example}
	$\Z - $ cчётное \\
	-2, -1, 0, 1, 2, 3\\
	5, 3, 1, 2, 4, 6 \\
	$ \N \rightarrow \Z $ \\
	$ n \mapsto \frac{n}{2}, 2 \divby n \\
	-\frac{n-1}{2}, 2 \divbynot n $ \\
\end{example}
\begin{example}
	X, Y - счётное Тогда $ X \times Y - $ счётное. \\
\end{example}
$ X \times Y = \bigcup_{i=1}^{\infty}\{(x_m, y_n)\mid m+n=i \} $ \\
$ |C_i| = i - 1 $ \\
Элементы $C_i$ получают номера заканчивающиеся на $ \frac{(i-2)(i-3)}{2}+1 \dots \frac{(i-1)(i-2)}{2} $ \\
$ \mathbb{Q}$ - счётно
$ \mathbb{Q} \rightarrow \Z \times \R$ \\
$ q \mapsto (a, b) $ \\
$ \alpha $ инъективно(очевидно) \\
$ |\mathbb{Q} \leq |\Z \times \N | = |\N| = \aleph_0 $\\
$ |X| < |Y| \Leftrightarrow |X| \leq |Y|, |X| \neq |Y| $ \\
X - любое множество, $ |X| < |2^X| $ (Упражнение)\\
Любые 2 промежутка в $ \R - $ равномощны \\
Cперва докажем, что $ [a, b] \sim [a', b'], a < b, a' < b' $ \\
$ \exists $ линейная ф-ция $ y = f(x)$ т.ч. $ f(a) = a', f(b) = b' $ Она даёт биекцию $ [a, b] \rightarrow [a', b']$\\
Аналогично: $  [a, b) \rightarrow [a', b')$ \\
$  (a, b] \rightarrow (a', b']$\\
$ (a, b) \rightarrow (a', b')$\\
Проверим $ [0, 1) \sim [0,1] $ \\
$ x \mapsto \left\{ \begin{array}{ll} x, x \neq \frac{1}{n}, n \in \N \\ \dfrac{1}{n-1}, x = \dfrac{1}{n} \end{array} \right. $\\
Аналогично $ (0, 1] \sim [0,1] $\\
$ (0, 1) \sim (0,1] $\\
$\R$ равномощно $ [0, 1] $ \\
$ \left( -\frac{\pi}{2}, \frac{\pi}{2} \right] \mapsto \R \\
x \mapsto tg(x) $\\
$ \left( -\frac{\pi}{2}, \frac{\pi}{2} \right]  \sim [0, 1] $ \\
\begin{theorem}
	$\R$ - несчётно \\
	\begin{proof}
		Достаточно доказать, что $ (0, 1) $ - несчётный.\\
		Предположим, что это не так\\
		$ (0, 1) = \{ a_1, a_2, \dots \} $ \\
		$ a_i = 0,a_1a_2a_3a_4 $ Без 9 в периоде \\
		$ \beta_i $ - любая цифра, отличная от $ a_{ii}, i=1,2,3,\dots$ 0 и 9\\
		$ \beta_i = 0,\beta_1\beta_2\dots$\\
		$\beta \notin \{ a_1, a_2, a_3, \dots \} $\\
		Число $ \beta$ нельзя пронумеровать 
	\end{proof}
\end{theorem}

Континуум-гипотеза - любое несчётное подмножество  $\R$  равномощно $\R$ \\



