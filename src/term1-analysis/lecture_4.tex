\begin{theorem}
	Пусть $x_n$ последовательность, $ x_n > 0, \lim \dfrac{x_{n+1}}{x_n} \leq 1 \Rightarrow \lim x_n = 0 $
	\begin{proof}
		Пусть $ c $ такое число, что $  \lim \dfrac{x_{n+1}}{x_n} \leq c \leq 1 $ \\
		$ \exists N, \forall n \geq N,  \dfrac{x_{n+1}}{x_n} < c $ \\
		$ \forall m \in \N :  \dfrac{x_{n+1}}{x_n} = \prod_{i=0}^{m-1}  \dfrac{x_{n+1}}{x_n} < c^m $ \\
		$ 0 < x_{n+m} < c^mx_n \underset{m \rightarrow \infty}{\rightarrow} 0 $ \\
		$ \Rightarrow \lim\limits_{n \rightarrow \infty} x_{n+m} = 0 =  \lim\limits_{n \rightarrow \infty} x_{n} $

	\end{proof}
	\begin{consequence}
		$ a > 1 \Rightarrow \dfrac{n^k}{a^n} \rightarrow 0 $
		\begin{proof}
			$ x_n = \dfrac{n^k}{a^n} $ \\
			$ \dfrac{x_{n+1}}{x^n} = \dfrac{(n+1)^k}{n^k \cdot a} = (1 + \dfrac{1}{n})^k \cdot \dfrac{1}{a} \rightarrow \dfrac{1}{a} \leq 1 $
		\end{proof}
	\end{consequence}
\end{theorem}
\begin{definition}
	Пусть $ x_n, y_n $ бесконечно большие. Говорят, что $ x_n $ бесконечно большая меньшего порядка, если $ \dfrac{x_n}{y_n} \rightarrow 0, x_n = O(y_n) $ \\
	$ n^k = O(a^k) $ 
\end{definition}
\begin{consequence}
$ \dfrac{a^n}{n!} \rightarrow 0 \forall a > 0 $ \\

$ \dfrac{x_{n+1}}{x_n} = a \cdot \dfrac{1}{n+1} \rightarrow 0 $ 
\end{consequence}
\begin{consequence}
	$ \dfrac{n!}{n^n} \rightarrow 0 $ \\
	\begin{proof}
		$ \dfrac{x_{n+1}}{x_n} = (n+1)\dfrac{n^n}{(n+1)^{n+1}} = \dfrac{n^n}{(n+1)^n} = (\dfrac{n}{n+1})^n = \dfrac{1}{(1+\frac{1}{n})^n} $
	\end{proof}
	
\end{consequence}

\Section{Теорема Больцано-Вейерштрасса и критерий Коши}

\begin{theorem}
	Теорема о стягивающихся отрезках \\
	Пусть заданы отрезки числовой прямой $ [a_i, b_i], i = 1, 2, 3,...$ \\
	\begin{enumerate}
		\item $ [a_i, b_i] \supset [a_{i+1}, b_{i+1}] $ 
		\item 
		% pic1
	\end{enumerate}
	\begin{proof}
		Знаем $ \cap [a_n, b_n] \neq \emptyset $ \\
		Предположим, что это не 1 элем. мн-во \\
		Тогда $ \exists c, d \in  \cap [a_n, b_n], c < d $ \\
		$ \forall n \in \N : c, d \in  [a_n, b_n] \Rightarrow a_n \leq c < d \leq b_n  \Rightarrow b_n - a_n \geq $
		% pic2,3
	\end{proof}
\end{theorem}
\begin{definition}
	Говорят, что $ b_n $ подпоследовательность $ a_n $ если \\
	возр. посл-ть натуральных чисел $ m_i $ \\
	$ \forall n \in \N$
	
\end{definition}
\begin{theorem}
	Теорема Больцано-Вейерштрасса \\

	Пусть $ x_n$ - ограниченная последовательность. Тогда в $ x_n $ можно выбрать сходящуюся подпоследовательность
	\begin{proof}
	$ (x_n) \Rightarrow \exists a, b \in \R : a \leq b, \forall n : x_n \in [a, b] $ \\
	Из отрезков $ [a, a+b / 2] $ и $ [a+b/2, b] $ выберем $ [a_1, b_1] $ такой, что $ \{i \mid x_i \in [a_1, b_1]\} $ бесконечно. \\
 	Из отрезков $ [a_1, a_1+b_1 / 2] $ и $ [a_1+b_1/2, b_1] $ выберем $ [a_2, b_2] $ такой, что $ \{i \mid x_i \in [a_2, b_2]\} $ бесконечно. \\
    $ b_i - a_i = \frac{1}{2^n} (b-a) $ \\
    По теореме о стягивающихся отрезках 
    $ \cap [a_n, b_n] = \{c\}$ \\
    $ m_1 $ такое число, что $ x_{m_1} \in [a_1, b_1] $ \\
    $ m_2 $ такое число, что $ x_{m_2} \in [a_2, b_2] $ \\
    ...\\
    $ x_{m_i} $ - подпоследовательность \\
    $ a_i \leq x_{m_i} \leq b_i  \Rightarrow x_m \rightarrow c$ 
    \end{proof}
	Зам. Легко видеть, что  если $ x_n \rightarrow c \in \R $ \\
	То любая подпоследовательность $ x_{m_i} \rightarrow c $ \\
\end{theorem}
Дополнение \\
Пусть $x_n$ - неограниченная последовательность. Тогда в ней есть подпоследовательность $ x_{n_i}  \rightarrow +\infty $ или $ \rightarrow -\infty $ \\
\begin{proof}
	Пусть $ x_n $ не ограничена сверху \\
	Тогда легко видеть, что $ \forall n \in \N $ существует бесконечно много $ i \in \N : x_i > n $ \\
	Иначе существуеть лишь конечн. i \\
	Пусть это так, тогда $ max(x_1, x_2, ..., n) $ - верхняя граница $x_n$\\
	$ m_1 $ - любое нат. число, $ x_{m_1} > 1 $ \\
	$ m_2, x_{m_2} > 2, m_2 > m_1 $ \\
	$ (x_{m_i})  -$ посл-ть \\
	$ \{i \mid x_{m_i} \notin (n, +\infty) \} \subset \{ 1,2,..., n-1 \} $ \\
	То $ x_{m_i} \rightarrow +\infty $\\
	Аналогично для $ -\infty $ 
\end{proof} 
\begin{consequence}
	%pic5
\end{consequence}

\begin{definition}
	Пусть $x_n$ - последовательность. Она называется фундаментальной, если выполненое след. св-во \\
	$ \forall \eps,  \exists N \in \N, \forall m,n \geq N : | x_m - x_n | < \eps $\\
	Очевидно, сходящаяся последовательность фундаментальна. Можно взять предел и окресность $ \frac{\eps}{2} $ \\
	Фундаментальная $=$ сходящаяся в себе $=$ последовательность Коши.
\end{definition}
\begin{theorem}
	Теорема Больцано-Коши(Критерий Коши) \\
	Последовательность $x_n$ фундаментальна $ \Leftrightarrow $ сходится
	\begin{proof}
		Пусть $ (x_n) $ - фундаментальна \\
		1. Докажем, что $(x_n)$ - ограничена \\
		По определению, $  \exists N \in \N \forall n \geq N : x_n \in (x_{N} - 1, x_N +1)$\\
		$ a = min(x_N - 1, x_n, ..., )$ %pic6
		2. По т. Б-В, $ (x_n)$ содержит сх. подпоследовательность $(x_{m_i}) $ \\
		Пусть $ x_{m_i} \rightarrow c $ \\
		Д-м, что $ x_n \rightarrow c $ \\
		По опр фунд. посл $ \exists N \in \N  : \forall m, n \geq N : |x_m-x_n| < \frac{\eps}{2}$ \\
		$\exists N' \in \N : \forall i \geq N' : |x_{m_i} -c | < \frac{\eps}{2} $ \\
		Пусть i таково, что %pic7,8,9,10
		
	\end{proof}
\end{theorem}
I R как множество дедекиндовых сечений\\
2 R как множество классов посл-тей коши\\
M - мн-во последовательностей Коши рац. чисел %pic11

