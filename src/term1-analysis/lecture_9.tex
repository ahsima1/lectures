$ a^b\ $ \\
$ a > 0, b \in \R $ \\
$ 0^{-x} = \dfrac{1}{0^x} $ \\
$ a < 0, b = \dfrac{p}{q} $\\
$ \sqrt[q]{x} - $ обр ф-ция к 	$ y = x^q $ 
\begin{lemma}
	$ \lim\limits_{x \rightarrow 0} a^x = 1 \ \ a > 0 $ \\
	$ \lim\limits_{n \rightarrow \infty} a^{\frac{1}{n}} = 1 $ \\ 
	$ x \in [-1; 1] \Rightarrow \left\{ \begin{array}{l}
		a^x \in [ a^{-1}; a ], a > 1 \\ a^x \in [a; a^{-1}], a < 1 
	\end{array} \right. $\\
	$ y = a^x $ %pic1
\end{lemma} 
\begin{theorem} Пусть $ a > 0 $ тогда $ y = a^x $ непрерывна на $ \R $ \\
	$ \lim\limits_{x \rightarrow x_0} a^x = \lim\limits_{x \rightarrow x_0}  a^{x_0} \cdot a^{x-x_0} = a^{x_0} \cdot \lim\limits_{x \rightarrow x_0} a^{x-x_0} = a^{x_0} $ 
\end{theorem}
Предл Пусть $ a > 0, a \neq 1 $\\
Тогда $ y, y=a^x $ суш обр ф-ция $ \log_a : (0, +\infty) \rightarrow \R $ которая непрерывна и строго монотонна \\
Нужно проверить, что $ Im \ f = (0, +\infty) $

Из св-в возведения в степень \\
$ \log_a x= \dfrac{\ln x}{\ln a} $\\
$ f(x) = a^x $ - показательная ф-ция \\
$ f(x) = x^t $ - степенная ф-ция \\
$ x^t = e^{t \cdot \ln x}$ \\
$ \Rightarrow f(x) = x^t $ - непрерывная, возраст при $ t > 0 $, убывает при $ t < 0 $ \\

\Subsection{Эквивалентные ф-ции и замечательные пределы}

Пусть $ f, g : E \rightarrow \R, x_0 - $ пред. точка E, f,g экв-ны в окресности $ x_0 $ если $ \exists \dot{U}(x_0) $ и $ \phi : \dot{U} (x_0) \cap E \rightarrow \R $ т.ч. $ g(x) = f(x) \cdot \phi(x), x\in \dot{U} (x_0) \cap E $ \\
Если $ \exists \dot{U}(x_0) $ т.ч. $ \forall x \in \dot{U} (x_0) \cap E : g(x_0) \neq 0 $\\
То. $ f \sim_{x \rightarrow x_0} g \Leftrightarrow \dfrac{ f(x)}{g(x)} \rightarrow_{x \rightarrow x_0} 1 $ \\
 $ \Rightarrow $ Очев $ \phi (x) \neq 0, \forall x \in U(x_0) \cap E \Rightarrow \dfrac{f(x)}{g(x)} = \dfrac{1}{\phi (x)} \rightarrow 1 $ \\
$ \Leftarrow \dfrac{f(x)}{g(x)} \rightarrow_{x \rightarrow x_0} 1\Rightarrow g(x) = f(x) \cdot \dfrac{g(x)}{f(x)} \rightarrow 1 $ \\
$ g(x) \neq 0 $ в окр $ x_0 \Rightarrow f(x) \neq 0 $ в окр $ x_0 $\\
Св 1. $ f \sim g \Rightarrow g \sim f $ \\
2. $ f \fim g , g \sim h \Rightarrow f \sim h $\\
$ g(x) = f(x) \cdot \phi(x) $
$ h(x) = g(x) \cdot \psi(x) \Rightarrow h(x) = f(x) (\phi(x)\psi(x))_{\Rightarrow 1} $\\
1. Пусть $ g(x) = f(x) \cdot \phi(x), \phi(x) \rightarrow 1$%pic2
3. 

%pic3-16

Предл. 4 $ \lim\limits_{x \rightarrow 0} \dfrac{(1+x)^p-1}{x} = p \ \ p \in \R $ \\
$   \dfrac{(1+x)^p-1}{x} = \dfrac{\ln(1+x)}{x}_{\rightarrow 1} \cdot \dfrac{(1+x)^p - 1}{\ln((1+x)^p)}_{\rightarrow 1} \cdot \dfrac{\ln((1+x)^p)}{\ln(1+x)}_{=p} \rightarrow p$ \\
Предл. 5 $ \lim\limits_{x \rightarrow 0} \dfrac{a^x-1}{x} = \ln a $ \\
$ \dfrac{a^x-1}{x} = \dfrac{a^2 - 1}{\ln(a^x)}_{\rightarrow 1} \cdot \dfrac{\ln(a^x)}{x}_{=\ln(a)} $
\begin{lemma}
	Пусть $ f <a,b> \rightarrow <c,d> $ \\
	$ x_0 \in <a,b>, \lim\limits_{x \rightarrow x_0} = y_0 \in <c,d> $ \\
	$ g <c, d> \rightarrow \R $ непрерывна \\
	Тогда $ \limits_{x \rightarrow x_0} (g \circ f)(x) = A $ \\
	$ \lim\limits_{x \rightarrow x_0} g(f(x)) = g(\lim\limits_{x \rightarrow x_0} f(x)) $ 
\end{lemma}

\Subsection{Сравнение функций}

$ f,g : E \rightarrow \R, a - $ пред тчк E \\
Пусть $ \exists \dot{U} (a) $ и ф-я $ \phi E \rightarrow \R $ \\
$ f(x) = \phi(x) g(x) $ на $ E \cap \dot{U} (a) $ \\
$ \phi(x) \rightarrow_{x \rightarrow a} 0 $ \\
Говорят $ f = o(g) $\\
Предл сл усл экв (при $ x \rightarrow a $) \\
1. $ f \sim g $ \\
2. $ f = g + o(g) $\\
3. $ f = g + o(f) $\\
$1 \Rightarrow 2 \ \ f(x)= \phi(x) g(x) \ \ \phi(x) \rightarrow 1 $ \\
$ = g(x) + (\phi(x) - 1)g(x) $ \\
$ 2 \Rightarrow 1 \ \ $ ан-но \\
$ 1 \Rightarrow 3 \ \ $ ан-но \\
Примеры \\
$ \sin(x) - x = o(x) $\\
$ \sin(x) = x + o(x) $ \\
$ \ln(1+x) = x + o(x) $ \\
$ \cos(x) = 1 - \dfrac{x^2}{2} + o(x^2) $ \\
Предл \\
1. $ f \cdot o(g) = o(fg) $ \\
2. $ f \sim g  \Rightarrow o(f) = o(g) $ \\
1. $ o(g) (x) = g(x) \phi(x) \ \ \phi(x) \rightarrow 0 $ \\
$ f(x) \cdot g(x) \cdot \phi(x) = o(fg) $\\
2. $ h= o(f) $ \\
$ h(x) = f(x) \phi(x), \phi(x) \rightarrow 0 $ \\
$ f(x) = g(x) \psi (x), \psi(x) \rightarrow 1 $ \\
$ \Rightarrow h(x) = g(x) (\phi(x) \psi(x))_{\rightarrow 0} = o(g) $

\begin{definition}
	$ f, g : E \rightarrow \R $ a - пред тчк E \\
	Eсли сущ $ \dot{U} (a) $ и \\
	$ \phi : \dot{U} (a) \cap E \rightarrow \R+ $ \\
	$ f(x) = \phi(x) g(x) \forall x \in \dot{U} (a) \cap E $ \\
	то $ f = O(g) ( $ при $  x \rightarrow a) $ \\
	% pic17
\end{definition}

\Section{Производная}

\begin{definition}
	Пусть $ f : <a, b> \rightarrow \R $ \\
	f называется дифференцируемой в $ x_0 \in <a, b> $ \\
	если $ \exists k \in \R $ \\
	$ f(x) = f(x_0) + k(x - x_0) + o(x - x_0) $ при $ x \rightarrow x_0 $\\
	k - производная f в точке $ x_0 $
	\begin{lemma}
		Если f дифф в $ x_0 $ то $ f'(x_0) $ корректно определена 
		\begin{proof}
			$ f(x) = f(x_0) + k(x - x_0) + o(x - x_0) $ \\
			$ = f(x_0) + \tilde{k}(x - x_0) + o(x - x_0) $ \\
			$ \Rightarrow k(x-x_0) + o(x-x_0) = \tilde{k} (x -x_0) + o(x-x_0) $\\
			$ \Rightarrow k+\dfrac{o(x-x_0)}{x - x_0} = \tilde{k} + \dfrac{o(x-x_0)}{x-x_0} $ 
		\end{proof}$ \lim\lim\limits_{y \rightarrow y_0} g(y) = A $\\
	\end{lemma}
\end{definition}
\begin{definition}
	Произв $ f<a,b> \rightarrow \R $ в $ x_o \in <a,b> $ \\
	наз $ \lim\limits_{x \rightarrow x_0} \dfrac{f(x) - f(x_0)}{x - x_0} = f'(x_0) $ \\
	Если $ f'(x_0) \in \R$, то дифф 
\end{definition}



