

\Subsection{Язык теории множеств}


\begin{definition}
	Конечные множества \\
	$ \{ 3, 1, 17 \} $ \\
	$ A = \{ 1, 2, 3... 100 \} $ \\
	$ \{ n^2 \mid n = 1, 2, ..., 50 \} $ \\
	$ \{ n \in A \mid n \divby 3 \} $  \\
	
\end{definition}

\begin{definition}Подмножество \\ 
	$ B \subset A \Leftrightarrow \forall b \in B : b \subset A $ \\
\end{definition}

\begin{definition} P - предикат \\
	$ A = \rm \mathbb{Z} $ \\
	$ P(n) - n^2 \leq 10 $ \\
	Множество А, Р - предикат на А \\
	$ \forall x \in A : P(x) $  \\
	$ \exists x \in A : P(x) $ \\
\end{definition}

\begin{definition} Квантор \\
	Квантор всеобщности: $ \forall $ \\
	Квантор существования: $ \exists $ \\
\end{definition}
\begin{definition} Операции над множествами \\ 
	$ A, B $ - множества \\
	$ A \cap B = \{ a \in A \mid a \in B \} $\\
	$ A \cup B = \{ a \in M \mid a \in A,  a \in B \} $ \\
	$ A \setminus B = \{a \in A \mid a \notin B \}  $ \\
	$ A \oplus B = ( A \setminus B ) \cup ( B \setminus A ) $ \\
	$ A \times B = { (a, b) \mid a \in A, b \in B } $ \\
\end{definition}

\begin{definition}
	Отношения \\
	$ a, b \in \mathbb{Z} $ \\
	$ a \divby b $ \\
	$ a \divbynot b $ \\
	Отношение делимости\\
\end{definition}

\begin{definition} Декартов квадрат \\
	$ M = \{1, 2, 3, 4\} $\\
	$ |M \times M | = 16 $ - декартов квадрат (произведение множества на себя)\\
	$ |x| $ - Мощность мн-ва Х (кол-во элементов в конечном мн-ве)\\
	$ \divby R \subset M \times M $\\
	$ R =  \{ (1, 1), (2, 1), (2, 2), (3, 1), (3, 3), (4, 1), (4, 2), (4, 4) \}$ \\
	$ |M| = n \rightarrow |2^M| = 2^n $ - количество подмножеств на множестве
	$ 2^{(n^2)} $ - количество отношений на множестве \\
\end{definition}

\noindent
$ \mathbb{N} $ - Натуральные числа\\
$ \mathbb{Z} $ - Целые числа\\
$ \mathbb{Q} = \{ \frac{a}{b} \mid a \in \mathbb{Z}, b \in \mathbb{N} \} $ - Рациональные числа\\
$ \mathbb{R} $ - Вещественные числа\\
$ \mathbb{C} $ - Комплексные числа\\

\subsection{Отображения}

\begin{definition} Отображение \\
X, Y - множества\\
$ f: X \rightarrow Y $ f - отображение из X в Y \\
Каждому $ х \in X $ сопоставлен $ y \in Y $ \\
$ X = \{ x_1, ..., x_n \} $ \\
$\hspace*{11mm} y_1, ..., y_n $\\
$ |X| = 3, |Y| = 5 $\\
$ f: X \rightarrow Y $\\
Количество отображений f - $ Y^X = 5^3 $ \\
$ X = Y = \mathbb{N} $ \\
$ f(x) = 3x + 5 $ \\
$ \mathbb{N} \Rightarrow \mathbb{N} $ \\
$ x \Rightarrow 3x+5 $ 
$ f(x) = |\{ d \in \mathbb{N} \mid x \divby d \}| $ 
\end{definition}

\begin{definition}
Пусть задано $ f : X \Rightarrow Y $ \\
$ X_0 \subset X $ \\
$ f(X_0) = \{ f(x) \mid x \in X_0 \} \in Y $ - образ подмножества \\
$ f^{-1}(y) = \{ x \in  X \mid f(x) = y \} $ - прообраз элемента \\
$ f^{-1}(Y_0) = \{ x \in X \mid f(x) \in Y_0 \} $  - прообраз множества 
\begin{example}
	$ f: \mathbb{Z} \Rightarrow \mathbb{Z} $ \\
	$ a \Rightarrow a^2 $ \\
	$ f( \{ -1, 0, 1, 2 \} ) = \{ 0, 1, 4 \} $ \\
	$ f^{-1} (1)= \{-1, 1\}$ \\
    $ f^{-1} (0)= \{0\}$ \\
	$ f^{-1} (2)= \varnothing$ \\
	$ f^{-1} (\{0,1\})= \{-1, 0, 1\}$ 
\end{example}
\end{definition}



\begin{definition}
Отображение $ f: X \Rightarrow Y $ называется сюръективным (сюръекцией), если $ f(X) = Y $ $ \forall y \in Y \exists x \in X : f(x) = y $ \\
f сюръекция $ \Leftrightarrow  \forall y \in Y |f^{-1}(y)| \geq 1 $ \\
\end{definition}
\begin{definition}
Отображение $ f: X \rightarrow Y $ называется инъективным (инъекцией), если $ \forall x_1, x_2 \in X : x_1 \neq x_2 \rightarrow f(x_1) \neq f(x_2) $ \\
f инъекция $ \Leftrightarrow  \forall y \in Y |f^{-1}(y)| \leq 1 $ 
\end{definition}
\begin{definition}
Отображение $ f: X \rightarrow Y $ называется биективным (биекцией), если оно сюръективно и инъективно 
\end{definition}

\begin{definition}
Пусть X, Y, Z - мн-ва, $f: X \rightarrow Y $ и $ g: Y \rightarrow Z $ \\
Композиция отображений $ g \circ f $ $ x \mapsto g(f(x)) $ 
\end{definition}

\begin{definition}
Тождественное отображение \\$ id_x : X \rightarrow X $ \\
$\hspace*{10mm} x \mapsto x $
\end{definition}

\begin{properties}
1. Композиция инъективных отображений инъективна\\
2. Композиция сюръективных предложений сюръективна\\
3. Композиция биективных отображений биективна
\begin{proof}
1.\\
$ f: X \rightarrow Y $\\
$ g: Y \rightarrow Z $\\
Нужно $x_1 \neq x_2 \Rightarrow (g \circ f)(x_1) \neq (g\circ f) x_2 $\\
$ x_1 \neq x_2 \Rightarrow f(x_1) \neq f(x_2) \Rightarrow g(f(x_1)) \neq g(f(x_2))  $
\end{proof}
\begin{proof}
2. $ \forall z \in Z $\\
$g$ сюръективно $ \Rightarrow \exists y \in Y : g(y) = z $\\
$f$ сюръективно $ \Rightarrow \exists x \in X : f(x) = y $\\
$ (g \circ f)(x) = g(y) = z $ 
\end{proof}
\end{properties}
\begin{definition}
Классы множеств:
Два множества принадлежат одному классу если между ними есть биекция
\end{definition}
\begin{definition}
Обратное отображение \\ $ f : Y \rightarrow Y $ \\ 
$ g: Y \rightarrow X $ называется обратным, если $ g \circ f = id_x, f \circ g = id_y $ \\
Если у отображения $ f $ cуществует обратное отображение $ f^{-1} $, то оно единственное
$ f^{-1} (y) = {x} $ - прообраз \\
$ f^{-1} (y) = x $ - обратное отображение
\end{definition}

\begin{theorem}
Пусть $ f : X \rightarrow Y $ отображение, то есть обратное отображение $g$ $ \Leftrightarrow $ $f$ - биекция 
\begin{proof}
$\Rightarrow$ - Пусть $x_1, x_2 \in X, f(x_1) = f(x_2) \Rightarrow g(f(x_1)) = g(f(x_2))$ - $f$ - инъективно\\
Пусть $ y \in Y $ тогда $ y = f(g(y)) \in f(X) $ тк $f \circ g = id_y $
$f$ - сюръективно\\
$\Leftarrow$ - Построим отображение $g$. Пусть $y \in Y$, тогда $f^{-1}(y) = \{x\}  g(y) = x$, следовательно $f \circ g = id_y, g \circ f = id_x$
\end{proof}
\end{theorem}

\section{Аксиомы вещественных чисел}
\subsection{Аксиомы поля}
\begin{definition}
Бинарные операции \\
$ M \times M \rightarrow M $ \\
$ \mathbb{Z} \times \mathbb{Z} \rightarrow \mathbb{Z} $ 
\end{definition}

Мы будем рассматривать множество $\mathbb{R}$ с бинарными операциями $+$ и $-$ и отношением $\leq$, для которых выполняются следующие условия
\begin{enumerate}
	\item $ \forall x, y \in \mathbb{R} : x + y = y + x $ 
	\item $ \forall x, y, z \in \mathbb{R} (x + y) + z = x + (y + z) $ 
	\item $ \exists 0 in \mathbb{R}, \forall x \in \mathbb{R}: x+0 = x $ 
	\item $ \forall x \in \mathbb{R} \exists x' \in R x + x' = 0 $ 
	\begin{lemma}
		0 - единственный 
		\begin{proof}
			Пусть $0' , 0 = 0 + 0' = 0, 0' + 0 = 0'$
		\end{proof} 
	\end{lemma}
	\begin{lemma}
	У любого элемента есть единственный противоположный
	\begin{proof}
		Пусть $0' , 0 = 0 + 0' = 0, 0' + 0 = 0'$
	\end{proof} 
\end{lemma}
	\item $ \forall x, y \in \mathbb{R} : x \cdot y = y \cdot x $ 
    \item $ \forall x, y, z \in \mathbb{R} (x \cdot y) + z = x + (y \cdot z) $ 
	\item $ \exists 1 \in \mathbb{R} \setminus \{0\}, \forall x \in \mathbb{R}: x \cdot 1 = x $ 
	\item  $ \forall x \in \mathbb{R} \setminus \{0\} \exists x' \in R x \cdot x' = 1 $
	\item $ \forall x, y, z \in \mathbb{R} x(y+z) = xy + xz $ 
\end{enumerate}

\begin{example}
	Примеры полей
	\begin{enumerate}
		\item $\mathbb{Q}$ 
		\item  $\mathbb{R}$ 
		\item  $\mathbb{C}$ 
		\item  $\mathbb{Q}(\sqrt{2}) = \{a + b \cdot \sqrt{2} \mid a, b \in \mathbb{Q} \}$ 
		\item  $ {0, 1} = \mathbb{F}_2 $ 
	\end{enumerate}

\end{example}

\begin{lemma}
	$ \forall x \in R 0 \cdot x = 0$ 
	\begin{proof}
		$ 0 + 0 = 0 $ \\
		$ (0+0) \cdot x = 0 \cdot x = $ \\
		$ 0 \cdot x + 0 \cdot x $ \\
		$ 0 \cdot x + \underbrace{0\cdot x + (-0 \cdot x)}_0 = \underbrace{0\cdot x + (-0 \cdot x)}_0 $ 
	\end{proof}
\end{lemma}

\subsection{Аксиомы порядка}
\noindent
Аксиомы порядка $ \ \ $ 
$\left\{
\begin{tabular}{p{.8\textwidth}}
\begin{enumerate}
	\item $ \forall x \in \mathbb{R} : x \leq x $ - рефлексивность 
	\item $ \forall x, y in \mathbb{R} : x \leq y \and y \leq x \Rightarrow x = y $ антисимметричность 
	\item $ \forall x, y, z \in \mathbb{R}, x \leq y, y \leq z \Rightarrow x \leq z $ Транзитивность.
\end{enumerate}
\end{tabular}
\right.$
Линейный порядок
$\left\{
\begin{tabular}{p{.8\textwidth}}
\begin{enumerate}[4.]
	\item $ \forall x, y \in \mathbb{R}, x \leq y $ или  $  y \leq x $ 
\end{enumerate}
	\end{tabular}
\right.$

\subsection{Аксиома полноты}
Пусть $ X, Y \subset R, \forall x \in X \forall y \in Y x /leq y $ \\
Тогда $ \exists c \in R $ 
\begin{enumerate}
\item $ \forall x \in X : x \leq c $ 
\item $ \forall y \in Y c \leq y $
\end{enumerate}

\section{Ограниченные множества}

$ a, b \in R, a < b $ \\
Отрезок $ [a, b] = \{ x \in R \mid a \leq x \leq b \} $
$ x < y \Leftrightarrow x \leq y x \neq y $ \\
$ [a, b) $,
$ (a, b] $,
$ (a, b) $\\
\begin{definition}
	Множество называется ограниченным сверху, если $ \exists c \in R \forall x \in M x \leq c $ \\
	Множество называется ограниченным снизу, если $ \exists c \in R \forall x \in M x \geq c $ 
\end{definition}

\begin{definition}
	Если $ \forall x \in M, x \leq c$, то $c$ называется вехней границей \\
	Если $ \forall x \in M, x \geq c$, то $c$ называется нижней границей 
\end{definition}
\begin{definition}
	$ c \in R $ называется точной верхней гранью или супремумом множества $M$, если
	\begin{enumerate}
		\item $c$ - верхняя граница
		\item Для любой верхней границы $c'$ мноества $M$ выполняется $ c \leq c'$
	\end{enumerate}
\end{definition}
\begin{definition}
Опр. $ c \in R $ называется точной нижней гранью или инфимумом множества $M$, если 
\begin{enumerate}
	\item $c$ - нижняя граница
	\item Для любой нижней границы $c'$ мноества $M$ выполняется $ c \geq c'$ 
\end{enumerate}
\end{definition}
Из определения очевидно, что у множества может быть не более одного супремума и одного инфимума 
\begin{theorem}
	Пусть $ M \subset R, M \neq \emptyset $ 
	\begin{enumerate}
		\item Если M ограничено сверху, то у M есть супремум
		\item Если M ограничено снизу, то у M есть инфимум

	\end{enumerate}
	\begin{proof}
		1. $ B = \{ b \in R \mid  \} $
	\end{proof}
\end{theorem}

\begin{theorem}
	Пусть $ X, Y \in \mathbb{R} $ ограниченные сверху мн-ва\\
	Тогда $sup(X+Y) = sup(X) + sup(Y)$ \\
	$ X + Y = \{ x + y \mid x \in X, y \in Y \} $
	\begin{consequence}
		Пусть $ a \in \mathbb{R} $  Тогда  $\exists n \in \mathbb{N} n > a $ 
		\begin{proof}
			Пусть это не так \\
			$ \forall n \in \mathbb{N} : n \leq a $ , т.е. $ \mathbb{N} $ ограничено сверху. \\
			$ \exists c = sup(\mathbb{N}) $ \\
			$ c - 1 < sup (\mathbb{N}) $ 
		\end{proof}
	\end{consequence}
\end{theorem}









