
\Section{Числовые последовательности}

\Subsection{Основные определения}

Числовая последовательность - отображение $ a: \N \rightarrow \R $ \\
$ \N \rightarrow \R $ \\
$ n \rightarrow a_n $ \\
$ a_n = 3n^2 - \sqrt{2} n+ \frac{5}{2} $ \\
$ a_n = a_{n-1} + n^2, a_1 = 1$\\

Последовательность ограничена сверху, если существует  $ C \in \R : \forall i \in \N : a_i < C $ \\
Последовательность ограничена снизу, если существует  $ C \in \R : \forall i \in \N : a_i > C $ \\
Последовательность ограничена, если она ограничена и сверху и снизу.

Монотонные последовательности \\
Неубывающая последовательность: $ \forall i \in \N : a_i \leq a_{i+1} $\\
Возрастающая $ < $ \\
Невозрастающая $ a_i \geq a_{i+1} $\\
Убывающая $ > $ \\
2, 4 - строго монотонные.

Монотонные, начиная с некоторого места \\
Неубывающая $ \exists n \in \N \forall i > n : a_i \leq a_{i+1} $\\

\Section{Предел последовательности}

$ \eps$-окрестность - $ (x - \eps, x+\eps) $\\
\begin{enumerate}
	\item  Окрестность x - это $ \eps $ окрестность x для $ \eps > 0 $
	\item Окрестность x - произвольный интервал $ (a ,b) : a < x < b$
	\item Окрестность x - $U \subset \R :  U \supset (a, b) , a, b \in \R, a < x < b $ 
\end{enumerate}
Любая окрестность х содержит $\eps $ окрестноть х с $ \eps > 0 $ \\
Пусть $(a_i)$ - последовательность, $ x \in \R $ Говорят, что $a_i$ сходится к x(имеет х своим пределом) если $ \forall U $ окрестности $x \exists N \in \N : a_i \in U $ при всех $ i \geq N $ \\
Понятие предела не изменится, если мы определения окрестности 1 перейдём к определению 3. Если нужное свойство выполняется для класса 3, то оно, в частности, выполняется и для $ \eps $ окрестностей. \\
Предположим оно выполнено для всех $ \eps $ окрестностей. Тогда $ U \supset(a, b), (a,b) > (x-\eps, x+ \eps) \eps > 0 \Rightarrow \exists N, \forall i \in \N, a_i \in U$  \\
Определение 2: Пусть $(a_i), X \in \R$ сходится к Х, если $\forall U $ 	числа Х $, \{ i \in \N \mid a_i \notin U \} $ конечно \\
% -------pic1---------
$ \lim a_i = x $ \\
$ a_i \rightarrow x $ \\
Последовательность имеет не больше одного предела.
Предп: $ a_i \rightarrow x, a_i \rightarrow y, x < y $ \\
$ \eps = \dfrac{y-x}{2} $ \\
$Y_1 = \{ i \mid a_i \notin U_{\eps} (x) \} $конечно\\
$ Y_2 = \{ i \mid a_i \notin U_{\eps} (y) \} $конечно\\
$ \N = Y_1 \cup Y_2 $ конечно

\Subsection{Свойства пределов}
Предл(предельный переход в неравенстве) \\
Пусть $ a_i \rightarrow x, b_i \rightarrow y $ \\
Предп. $ \forall i \in \N : a_i \leq b_i. $Тогда $ x\leq y $ \\
Пусть $ y < x, \eps = \dfrac{x-y}{2} $ 
$ \exists N_1 : \forall n \geq N_1 a_n \in U_{\eps}(x) $
$ \exists N_2 : \forall n \geq N_2 b_n \in U_{\eps}(y) $
$  N = max(N_1, N_2) \Rightarrow \forall n \geq N : 
a_n \in U_{\eps}(x)\\
b_n \in U_{\eps}(y)$\\
$ x- \eps <  a_n \leq b_n < y+\eps = \eps x-\eps$\\
Противоречие \\
Следствие: Пусть $ a_i \leq C \forall i; a_i \rightarrow x \Rightarrow x \leq C $\\

\begin{theorem} Теорема о сжатой последовательности\\
Пусть $ a_i \rightarrow x, b_i \rightarrow x $ \\
$ c_i \ \ \forall i, a_i \leq c_i \leq b_i $ \\
\begin{proof}
Возьмём $ \eps > 0 $ и док-м, что $ \exists N \in \N $ \\
$ \forall i \geq N : c_i \in U_{\eps} (x) $ \\
$ \exists N_1 : \forall i \geq N_1 : a_i \in U_{\eps} (x) $\\
$ \exists N_2 : \forall i \geq N_2 : b_i \in U_{\eps} (x) $\\
$ \Rightarrow N = max(N_1, N_2) $ обладает нужным свойством. \\
Начиная с некоторого номера - менять/откидывать некоторые члены, предел от этого не ипзменится. \\

Если сушествует $ lim a_i = x $, то $(a_i)$ - ограничена\\
% ---pic2 ----
\end{proof}
\end{theorem}

Если последовательность неубывающая и ограничена свеху, то она сходится
Если последовательность невозрастающая и ограничена снизу, то она сходится

$ x = \sup\{a_i | i \in \N\} $ \\
$ a_i \rightarrow x $\\
Возьмём $ \eps > 0 $\\
$ x - \eps < x \Rightarrow x - \eps  - $ не верхняя граница $ \Rightarrow \exists N \in \N, a_N > x-\eps $ \\
$ a_i $ - неубывающая $ \Rightarrow \forall i \geq N : a_i > x-\eps \Rightarrow \forall i \geq N : a_i \in U_{\eps} (x)$\\

\Subsection{Арифметические действия с пределами} 

Пусть есть $ a_i \rightarrow x, b_i \rightarrow y $ \\
$ |a_i| \rightarrow |x| $\\
$ a_i + b+i \rightarrow x+y $ \\
$ a_i - b_i \rightarrow x - y$\\
$ a_i \cdot b_i \rightarrow xy $ \\
$ b_i \neq 0 \ \  \forall i, \dfrac{ a_i}{b_i} \rightarrow \dfrac{x}{y} $\\

$ | | a_i | - | x| | \leq | a_i - x | $ \\
2. $ \eps > 0$
$\exists N:  \forall i > N: a_i \in U_{\frac{\eps}{2}} (x) , b_i \in U_{\frac{\eps}{2}} (y) \Rightarrow a_i + b_i \in U_{\eps}(x+y) $ \\
4. $  a_i b_i - xy = (a_i b_i - xb_i) + ( xb_i - xy )$ \\
$ | a_i b_i - xy | \leq  |a_i b_i - xb_i| + | xb_i - xy | $ \\ 
$ \exists C > 0 : | b_i | \leq C \forall i \in \N $ \\
$ | a_i b_i - xy | \leq C | a_i - x| + |x | \cdot | b_i - y | $\\
Возьмём $\eps > 0 $ 
Тогда $ \exists N \in \N \forall i \geq N :\\ |a - x| < 
\frac{\eps}{2C} \\
|b_i - y | < \frac{\eps}{2|x|} $(при х $\neq 0$)\\
%----pic3----											
$ |ab_i - xy| < C \cdot \frac{\eps}{2C} + \frac{\eps}{2} = \eps $

5. Достаточно доказать, что $\frac{1}{b_i} \rightarrow \frac{1}{y}$ \\
$|\dfrac{1}{b_i} - \dfrac{1}{y}| = |\dfrac{y-b_i}{b_i y}| = \dfrac{|b_i - y|}{|b_i||y|}$ 	\\
$ \exists  N_0 \in \N \ \forall i > N_0 : |b_i| > \frac{|y|}{2} $\\
$ y > 0 : b_i \in U_{\frac{y}{2}}(y) \Rightarrow b_i  > \frac{y}{2} > 0 $\\
$ y < 0	:  b_i \in U_{-\frac{y}{2}}(y) \Rightarrow b_i  < \frac{y}{2} < 0 $ \\
$ |\dfrac{1}{b_i} - \dfrac{1}{y}| \leq |\dfrac{b_i - y}{|y|^2 / 2}| $ \\
Возьмём $ \eps > 0 $\\
$ \exists N \geq N_0 : \forall i \geq N_0 $
%-----pic4-----

\subsection{Бесконечные пределы} 
Пусть $a_i$ последовательность.
$ a_i $ стремится к $ +\infty$ если $ \forall C \in \R \  \exists N \in \N : \forall i \geq N : a_i > C $ \\
Определение станет аналогичным определению конечного предела, если определить окресность $ +\infty $ как произвольный открытый луч $ (C, +\infty)$ \\
Равносильное определение: $ \{i \mid a_i \neq (C, +\infty) \} $ - конечно \\
Последовательность не имеет конечного предела - последовательность расходится \\
Пусть $a_i$ последовательность.
$ a_i $ стремится к $ -\infty$ если $ \forall C \in \R \  \exists N \in \N : \forall i \geq N : a_i < C $ \\
$ a_i \rightarrow +\infty \Leftrightarrow -a_i \rightarrow -\infty $\\
$ a_i \rightarrow \infty $ если $ \forall C \in \R \ \exists N \in \N : \forall i > N, |a_i| > C $ \\
$ ( C, +\infty )$ - окрестность $+\infty$ \\
$ (-\infty, C) $ \\
$ (-\infty, C) \cup ( C, +\infty), C > 0 $  \\

Теорема: Не ограниченная сверху неубывающая последовательность стремится к $ + \infty $ \\
Не ограниченная снизу невозрастающая последовательность стремится к $ -\infty $ \\
% ---------pic5---------
$ a_i \rightarrow \infty \Leftrightarrow |a_i | \rightarrow +\infty $ \\
Бесконечно малая посл-ть $ a_i \rightarrow 0 $ \\
Бесконечно большая посл-ть $ a_i \rightarrow \infty $ \\
Теорема $\forall i : a_i \neq 0 \Rightarrow (a_i) \rightarrow 0 \Leftrightarrow (\dfrac{1}{a_i}) \rightarrow \infty $\\
%----pic6-----
Замечание - сумма, разность, произведение бесконечно малых - бесконечно малая посл. \\
Пусть $(a_i) $ бесконечно большая, $(b_i) $ ограниченная, тогда их сумма бесконечно большая. \\
$ U_C(\infty) =  (-\infty, C) \cup ( C, +\infty) $ \\
Пусть $ C > 0 $ \\
Проверить $\{ i \mid a_i + b_i \notin U_C{\infty} \}$\\
$ \exists D > 0 : \forall i \in \N: |b_i | < D $ \\
$ \exists N \in \N : \forall i \geq N: |a_i| > C+D $ \\
При $ i \geq N : | a_i + b_i | \geq |a_i| + |b_i|  > C+D-D = C $ \\
$ a_i + b_i \in U_C(\infty) $ \\
Произведение бесконечно малой посл-ти $a_i$ на ограниченную $b_i$ - бесконечно малая \\
$ \exists C, \forall i, |b_i| < C $ \\
$ |a_i b_i | \leq C \cdot |a_i| $ 
$ a_i \rightarrow 0 \Rightarrow  |a_i| \rightarrow 0 \Rightarrow |a_i b_i| \rightarrow 0 $ \\
$ \overline{\R} = \R \cup \{+\infty, -\infty\} $ \\
$ +\infty + (-\infty) $ - не определена \\
$ a \cdot (+\infty) = (+\infty), a > 0 $\\
$ a \cdot (+\infty) = (-\infty), a < 0 $\\
$ a_n \leq b_n, a_n \rightarrow x \in \overline{\R}, b_n \rightarrow y \in \overline{\R} $ \\
Тогда $ x \leq y $\\
% ----------pic7--------------
Пр. \\
1. $lim x_n = +\infty, y_n $ огр. снизу $ \Rightarrow x_n + y_n \rightarrow +\infty $
2. $lim x_n = -\infty, y_n $ огр. сверху $ \Rightarrow x_n + y_n \rightarrow -\infty $
3. $ x_n \rightarrow +\infty (-\infty) \ y_n \geq C > 0 \ \forall n \in  \N \Rightarrow x_n y_n \rightarrow +\infty ( -\infty) $ \\
4. $ x_n \rightarrow a \neq 0, y_n \neq 0, y_n \rightarrow 0 \Rightarrow \dfrac{x_n}{y_n} \rightarrow \infty$ \\
5. $ x_n \rightarrow a \in \R y_n \rightarrow \infty \Rightarrow \dfrac{x_n}{y_n} \rightarrow 0 $ \\
6. 
7. % ---- pic8---------
3. Пусть $ E > 0 \  \exists N, \forall i \geq N, x_i > \dfrac{E}{C}, x_i y_i > \dfrac{E}{C} C = E $ \\
4. $ \exists E > 0, x_n \rightarrow a \Rightarrow \exists N_0 : |x_i| \geq \dfrac{|a|}{2} $ при  $ i \geq N_0$ \\
$ \dfrac{x_n}{y_n} $ беск. больш $ \Leftrightarrow \dfrac{y_n}{x_n} $ беск мал. $ \dfrac{1}{|x_i|} \leq \dfrac{2}{a} \Rightarrow \dfrac{1}{x_i}$ ограниченная \\
$ y_n \cdot \left( \dfrac{1}{x_n} \right)  $\\

Неравенство Бернулли \\
$ x > -1, n \in \N $ \\
$ (1+x)^n \geq 1 + nx $ \\
$ k \rightarrow k + 1 $ \\
$ (1+x)^{k+1} = (1+x)^k(1+x) \geq (1+kx)(1+x) = 1 + (k+1)x + kx^2 \geq 1 + (k+1)x $ \\
Сл. 1 \\
1. $ lim a^n = +\infty $ при $ a > 1 $ \\
2. $ lim a^n = 0$ при $ |a| < 1$ \\
Д-во \\
$ a^n = (1 + (a-1))^n \geq 1 + n(a-1) \rightarrow +\infty $\\
Сл. 2 Пусть $ a > 1 \Rightarrow \sqrt[n]{a} \rightarrow 1$\\
$ \sqrt[n]{a} = 1 + x_n $ \\
$ a = (1 + x_n) ^n > 1 + n \cdot x_n $ \\
$ x_n \leq \dfrac{1}{n} (a-1)  $ \\
$ 1 \leq \sqrt[n]{a} \leq  \dfrac{1}{n} (a-1) $ \\
По зажатой последовательности $ \sqrt[n]{a} \rightarrow 1 $\\

$ x_n = \left(1 + \dfrac{1}{n}\right)^n $\\
$ y_n =  \left(1 + \dfrac{1}{n}\right)^{n+1} $ \\
$ \dfrac{y_{n-1}}{y_n} = \dfrac{ \left(1 + \dfrac{1}{n-1}\right)^{n}}{ \left(1 + \dfrac{1}{n}\right)^{n+1}} = \left( \dfrac{1+\dfrac{1}{n-1}}{1 + \dfrac{1}{n}} \right)^{n-1} \left( 1 + \dfrac{1}{n-1} \right)^{-1}  = \left( \dfrac{n+\dfrac{n}{n-1}}{n + 1} \right)^{n+1} \dfrac{n-1}{n}  =  \left( \dfrac{n+1+\dfrac{n}{n-1}}{n + 1} \right)^{n+1} \dfrac{n-1}{n}  \geq \left(1 + (n+1)\dfrac{1}{n-1} \right)\dfrac{n-1}{n} = \left( 1+ \dfrac{1}{n-1} \right) \dfrac{n-1}{n}  = 1$\\
Т.е. $ y_n $ невозрастающ. $ \Rightarrow y_n \rightarrow y$\\
$ x_n = y_n \cdot ( 1 + \dfrac{1}{n})^{-1} \rightarrow y $ \\
