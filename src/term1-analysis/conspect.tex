% !TeX program = xelatex
% Core.tex, version=12.1
% Sayutin Dmitry, 2016 & 2017.

% Changelog:
% 12.1: Improve \N, \Q, \R, \Z, \CC definitions a bit, add some customizations for math mode
% 12: Improve \TODO, \skip[sub][sub]section (takes optional counter now), fix code a bit, add exerc env.
% 11: Add cmap
% 10.6: Styling changes
% 10.5: Add \slashn stuff, custom theorem namespace support, style changes.
% 10.4: replace \emptyset in math mode.
% 10.3: add properties(theorems), add \ref example, add \ThmSpacing configure var.
% 10.2: fix for proofs
% 10.1: better style for theorems, thmslashn
% 10: more math symbols, theorems support (\enablemath), using paralist, using cancel, moved code support to \enablecode
% 9: new math symbols (\enablemath), \skip[sub][sub]section, \CustomTitle support
% 8: better source code support, padding.
% 7: custom head, foot, sections.

\documentclass[a4paper,12pt]{article}
\usepackage{cmap}
\usepackage[usenames,dvipsnames]{xcolor}
\usepackage{polyglossia}
\usepackage{xspace}
\setdefaultlanguage[spelling=modern]{russian}
\setotherlanguage{english}
\defaultfontfeatures{Ligatures={TeX},Renderer=Basic}
%\setmainfont[Ligatures={TeX,Historic}]{Liberation Serif}
%\setsansfont{Liberation Sans}
%\setmonofont{Liberation Mono}
\setmainfont[Ligatures={TeX,Historic}]{CMU Serif}
\setsansfont{CMU Sans Serif}
\setmonofont{CMU Typewriter Text}


\usepackage{graphicx}

\pagestyle{plain}
\usepackage[
  left=0.50in,
  right=0.50in,
  top=0.8in,
  bottom=0.7in,
  headheight=0.8in]{geometry}
\pagenumbering{gobble}

\setlength{\parskip}{0.15cm}

\usepackage{indentfirst}

\usepackage{hyperref}
\hypersetup{
  colorlinks,
  citecolor=black,
  filecolor=black,
  linkcolor=blue,
  urlcolor=blue
}

\usepackage{paralist}
\usepackage{cancel}
\usepackage{textcomp}
\usepackage{gensymb}
\usepackage{mdframed}
\usepackage{lastpage}
\usepackage{microtype}
\usepackage[super]{cite}
\usepackage{fancyhdr}
\pagestyle{fancy}

% Основано на коде С. Копелиовича.
\newcommand\Section[2]{
  \newpage % new page
  \stepcounter{section}
  \bigskip
  \phantomsection
  \addcontentsline{toc}{section}{\arabic{section}. #1}
  \begin{center}
    {\huge \bf \arabic{section}. #1}\\
  \end{center}
  \bigskip
  \gdef\SectionName{#1}
  \gdef\AuthorName{#2}

  \lhead{\ShortCourseName}
  \chead{}
  \rhead{\SectionName}

  \cfoot{
%    \topskip0pt\vspace*{\fill}
    \thepage~из~\pageref*{LastPage}
%    \vspace*{\fill}
  }
  
  \renewcommand{\headrulewidth}{0.15 mm}

  \ifx\LaconicFooter\undefined
  \lfoot{
%    \topskip0pt\vspace*{\fill}
    Глава \texttt{\#\arabic{section}}
%    \vspace*{\fill}
  }
  \rfoot{
%    \topskip0pt\vspace*{\fill}
    Автор: \AuthorName
%    \vspace*{\fill}
  }
  \renewcommand{\footrulewidth}{0.15 mm}
  \fi
}

\newcommand\Subsection[1]{
  % Пока здесь нет никаких кастомизаций,
  % Но рекомендуется использовать имено вариацию с большой буквы,
  % На случай, если в будущем они появятся.
  \subsection{#1}
}

\newcommand\Subsubsection[1]{
  % Пока здесь нет никаких кастомизаций,
  % Но рекомендуется использовать имено вариацию с большой буквы,
  % На случай, если в будущем они появятся.
  \subsubsection{#1}
}

\DeclareRobustCommand{\divby}{%
	\mathrel{\vbox{\baselineskip.65ex\lineskiplimit0pt\hbox{.}\hbox{.}\hbox{.}}}%
	%\scalebox{0.75}{\vdots}
}
\DeclareRobustCommand{\divbynot}{%
	%\mathrel{\vbox{\baselineskip.65ex\lineskiplimit0pt\hbox{.}\hbox{.}\hbox{.}}}%
	\centernot\divby
}


\newcommand\slashnl{~\\*[-26pt]}
\newcommand\slashn{~\\*[-22pt]}
\newcommand\slashns{~\\*[-18pt]}
\newcommand\slashnss{~\\*[-14pt]}
\newcommand\slashnsss{~\\*[-10pt]}

\newcommand{\makegood}{
  \ifx\ShortCourseName\undefined
  \gdef\ShortCourseName{\CourseName}
  \fi
  
  % \newcommand\CustomTitle{...} до \makegood для того, чтобы
  % переопределить содержимое титульной страницы до содержания.
  \ifx\CustomTitle\undefined
    \title{\CourseName}
    \maketitle
  \else
    \pagestyle{empty}
    \CustomTitle
  \fi
  
  \tableofcontents
  \pagebreak
  \pagestyle{fancy}
  \pagenumbering{arabic}
  \setcounter{page}{1}

  \ifdefined\ENABLEDMATH
  \renewcommand\proofname{\em\textbf{Доказательство}}
  \else
  \fi
}

% Использовать как \skipsection или \skipsection[2]

\newcommand\skipsection[1][1]{
  \addtocounter{section}{#1}
}

\newcommand\skipsubsection[1][1]{
  \addtocounter{subsection}{#1}
}

\newcommand\skipsubsubsection[1][1]{
  \addtocounter{subsubsection}{#1}
}

\newcommand{\TODO}[1][]{
  \vspace{0.2em}
  \textbf{{\bf\color{red} TODO:} #1}
  \vspace{0.2em}
}

% Должна использоваться вне \document.
\newcommand\enablemath{
  \usepackage{amsmath,amsthm,amssymb,mathtext}
  \usepackage{thmtools}
  \usepackage{tikz}

  \newcommand\R{\ensuremath{\mathbb{R}}\xspace}
  \newcommand\Q{\ensuremath{\mathbb{Q}}\xspace}
  \newcommand\N{\ensuremath{\mathbb{N}}\xspace}
  \newcommand\Z{\ensuremath{\mathbb{Z}}\xspace}
  \newcommand\CC{\ensuremath{\mathbb{C}}\xspace} % C.C. from Code Geass

  \DeclareRobustCommand{\divby}{%
    \mathrel{\vbox{\baselineskip.65ex\lineskiplimit0pt\hbox{.}\hbox{.}\hbox{.}}}%
  }
  \newcommand\notmid{\centernot\mid}
  
  \let\Im\relax % Переопределяем стрёмные значки для понятных вещей.
  \let\Re\relax
  \DeclareMathOperator\Im{Im}
  \DeclareMathOperator\Re{Re}
  
  \DeclareMathOperator*{\lcm}{lcm}
  \newcommand\vphi{\varphi}
  
  \usepackage{
    nameref,
    hyperref,
    cleveref}

  \ifx\ThmSpacing\undefined
  \def\ThmSpacing{9pt}
  \fi

  \ifx\ThmNamespace\undefined
  \def\ThmNamespace{section}
  \fi
  
  \declaretheoremstyle[
    spaceabove=\ThmSpacing, spacebelow=\ThmSpacing,
    headfont=\slshape\bfseries,
    bodyfont=\normalfont,
    postheadspace=0.5em,
  ]{thmstyle_def}

  \declaretheoremstyle[
    spaceabove=\ThmSpacing, spacebelow=\ThmSpacing,
    postheadspace=0.5em,
  ]{thmstyle_thm}

  \declaretheoremstyle[
    spaceabove=\ThmSpacing, spacebelow=\ThmSpacing,
    headfont=\itshape\bfseries,
    notefont=\itshape\bfseries, notebraces={}{},
    bodyfont=\normalfont,
    postheadspace=0.5em,
  ]{thmstyle_cons}

  \declaretheoremstyle[
    spaceabove=\ThmSpacing, spacebelow=\ThmSpacing,
    headfont=\bfseries,
    notefont=\bfseries, notebraces={}{},
    bodyfont=\normalfont,
    postheadspace=0.5em,
  ]{thmstyle_examp}  

  \declaretheoremstyle[
    spaceabove=\ThmSpacing, spacebelow=\ThmSpacing,
    headfont=\ttfamily\itshape,
    notefont=\ttfamily\itshape, notebraces={}{},
    bodyfont=\normalfont,
    postheadspace=0.5em,
  ]{thmstyle_remark}
  
  \declaretheorem[numberwithin=\ThmNamespace, name=Теорема, style=thmstyle_thm]{theorem}
  \declaretheorem[numberwithin=\ThmNamespace, name=Определение, style=thmstyle_def]{definition}
  \declaretheorem[sibling=theorem, name=Утверждение, style=thmstyle_thm]{statement}
  \declaretheorem[numbered=no, name=Замечание, style=thmstyle_remark]{remark}
  \declaretheorem[numbered=no, name=Лемма, style=thmstyle_thm]{lemma}
  \declaretheorem[numbered=no, name=Следствие, style=thmstyle_cons]{consequence}
  \declaretheorem[numbered=no, name=Пример, style=thmstyle_examp]{example}
  \declaretheorem[numbered=no, name=Свойства, style=thmstyle_cons]{properties}
  \declaretheorem[numbered=no, name=Свойство, style=thmstyle_cons]{property}
  \declaretheorem[numbered=no, name=Упражнение, style=thmstyle_examp]{exerc}
  
  % написать после \begin{proof} и т.п., чтобы
  % продолжить на новой строчке.
  % Вообще рекомендуется использовать \slashn[...].
  \newcommand\thmslashn{\slashn}
  
  % Примеры:
  % Первый, простейший:
  % \begin{theorem} theorem-statement \end{theorem}
  %
  % Второй, использовать название теоремы в заголовке.
  % \begin{definition}[My name] the definition \end{definition}
  %
  % Третий, создать метку, чтобы потом можно было сделать сюда ссылку.
  % \begin{statement}\label{stm:identifier} the statement \end{statement}
  %
  % Четвёртый: ссылки (чтобы понять, в чём разница, нужно собрать и посмотреть).
  % \begin{statement}\label{otherlabel}
  %   Согласно \hyperref[stm:identifier]{Теореме о волшебных палочках} магия существует.
  %   \ref{stm:identifier}
  %   \autoref{stm:identifier} ‘‘\nameref{stm:identifier}’’,
  % \end{statement}
    
  \newcommand\eps{\varepsilon}
  \renewcommand\le{\leqslant}
  \renewcommand\ge{\geqslant}
  \newcommand\empysetold{\emptyset}
  \renewcommand\emptyset{\varnothing}
  
  \newcommand\ENABLEDMATH{YES}
}

\newcommand\enablecode{
  \usepackage{listings}
  \lstset{
    belowcaptionskip=1\baselineskip,
    breaklines=true,
    frame=L,
    xleftmargin=\parindent,
    showstringspaces=false,
    basicstyle=\footnotesize\ttfamily,
    keywordstyle=\bfseries\color{blue},
    commentstyle=\itshape\color{Maroon},
    identifierstyle=\color{black},
    stringstyle=\color{orange},
    numbers=left
  }

  % Some voodoo magic:
  % При желании можно адаптировать цвета под себя,
  % Для этого скопируйте это в conspect.tex с новым названием стиля и отредактируйте
  % Соответствующие куски.
  \lstdefinestyle{supercpp} {
    language=C++,
    deletekeywords={int, long, char, short, unsigned, signed,
      uint64\_t, int64\_t, uint32\_t, int32\_t, uint16\_t, int16\_t, uint8\_t, int8\_t,
      size\_t, ptrdiff\_t, \#include,\#define,\#if,\#ifdef,\#ifndef},
    classoffset=1,
    morekeywords={vector,stack,queue,set,map,unordered\_set,unordered\_map,deque,array,string,multiset,multimap,
      int, long, char, short, unsigned, signed,
      uint64\_t, int64\_t, uint32\_t, int32\_t, uint16\_t, int16\_t, uint8\_t, int8\_t,
      size\_t, ptrdiff\_t
    },
    keywordstyle=\bfseries\color{green!40!black},
    classoffset=0,
    classoffset=2,
    morekeywords={std},
    keywordstyle=\bfseries\color{ForestGreen},
    classoffset=0,
    morecomment=[l][\bfseries\color{purple!99!black}]{\#}
  }
}

\usepackage{multicol}
\usepackage{centernot}
\enablemath

\renewcommand*{\t}{\texttt}
\newcommand*{\liminfty}[1]{\lim\limits_{{#1} \to \infty}}
\newcommand*{\abs}[1]{\left|\,{#1}\,\right|}
\newcommand*{\ceil}[1]{\lceil \, {#1} \, \rceil}
\newcommand*{\floor}[1]{\lfloor \, {#1} \, \rfloor}
\newcommand*{\rpm}{\raisebox{.2ex}{$\scriptstyle\pm$}}
\newcommand*{\notdivby}{\centernot\divby}
\newcommand*{\integer}[1]{\left[{#1}\right]}
\newcommand*{\real}[1]{\left\{{#1}\right\}}
\newcommand*{\action}{\curvearrowright}
\renewcommand*{\TODO}{\textcolor{red}{TODO}}
\renewcommand*{\thmslashn}{\slashns}

\declaretheorem[numbered=no, name=Упражнение, style=thmstyle_cons]{exercise}
\declaretheorem[numbered=no, name=Следствия, style=thmstyle_cons]{consequences}

\newcommand*{\Ker}[1]{Ker \, {#1}}
\renewcommand*{\Im}[1]{Im \, {#1}}

\let\epsilon\varepsilon
\let\emptyset\varnothing
\let\phi\varphi



\begin{document}
	\gdef\CourseName{Математический анализ}
	\author{Печин Михаил}
	\makegood

	\Section{Вещественные числа}{Печин Михаил}
	\Subsection{Язык теории множеств}


\begin{definition}
	Конечные множества \\
	$ \{ 3, 1, 17 \} $ \\
	$ A = \{ 1, 2, 3... 100 \} $ \\
	$ \{ n^2 \mid n = 1, 2, ..., 50 \} $ \\
	$ \{ n \in A \mid n \divby 3 \} $  \\
	
\end{definition}

\begin{definition}Подмножество \\ 
	$ B \subset A \Leftrightarrow \forall b \in B : b \subset A $ \\
\end{definition}

\begin{definition} P - предикат \\
	$ A = \rm \Z $ \\
	$ P(n) - n^2 \leq 10 $ \\
	Множество А, Р - предикат на А \\
	$ \forall x \in A : P(x) $  \\
	$ \exists x \in A : P(x) $ \\
\end{definition}

\begin{definition} Квантор \\
	Квантор всеобщности: $ \forall $ \\
	Квантор существования: $ \exists $ \\
\end{definition}
\begin{definition} Операции над множествами \\ 
	$ A, B $ - множества \\
	$ A \cap B = \{ a \in A \mid a \in B \} $\\
	$ A \cup B = \{ a \in M \mid a \in A,  a \in B \} $ \\
	$ A \setminus B = \{a \in A \mid a \notin B \}  $ \\
	$ A \oplus B = ( A \setminus B ) \cup ( B \setminus A ) $ \\
	$ A \times B = { (a, b) \mid a \in A, b \in B } $ \\
\end{definition}

\begin{definition}
	Отношения \\
	$ a, b \in \Z $ \\
	$ a \divby b $ \\
	$ a \divbynot b $ \\
	Отношение делимости\\
\end{definition}

\begin{definition} Декартов квадрат \\
	$ M = \{1, 2, 3, 4\} $\\
	$ |M \times M | = 16 $ - декартов квадрат (произведение множества на себя)\\
	$ |x| $ - Мощность мн-ва Х (кол-во элементов в конечном мн-ве)\\
	$ \divby R \subset M \times M $\\
	$ R =  \{ (1, 1), (2, 1), (2, 2), (3, 1), (3, 3), (4, 1), (4, 2), (4, 4) \}$ \\
	$ |M| = n \rightarrow |2^M| = 2^n $ - количество подмножеств на множестве
	$ 2^{(n^2)} $ - количество отношений на множестве \\
\end{definition}

\noindent
$ \N $ - Натуральные числа\\
$ \Z $ - Целые числа\\
$ \mathbb{Q} = \{ \frac{a}{b} \mid a \in \Z, b \in \N \} $ - Рациональные числа\\
$ \R $ - Вещественные числа\\
$ \mathbb{C} $ - Комплексные числа\\

\Subsection{Отображения}

\begin{definition} Отображение \\
X, Y - множества\\
$ f: X \rightarrow Y $ f - отображение из X в Y \\
Каждому $ х \in X $ сопоставлен $ y \in Y $ \\
$ X = \{ x_1, ..., x_n \} $ \\
$\hspace*{11mm} y_1, ..., y_n $\\
$ |X| = 3, |Y| = 5 $\\
$ f: X \rightarrow Y $\\
Количество отображений f - $ Y^X = 5^3 $ \\
$ X = Y = \N $ \\
$ f(x) = 3x + 5 $ \\
$ \N \Rightarrow \N $ \\
$ x \Rightarrow 3x+5 $ 
$ f(x) = |\{ d \in \N \mid x \divby d \}| $ 
\end{definition}

\begin{definition}
Пусть задано $ f : X \Rightarrow Y $ \\
$ X_0 \subset X $ \\
$ f(X_0) = \{ f(x) \mid x \in X_0 \} \in Y $ - образ подмножества \\
$ f^{-1}(y) = \{ x \in  X \mid f(x) = y \} $ - прообраз элемента \\
$ f^{-1}(Y_0) = \{ x \in X \mid f(x) \in Y_0 \} $  - прообраз множества 
\begin{example}
	$ f: \Z \Rightarrow \Z $ \\
	$ a \Rightarrow a^2 $ \\
	$ f( \{ -1, 0, 1, 2 \} ) = \{ 0, 1, 4 \} $ \\
	$ f^{-1} (1)= \{-1, 1\}$ \\
    $ f^{-1} (0)= \{0\}$ \\
	$ f^{-1} (2)= \varnothing$ \\
	$ f^{-1} (\{0,1\})= \{-1, 0, 1\}$ 
\end{example}
\end{definition}



\begin{definition}
Отображение $ f: X \Rightarrow Y $ называется сюръективным (сюръекцией), если $ f(X) = Y $ $ \forall y \in Y \exists x \in X : f(x) = y $ \\
f сюръекция $ \Leftrightarrow  \forall y \in Y |f^{-1}(y)| \geq 1 $ \\
\end{definition}
\begin{definition}
Отображение $ f: X \rightarrow Y $ называется инъективным (инъекцией), если $ \forall x_1, x_2 \in X : x_1 \neq x_2 \rightarrow f(x_1) \neq f(x_2) $ \\
f инъекция $ \Leftrightarrow  \forall y \in Y |f^{-1}(y)| \leq 1 $ 
\end{definition}
\begin{definition}
Отображение $ f: X \rightarrow Y $ называется биективным (биекцией), если оно сюръективно и инъективно 
\end{definition}

\begin{definition}
Пусть X, Y, Z - мн-ва, $f: X \rightarrow Y $ и $ g: Y \rightarrow Z $ \\
Композиция отображений $ g \circ f $ $ x \mapsto g(f(x)) $ 
\end{definition}

\begin{definition}
Тождественное отображение \\$ id_x : X \rightarrow X $ \\
$\hspace*{10mm} x \mapsto x $
\end{definition}

\begin{properties}
1. Композиция инъективных отображений инъективна\\
2. Композиция сюръективных предложений сюръективна\\
3. Композиция биективных отображений биективна
\begin{proof}
1.\\
$ f: X \rightarrow Y $\\
$ g: Y \rightarrow Z $\\
Нужно $x_1 \neq x_2 \Rightarrow (g \circ f)(x_1) \neq (g\circ f) x_2 $\\
$ x_1 \neq x_2 \Rightarrow f(x_1) \neq f(x_2) \Rightarrow g(f(x_1)) \neq g(f(x_2))  $
\end{proof}
\begin{proof}
2. $ \forall z \in Z $\\
$g$ сюръективно $ \Rightarrow \exists y \in Y : g(y) = z $\\
$f$ сюръективно $ \Rightarrow \exists x \in X : f(x) = y $\\
$ (g \circ f)(x) = g(y) = z $ 
\end{proof}
\end{properties}
\begin{definition}
Классы множеств:
Два множества принадлежат одному классу если между ними есть биекция
\end{definition}
\begin{definition}
Обратное отображение \\ $ f : Y \rightarrow Y $ \\ 
$ g: Y \rightarrow X $ называется обратным, если $ g \circ f = id_x, f \circ g = id_y $ \\
Если у отображения $ f $ cуществует обратное отображение $ f^{-1} $, то оно единственное
$ f^{-1} (y) = {x} $ - прообраз \\
$ f^{-1} (y) = x $ - обратное отображение
\end{definition}

\begin{theorem}
Пусть $ f : X \rightarrow Y $ отображение, то есть обратное отображение $g$ $ \Leftrightarrow $ $f$ - биекция 
\begin{proof}
$\Rightarrow$ - Пусть $x_1, x_2 \in X, f(x_1) = f(x_2) \Rightarrow g(f(x_1)) = g(f(x_2))$ - $f$ - инъективно\\
Пусть $ y \in Y $ тогда $ y = f(g(y)) \in f(X) $ тк $f \circ g = id_y $
$f$ - сюръективно\\
$\Leftarrow$ - Построим отображение $g$. Пусть $y \in Y$, тогда $f^{-1}(y) = \{x\}  g(y) = x$, следовательно $f \circ g = id_y, g \circ f = id_x$
\end{proof}
\end{theorem}

\Section{Аксиомы вещественных чисел}

\Subsection{Аксиомы поля}

\begin{definition}
Бинарные операции \\
$ M \times M \rightarrow M $ \\
$ \Z \times \Z \rightarrow \Z $ 
\end{definition}

Мы будем рассматривать множество $\R$ с бинарными операциями $+$ и $-$ и отношением $\leq$, для которых выполняются следующие условия
\begin{enumerate}
	\item $ \forall x, y \in \R : x + y = y + x $ 
	\item $ \forall x, y, z \in \R (x + y) + z = x + (y + z) $ 
	\item $ \exists 0 in \R, \forall x \in \R: x+0 = x $ 
	\item $ \forall x \in \R \exists x' \in R x + x' = 0 $ 
	\begin{lemma}
		0 - единственный 
		\begin{proof}
			Пусть $0' , 0 = 0 + 0' = 0, 0' + 0 = 0'$
		\end{proof} 
	\end{lemma}
	\begin{lemma}
	У любого элемента есть единственный противоположный
	\begin{proof}
		Пусть $0' , 0 = 0 + 0' = 0, 0' + 0 = 0'$
	\end{proof} 
\end{lemma}
	\item $ \forall x, y \in \R : x \cdot y = y \cdot x $ 
    \item $ \forall x, y, z \in \R (x \cdot y) + z = x + (y \cdot z) $ 
	\item $ \exists 1 \in \R \setminus \{0\}, \forall x \in \R: x \cdot 1 = x $ 
	\item  $ \forall x \in \R \setminus \{0\} \exists x' \in R x \cdot x' = 1 $
	\item $ \forall x, y, z \in \R x(y+z) = xy + xz $ 
\end{enumerate}

\begin{example}
	Примеры полей
	\begin{enumerate}
		\item $\mathbb{Q}$ 
		\item  $\R$ 
		\item  $\mathbb{C}$ 
		\item  $\mathbb{Q}(\sqrt{2}) = \{a + b \cdot \sqrt{2} \mid a, b \in \mathbb{Q} \}$ 
		\item  $ {0, 1} = \mathbb{F}_2 $ 
	\end{enumerate}

\end{example}

\begin{lemma}
	$ \forall x \in R 0 \cdot x = 0$ 
	\begin{proof}
		$ 0 + 0 = 0 $ \\
		$ (0+0) \cdot x = 0 \cdot x = $ \\
		$ 0 \cdot x + 0 \cdot x $ \\
		$ 0 \cdot x + \underbrace{0\cdot x + (-0 \cdot x)}_0 = \underbrace{0\cdot x + (-0 \cdot x)}_0 $ 
	\end{proof}
\end{lemma}

\subsection{Аксиомы порядка}
\noindent
Аксиомы порядка $ \ \ $ 
$\left\{
\begin{tabular}{p{.8\textwidth}}
\begin{enumerate}
	\item $ \forall x \in \R : x \leq x $ - рефлексивность 
	\item $ \forall x, y in \R : x \leq y \and y \leq x \Rightarrow x = y $ антисимметричность 
	\item $ \forall x, y, z \in \R, x \leq y, y \leq z \Rightarrow x \leq z $ Транзитивность.
\end{enumerate}
\end{tabular}
\right.$
Линейный порядок
$\left\{
\begin{tabular}{p{.8\textwidth}}
\begin{enumerate}[4.]
	\item $ \forall x, y \in \R, x \leq y $ или  $  y \leq x $ 
\end{enumerate}
	\end{tabular}
\right.$

\subsection{Аксиома полноты}
Пусть $ X, Y \subset R, \forall x \in X \forall y \in Y x /leq y $ \\
Тогда $ \exists c \in R $ 
\begin{enumerate}
\item $ \forall x \in X : x \leq c $ 
\item $ \forall y \in Y c \leq y $
\end{enumerate}

\section{Ограниченные множества}

$ a, b \in R, a < b $ \\
Отрезок $ [a, b] = \{ x \in R \mid a \leq x \leq b \} $
$ x < y \Leftrightarrow x \leq y x \neq y $ \\
$ [a, b) $,
$ (a, b] $,
$ (a, b) $\\
\begin{definition}
	Множество называется ограниченным сверху, если $ \exists c \in R \forall x \in M x \leq c $ \\
	Множество называется ограниченным снизу, если $ \exists c \in R \forall x \in M x \geq c $ 
\end{definition}

\begin{definition}
	Если $ \forall x \in M, x \leq c$, то $c$ называется вехней границей \\
	Если $ \forall x \in M, x \geq c$, то $c$ называется нижней границей 
\end{definition}
\begin{definition}
	$ c \in R $ называется точной верхней гранью или супремумом множества $M$, если
	\begin{enumerate}
		\item $c$ - верхняя граница
		\item Для любой верхней границы $c'$ мноества $M$ выполняется $ c \leq c'$
	\end{enumerate}
\end{definition}
\begin{definition}
Опр. $ c \in R $ называется точной нижней гранью или инфимумом множества $M$, если 
\begin{enumerate}
	\item $c$ - нижняя граница
	\item Для любой нижней границы $c'$ мноества $M$ выполняется $ c \geq c'$ 
\end{enumerate}
\end{definition}
Из определения очевидно, что у множества может быть не более одного супремума и одного инфимума 
\begin{theorem}
	Пусть $ M \subset R, M \neq \emptyset $ 
	\begin{enumerate}
		\item Если M ограничено сверху, то у M есть супремум
		\item Если M ограничено снизу, то у M есть инфимум

	\end{enumerate}
	\begin{proof}
		1. $ B = \{ b \in R \mid  \} $
	\end{proof}
\end{theorem}

\begin{theorem}
	Пусть $ X, Y \in \R $ ограниченные сверху мн-ва\\
	Тогда $sup(X+Y) = sup(X) + sup(Y)$ \\
	$ X + Y = \{ x + y \mid x \in X, y \in Y \} $
	\begin{consequence}
		Пусть $ a \in \R $  Тогда  $\exists n \in \N n > a $ 
		\begin{proof}
			Пусть это не так \\
			$ \forall n \in \N : n \leq a $ , т.е. $ \N $ ограничено сверху. \\
			$ \exists c = sup(\N) $ \\
			$ c - 1 < sup (\N) $ 
		\end{proof}
	\end{consequence}
\end{theorem}










	
\begin{theorem} Принцип вложенных отрезков\\

Пусть $A_i = [a_i, b_i], i = 1,2,3,\dots$ - отрезок \\
$ a_i, h \in \R, a_i < b_i $ и $ A_i \supset A_{i+1} , i=1,2,\dots $ \\
Тогда $ \bigcap_{i=1}^{\infty} A_i \neq \emptyset $\\
\begin{proof} $ X = \{ a_i | i=1,2\dots \}, Y=\{b_i | i = 1,2,\dots\} $ \\
$ \forall x \in X, \forall y \in Y: x \leq y $ \\
$ x = a_i \in N $\\
$ y = b_j  \in N$ \\
$ \exists i \leq j \ a_i \leq a_j < b_j \ a_i \leq a_{i+1} \leq a_{i+2} $ \\
$ \exists i > j \ a_i < b_i \leq b_j \ b_j \geq b_{j+1} \geq b_{g+2} $\\
Т. е. $ a_i < b_j $ \\
$ \exists c \in \R : \forall x \in X : x \leq c $ \\
$ \hspace*{16mm} \forall y \in Y : y \geq c $ \\
$ a_i \leq c \leq b_i $ т. е. $c \in A_i$ \\ 
$\Rightarrow \bigcap_{i=1}^{\infty} A_i \neq \emptyset$ \\
$ \bigcap_{i=1}^{\infty} A_i $ - точка или отрезок \\
\end{proof}
Аналогичное утверждение не верно для интервалов. $ A_i = (0, \frac{1}{i} ]$ \\
Аналогичное утверждение не верно для $\mathbb{Q}$ - $ \{a_i\} = \{1, 1.4, 1,41, \dots\}   \ \{b_i\} = \{2, 1.5, 1,42, \dots\},   \bigcap_{i=1}^{\infty} A_i = \sqrt{2} = \emptyset$
\end{theorem}

\Subsection{Мощность множества}

\begin{definition}
	Множества называются равномощными если существует биекция $ f: X \rightarrow Y $. Равномощность - отношение эквивалентности.\\
\end{definition}

\begin{definition} $ \sim $ - отношение эквивалентности \\
	\begin{enumerate}
		\item $X \sim X \ id_X$
		\item $ X \sim Y \Rightarrow Y \sim X$ 
		\item $ X \sim Y, Y \sim Z \Rightarrow X \sim Z $ 
		\item $ f: X \rightarrow Y, g: Y \rightarrow Z $ - биекции $\Rightarrow g \circ f $ - биекция 
	\end{enumerate}
\end{definition}
Mощность конечного множества - неотрицательное целое число. \\

\begin{definition}
	Множество X называется счётным, если оно равномощно множеству $ \N $
\end{definition}

\begin{example} $ $ \\
	$ \N \setminus \{1\} $ - счётное \\
	$ \N \setminus \{1\} \mapsto \N $\\
	$ x \mapsto x - 1 $ 
\end{example}

\begin{definition}
	Собственное подмножество - подмножество отличное от всего множества. 
\end{definition}

\begin{theorem}
	Любое подмножество счётного множества конечно или счётно \\
	$ x = \{ x, x_1, x_2, x_3, \dots | x_i \neq x_j, i \neq j \}$ \\
	\begin{proof}
		Пусть $ Y \subset X $ \\
		Y конечнно - очевидно \\
		Y бесконечно $ y_1 = x_{i_1}, i_1 $ - минимально \\
		$ y_2 = x_{i_2} $ т.ч. $ x_{i_2} \in Y \setminus \{y_1\} $ и $ i_2 $ - минимально\\
		$ y_3 = x_{i_3} $ т.ч. $ x_{i_3} \in Y \setminus \{y_1, y_2\} $ и $ i_3 $ - минимально\\
		И т.д. Все $ Y \setminus \{y_1, \dots, y_n \} $ непусты т.к. Y бесконечно. \\
		Проверим $ Y = \{y_1, y_2, \dots, y_n \} $ \\
		Пусть $ y \in Y \setminus \{ y_1, \dots, y_n \} $ \\
		$ \exists j : y = x_j $ \\
		$ y \in Y \setminus \{ y_1, \dots, y_n \} $ при любом n \\
		$ \Rightarrow \forall n \in \N, i_n < j $ В силу процедуры выбора $ y_n $ \\
		$\underbrace{i_1, i_2, \dots, i_j}_{\in \N} < j $ - не бывает \\
		$ \N \rightarrow Y - $ биекция
	\end{proof}
\end{theorem}


Мощность X не превосходит мощности Y если X равномощно подмножеству Y. $ |X| \leq |Y| $ \\
Равносильно инъекции $X \rightarrow Y$ \\
$ |X| \leq |\N \Rightarrow |X| \in \N_0 \mid  |X| = |N| $\\
$ |X| \leq Y, |Y| \leq |Z| \Rightarrow |X| \leq |Z|$ \\
Задача: доказать, если $|X| \leq |Y|, |Y| \leq |X| $, то $ |X| = |Y|$ \\
\begin{example}
	$\Z - $ cчётное \\
	-2, -1, 0, 1, 2, 3\\
	5, 3, 1, 2, 4, 6 \\
	$ \N \rightarrow \Z $ \\
	$ n \mapsto \frac{n}{2}, 2 \divby n \\
	-\frac{n-1}{2}, 2 \divbynot n $ \\
\end{example}
\begin{example}
	X, Y - счётное Тогда $ X \times Y - $ счётное. \\
\end{example}
$ X \times Y = \bigcup_{i=1}^{\infty}\{(x_m, y_n)\mid m+n=i \} $ \\
$ |C_i| = i - 1 $ \\
Элементы $C_i$ получают номера заканчивающиеся на $ \frac{(i-2)(i-3)}{2}+1 \dots \frac{(i-1)(i-2)}{2} $ \\
$ \mathbb{Q}$ - счётно
$ \mathbb{Q} \rightarrow \Z \times \R$ \\
$ q \mapsto (a, b) $ \\
$ \alpha $ инъективно(очевидно) \\
$ |\mathbb{Q} \leq |\Z \times \N | = |\N| = \aleph_0 $\\
$ |X| < |Y| \Leftrightarrow |X| \leq |Y|, |X| \neq |Y| $ \\
X - любое множество, $ |X| < |2^X| $ (Упражнение)\\
Любые 2 промежутка в $ \R - $ равномощны \\
Cперва докажем, что $ [a, b] \sim [a', b'], a < b, a' < b' $ \\
$ \exists $ линейная ф-ция $ y = f(x)$ т.ч. $ f(a) = a', f(b) = b' $ Она даёт биекцию $ [a, b] \rightarrow [a', b']$\\
Аналогично: $  [a, b) \rightarrow [a', b')$ \\
$  (a, b] \rightarrow (a', b']$\\
$ (a, b) \rightarrow (a', b')$\\
Проверим $ [0, 1) \sim [0,1] $ \\
$ x \mapsto \left\{ \begin{array}{ll} x, x \neq \frac{1}{n}, n \in \N \\ \dfrac{1}{n-1}, x = \dfrac{1}{n} \end{array} \right. $\\
Аналогично $ (0, 1] \sim [0,1] $\\
$ (0, 1) \sim (0,1] $\\
$\R$ равномощно $ [0, 1] $ \\
$ \left( -\frac{\pi}{2}, \frac{\pi}{2} \right] \mapsto \R \\
x \mapsto tg(x) $\\
$ \left( -\frac{\pi}{2}, \frac{\pi}{2} \right]  \sim [0, 1] $ \\
\begin{theorem}
	$\R$ - несчётно \\
	\begin{proof}
		Достаточно доказать, что $ (0, 1) $ - несчётный.\\
		Предположим, что это не так\\
		$ (0, 1) = \{ a_1, a_2, \dots \} $ \\
		$ a_i = 0,a_1a_2a_3a_4 $ Без 9 в периоде \\
		$ \beta_i $ - любая цифра, отличная от $ a_{ii}, i=1,2,3,\dots$ 0 и 9\\
		$ \beta_i = 0,\beta_1\beta_2\dots$\\
		$\beta \notin \{ a_1, a_2, a_3, \dots \} $\\
		Число $ \beta$ нельзя пронумеровать 
	\end{proof}
\end{theorem}

Континуум-гипотеза - любое несчётное подмножество  $\R$  равномощно $\R$ \\




	
\Section{Числовые последовательности}

\Subsection{Основные определения}

\begin{definition}
	Числовая последовательность - отображение $ a: \N \rightarrow \R $ \\
	$ \N \rightarrow \R $ \\
	$ n \rightarrow a_n $ \\
	$ a_n = 3n^2 - \sqrt{2} n+ \frac{5}{2} $ \\
	$ a_n = a_{n-1} + n^2, a_1 = 1$
\end{definition}

\begin{definition}
	Последовательность ограничена сверху, если существует  $ C \in \R : \forall i \in \N : a_i < C $  \\
	Последовательность ограничена снизу, если существует  $ C \in \R : \forall i \in \N : a_i > C $ \\
	Последовательность ограничена, если она ограничена и сверху и снизу.
\end{definition}

\begin{definition}
	Монотонные последовательности \\
	Неубывающая последовательность: $ \forall i \in \N : a_i \leq a_{i+1} $\\
	Возрастающая $ < $ \\
	Невозрастающая $ a_i \geq a_{i+1} $\\
	Убывающая $ > $ \\
	2, 4 - строго монотонные.\\
	Монотонные, начиная с некоторого места \\
	Неубывающая $ \exists n \in \N \forall i > n : a_i \leq a_{i+1} $\\
\end{definition}


\Section{Предел последовательности}

\begin{definition}
	$ \eps$-окрестность - $ (x - \eps, x+\eps) $
	\begin{enumerate}
		\item  Окрестность x - это $ \eps $ окрестность x для $ \eps > 0 $
		\item Окрестность x - произвольный интервал $ (a ,b) : a < x < b$
		\item Окрестность x - $U \subset \R :  U \supset (a, b) , a, b \in \R, a < x < b $ 
	\end{enumerate}
\end{definition}

Любая окрестность х содержит $\eps $ окрестноть х с $ \eps > 0 $ \\
\begin{definition}
	Пусть $(a_i)$ - последовательность, $ x \in \R $ Говорят, что $a_i$ сходится к x(имеет х своим пределом) если $ \forall U $ окрестности $x \exists N \in \N : a_i \in U $ при всех $ i \geq N $ \\
	Понятие предела не изменится, если мы определения окрестности 1 перейдём к определению 3. Если нужное свойство выполняется для класса 3, то оно, в частности, выполняется и для $ \eps $ окрестностей. \\
	Предположим оно выполнено для всех $ \eps $ окрестностей. Тогда $ U \supset(a, b), (a,b) > (x-\eps, x+ \eps) \eps > 0 \Rightarrow \exists N, \forall i \in \N, a_i \in U$  \\
\end{definition}
\begin{definition}
	Пусть $(a_i), X \in \R$ сходится к Х, если $\forall U $ 	числа Х $, \{ i \in \N \mid a_i \notin U \} $ конечно 
\end{definition}
\noindent Определение 1 $\Rightarrow$ Определение 2 \\
$ Y = \{ i \in \N | a_i \notin U \} \subset \{ 1, 2, \dots, N - 1 \} \Rightarrow $ конечность Y \\
Определение 2 $\Rightarrow$ Определение 1 \\ 
$ N = \max Y + 1, \forall i \geq N : i \notin Y$ т.е. $ a_i \in U$
\begin{definition}
	$ \lim a_i = x  \Leftrightarrow a_i \rightarrow x $ \\
\end{definition}

Последовательность имеет не больше одного предела.
\begin{proof}
	Предположим $ a_i \rightarrow x, a_i \rightarrow y, x < y $ \\
	$ \eps = \dfrac{y-x}{2} $ \\
	$Y_1 = \{ i \mid a_i \notin U_{\eps} (x) \} $конечно\\
	$ Y_2 = \{ i \mid a_i \notin U_{\eps} (y) \} $конечно\\
	$ \N = Y_1 \cup Y_2 $ конечно
\end{proof}


\Subsection{Свойства пределов}

Предл(предельный переход в неравенстве) \\
Пусть $ a_i \rightarrow x, b_i \rightarrow y $ \\
Предп. $ \forall i \in \N : a_i \leq b_i. $Тогда $ x\leq y $ \\
Пусть $ y < x, \eps = \dfrac{x-y}{2} $ 
$ \exists N_1 : \forall n \geq N_1 a_n \in U_{\eps}(x) $
$ \exists N_2 : \forall n \geq N_2 b_n \in U_{\eps}(y) $
$  N = max(N_1, N_2) \Rightarrow \forall n \geq N : 
a_n \in U_{\eps}(x)\\
b_n \in U_{\eps}(y)$\\
$ x- \eps <  a_n \leq b_n < y+\eps = x-\eps$\\
Противоречие \\
Следствие: Пусть $ a_i \leq C \forall i; a_i \rightarrow x \Rightarrow x \leq C $\\

\begin{theorem} Теорема о сжатой последовательности\\
	Пусть $ a_i \rightarrow x, b_i \rightarrow x $ \\
	$ c_i \ \ \forall i, a_i \leq c_i \leq b_i $ 
	\begin{proof}
		Возьмём $ \eps > 0 $ и док-м, что $ \exists N \in \N $ \\
		$ \forall i \geq N : c_i \in U_{\eps} (x) $ \\
		$ \exists N_1 : \forall i \geq N_1 : a_i \in U_{\eps} (x) $\\
		$ \exists N_2 : \forall i \geq N_2 : b_i \in U_{\eps} (x) $\\
		$ \Rightarrow N = max(N_1, N_2) $ обладает нужным 		свойством. \\
		Начиная с некоторого номера - менять/откидывать некоторые члены, предел от этого не ипзменится. \\

		Если сушествует $ \lim a_i = x $, то $(a_i)$ - ограничена\\
		$ \lim a_i = x $ \\
		Проверим огр-ть сверху \\
		$ \exists N : \forall i \geq N, a_i \in U_1 (x) $ \\
		в части $ a_i < x + 1 $ \\
		$ C = \max( a_1, \dots, a_{N-1}, x + 1) $ \\
		Очевидно $ \forall i \in \N : a_i \leq C $ \\
		Аналогично $ a_i $ ограничено снизу 
	\end{proof}
\end{theorem}

Если последовательность неубывающая и ограничена свеху, то она сходится
Если последовательность невозрастающая и ограничена снизу, то она сходится

\begin{proof}
	$ x = \sup\{a_i | i \in \N\} $ \\
	$ a_i \rightarrow x $\\
	Возьмём $ \eps > 0 $\\
	$ x - \eps < x \Rightarrow x - \eps  - $ не верхняя граница $ \Rightarrow \exists N \in \N, a_N > x-\eps $ \\
	$ a_i $ - неубывающая $ \Rightarrow \forall i \geq N : a_i > x-\eps \Rightarrow \forall i \geq N : a_i \in U_{\eps} (x)$\\
\end{proof}

\Subsection{Арифметические действия с пределами} 

Пусть есть $ a_i \rightarrow x, b_i \rightarrow y $ 
\begin{enumerate}
\item $ |a_i| \rightarrow |x| $
\item $ a_i + b_i \rightarrow x+y $ 
\item $ a_i - b_i \rightarrow x - y$
\item $ a_i \cdot b_i \rightarrow xy $ 
\item $ b_i \neq 0 \ \  \forall i, \dfrac{ a_i}{b_i} \rightarrow \dfrac{x}{y} $
\end{enumerate}

$ | | a_i | - | x| | \leq | a_i - x | $ \\
2. $ \eps > 0$
$\exists N:  \forall i > N: a_i \in U_{\frac{\eps}{2}} (x) , b_i \in U_{\frac{\eps}{2}} (y) \Rightarrow a_i + b_i \in U_{\eps}(x+y) $ \\
4. $  a_i b_i - xy = (a_i b_i - xb_i) + ( xb_i - xy )$ \\
$ | a_i b_i - xy | \leq  |a_i b_i - xb_i| + | xb_i - xy | $ \\ 
$ \exists C > 0 : | b_i | \leq C \forall i \in \N $ \\
$ | a_i b_i - xy | \leq C | a_i - x| + |x | \cdot | b_i - y | $\\
Возьмём $\eps > 0 $ 
Тогда $ \exists N \in \N \forall i \geq N : \left\{ \begin{array}{ll} |a - x| < 
\dfrac{\eps}{2C} \ \ a \in U_{\frac{\eps}{2c}} (x) \\
\\
|b_i - y | < \dfrac{\eps}{2|x|} \text{(при х }\neq 0) \end{array} \right. $ \\											
$ |ab_i - xy| < C \cdot \frac{\eps}{2C} + \frac{\eps}{2} = \eps $

5. Достаточно доказать, что $\dfrac{1}{b_i} \rightarrow \dfrac{1}{y}$ \\
$\left|\dfrac{1}{b_i} - \dfrac{1}{y} \right| = \left|\dfrac{y-b_i}{b_i y}\right| = \dfrac{|b_i - y|}{|b_i||y|}$ 	\\
$ \exists  N_0 \in \N \ \forall i > N_0 : |b_i| > \dfrac{|y|}{2} $\\
$ y > 0 : b_i \in U_{\dfrac{y}{2}}(y) \Rightarrow b_i  > \dfrac{y}{2} > 0 $\\
$ y < 0	:  b_i \in U_{-\dfrac{y}{2}}(y) \Rightarrow b_i  < \dfrac{y}{2} < 0 $ \\
$ \left|\dfrac{1}{b_i} - \dfrac{1}{y}\right| \leq \left|\dfrac{b_i - y}{|y|^2 / 2}\right| $ \\
Возьмём $ \eps > 0 $\\
$ \exists N \geq N_0 : \forall i \geq N_0 $
$ \exists N \geq N_0 : \forall i \geq N_0, |b_i - y | < \eps \cdot \dfrac{|y|^2}{2} $ \\
Отсюда при $ i \geq N : \left| \dfrac{1}{b_i} - \dfrac{1}{y} \right| < \dfrac{\eps\cdot\dfrac{|y|^2}{2}}{\dfrac{|y|^2}{2}} = \eps $ \\
То $ \dfrac{1}{b_i} \rightarrow \dfrac{1}{y} $ 

\Subsection{Бесконечные пределы} 

Пусть $a_i$ последовательность.
$ a_i $ стремится к $ +\infty$ если $ \forall C \in \R \  \exists N \in \N : \forall i \geq N : a_i > C $ \\
Определение станет аналогичным определению конечного предела, если определить окрестность $ +\infty $ как произвольный открытый луч $ (C, +\infty)$ \\
Равносильное определение: $ \{i \mid a_i \neq (C, +\infty) \} $ - конечно \\
Последовательность не имеет конечного предела - последовательность расходится \\
Пусть $a_i$ последовательность.
$ a_i $ стремится к $ -\infty$ если $ \forall C \in \R \  \exists N \in \N : \forall i \geq N : a_i < C $ \\
$ a_i \rightarrow +\infty \Leftrightarrow -a_i \rightarrow -\infty $\\
$ a_i \rightarrow \infty $ если $ \forall C \in \R \ \exists N \in \N : \forall i > N, |a_i| > C $ \\
$ ( C, +\infty )$ - окрестность $+\infty$ \\
$ (-\infty, C) $ \\
$ (-\infty, C) \cup ( C, +\infty), C > 0 $  \\

\begin{theorem}
	Не ограниченная сверху неубывающая последовательность стремится к $ + \infty $ \\
	Не ограниченная снизу невозрастающая последовательность стремится к $ -\infty $ \\
	\begin{proof}
		Пусть $ c \in \R $ \\
		a не огр сверху $ \Rightarrow \exists N : a_N > c \Rightarrow \forall i \geq N, a_i > c $
	\end{proof}
\end{theorem}

$ a_i \rightarrow \infty \Leftrightarrow |a_i | \rightarrow +\infty $ \\
\begin{definition}
	Бесконечно малая посл-ть $ a_i \rightarrow 0 $ \\
	Бесконечно большая посл-ть $ a_i \rightarrow \infty $ \\
\end{definition}

\begin{theorem}
	$ (a_i) $ - посл-ть, $\forall i : a_i \neq 0 \Rightarrow (a_i) $ бесконечно малая $ \Leftrightarrow (\dfrac{1}{a_i}) \rightarrow \infty $ \\
	$ (a_i) - $ б.м. $ \begin{array}{ll}
		 \Leftrightarrow \forall \eps > 0 \  \{ i | a_i \notin U_{\eps} (0) \} \text{кон} \\
		 \Leftrightarrow \forall \eps > 0 \ \{ i | |a_i|>\eps \} \text{кон} \\
		 \Leftrightarrow \forall \eps > 0 \  \{ i | |a_i^{-1}| < \eps^{-1} \} \text{кон} \\
		 \Leftrightarrow \forall c > 0 \  \{ i | |a_i^{-1}| < c \} \text{кон} \\
	\end{array}$ \\
	$ \Leftrightarrow a_i^{-1} \rightarrow \infty $
\end{theorem}
Замечание - сумма, разность, произведение бесконечно малых - бесконечно малая посл. \\
Пусть $(a_i) $ бесконечно большая, $(b_i) $ ограниченная, тогда их сумма бесконечно большая. \\
$ U_C(\infty) =  (-\infty, C) \cup ( C, +\infty) $ \\
Пусть $ C > 0 $ \\
Проверить $\{ i \mid a_i + b_i \notin U_C{\infty} \}$\\
$ \exists D > 0 : \forall i \in \N: |b_i | < D $ \\
$ \exists N \in \N : \forall i \geq N: |a_i| > C+D $ \\
При $ i \geq N : | a_i + b_i | \geq |a_i| + |b_i|  > C+D-D = C $ \\
$ a_i + b_i \in U_C(\infty) $ \\
Произведение бесконечно малой посл-ти $a_i$ на ограниченную $b_i$ - бесконечно малая \\
$ \exists C, \forall i, |b_i| < C $ \\
$ |a_i b_i | \leq C \cdot |a_i| $ 
$ a_i \rightarrow 0 \Rightarrow  |a_i| \rightarrow 0 \Rightarrow |a_i b_i| \rightarrow 0 $ \\
$ \overline{\R} = \R \cup \{+\infty, -\infty\} $ \\
$ +\infty + (-\infty) $ - не определена \\
$ a \cdot (+\infty) = (+\infty), a > 0 $\\
$ a \cdot (+\infty) = (-\infty), a < 0 $\\
$ a_n \leq b_n, a_n \rightarrow x \in \overline{\R}, b_n \rightarrow y \in \overline{\R} $ \\
Тогда $ x \leq y $\\
% ----------pic7--------------

\begin{enumerate}
	\item $lim x_n = +\infty, y_n $ огр. снизу $ \Rightarrow x_n + y_n \rightarrow +\infty $
	\item $lim x_n = -\infty, y_n $ огр. сверху $ \Rightarrow x_n + y_n \rightarrow -\infty $
	\item $ x_n \rightarrow +\infty (-\infty) \ y_n \geq C > 0 \ \forall n \in  \N \Rightarrow x_n y_n \rightarrow +\infty ( -\infty) $
	\item $ x_n \rightarrow a \neq 0, y_n \neq 0, y_n \rightarrow 0 \Rightarrow \dfrac{x_n}{y_n} \rightarrow \infty$
	\item $ x_n \rightarrow a \in \R y_n \rightarrow \infty \Rightarrow \dfrac{x_n}{y_n} \rightarrow 0 $
	\item $ x_n \rightarrow \infty, y_n \rightarrow b \in \R  \Rightarrow \dfrac{x_n}{y_n} \rightarrow \infty $ 
	\item $ (a_i) $ - бесконечно большая, $ (b_i) $ - ограниченная. Тогда $ (a_i + b_i) $ - бесконечно большая
\end{enumerate}

\noindent 3. Пусть $ E > 0 \  \exists N, \forall i \geq N, x_i > \dfrac{E}{C}, x_i y_i > \dfrac{E}{C} C = E $ \\
4. $ \exists E > 0, x_n \rightarrow a \Rightarrow \exists N_0 : |x_i| \geq \dfrac{|a|}{2} $ при  $ i \geq N_0$ \\
$ \dfrac{x_n}{y_n} $ беск. больш $ \Leftrightarrow \dfrac{y_n}{x_n} $ беск мал. $ \dfrac{1}{|x_i|} \leq \dfrac{2}{a} \Rightarrow \dfrac{1}{x_i}$ ограниченная \\
$ y_n \cdot \left( \dfrac{1}{x_n} \right)  $\\

Неравенство Бернулли \\
$ x > -1, n \in \N $ \\
$ (1+x)^n \geq 1 + nx $ \\
$ k \rightarrow k + 1 $ \\
$ (1+x)^{k+1} = (1+x)^k(1+x) \geq (1+kx)(1+x) = 1 + (k+1)x + kx^2 \geq 1 + (k+1)x $ \\
Сл. 1 \\
1. $ lim a^n = +\infty $ при $ a > 1 $ \\
2. $ lim a^n = 0$ при $ |a| < 1$ \\
Д-во \\
$ a^n = (1 + (a-1))^n \geq 1 + n(a-1) \rightarrow +\infty $\\
Сл. 2 Пусть $ a > 1 \Rightarrow \sqrt[n]{a} \rightarrow 1$\\
$ \sqrt[n]{a} = 1 + x_n $ \\
$ a = (1 + x_n) ^n > 1 + n \cdot x_n $ \\
$ x_n \leq \dfrac{1}{n} (a-1)  $ \\
$ 1 \leq \sqrt[n]{a} \leq  \dfrac{1}{n} (a-1) $ \\
По зажатой последовательности $ \sqrt[n]{a} \rightarrow 1 $\\

$ x_n = \left(1 + \dfrac{1}{n}\right)^n $\\
$ y_n =  \left(1 + \dfrac{1}{n}\right)^{n+1} $ \\
$ \dfrac{y_{n-1}}{y_n} = \dfrac{ \left(1 + \dfrac{1}{n-1}\right)^{n}}{ \left(1 + \dfrac{1}{n}\right)^{n+1}} = \left( \dfrac{1+\dfrac{1}{n-1}}{1 + \dfrac{1}{n}} \right)^{n-1} \left( 1 + \dfrac{1}{n-1} \right)^{-1}  = \left( \dfrac{n+\dfrac{n}{n-1}}{n + 1} \right)^{n+1} \dfrac{n-1}{n}  =  \left( \dfrac{n+1+\dfrac{n}{n-1}}{n + 1} \right)^{n+1} \dfrac{n-1}{n}  \geq \left(1 + (n+1)\dfrac{1}{n-1} \right)\dfrac{n-1}{n} = \left( 1+ \dfrac{1}{n-1} \right) \dfrac{n-1}{n}  = 1$\\
Т.е. $ y_n $ невозрастающ. $ \Rightarrow y_n \rightarrow y$\\
$ x_n = y_n \cdot ( 1 + \dfrac{1}{n})^{-1} \rightarrow y $ \\

	\begin{theorem}
	Пусть $x_n$ последовательность, $ x_n > 0, \lim \dfrac{x_{n+1}}{x_n} \leq 1 \Rightarrow \lim x_n = 0 $
	\begin{proof}
		Пусть $ c $ такое число, что $  \lim \dfrac{x_{n+1}}{x_n} \leq c \leq 1 $ \\
		$ \exists N, \forall n \geq N,  \dfrac{x_{n+1}}{x_n} < c $ \\
		$ \forall m \in \N :  \dfrac{x_{n+1}}{x_n} = \prod_{i=0}^{m-1}  \dfrac{x_{n+1}}{x_n} < c^m $ \\
		$ 0 < x_{n+m} < c^mx_n \underset{m \rightarrow \infty}{\rightarrow} 0 $ \\
		$ \Rightarrow \lim\limits_{n \rightarrow \infty} x_{n+m} = 0 =  \lim\limits_{n \rightarrow \infty} x_{n} $

	\end{proof}
	\begin{consequence}
		$ a > 1 \Rightarrow \dfrac{n^k}{a^n} \rightarrow 0 $
		\begin{proof}
			$ x_n = \dfrac{n^k}{a^n} $ \\
			$ \dfrac{x_{n+1}}{x^n} = \dfrac{(n+1)^k}{n^k \cdot a} = (1 + \dfrac{1}{n})^k \cdot \dfrac{1}{a} \rightarrow \dfrac{1}{a} \leq 1 $
		\end{proof}
	\end{consequence}
\end{theorem}
\begin{definition}
	Пусть $ x_n, y_n $ бесконечно большие. Говорят, что $ x_n $ бесконечно большая меньшего порядка, если $ \dfrac{x_n}{y_n} \rightarrow 0, x_n = O(y_n) $ \\
	$ n^k = O(a^k) $ 
\end{definition}
\begin{consequence}
$ \dfrac{a^n}{n!} \rightarrow 0 \forall a > 0 $ \\

$ \dfrac{x_{n+1}}{x_n} = a \cdot \dfrac{1}{n+1} \rightarrow 0 $ 
\end{consequence}
\begin{consequence}
	$ \dfrac{n!}{n^n} \rightarrow 0 $ \\
	\begin{proof}
		$ \dfrac{x_{n+1}}{x_n} = (n+1)\dfrac{n^n}{(n+1)^{n+1}} = \dfrac{n^n}{(n+1)^n} = (\dfrac{n}{n+1})^n = \dfrac{1}{(1+\frac{1}{n})^n} $
	\end{proof}
	
\end{consequence}

\Section{Теорема Больцано-Вейерштрасса и критерий Коши}

\begin{theorem}
	Теорема о стягивающихся отрезках \\
	Пусть заданы отрезки числовой прямой $ [a_i, b_i], i = 1, 2, 3,...$ \\
	\begin{enumerate}
		\item $ [a_i, b_i] \supset [a_{i+1}, b_{i+1}] $ 
		\item 
		% pic1
	\end{enumerate}
	\begin{proof}
		Знаем $ \cap [a_n, b_n] \neq \emptyset $ \\
		Предположим, что это не 1 элем. мн-во \\
		Тогда $ \exists c, d \in  \cap [a_n, b_n], c < d $ \\
		$ \forall n \in \N : c, d \in  [a_n, b_n] \Rightarrow a_n \leq c < d \leq b_n  \Rightarrow b_n - a_n \geq $
		% pic2,3
	\end{proof}
\end{theorem}
\begin{definition}
	Говорят, что $ b_n $ подпоследовательность $ a_n $ если \\
	возр. посл-ть натуральных чисел $ m_i $ \\
	$ \forall n \in \N$
	
\end{definition}
\begin{theorem}
	Теорема Больцано-Вейерштрасса \\

	Пусть $ x_n$ - ограниченная последовательность. Тогда в $ x_n $ можно выбрать сходящуюся подпоследовательность
	\begin{proof}
	$ (x_n) \Rightarrow \exists a, b \in \R : a \leq b, \forall n : x_n \in [a, b] $ \\
	Из отрезков $ [a, a+b / 2] $ и $ [a+b/2, b] $ выберем $ [a_1, b_1] $ такой, что $ \{i \mid x_i \in [a_1, b_1]\} $ бесконечно. \\
 	Из отрезков $ [a_1, a_1+b_1 / 2] $ и $ [a_1+b_1/2, b_1] $ выберем $ [a_2, b_2] $ такой, что $ \{i \mid x_i \in [a_2, b_2]\} $ бесконечно. \\
    $ b_i - a_i = \frac{1}{2^n} (b-a) $ \\
    По теореме о стягивающихся отрезках 
    $ \cap [a_n, b_n] = \{c\}$ \\
    $ m_1 $ такое число, что $ x_{m_1} \in [a_1, b_1] $ \\
    $ m_2 $ такое число, что $ x_{m_2} \in [a_2, b_2] $ \\
    ...\\
    $ x_{m_i} $ - подпоследовательность \\
    $ a_i \leq x_{m_i} \leq b_i  \Rightarrow x_m \rightarrow c$ 
    \end{proof}
	Зам. Легко видеть, что  если $ x_n \rightarrow c \in \R $ \\
	То любая подпоследовательность $ x_{m_i} \rightarrow c $ \\
\end{theorem}
Дополнение \\
Пусть $x_n$ - неограниченная последовательность. Тогда в ней есть подпоследовательность $ x_{n_i}  \rightarrow +\infty $ или $ \rightarrow -\infty $ \\
\begin{proof}
	Пусть $ x_n $ не ограничена сверху \\
	Тогда легко видеть, что $ \forall n \in \N $ существует бесконечно много $ i \in \N : x_i > n $ \\
	Иначе существуеть лишь конечн. i \\
	Пусть это так, тогда $ max(x_1, x_2, ..., n) $ - верхняя граница $x_n$\\
	$ m_1 $ - любое нат. число, $ x_{m_1} > 1 $ \\
	$ m_2, x_{m_2} > 2, m_2 > m_1 $ \\
	$ (x_{m_i})  -$ посл-ть \\
	$ \{i \mid x_{m_i} \notin (n, +\infty) \} \subset \{ 1,2,..., n-1 \} $ \\
	То $ x_{m_i} \rightarrow +\infty $\\
	Аналогично для $ -\infty $ 
\end{proof} 
\begin{consequence}
	%pic5
\end{consequence}

\begin{definition}
	Пусть $x_n$ - последовательность. Она называется фундаментальной, если выполненое след. св-во \\
	$ \forall \eps,  \exists N \in \N, \forall m,n \geq N : | x_m - x_n | < \eps $\\
	Очевидно, сходящаяся последовательность фундаментальна. Можно взять предел и окресность $ \frac{\eps}{2} $ \\
	Фундаментальная $=$ сходящаяся в себе $=$ последовательность Коши.
\end{definition}
\begin{theorem}
	Теорема Больцано-Коши(Критерий Коши) \\
	Последовательность $x_n$ фундаментальна $ \Leftrightarrow $ сходится
	\begin{proof}
		Пусть $ (x_n) $ - фундаментальна \\
		1. Докажем, что $(x_n)$ - ограничена \\
		По определению, $  \exists N \in \N \forall n \geq N : x_n \in (x_{N} - 1, x_N +1)$\\
		$ a = min(x_N - 1, x_n, ..., )$ %pic6
		2. По т. Б-В, $ (x_n)$ содержит сх. подпоследовательность $(x_{m_i}) $ \\
		Пусть $ x_{m_i} \rightarrow c $ \\
		Д-м, что $ x_n \rightarrow c $ \\
		По опр фунд. посл $ \exists N \in \N  : \forall m, n \geq N : |x_m-x_n| < \frac{\eps}{2}$ \\
		$\exists N' \in \N : \forall i \geq N' : |x_{m_i} -c | < \frac{\eps}{2} $ \\
		Пусть i таково, что %pic7,8,9,10
		
	\end{proof}
\end{theorem}
I R как множество дедекиндовых сечений\\
2 R как множество классов посл-тей коши\\
M - мн-во последовательностей Коши рац. чисел %pic11


	
7-адический показатель \\
$ v_7\left( 7^b \dfrac{p}{l} \right) $ \\
$ v_7 \left( \dfrac{3}{14} \right)  = -1 $ \\
7-адическое расстояние \\
$ \rho (a, b) = 7  \ \ \ -v_7(a - b) $ \\
$ \rho (a, a) = 0 $ \\
$ a_i - $ последовательность Коши \\
$ \forall \varepsilon > 0, \exists N \in \mathbb{N}, \forall i,j > N, v_7(a, b) < \varepsilon$ \\
$ M \setminus \sim = \mathbb{Q}_2 $ \\
\Section{Верхние и нижние пределы. Частичные пределы}

\begin{definition}
$ x_k - $ ограничено \\
$ y_n = \sup\limits_{k \geq n} x_k \in \mathbb{R} $ \\
$ y_1 \geq y_2 \geq y_3 ...$\\
$ y_n - $ ограничено \\
$ x_k \geq C \Rightarrow \forall n : y_n \geq C $ \\
$ \lim y_n = \varlimsup\limits_{n \rightarrow \infty} x_n = \lim\sup\limits_{n \rightarrow \infty} x_n  - $ верхний предел посл. $ x_n $ \\
$ x_n $ не огр. сверху $ \Rightarrow \varlimsup x_n = -\infty $\\
$ x_n $ огр сверху, но не огр. снизу $ \Rightarrow \varlimsup \in \mathbb{N} \cup -\infty $ \\
$ z_n = \inf\limits_{k \geq n} x_k $ \\
$ z_1 \leq z_2 ... $\\
$ \lim z_n = \varliminf x_n = \lim\inf x_n $ - нижний предел 
\end{definition}
Прежлож . Пусть $x_n$ - произв. послед \\
$ \varliminf \leq \varlimsup $\\
$ y_n \leq z_n \Rightarrow \lim y_n \leq \lim z_n $ \\

\begin{definition}
	Пусть $ x_n $ посл-ть, $ a \in \mathbb{R} $ \\
	$ a $ - частичный предел $ x_n $, если в $ x_n $ есть подпосл-ть, стремящаяся к $ a $ \\
	\begin{theorem}
		$ x_n $ - посл-ть \\
		1. Её верхний предел - наибольший частичный предел \\
		2. Её нижний предел - наименьший частичный предел \\
		3. $ \exists \lim x_n \Leftrightarrow \varlimsup x_n = \varliminf x_n $ \\
		$  \lim x_n =  \varlimsup x_n = \varliminf x_n $ 
		\begin{proof}
			Пусть $ A = \varlimsup x_n $ Предположим $ A \in \mathbb{R} $\\
			$ u_n = \left( A - \dfrac{1}{n}, A + \dfrac{1}{n} \right) $ \\
			$ x_{m_1} \in u_1, x_{m_2} \in u_2 $ \\
			$ y_n = \sup\limits_{k \geq n} x_k $ \\
			Предпол. $ x_1 \notin u_1 $ % --pic3,4 
			
			Предп $ \exists A' > A : A' - $ частичный предел $ x_n $, т.е. $ \exists x_{m_k} \rightarrow A' $ \\
			$ \varepsilon = (A' - A) / 2 $ \\
			% pic5
		\end{proof}
		\begin{proof}
			Пусть $ \exists \lim x_n = A \Rightarrow $ любой частичный предел равен $ A $  \\
			$ \varliminf x_n = \varlimsup x_n = A $\\
			Пусть $ \varliminf x_n = \varlimsup x_n = A $ \\
			$ A \in \mathbb{R} $ \\
			$ \varliminf x_n = A \Rightarrow \exists N_1 \forall n \geq N_1 : z_n \in  $ \\% pic6]
		\end{proof}
	\end{theorem}
\end{definition}
Предл. Пусть $ \forall n \in \mathbb{N} : a_n \leq b_n $ Тогда $\varlimsup a_n \leq \varlimsup b_n, \varliminf a_n \leq \varliminf b_n $ \\
\begin{theorem}
	1. $ A = \varliminf x_n  \Leftrightarrow \left\{ \begin{matrix}
	\forall \varepsilon > 0 \  \exists N, \forall n \geq N, x_n > a - \varepsilon \\
	\forall \varepsilon > 0 \  \forall N, \exists n \geq N , x_n  < a + \varepsilon 
	\end{matrix}  \right. $\\
	2. $ A = \varlimsup x_n  \Leftrightarrow \left\{ \begin{matrix}
	\forall \varepsilon > 0 \ \forall N, \exists n \geq N, x_n > a - \varepsilon \\
	\forall \varepsilon > 0 \  \exists N, \forall n \geq N , x_n  < a + \varepsilon 
	\end{matrix} \right.  $ \\
	\begin{proof}
		$ \Rightarrow x_n \geq z_n \rightarrow a (z_n  \inf\limits_{k \geq n} x_k) $\\
		$ \exists N_0 \forall n \geq N_0 : z_n < a + \varepsilon  $ \\
		т. е. $ \exists k \geq n : x_k < a + \varepsilon $ \\
		$ \Leftarrow \varepsilon > 0, \exists n \forall n \geq N : z_n \in u_{\varepsilon}(a) $\\
		$ z_n > a - \varepsilon $ \\
		По (1) $ \exists N_1 \forall n \geq N_1 : x_n > a - \frac{\varepsilon}{2} \Rightarrow \forall n \geq N_1 : z_n \geq a - \frac{\varepsilon }{2} > a - \varepsilon $ \\
		$ \forall n \in \mathbb{N} : z_n = \inf\limits_{k \geq n} x_k < a + \varepsilon $ (при нек. k) \\
		Т.е. $ z_n \rightarrow k $
		% pic7
	\end{proof}
\end{theorem} 
Предложение. Пусть $ a_n b_n $ последовательности
\begin{enumerate}
	\item $ \varliminf(a_n + b_n) \geq \varliminf a_n + \varliminf b_n $ 
	\begin{proof}
		$ \inf\limits_{k \geq n} (a_k + b_k) \geq  \inf\limits_{k \geq n} a_k +  \inf\limits_{k \geq n} b_k $ \\
		$ \forall k \geq n : a_k + b_k \geq  \inf\limits_{k \geq n} a_k +  \inf\limits_{k \geq n} b_k \Rightarrow \lim\inf\limits_{k \geq n} (a_k + b_k ) \geq \lim( \inf\limits_{k \geq n} a_k +  \inf\limits_{k \geq n} b_k) $\\
	\end{proof}
	\item $ \varlimsup(a_n+b_n) \leq \varlimsup a_n + \varlimsup b_n $ 
\end{enumerate}

\section{Ряды}

\begin{definition}
	$ \sum_{k=1}^{\infty} a_k $\\
	Говорят, что сумма ряда $  \sum_{k=1}^{\infty} a_k $ равна  $ b \in \mathbb{R} $ если $ S_n \rightarrow b, $ где $ S_n =  \sum_{k=1}^{n} a_k$
	Посл-ть частичных сумм ряда $  \sum_{k=1}^{\infty} a_k$ \\
	$ (c_n) a_n = \left\{ \begin{matrix}
		c_1, n = 1 \\
		c_n - c_1, n > 1 
	\end{matrix} \right. $ 
	Если $  \sum_{k=1}^{\infty} a_k = b \in \mathbb{R}, $ говорят, что ряд сходится.
\end{definition}
Предл. Пусть $  \sum_{k=1}^{\infty} a_k $ сходится, тогда $ a_k \rightarrow 0 $ 
Д-во $ a_k = S_k - S_k-1 = b - b = 0$\\
\begin{example}
	 $ \sum_{k=1}^{\infty} q^n $\\
	 $ s_k = \dfrac{q - q^n}{1 - q} $ \\
	 $ |q | > 1\ \ S_k \rightarrow \infty $ \\
	 $ q = 1 \ \ S_k = k \rightarrow \infty $ \\
	 $ | q | < 1, S_k \rightarrow \dfrac{q}{1- q} $ \\
\end{example}
\begin{example}
	$  \sum_{k=1}^{\infty} \dfrac{1}{k} = +\infty $ \\
\end{example}
\begin{properties}
	1. $ \sum(a_k + b_k) = \sum a_k + \sum b_k $ если $a_k $ $ b_k $ сходятся.\\
	2. $ \sum (c \cdot a_k ) = c \cdot \sum a_k $ \\
	3. Если ряд сходится, то сходится и имеет ту же самую сумму ряд, полученный расстановкой скобок. \\
	
\end{properties}
\begin{theorem} Теорема Штольца \\
	Пусть $ y_1 < y_2, \lim y_n =  +\infty $ \\
	$ x_n, \lim \dfrac{x_n - x_{n-1}}{y_n - y_{n-1}} = l \in \mathbb{R} \Rightarrow \dfrac{x_n}{y_n} = l $
	\begin{proof}
		1. $ l = 0, \varepsilon_n = \dfrac{x_n - x_{n-1}}{y_n - y_{n-1}} $ \\
		$ \varepsilon > 0, \exists m \forall n \geq m |\varepsilon_n | < \varepsilon $ \\
		$ x_n - x_m = $
		%pic8,9,10
	\end{proof}
\end{theorem}
$ \lim\limits_{n \rightarrow \infty} \dfrac{1}{n^{m+1}} \sum_{k=1}^{n} k^m $ \\
$ y_n = n^{m+1} $ \\
$ x_n = \sum_{k=1}^{n} k^m $ \\
$ \dfrac{x_n - x_{n-1}}{y_n - y_{n-1}} = \dfrac{n^m}{n^{m+1}- (n-1)^{m+1}} = \dfrac{n^m}{n^{m+1} - (n^{m+1} - (m+1)n^m) + \frac{(m+1)m}{2} n^{m-1}} = \dfrac{n^m}{(m+1) n^m - \frac{(m+1)m}{2} n^{m-1} } = \dfrac{1}{m+1} $ \\
	
	
\begin{theorem}
	$ y_1 > y_2 > ... > 0 $ \\
	$ \lim x_n = \lim y_n = 0 $ \\
	$ \lim \dfrac{x_n - x_{n-1}}{y_n - y_{n-1}} = l \in \mathbb{R} $ \\
	$ \lim \dfrac{x_n}{y_n} = l $ \\
	\begin{proof}
		1. $l = 0 \ \ \ | \varepsilon_k | < \varepsilon $ при $ k \geq m $ \\
		$ | x_n - x_m | \leq \varepsilon (y_m - y_n) $ \\
		2. $ l \in \mathbb{R} $ сводится к п.1
		3. $ l = + \infty $ \\
		$ \dfrac{ x_k - x_{k-1}}{y_k - y_{k-1}} > 1 \Rightarrow x_k \leq x_{k-1} $\\
		$ x_k \searrow, x_k \rightarrow 0 \Rightarrow x_k > 0$ начиная с нек. $k$ \\
		$ \dfrac{y_k - y_{k-1}}{x_k - x_{k-1}} \rightarrow 0 \Rightarrow \dfrac{y_k}{x_k} \Rightarrow 0 \Rightarrow \dfrac{x_k}{y_k} \rightarrow +\infty $\\
		4. 
 	\end{proof}
\end{theorem}

\section{Пределы функций}

\subsection{Предельные точки множеств}

$ E \subset \mathbb{R} $ \\
$ a \in \mathbb{R} - $ предельная точка мн-ва E \\
$ \forall \varepsilon > 0 \exists x \in E : \left\{ \begin{matrix}
|x-a| < \varepsilon \\
x \neq a 
\end{matrix}\right. 
$\\
1. $ (a, b') = [a, b] $ \\
2. $ \mathbb{Q}' =  \mathbb{R} $\\
3. $\dfrac{1}{\mathbb{R}}' = {0} $ \\
Предл. След 3 условия эквивалентны
1. а предельная точка Е \\
2. В любой окресности а есть бесконечно много точек Е \\
3. Сущ посл-ть $(x_n), \forall n : x_n$
\begin{proof}
	$ 1 \Rightarrow 2 $ \\
	Пусть $ \dot{u}_{\varepsilon} (a) \cap = {x_1 .. x_n} $ \\
	Пусть $ \varepsilon' = \min (\varepsilon, |x_i - a | ) $\\
	$ \dot{u}_{\varepsilon'} (a) \cap E = \emptyset $ \\
\end{proof}
\Subsection{Предел функции}
Пусть $ E \subset \mathbb{R} $ \\
$ f : E \rightarrow \mathbb{R} $ \\
a - пред. точка \\
Говорят, предел $ f $ в а равен $y$ 
Если $ \forall $ окресн. $ U_{\varepsilon} (y) $ сущ $ \delta > 0 $ \\

	\Section{Матрицы и операции над ними}

R - кольцо\\
$ \begin{pmatrix}
	a_{11} & ... & a_{1m} \\
	...& ...& ... \\
	a_{n1} & ... & a_{nm}
\end{pmatrix}\\
Mat(R, n, m) \\
A = (a_{ij})$\\
$ R \times Mat(R, n, m) \rightarrow Mat(R, n, m) $\\
$ a \times A =  \begin{pmatrix}
a\cdot a_{11} & ... & a\cdot  a_{1m} \\
...& ...& ... \\
a\cdot  a_{n1} & ... & a\cdot a_{nm}
\end{pmatrix}$ \\
$ A + B = (a_{ij} + b_{ij})  \forall i < n, j < m$ \\
$ + : Mat(R, n, m) \times Mat(R, n, m) \rightarrow Mat(R, n, m) $\\
$ \cdot : Mat(R, n, m) \times Mat(R, m, l) \rightarrow Mat(R, n, l) $ \\
$ A (i, j), B(j, k) $ \\
$ AB = C = (c_{i k}) $ \\
$ c_{ik} = \sum_{j=1}^{m} a_{ij} b_{jk} $ \\
$ \begin{pmatrix}
	a_{i1} & ... & a_{im} \\
\end{pmatrix}  \begin{pmatrix}
	b_{1k}  \\
	...  \\
	b_{mk}
\end{pmatrix} = (c_{ik})\\$
$A - n \times m, B - m \times l $\\
$ x_1, ..., x_e $ \\
$y_1, ..., y_m $ \\
$z_1, ..., z_m $ \\
Линейное преобразование $ f(ax) = a \cdot f(x) $\\ 
y линейно зависит от х \\
z линейно зависит от y\\
$ y_1 = b_{11} x_1 + ... + b_{1l}x_l $ \\
$ \vdots ---- $\\
$ y_m = b_{m1}x_1 + ...+ b_{ml} x_l$

$ z_i = \sum_{r=1}^m a_{ir} y_r = \sum_{r=1}^{m} a_{ir}(\sum_{j=1}^{l} b_r x_j )  = \sum_{r=1}^{m} \sum_{j=1}^{l} a_{ir}b{rj}x_j =\\=  \sum_{i=1}^{l} \sum_{j=1}^{m} a_{ir}b{rj}x_j = \sum_{i=1}^{l}  (\sum_{r=1}^{m}a_{ir}b_{rj})x_j $ \\
Матрицы с одним столбцом называются векторами-столбцами, с 1 строкой - векторами-строками.
 
$ X =\begin{pmatrix}
X_1 \\
\vdots \\
X_l
\end{pmatrix}  Y=\begin{pmatrix}
Y_1 \\
\vdots \\
Y_m
\end{pmatrix}X =\begin{pmatrix}
Z_1 \\
\vdots \\
Z_n
\end{pmatrix}$\\
$ Y = B \cdot X, Z = A \cdot Y, Z = (A\cdot B) \cdot X $\\
$ A \in M(n,m,R)\\
B \in M(m,l,R)\\
C \in M(l,k,R)\\
(AB)C = A(BC) $\\
$ ((AB)C)_{ij} = \sum_{s=1}^{l} (AB)_{ij} C_{sj} $\\
$ \sum_{s=1}^{l} (\sum_{r=1}^{m} (A_{ir} B{rs} )) = \sum_{r=1}^{m} \sum_{s=1}^{l} A_{ir} (B_{rs} C_{sj}) = \sum_{r=1}^{m} A_{ir} (\sum_{s=1}^{l} B_{rs} C_{sj}) = \sum_{r=1}^{m} A_{ir} (BC)_{rj}  = (A(BC))_{ij}$\\
$ A \in M(n,m,R)\\
B,C \in M(m,l,R)$\\
$ D \in M(l,k,R) $ \\
Тогда $ A(B+C) = AB+AC$\\
$ (B+C)D = BC+CD $ - доказать упражнение \\

$ a\in R, B,C  - $ матрицы одного размера \\
1. $ a(B+C) = aB + aC $\\
2. $ a(BC) = (aB)C $\\
3. Если R -коммутативно \\
$ a(BC) = B(aC) $\\

Матричное умножение некоммутативно \\
$ A - n \times m$\\
$ B m \times l $ \\
$ AB, l \neq m \Rightarrow BA $ не определено \\
$ l = m $\\
$ AB - n \times n, BA - m \times m $\\
Если $ n \neq n $ то размеры получающихся матриц не совпадают
$ l = m = n $ \\
$ n = 1, R $- коммутативно, $ M(1,1,R) $ - коммутативно \\
$ n \geq 2 - $ не коммутативно.\\
$ A = \begin{pmatrix}
0 & 1\\
0 & 0 
\end{pmatrix}
B = \begin{pmatrix}
0 & 0 \\
1 & 0
\end{pmatrix} \\
AB = \begin{pmatrix}
1 & 0 \\
0 & 0 
\end{pmatrix}
BA = \begin{pmatrix}
0 & 0\\
0 & 1
\end{pmatrix} AB \neq BA$
\begin{theorem}
	R - кольцо с 1\\
	$ M_n (R) = M(n,n,R) - $ кольцо с 1, если $ n \geq 2 $ то не коммутативн\\
	$ \overline{M_n} (R)  $ - кольцо матриц размера n под кольцом R \\
	Для сложения свойства очевидны. 
	$ \mathds{O} = \begin{pmatrix}
		0 & ... & 0 \\
		\vdots & ... & \vdots \\
		0 & ... & 0
	\end{pmatrix} $\\
	 $ A = (a_{ij}) \\
	 -A = (-a_{ij}) $
	 $ \mathds{1} = \begin{pmatrix}
	 1 & ... & 0 \\
	 \vdots & 1 & \vdots \\
	 0 & ... & 1
	 \end{pmatrix} $\\
	 Если $ n \geq 2, $  в $ M_n(R) $ есть нетривиальные делители нуля - не всякий элемент обратим \\
	 $ A = \begin{pmatrix}
	 	0 & 1\\
	 	0 & 0 
	 \end{pmatrix} \neq \mathds{0} $ \\
	 $ A^2 = \begin{pmatrix}
	 0 & 1\\
	 0 & 0 
	 \end{pmatrix} \begin{pmatrix}
	 0 & 1\\
	 0 & 0 
	 \end{pmatrix} = \begin{pmatrix}
	 0 & 0\\
	 0 & 0 
	 \end{pmatrix} 
	 A^2 = 0 \Rightarrow A - $ необратим
 \end{theorem}

\subsection{Ещё раз о комплексных числах}

\begin{definition}
	R - кольцо \\
	$ \emptyset \neq R_1 \subseteq R $\\
	$ R_1 - $ подкольцо в R если оно является кольцом отностительно тех же операций, что и в R \\
	$ + : R \times R \rightarrow R $ \\
	$ + : R_1 \times R_1 \rightarrow R_1 $ \\
	Говорят, что $ R_1 $ замкнуто относительно сложения (умножения)
\end{definition}
Предложение $ \emptyset \neq R_1 \subseteq R $ является подкольцом если оно замкнуто по сложению, взятию обратного по сложению и умножения. 
$ a, b \in R_1, a+b \in R_1, -a \in R_1, ab \in R_1 $ \\
Рассмотрим $ M_2(\mathbb{R}) $ \\
$\left\{  \begin{pmatrix}
a & -b\\
b & a 
\end{pmatrix} \top a,b \in \mathbb{R} \right\} = C$ \\
$  \begin{pmatrix}
a & -b\\
b & a 
\end{pmatrix} +  \begin{pmatrix}
c & -d\\
d & c 
\end{pmatrix} =  \begin{pmatrix}
a+c & -(b+d)\\
b+d & a+c 
\end{pmatrix} \in C $ \\
$  \begin{pmatrix}
a &- b\\
b & a 
\end{pmatrix}  \begin{pmatrix}
c & -d\\
d & c 
\end{pmatrix} =  \begin{pmatrix}
ac-bd & -(ad+bc)\\
bc+ad & -bd+ac 
\end{pmatrix} \in C $ \\
$ \begin{pmatrix}
c & -d\\
d & c 
\end{pmatrix}  \begin{pmatrix}
a &- b\\
b & a 
\end{pmatrix} = 
 \begin{pmatrix}
ca-db & -(cb+da)\\
da+cb & -db+ca
\end{pmatrix} =  \begin{pmatrix}
a &- b\\
b & a 
\end{pmatrix}  \begin{pmatrix}
c & -d\\
d & c 
\end{pmatrix} $
C - коммутативное кольцо\\
$ (a,b) \neq (0,0) $ \\
$  \begin{pmatrix}
a &- b\\
b & a 
\end{pmatrix}  \cdot \dfrac{1}{a^2+b^2}  \begin{pmatrix}
a & b\\
-b & a 
\end{pmatrix} = \dfrac{1}{a^2+b^2} \begin{pmatrix}
a^2+b^2 & ab-ba\\
ba-ab & b^2+a^2
\end{pmatrix}   = \begin{pmatrix}
1 & 0\\
0 & 1 
\end{pmatrix}   \Rightarrow C - $ поле \\
Изоморфизм $ \mathbb{C} $ и C \\
$ (a, b) \rightarrow 
 \begin{pmatrix}
a & -b\\
b & a 
\end{pmatrix} $\\
$ \phi : \mathbb{C} \rightarrow C $ \\
$ z_1 = a+bi = (a,b) \\
z_2 = c + di = (c,d) $\\
$ z_1 + z_2 = a + c + (b+d)i $ \
$ \phi(z_1+z_2) = \phi(z_1) + \phi (z_2) $\\
$ z_1 z_2 = ac - bd + (ad + bc)i $ \\
$ \phi(z_1 z_2) = \phi(z_1) \phi(z_2) $\\
$ \mathbb{C} \cong C $

\subsection{Тело кватернионов}

$ M_2 (\mathbb{C}) $ \\
$ \mathcal{H} =\left\{ \begin{pmatrix}
z_1 & - \overline{z_2} \\
z_2 &  \overline{z_1}
\end{pmatrix}, z_1,z_2 \in \mathbb{C} \right\}$\\
$  -\begin{pmatrix}
z_1 & - \overline{z_2} \\
z_2 &  \overline{z_1}
\end{pmatrix} =  \begin{pmatrix}
-z_1 & -\overline{z_2} \\
-z_2 &  -\overline{z_1}
\end{pmatrix} \in \mathcal{H} $\\
$ -\overline{(z_2)} = \overline{z_2}$ \\
%pic2,3
$ \mathcal{H} $ - тело, но не поле \\






	\subsection{title}

$ 1 = \begin{pmatrix}
1 & 0 \\
0 & 1
\end{pmatrix}
I = \begin{pmatrix}
i & 0 \\
0 & -i
\end{pmatrix}
J = \begin{pmatrix}
0 & i \\
i & 0
\end{pmatrix}
K = \begin{pmatrix}
0 & -i \\
1 & 0
\end{pmatrix} $ \\
$ z = c_0 + c_1 I + c_2 J + c_3 K $ \\
$ I^2 = J^2 = K^2 = -1 $ \\
$ IJ = -JI = K $ \\
$ JK = -KJ = I $\\
$ KI = -IK = K $ \\
Вторая конструкция \\
$ \{ ( c_0, c_1, c_2, c_3 ) | c_i \in \R \}$ \\
$ 1 = (1, 0, 0, 0) $ \\
$ i = (0, 1, 0, 0) \\
j = (0, 0, 1, 0) \\
k = (0, 0, 0, 1) $ \\
$ (c_0, c_1, c_ 2, c_3) \cdot (d_0, d_1, d_2, d_3) = \begin{pmatrix}
c_0d_0 - c_1d_1 - c_2d_2 -c_3d_3\\
c_0d_1 + c_1d_0 + c_2d_3 - c_3d_2\\
c_0d_2 - c_1d_3 + c_2d_0 +c_3d_1 \\
c_0d_3 + c_1d_2 - c_2d_1 + c_3d_0 
\end{pmatrix}
$\\
$ \mathcal{H} \subseteq M_2(\mathbb{C}) $ \\
$ \mapsto \mathbb{H} $ \\
$ c_0\cdot 1 + c_1 I + c_2 J + c_3 K \mapsto (c_0, c_1,c_2, c_3) $\\
$ \{ (a_0, a_1, 0, 0) | a_i \in \R \} = \{ a_0 + a_1I \} $ - Изоморфно комплексным числам  $ \backsimeq $\\
$ \{ (a_0, 0, a_2, 0) | a_i \in \R \} = \{ a_0 + a_2О \}  \backsimeq $ \\
$ \{ (a_0, 0, 0, a_3) | a_i \in \R \} = \{ a_0 + a_3Л \} $ \\

$ \det \left( \begin{pmatrix} a & b \\ c & d \end{pmatrix} \cdot \begin{pmatrix} a' & b' \\ c' & d' \end{pmatrix} \right) = \det \begin{pmatrix} a & b \\ c & d \end{pmatrix} \cdot \det \begin{pmatrix} a' & b' \\ c' & d' \end{pmatrix} $ \\
$ \begin{pmatrix} a & b \\ c & d \end{pmatrix} \cdot \begin{pmatrix} a' & b' \\ c' & d' \end{pmatrix} = \begin{pmatrix} aa' + bc' & ab' + bd' \\ ca' + dc' & cb' + dd' \end{pmatrix} $ \\
$ (aa' + bc')(cb' + dd') - (ab' + bd')(ca'+dc') ?=(ad-bc)(a'd' -b'c') $\\ % pic1
$ \mathcal{H} =  \left\{\begin{pmatrix} z_1 & - \overline{z}_2 \\ z^2 &  \overline{z_1} \end{pmatrix} | z_i \in \mathbb{C}  \right\} $ \\
$ z_1 = a_0 + a_1 i, z_2 = a_3 + a_2 i $ \\
$ \det \begin{pmatrix}  z_1 & - \overline{z}_2 \\ z^2 &  \overline{z_1}  \end{pmatrix} = z_1 \overline{z_1} + z_2 \overline{z_2} = a_0^2 +a_1^2+a_2^2+a_3^2  $\\
$\sqrt{ a_0^2 +a_1^2+a_2^2+a_3^2 } - $ модуль кватерниона $ a_0 + a_1I + a_2J + a_3K $\\
$ (c_0 + c_1I + c_2J + c_3K) \cdot (d_0 + d_1I + d_2J + d_3 K)  $ \\
$ (c_0^2 + c_1^2 + c_2^2 + c_3^2) \cdot (d_0^2 + d_1^2 + d_2^2 + d_3^2) = \\
(c_0d_0 - c_1d_1 - c_2d_2 -c_3d_3 )^2\\
(c_0d_1 + c_1d_0 + c_2d_3 - c_3d_2)^2\\
(c_0d_2 - c_1d_3 + c_2d_0 +c_3d_1)^2 \\
(c_0d_3 + c_1d_2 - c_2d_1 + c_3d_0 )^2 $ \\
$ z = (c_0, c_1, c_2, c_3) = c_0 + c_1 i + c_2 j + c_3 k $ \\
$ Re(z) = c_0 $ - вещественная часть\\
$ \vec{V} = c_1i + c_2j + c_3k $ - векторная часть \\
$ z = c_0 + \vec{V} $ \\
$ \overline{z} = c_0 - \vec{V} = c_0 - c_1i -c_2j - c_3k $ \\
$ z \cdot \overline{z} = (c_0 + c_1i + c_2j + c_3k) (c_0 - c_1 -c_2j -c_3k) = c_0^2 + c_1^2 + c_2^2 + c_3^2 = | z | $ \\
2-й способ д-ва тождества Эйлера \\
$ \overline{\overline{z}} = z $ \\
$ \overline{z_1 + z_2 } = z_1 + \overline{z_2} $ \\
$ \overline{z_1z_2} = \overline{z_2} \cdot \overline{z_1} $ \\
$ a \in \R \ \ \ \overline{az} = a \cdot \overline{z} $\\
$ |z_1z_2|^2 = z_1z_2\cdot \overline{ z_1z_2} = z_1 z_2 \overline{ z_2} \overline{z_1} =  = z_1 (z_2 \overline{z_2}) \overline{ z_1} = |z_1|^2 \cdot |z_2|^2$ \\ 

$ (c_0 + \vec{V}) (d_0) + \vec{U} = c_0d_0 - \vec{v} \cdot  \vec{u} + c_0 \cdot \vec{u} + \vec{v} \cdot d_0  $ \\%pic2,3

\Section{Теория делимости в коммутативных кольцах}

$ R $ - комм кольцо \\
$ a | b $ a делит b $ \exists c : b = ac $ \\
$ b \divby a  $ b делится на a \\

\begin{properties}
	\begin{enumerate}
		\item  $ a | 0 $
		\item $ a | b , b | c \Rightarrow a | c $ 
		\item $ R > 1 \Rightarrow a | a, a = a \cdot 1 $ \\
		Замечание $ a | b \& b | a \centernot\Rightarrow a = b $
		\item $ a| b, a|c \Rightarrow a | (b \pm c ) $
	\end{enumerate}
\end{properties}
\begin{definition}
	Если $ a \cdot b = 0 $, но $ a, b \neq 0 $ то a и b называются нетривиальными делителями нуля \\
	R - область целостности если нет нетривиальных делителей нуля 
\end{definition}
Замечание: для некомм R : \\
Если $ b = ac $ ,то a - левый, b - правый делитель \\

$ R > 1 - $ комм \\
$ R^* = \{ a \in R : a | 1 \} = \{ a \in R : \exists b, ab=1 \} $ - мн-во обратимых элементов \\
$ R^*m \cdot $ - группа \\
$ R^* - $ мультипликативная группа кольца \\
$ R^* \times R^* \rightarrow R^* $\\
%pic5
$ a \in R^* \exists b ab=1 \Rightarrow ab = ba = 1 \Rightarrow b \in R^* $\\
Примеры: $ \Z^* = \{\pm 1\} $ \\
K - поле $ K^* = K \setminus \{0\} $\\
К - поле $ (k[x])^* = K^* $ \\
Упр. 1 $ \Z[i] = \{ a + bi | a,b \in \Z \} \subseteq \mathbb{C} $ - Гауссовы числа\\
Д-те что $ \Z[i] $ - кольцо и найдите $(\Z[i])^* $\\
Упр. 2 $ \omega = \dfrac{-1 + \sqrt{3} i}{2}, \omega^3 = 1 $ \\
$ \Z[\omega] = \{ a + b\omega | a,b \in \Z \} $ - Кольцо чисел Эйзенштейна \\
$ d \in \Z $ не явл целым квадратом \\
$ \Z[\sqrt{d}] \subseteq \mathbb{C} $ если $ d < 0 $ \\
$ \hspace*{12mm} \subseteq \R $ если $ d > 0 $ \\
$ \Z[\sqrt{d}] = \{ a + b\sqrt{d} | a, b \in \Z  \} $ \\
Каждый эл-т из $ \Z[\sqrt{d}] $ единственным образом представляется в виде $ a + b\sqrt{d}, a,b \in \Z $ \\
5. Д-те, что $ (\Z[\sqrt{2}])^*, (Z[\sqrt{3}])^*, (\Z[\sqrt{5}])^*, (\Z[\sqrt{6}])^* $ - бесконечны \\ % pic6

\Subsection{Ассоцированность} 

R - кольцо коммут, $ 1 \in R $ \\
a ассоцировано с b если $ a | b \& b | a $ \\
$ a \sim b $\\
$ \sim $ - отнош эквивалентности \\
\begin{theorem}
	1. Если $ \varepsilon \in R^* $, то $ a \sim a\eps $\\
	2. Если R - область целостности, $ a \sim b \Rightarrow \exists \eps \in R^*, b = a\eps $
	\begin{proof}
		1. %pic
		2. $ a | b \& b | a $ \\
		$ a = 0 \Rightarrow b = 0, 0 = 0 \cdot 1 \Rightarrow \eps = 1	$ \\
		$ a \neq 0, a | b \ \exists \eps \in R, b = a\eps $ \\
		$ b | a \ \exists r \in R \Rightarrow a = r \cdot b $ \\
		$ a = r \eps a  $ \\
		$ 0 = a (1 - re) \Rightarrow 1 - r \eps = 0 $ \\
		$ \eps | 1, \eps \in R^* $
	\end{proof}
\end{theorem}
Примеры \\
$ \Z, Z^* = \{\pm 1 \}$ \\
$ \{0\}, \{\pm 1\}, \{ \pm 2 \} $ - классы ассоцированности \\
$ K[X], K - $ поле $ \{0\} $ \\
Любой другой класс ассоцированности содержит ровно 1 многочлен со старш коэфф 1(приведённый, унитарный) \\

\Subsection{Идеал в кольце}

R - произвольное кольцо\\ %pic
$I = R$ \\
$ I = \{0\} $ - идеалы \\ 
Для некомм колец различают левые и правые идеалы \\
1. остаётся \\
2(левый). $ \forall a \in I, \forall r \in R, ra \in I, R\cdot I \subseteq I $ \\ 2(правый). $ \forall a \in I, \forall r \in R, ar \in I,  I \cdot R \subseteq  (\sum r_is_i + \sum m_js_j)I $ \\

Замеч. в усл 1 $ \forall a,b \in I, a \pm b \in I $\\
$ 1' : \forall a,b \in I, a-b \in I , \\
1' \Rightarrow 1 $ I - непусто \\
$ a \in I, I \neq 0 $ \\
$ a - a \in I, 0 \in I $\\
$ a \in I, 0 \in  I, 0 - a \in I, -a \in I $\\
$ \forall a,b, -b \in I, a - (-b) \in I $\\

$ R, S \subseteq R $ \\
Какой наименьш идеал содержит S \\
$ s \in S \subseteq I $ \\
$ \forall r \in R, r \cdot s \in I $\\
$ \forall m \in \Z, m \cdot s \in I $ \\
$ \{ \sum_{i=1}^{n} r_is_i + \sum_{j=1}^{k} m_js_j | n, k \in \N \cup \{0\}, r_i \in R, m_j \in \Z \} $ - это мн-во образует идеал \\
а значит это и есть минимальный идеал, содержащий S \\
$ (\sum r_is_i + \sum m_js_j) \pm  (\sum r'_is'_i + \sum m'_js'_j) - $ вновь сумма такого же вида \\
Идеал, порождённый мн-вом S $ (S) $ \\















\end{document}
