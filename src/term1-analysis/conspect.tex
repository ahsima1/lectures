% !TeX program = xelatex
\input{core}
\usepackage{multicol}
\usepackage{centernot}
\usepackage{amsmath}
\enablemath

\renewcommand*{\t}{\texttt}
\newcommand*{\liminfty}[1]{\lim\limits_{{#1} \to \infty}}
\newcommand*{\abs}[1]{\left|\,{#1}\,\right|}
\newcommand*{\ceil}[1]{\lceil \, {#1} \, \rceil}
\newcommand*{\floor}[1]{\lfloor \, {#1} \, \rfloor}
\newcommand*{\rpm}{\raisebox{.2ex}{$\scriptstyle\pm$}}
\newcommand*{\notdivby}{\centernot\divby}
\newcommand*{\integer}[1]{\left[{#1}\right]}
\newcommand*{\real}[1]{\left\{{#1}\right\}}
\newcommand*{\action}{\curvearrowright}
\renewcommand*{\TODO}{\textcolor{red}{TODO}}
\renewcommand*{\thmslashn}{\slashns}

\declaretheorem[numbered=no, name=Упражнение, style=thmstyle_cons]{exercise}
\declaretheorem[numbered=no, name=Следствия, style=thmstyle_cons]{consequences}

\newcommand*{\Ker}[1]{Ker \, {#1}}
\renewcommand*{\Im}[1]{Im \, {#1}}

\let\epsilon\varepsilon
\let\emptyset\varnothing
\let\phi\varphi



\begin{document}
	\gdef\CourseName{Математический анализ}
	\author{Печин Михаил}
	\makegood

	\Section{Вещественные числа}{Печин Михаил}
	\Section{Введение}

\noindent
$ (a, b, c, ) $ - упорядоченный набор \\
$ \{ a, b, c \} $ - неупорядоченный набор \\
$ i = 1 .. k = \overline{1, k} $ \\
$ \lceil x \rceil $ - наименьшее целое $ \geq x $ \\
$ \lfloor \rfloor $ - наибольшее целое $ \leq x $ \\
$ \N $ - натуральные числа, без 0 \\
$ \Z $ - целые числа \\
$ \R $ - вещественные числа \\
$ \N + 0 = \N \cup \{ 0 \}  $ \\
$ \forall $ - "для любого" \\
$ \exists $ - "существует" \\
$ \nexists $ - "не существует" \\
$ \exists ! $ - "существует и единственный" \\
$ \wedge $ - И \\
$ \vee $ - ИЛИ \\
$ \neg $ - НЕ \\
$ \{ x | ... \} $ - множество таких x, что ... \\

\begin{definition}
	Множество - любая определённая совокупность объектов.
\end{definition}
$ x \in M $ - принадлежит\\
$ x \notin M $ - не принадлежит \\
$ \emptyset $ - пустое множество \\
$ U $ - универсальное множество, универсум \\
\begin{enumerate}
	\item Перечислить элементы \\
	$ M := \{ a, b,c, ..., z \} $ 
	\item Характеристический предикат \\
	$ M = \{ x | P(x) \} \ \ \ M = \{ n | n \in N \& n < 10 \} $
	\item Порождающая процедура \\
	$ M := \{x| x := f \} $ \\
	$ \{a_i\}_{i=1}^{k} \rightarrow \{a_1, a_2, ..., a_k \} $ \\
	$ \mathcal{M} = \{M_{\alpha}\}_{\alpha \in A} $ \\
\end{enumerate}

$ A \cup \{x\} = A + x \stackrel{\triangle}{=} \{ y | y \in A \vee y = x \} $ \\
$ A \setminus \{x\} \stackrel{\triangle}{=} A - x =  \{ y | y \in A \wedge y \neq x \} $ \\
Противоречие Рассела \\
$ Y = \{ X | X \neq X \} $ \\
$ \letus \exists Y \rightarrow Y \in Y $ или $ Y \notin Y $ \\
Если $ Y \in Y $ то $ Y \in Y \Rightarrow Y \centernot\subset Y $ \\
Если $ Y \notin Y $ то $ Y \notin Y $ \\
\begin{definition}
	Мультимножества \\
	$ \letus X = \{  \} $
	$ \stackrel{\triangle}{X} = $ 
\end{definition}

$ \forall i \in \overline{1, n} $ \\
$ a_i : ( a_i = 0, a_i = 1 ) $ - мультимн-во явл индикатором \\
$ \stackrel{\wedge}{X} = < a_1(x_1), ... > $ над $ X = \{x_1, ...\} $ \\
Конечные последовательности \\
$ \letus X = \{x_1, x_2, ..., x_n \} $ \\
$ \stackrel{\wedge}{X} = $\\
$ X = \{a, b\}, \ \ \multiset{X} = <2(a), 2(b) > $\\
$ aabb, abab, baba, bbaa, abba, baab $ - вектор \\

$ A \subset B \defeq x \in A \Rightarrow x \in B \forall x \in A $ \\
A - подмн-во, B - надмн-во \\
$ \forall M, \emptyset \subset M $ \\
$ \letus X = \{ x_1, ..., x_n \} Y \subset X, \exists! $ индикатор $ \multiset{Y} = < a_1(x_1), ..., a_n(x_n) > $ над $ X, \forall i \in \overline{1, n} (a_i = 1 \Leftrightarrow x_i \in Y)  $ \\
$ \forall A,B (A \subset B, B \subset A \Leftrightarrow A = B) $ \\
$ \forall A,B,C ( A \subset B, B \subset C, A \subset C) $ \\
$ (A \varsubsetneq B \subseteq C \Rightarrow A \varsubsetneq C)  $\\
\begin{definition}
	Равномощные множества \\
	A и B взаимно однозначное соответствие (биекция) \\
	A и B изоморфны, $ A \sim B $ \\
	$ a \in A, b \in B, a \mapsto b $ \\
	$ \N, 2\N $ \\
	$ N \sim 2\N $ \\
	$ |A| =|B| \defeq A \sim B $\\
	$ \forall A, |A| = |A| $ \\
	$ \forall A,B, |A| = |B| \Rightarrow |B| = |A| $ \\
	$\forall A,B,C, |A|=|B|, |B|=|C| \Rightarrow |A| = |C| $ 
\end{definition}
\Subsection{Конечные и бесконечные множества}
Часть меньше целого - не работает для бесконечных \\
A - конечное, если у него нет равномощного собственного подмножества \\
$ \forall B, (B \subset A \& |B|=|A|) \Rightarrow B = A  \ \ |B| < \infty$ \\
$ \exists B, (B \subset A \& |B| = |A| \& B \neq A) $ - бесконечное множество \\
$ |\N| = \infty $ \\
$ 2 \N = \N $ \\
\begin{theorem}
	Множество,  имеющее бесконечные подмножества само является бесконечным \\
	$ (B \subset A \& B = \infty \Rightarrow |A| = \infty) $ 	
\end{theorem}
\Subsection{Счётные и несчётные множества} 

Счётное - бесконечное, равномощное $\N$ \\
$ \Z \ \ n = 0, n \rightarrow 1 $ \\
$ n > 0, n \rightarrow 2n $ \\
$ n < 0, n \rightarrow 2n + 1 $ \\
$ T(1) = 1, T(2) = 3, T(3) = 6 $ \\
$ T(k) = \sum_{i=1}^{k} i = \dfrac{(k+1)k}{2} = T(k-1) + k $ \\
\begin{theorem}
	Множество всех подмножеств натурального ряда несчётно \\
	\begin{proof}
	Пусть множество счётно $ |X| < \infty, X \sim \N $ \\
	$ \letus N(X) $ номер мн-ва X \\
	$ Y $ содержит в себе номера множеств, не содержащих в себе этот номер \\
	$ Y := \{ x \in N | \exists X \subset N (x = N(x), x \notin X) \} $ \\
	$ y = N(Y) $ Если $y \notin Y, y \in Y $ \\
	Если $ y \in Y \Rightarrow y \notin Y $ \\
	Противоречие 
	\end{proof}
\end{theorem}
Счётным является бесконечное мн-во всех конечных подмножеств $\N$ \\
\begin{theorem}
	Любое непустое конечное множество равномощно некоторому отрезку натурального ряда \\
	$ \forall A ( |A| \neq 0, \& |A| < \infty, \rightarrow  \exists k \subset \N, (|A| = |1, ..., k|)) $\\
\end{theorem}
\begin{theorem}
	Любой отрезок натурального ряда конечен \\
	Пусть есть бесконечный отрезок.
	Наименьшее n, $ 1 ... n | = \infty $ \\
	$ \exists A : |1...N| 1..n \sim A $ \\
	$ n \rightarrow i $ \\
	
\end{theorem}
 
\subsection{Операции над множествами} 
\begin{enumerate}
	\item Объединение $ A \cup B \defeq \{ x | x\in A \vee x \in B \} $ 
	\item Пересечение $ A \cap B \defeq \{ x | x \in A \& x \in B \}$
	\item Разность $ A \setminus B \defeq \{x| x \in A \& x \notin B \} $
	\item Симметрическая разность $ A \triangle B \defeq (A \cup B) \setminus (A \cap B) = \{ x | (x \in A \& x \notin B) \vee (x \notin A \& x \in B)  \}$ 
	\item $ \overline{A} \defeq =  \{ x | x \notin A \} $ \\
	$ \overline{A} = U \setminus A $
\end{enumerate}
$
\begin{array}{|c|c|c|c|c|c|c|}
	\hline 
	& \cup \cap  & \cup \setminus  & \cup \triangle  & \cap \setminus & \cap \triangle & \setminus \triangle  \\ 
	\hline 
	A \cup B & A \cup B & A \cup B  & A \cup B  &  &  &  \\ 
	\hline 
	A \cap B& A \cap B &  &  & A \cap B & A \cap B &  \\ 
	\hline 
	A \setminus B &  & A \setminus B &  & A \setminus B &  &  A \setminus B\\ 
	\hline 
	A \triangle B &  &  & A \triangle B  &  & A \triangle B  & A \triangle B   \\ 
	\hline 

\end{array} 
$


\Subsection{Разбиение и покрытие}

$ \letus \eps = \{ E_i \}_{i\in I} - $ семейство подмножеств \\
$ \eps $ - покрытие M $ \forall x \in M, (\exists i \in I, x = E_i) $ \\
$ \eps $ - дизъюнктным если $ \forall i, j \in I, i \neq j, E_i \cap E_j = \emptyset $
$ \eps $ - рабиение - дизъюнктное покрытие \\

\begin{theorem}
	$ \eps = \{ E_i \}_{i\in I} $ - дизъюнктное семейство подмн-ва M то существует разбиение $ B = \{ B_i \}_{i\in I} $ Каждый элемент  $ \eps $ подмножество блока разбиения B 
	\begin{proof}
		$ i_0 \in I, B = \{ B_i \}_{i\in I} B_{i_0} = M \setminus \cup_{i\in I - i_0} E_i  \forall i \in I - i_0 ( B_i = E_c) $ 
	\end{proof}
\end{theorem}
$ M = \{1,2,3\}, \eps = \{\{ 1 \}, \{2\}, \} $ \\
$ B = \{ \{1\}, \{ 2,3\} \} $ \\

\Subsection{Булеан} 
Множество всех подмножеств мн-ва M - булеан мн-ва M и обозначается $ 2^M $\\
$ 2^M \defeq \{ A | A \subset M \} $\\
\begin{theorem}
	M - конечно $ \Leftrightarrow |2^M| = 2^{|M|} $
	\begin{proof}
		$ |M| = 0, 2^M = \{ \emptyset \}, |2^M| = 0 $ \\
		$ M (M < k, |2^{M}| = 2^{|M|}  ) $ \\
		$ M = \{ a_1, ..., a_k \}, |M| = k $
		$ M_1 = \{ x \in 2^M | a_k \notin x \} $ \\
		$ M_2 = \{ x \in 2^M | a_k \in x \}  $
	\end{proof}
\end{theorem}
В булеане можно выделить семейство подмн-в \\
$ C^k(M) = \{ S \subset M | |S| = k \} $ \\
$ C^0 (M) = \{\emptyset\} $ \\
$ C^1 (M)  $ - разбиение \\
$ |M| = n, 2^M = \cup_{x=0}^{n} C^k(M), |C^k (M)| = C_k^n $ \\
Булеан множества мощнее самого множества \\
\Subsection{Свойства операций над мн-вами}
Идемпотентность $ A \cup A = A, A \cap A = A $\\
Коммутативность $ A \cup B = B \cup A $ \\
Ассоциативность $ A \cup B \cup C  = A \cup (A \cup B )$\\
Дистрибутивность $ A \cup (B \cap C) = (A \cup B) \cap (A \cup C) $ \\
Поглощение $ (A \cap B) \cup A = A, (A \cup B) \cap B = A $ \\
Св-ва нуля $ A \cup \emptyset = A, A \cap \emptyset = \emptyset $ \\
Св-ва единицы $ A \cup U = U, A \cap U = A $ \\
Инволютивность $ \overline{\overline{A}} = A $ \\
Законы Де-Моргана $ \overline{A \cap B} = \overline{A} \cup \overline{B} $\\
$ \overline{A \cup B} = s $
Свойства дополнения $ A \cup \overline{A} U, A \cap \overline{A} = \emptyset $\\
Знак для разности $ A \setminus B = A \cap \overline{B} $ 












	\noindent
$\{A_i\}_{i\in I}$ \\
$ M $ - множество множеств \\
$ I $ - индексное множество \\
$ f: I \rightarrow M $ \\
$ i \mapsto A $

\Section{Алгебраические структуры. Группы, кольца, поля.}

\Subsection{Группы}

Опр: На множестве А задана n-арная операция, если задано отображение $ f: A^n \rightarrow A $\\
Бинарные операции n=2, унарные n=1 \\
Опр. Непустое мн-во G с одной бинарной операцией $ * : G \times G \rightarrow G $ если выполнено три аксиомы. \\
1. $ \forall a,b,c \in G \ a * (b * c) = (a *b) * c) \\$
2. $ \exists e \in G \forall a \in G \ e*a=a*e=a $ \\
3. $ \forall a \in G \exists a' \in G \ a*a'=a'*a=e$ \\
Опр: Если G - группа и $ \forall a,b \in G \ a*b=b*a $ G называется абелевой группой. \\
Если $*$ обозначается как $ \cdot$ или знак опускается, то говорят о мультипликативной записи, $e = 1, a' = a^{-1} $. Нейтральный элемент называют единицей\\
Если $*$ обозначается через $+$, говорят об аддитивной записи, $ e=0$, называют нулём группы, $ a' = -a$\\
Для неабелевых групп обычно  используется мультипликативная запись, для абелевых и та и другая. \\
Примеры: \\
1. $ \Z, +, 0 $ \\
$ \mathbb{Q}, + $ \\
$ \R, + $ \\
$ \N, + $ - не группа \\
3. $\mathbb{Q} \setminus \{A\}, \cdot $\\
$\R \setminus \{A\}, \cdot $\\
$\R_{> 0}, \cdot $\\
4. $ \Z \setminus 0, \cdot $ не группа. \\
5. $ \{ \pm 1 \}, \cdot $ группа из двух элементов \\
6. $ X$ - какое-то множество $ X \neq \emptyset $\\
$ S(X) -$ множество всех биекций на $X$\\
$ * - \circ $ - композиция \\
$ S(X) \neq \emptyset \ id_X \in S(X) $ \\
$ \circ $ -ассоциативна \\
$ id_X \circ f = f \circ id_X = f \Rightarrow e = id_X $ \\
$ f \in S(X) \Rightarrow \exists $ обратное отображение \\
$ f \circ f^{-1} = f^{-1} \circ f = id_X $ \\
$ |X| = 1, X=\{X_1\} , |S(X)| = 1, x_1 \mapsto x_1 $ \\
$ |X| = 2, X = \{x_1, x_2\}, id: x_1 \mapsto x_1, x_2 \mapsto x_2 $ \\
$ g: x_1 \mapsto x_2, x_2 \mapsto x_1 $ \\
$ g * g = id $ \\
$ |X| \geq 3, S(X) -$ не абелева. \\
$ f: x_1 \mapsto x_2, x_2 \mapsto x_3, x_3 \mapsto x_3 $ \\
% ------pic1
$ g \circ f: x_1 \mapsto x_3, x_2 \mapsto x_1, x_3 \mapsto x_2  $\\
$ f \circ g: x_1 \mapsto x_2, x_2 \mapsto x_3, x_3 \mapsto x_1 $ \\
$ S(X) $ - симметрическая группа на множестве $X$ \\
$ X = \{ x_1, \dots, x_n \} \ X = \{ 1, \dots, n \} $ \\
$ S_n - $ cимметрическая группа степени n. \\
$ X $ - мн-во \\
$ G = P(X) $ - мн-во подмножеств X
$ \triangle : G * G \rightarrow G $ \\
$ A \triangle B = (A \setminus B) \cup (B \setminus A) = (A \cup B) \setminus (A \cap B)$ \\
$ G, \triangle $ - группа. \\
Найдите нейтральный и обратный. Будет ли группа абелевой. \\
\subsubsection{Свойства групп}
1. Единственность нейтрального. \\
Пусть $e_1 $ и $e_2$ - нейтральные. \\
$ e_1 * e_2 = e_1$ \\
$ \forall a \in G, e*a=a*e=a $\\
2. Единственность противоположного \\ 
$ a \in G $ пусть есть два обратных $ a'$ и $a'' $\\
$ a' * a * a'' = a'*(a*a'') = a'*e=a' $\\
$ (a' *a) *a'' = e * a'' = a'' \Rightarrow a' = a'' $
\subsection{Кольца и поля}

$ R, +, \cdot \hspace*{5cm} R \neq \emptyset $ \\
1. $ \forall a,b,c \in R \ a + (b + c) = (a +b) + c) \\$
2. $ \exists 0 \in R \forall a \in R \ 0+a=a+0=a $ \\
3. $ \forall a \in R \exists -a \in R \ a+(-a)=(-a)+a=0$ \\
4. $ \forall a,b \in R \ a+b=b+a $ \\
5. $ \forall a, b, c \in R \ (a \cdot b) \cdot c = a \cdot (b \cdot c) $ \\
6. $ \exists 1 \in R \ \forall a \in R \ 1 \cdot a = a \cdot 1 = a $ \\
7. $ \forall a \in R \setminus \{0\} \exists a^{-1} \ a \cdot a^{-1} = a^{-1} \cdot a = 1 $ \\
8. $ \forall a, b \in R \ ab=ba$ \\
9. $ \forall a,b,c \in R \ a(b+c) = ab+ac $ \\
10 $ 1 \neq 0 $ \\
Кольцо - выполнены 1-5 и 9 \\
Кольцо с 1 - выполнены 1-6, 9 \\
Коммутативное кольцо - выполнены 1-5, 8, 9 \\
Коммутативное кольцо с 1 - выполнены 1-6, 8, 9 \\
Тело, если выполнены 1-7, 9-10 \\
Поле, если выполнены все 10 аксиом \\
Примеры:
$ \Z, +, \cdot $ Кольцо, коммутативное с 1, не поле. \\
$ \mathbb{Q}, +, \cdot $ Поле \\
$ \R, +, \cdot $ Поле \\
$ 2\Z , +, \cdot $ Кольцо, коммутативное без 1 \\
Существуют конечные поля.\\
$ |F| = 2,3,4$ \\
\begin{tabular}{|c|c|c|}
	\hline 
+	& 0 & 1 \\ 
	\hline 
0	& 0 & 1 \\ 
	\hline 
1	& 1 & 0 \\ 
	\hline 
\end{tabular} 
\begin{tabular}{|c|c|c|}
	\hline 
	$\cdot$	& 0 & 1 \\ 
	\hline 
	0	& 0 & 0  \\ 
	\hline 
	1	&  0 & 1 \\ 
	\hline 
\end{tabular} \\
%-----pic2------
$ R $ -кольцо \\
$ (-(0\cdot a)) \mid 0 \cdot a = (0+0) \cdot a = 0 \cdot a + 0 \cdot a $\\
\hspace*{15mm} $ -(0 \cdot a) + 0 \cdot a = -(0 \cdot a) +  0 \cdot a + 0\cdot a$ \\
\hspace*{15mm}$ 0 = 0 + 0 \cdot a = 0 \cdot a $\\
Упражнение: проверить, что заданное $ \mathbb{F}_2 $ - поле \\
\begin{tabular}{|c|c|c|c|}
	\hline 
+	& 0 & 1 & a \\ 
	\hline 
0	& 0 & 1 & a \\ 
	\hline 
1	& 1 & a & 0 \\ 
	\hline 
a	& a & 0 & 1 \\ 
	\hline 
\end{tabular} 
 \begin{tabular}{|c|c|c|c|}
	\hline 
	$\cdot$	& 0 & 1 & a \\ 
	\hline 
	0	& 0 & 0 & 0 \\ 
	\hline 
	1	& 0 & 1 & a \\ 
	\hline 
	a	& 0 & a & 1 \\ 
	\hline 
\end{tabular} \\
%---pic3, 4
F - поле, $ a \in F \setminus \{0\} $ \\
$ F \setminus \{0\} \rightarrow   $\\
$ x \mapsto ax $
Упражнение: проверить, что заданное $ \mathbb{F}_3 $ - поле \\


	
\Section{Числовые последовательности}

\Subsection{Основные определения}

Числовая последовательность - отображение $ a: \N \rightarrow \R $ \\
$ \N \rightarrow \R $ \\
$ n \rightarrow a_n $ \\
$ a_n = 3n^2 - \sqrt{2} n+ \frac{5}{2} $ \\
$ a_n = a_{n-1} + n^2, a_1 = 1$\\

Последовательность ограничена сверху, если существует  $ C \in \R : \forall i \in \N : a_i < C $ \\
Последовательность ограничена снизу, если существует  $ C \in \R : \forall i \in \N : a_i > C $ \\
Последовательность ограничена, если она ограничена и сверху и снизу.

Монотонные последовательности \\
Неубывающая последовательность: $ \forall i \in \N : a_i \leq a_{i+1} $\\
Возрастающая $ < $ \\
Невозрастающая $ a_i \geq a_{i+1} $\\
Убывающая $ > $ \\
2, 4 - строго монотонные.

Монотонные, начиная с некоторого места \\
Неубывающая $ \exists n \in \N \forall i > n : a_i \leq a_{i+1} $\\

\Section{Предел последовательности}

$ \eps$-окрестность - $ (x - \eps, x+\eps) $\\
\begin{enumerate}
	\item  Окрестность x - это $ \eps $ окрестность x для $ \eps > 0 $
	\item Окрестность x - произвольный интервал $ (a ,b) : a < x < b$
	\item Окрестность x - $U \subset \R :  U \supset (a, b) , a, b \in \R, a < x < b $ 
\end{enumerate}
Любая окрестность х содержит $\eps $ окрестноть х с $ \eps > 0 $ \\
Пусть $(a_i)$ - последовательность, $ x \in \R $ Говорят, что $a_i$ сходится к x(имеет х своим пределом) если $ \forall U $ окрестности $x \exists N \in \N : a_i \in U $ при всех $ i \geq N $ \\
Понятие предела не изменится, если мы определения окрестности 1 перейдём к определению 3. Если нужное свойство выполняется для класса 3, то оно, в частности, выполняется и для $ \eps $ окрестностей. \\
Предположим оно выполнено для всех $ \eps $ окрестностей. Тогда $ U \supset(a, b), (a,b) > (x-\eps, x+ \eps) \eps > 0 \Rightarrow \exists N, \forall i \in \N, a_i \in U$  \\
Определение 2: Пусть $(a_i), X \in \R$ сходится к Х, если $\forall U $ 	числа Х $, \{ i \in \N \mid a_i \notin U \} $ конечно \\
% -------pic1---------
$ \lim a_i = x $ \\
$ a_i \rightarrow x $ \\
Последовательность имеет не больше одного предела.
Предп: $ a_i \rightarrow x, a_i \rightarrow y, x < y $ \\
$ \eps = \dfrac{y-x}{2} $ \\
$Y_1 = \{ i \mid a_i \notin U_{\eps} (x) \} $конечно\\
$ Y_2 = \{ i \mid a_i \notin U_{\eps} (y) \} $конечно\\
$ \N = Y_1 \cup Y_2 $ конечно

\Subsection{Свойства пределов}
Предл(предельный переход в неравенстве) \\
Пусть $ a_i \rightarrow x, b_i \rightarrow y $ \\
Предп. $ \forall i \in \N : a_i \leq b_i. $Тогда $ x\leq y $ \\
Пусть $ y < x, \eps = \dfrac{x-y}{2} $ 
$ \exists N_1 : \forall n \geq N_1 a_n \in U_{\eps}(x) $
$ \exists N_2 : \forall n \geq N_2 b_n \in U_{\eps}(y) $
$  N = max(N_1, N_2) \Rightarrow \forall n \geq N : 
a_n \in U_{\eps}(x)\\
b_n \in U_{\eps}(y)$\\
$ x- \eps <  a_n \leq b_n < y+\eps = \eps x-\eps$\\
Противоречие \\
Следствие: Пусть $ a_i \leq C \forall i; a_i \rightarrow x \Rightarrow x \leq C $\\

\begin{theorem} Теорема о сжатой последовательности\\
Пусть $ a_i \rightarrow x, b_i \rightarrow x $ \\
$ c_i \ \ \forall i, a_i \leq c_i \leq b_i $ \\
\begin{proof}
Возьмём $ \eps > 0 $ и док-м, что $ \exists N \in \N $ \\
$ \forall i \geq N : c_i \in U_{\eps} (x) $ \\
$ \exists N_1 : \forall i \geq N_1 : a_i \in U_{\eps} (x) $\\
$ \exists N_2 : \forall i \geq N_2 : b_i \in U_{\eps} (x) $\\
$ \Rightarrow N = max(N_1, N_2) $ обладает нужным свойством. \\
Начиная с некоторого номера - менять/откидывать некоторые члены, предел от этого не ипзменится. \\

Если сушествует $ lim a_i = x $, то $(a_i)$ - ограничена\\
% ---pic2 ----
\end{proof}
\end{theorem}

Если последовательность неубывающая и ограничена свеху, то она сходится
Если последовательность невозрастающая и ограничена снизу, то она сходится

$ x = \sup\{a_i | i \in \N\} $ \\
$ a_i \rightarrow x $\\
Возьмём $ \eps > 0 $\\
$ x - \eps < x \Rightarrow x - \eps  - $ не верхняя граница $ \Rightarrow \exists N \in \N, a_N > x-\eps $ \\
$ a_i $ - неубывающая $ \Rightarrow \forall i \geq N : a_i > x-\eps \Rightarrow \forall i \geq N : a_i \in U_{\eps} (x)$\\

\Subsection{Арифметические действия с пределами} 

Пусть есть $ a_i \rightarrow x, b_i \rightarrow y $ \\
$ |a_i| \rightarrow |x| $\\
$ a_i + b+i \rightarrow x+y $ \\
$ a_i - b_i \rightarrow x - y$\\
$ a_i \cdot b_i \rightarrow xy $ \\
$ b_i \neq 0 \ \  \forall i, \dfrac{ a_i}{b_i} \rightarrow \dfrac{x}{y} $\\

$ | | a_i | - | x| | \leq | a_i - x | $ \\
2. $ \eps > 0$
$\exists N:  \forall i > N: a_i \in U_{\frac{\eps}{2}} (x) , b_i \in U_{\frac{\eps}{2}} (y) \Rightarrow a_i + b_i \in U_{\eps}(x+y) $ \\
4. $  a_i b_i - xy = (a_i b_i - xb_i) + ( xb_i - xy )$ \\
$ | a_i b_i - xy | \leq  |a_i b_i - xb_i| + | xb_i - xy | $ \\ 
$ \exists C > 0 : | b_i | \leq C \forall i \in \N $ \\
$ | a_i b_i - xy | \leq C | a_i - x| + |x | \cdot | b_i - y | $\\
Возьмём $\eps > 0 $ 
Тогда $ \exists N \in \N \forall i \geq N :\\ |a - x| < 
\frac{\eps}{2C} \\
|b_i - y | < \frac{\eps}{2|x|} $(при х $\neq 0$)\\
%----pic3----											
$ |ab_i - xy| < C \cdot \frac{\eps}{2C} + \frac{\eps}{2} = \eps $

5. Достаточно доказать, что $\frac{1}{b_i} \rightarrow \frac{1}{y}$ \\
$|\dfrac{1}{b_i} - \dfrac{1}{y}| = |\dfrac{y-b_i}{b_i y}| = \dfrac{|b_i - y|}{|b_i||y|}$ 	\\
$ \exists  N_0 \in \N \ \forall i > N_0 : |b_i| > \frac{|y|}{2} $\\
$ y > 0 : b_i \in U_{\frac{y}{2}}(y) \Rightarrow b_i  > \frac{y}{2} > 0 $\\
$ y < 0	:  b_i \in U_{-\frac{y}{2}}(y) \Rightarrow b_i  < \frac{y}{2} < 0 $ \\
$ |\dfrac{1}{b_i} - \dfrac{1}{y}| \leq |\dfrac{b_i - y}{|y|^2 / 2}| $ \\
Возьмём $ \eps > 0 $\\
$ \exists N \geq N_0 : \forall i \geq N_0 $
%-----pic4-----

\subsection{Бесконечные пределы} 
Пусть $a_i$ последовательность.
$ a_i $ стремится к $ +\infty$ если $ \forall C \in \R \  \exists N \in \N : \forall i \geq N : a_i > C $ \\
Определение станет аналогичным определению конечного предела, если определить окресность $ +\infty $ как произвольный открытый луч $ (C, +\infty)$ \\
Равносильное определение: $ \{i \mid a_i \neq (C, +\infty) \} $ - конечно \\
Последовательность не имеет конечного предела - последовательность расходится \\
Пусть $a_i$ последовательность.
$ a_i $ стремится к $ -\infty$ если $ \forall C \in \R \  \exists N \in \N : \forall i \geq N : a_i < C $ \\
$ a_i \rightarrow +\infty \Leftrightarrow -a_i \rightarrow -\infty $\\
$ a_i \rightarrow \infty $ если $ \forall C \in \R \ \exists N \in \N : \forall i > N, |a_i| > C $ \\
$ ( C, +\infty )$ - окрестность $+\infty$ \\
$ (-\infty, C) $ \\
$ (-\infty, C) \cup ( C, +\infty), C > 0 $  \\

Теорема: Не ограниченная сверху неубывающая последовательность стремится к $ + \infty $ \\
Не ограниченная снизу невозрастающая последовательность стремится к $ -\infty $ \\
% ---------pic5---------
$ a_i \rightarrow \infty \Leftrightarrow |a_i | \rightarrow +\infty $ \\
Бесконечно малая посл-ть $ a_i \rightarrow 0 $ \\
Бесконечно большая посл-ть $ a_i \rightarrow \infty $ \\
Теорема $\forall i : a_i \neq 0 \Rightarrow (a_i) \rightarrow 0 \Leftrightarrow (\dfrac{1}{a_i}) \rightarrow \infty $\\
%----pic6-----
Замечание - сумма, разность, произведение бесконечно малых - бесконечно малая посл. \\
Пусть $(a_i) $ бесконечно большая, $(b_i) $ ограниченная, тогда их сумма бесконечно большая. \\
$ U_C(\infty) =  (-\infty, C) \cup ( C, +\infty) $ \\
Пусть $ C > 0 $ \\
Проверить $\{ i \mid a_i + b_i \notin U_C{\infty} \}$\\
$ \exists D > 0 : \forall i \in \N: |b_i | < D $ \\
$ \exists N \in \N : \forall i \geq N: |a_i| > C+D $ \\
При $ i \geq N : | a_i + b_i | \geq |a_i| + |b_i|  > C+D-D = C $ \\
$ a_i + b_i \in U_C(\infty) $ \\
Произведение бесконечно малой посл-ти $a_i$ на ограниченную $b_i$ - бесконечно малая \\
$ \exists C, \forall i, |b_i| < C $ \\
$ |a_i b_i | \leq C \cdot |a_i| $ 
$ a_i \rightarrow 0 \Rightarrow  |a_i| \rightarrow 0 \Rightarrow |a_i b_i| \rightarrow 0 $ \\
$ \overline{\R} = \R \cup \{+\infty, -\infty\} $ \\
$ +\infty + (-\infty) $ - не определена \\
$ a \cdot (+\infty) = (+\infty), a > 0 $\\
$ a \cdot (+\infty) = (-\infty), a < 0 $\\
$ a_n \leq b_n, a_n \rightarrow x \in \overline{\R}, b_n \rightarrow y \in \overline{\R} $ \\
Тогда $ x \leq y $\\
% ----------pic7--------------
Пр. \\
1. $lim x_n = +\infty, y_n $ огр. снизу $ \Rightarrow x_n + y_n \rightarrow +\infty $
2. $lim x_n = -\infty, y_n $ огр. сверху $ \Rightarrow x_n + y_n \rightarrow -\infty $
3. $ x_n \rightarrow +\infty (-\infty) \ y_n \geq C > 0 \ \forall n \in  \N \Rightarrow x_n y_n \rightarrow +\infty ( -\infty) $ \\
4. $ x_n \rightarrow a \neq 0, y_n \neq 0, y_n \rightarrow 0 \Rightarrow \dfrac{x_n}{y_n} \rightarrow \infty$ \\
5. $ x_n \rightarrow a \in \R y_n \rightarrow \infty \Rightarrow \dfrac{x_n}{y_n} \rightarrow 0 $ \\
6. 
7. % ---- pic8---------
3. Пусть $ E > 0 \  \exists N, \forall i \geq N, x_i > \dfrac{E}{C}, x_i y_i > \dfrac{E}{C} C = E $ \\
4. $ \exists E > 0, x_n \rightarrow a \Rightarrow \exists N_0 : |x_i| \geq \dfrac{|a|}{2} $ при  $ i \geq N_0$ \\
$ \dfrac{x_n}{y_n} $ беск. больш $ \Leftrightarrow \dfrac{y_n}{x_n} $ беск мал. $ \dfrac{1}{|x_i|} \leq \dfrac{2}{a} \Rightarrow \dfrac{1}{x_i}$ ограниченная \\
$ y_n \cdot \left( \dfrac{1}{x_n} \right)  $\\

Неравенство Бернулли \\
$ x > -1, n \in \N $ \\
$ (1+x)^n \geq 1 + nx $ \\
$ k \rightarrow k + 1 $ \\
$ (1+x)^{k+1} = (1+x)^k(1+x) \geq (1+kx)(1+x) = 1 + (k+1)x + kx^2 \geq 1 + (k+1)x $ \\
Сл. 1 \\
1. $ lim a^n = +\infty $ при $ a > 1 $ \\
2. $ lim a^n = 0$ при $ |a| < 1$ \\
Д-во \\
$ a^n = (1 + (a-1))^n \geq 1 + n(a-1) \rightarrow +\infty $\\
Сл. 2 Пусть $ a > 1 \Rightarrow \sqrt[n]{a} \rightarrow 1$\\
$ \sqrt[n]{a} = 1 + x_n $ \\
$ a = (1 + x_n) ^n > 1 + n \cdot x_n $ \\
$ x_n \leq \dfrac{1}{n} (a-1)  $ \\
$ 1 \leq \sqrt[n]{a} \leq  \dfrac{1}{n} (a-1) $ \\
По зажатой последовательности $ \sqrt[n]{a} \rightarrow 1 $\\

$ x_n = \left(1 + \dfrac{1}{n}\right)^n $\\
$ y_n =  \left(1 + \dfrac{1}{n}\right)^{n+1} $ \\
$ \dfrac{y_{n-1}}{y_n} = \dfrac{ \left(1 + \dfrac{1}{n-1}\right)^{n}}{ \left(1 + \dfrac{1}{n}\right)^{n+1}} = \left( \dfrac{1+\dfrac{1}{n-1}}{1 + \dfrac{1}{n}} \right)^{n-1} \left( 1 + \dfrac{1}{n-1} \right)^{-1}  = \left( \dfrac{n+\dfrac{n}{n-1}}{n + 1} \right)^{n+1} \dfrac{n-1}{n}  =  \left( \dfrac{n+1+\dfrac{n}{n-1}}{n + 1} \right)^{n+1} \dfrac{n-1}{n}  \geq \left(1 + (n+1)\dfrac{1}{n-1} \right)\dfrac{n-1}{n} = \left( 1+ \dfrac{1}{n-1} \right) \dfrac{n-1}{n}  = 1$\\
Т.е. $ y_n $ невозрастающ. $ \Rightarrow y_n \rightarrow y$\\
$ x_n = y_n \cdot ( 1 + \dfrac{1}{n})^{-1} \rightarrow y $ \\

	\begin{theorem}
	Пусть $x_n$ последовательность, $ x_n > 0, \lim \dfrac{x_{n+1}}{x_n} \leq 1 \Rightarrow \lim x_n = 0 $
	\begin{proof}
		Пусть $ c $ такое число, что $  \lim \dfrac{x_{n+1}}{x_n} \leq c \leq 1 $ \\
		$ \exists N, \forall n \geq N,  \dfrac{x_{n+1}}{x_n} < c $ \\
		$ \forall m \in \N :  \dfrac{x_{n+1}}{x_n} = \prod_{i=0}^{m-1}  \dfrac{x_{n+1}}{x_n} < c^m $ \\
		$ 0 < x_{n+m} < c^mx_n \underset{m \rightarrow \infty}{\rightarrow} 0 $ \\
		$ \Rightarrow \lim\limits_{n \rightarrow \infty} x_{n+m} = 0 =  \lim\limits_{n \rightarrow \infty} x_{n} $

	\end{proof}
	\begin{consequence}
		$ a > 1 \Rightarrow \dfrac{n^k}{a^n} \rightarrow 0 $
		\begin{proof}
			$ x_n = \dfrac{n^k}{a^n} $ \\
			$ \dfrac{x_{n+1}}{x^n} = \dfrac{(n+1)^k}{n^k \cdot a} = (1 + \dfrac{1}{n})^k \cdot \dfrac{1}{a} \rightarrow \dfrac{1}{a} \leq 1 $
		\end{proof}
	\end{consequence}
\end{theorem}
\begin{definition}
	Пусть $ x_n, y_n $ бесконечно большие. Говорят, что $ x_n $ бесконечно большая меньшего порядка, если $ \dfrac{x_n}{y_n} \rightarrow 0, x_n = O(y_n) $ \\
	$ n^k = O(a^k) $ 
\end{definition}
\begin{consequence}
$ \dfrac{a^n}{n!} \rightarrow 0 \forall a > 0 $ \\

$ \dfrac{x_{n+1}}{x_n} = a \cdot \dfrac{1}{n+1} \rightarrow 0 $ 
\end{consequence}
\begin{consequence}
	$ \dfrac{n!}{n^n} \rightarrow 0 $ \\
	\begin{proof}
		$ \dfrac{x_{n+1}}{x_n} = (n+1)\dfrac{n^n}{(n+1)^{n+1}} = \dfrac{n^n}{(n+1)^n} = (\dfrac{n}{n+1})^n = \dfrac{1}{(1+\frac{1}{n})^n} $
	\end{proof}
	
\end{consequence}

\Section{Теорема Больцано-Вейерштрасса и критерий Коши}

\begin{theorem}
	Теорема о стягивающихся отрезках \\
	Пусть заданы отрезки числовой прямой $ [a_i, b_i], i = 1, 2, 3,...$ \\
	\begin{enumerate}
		\item $ [a_i, b_i] \supset [a_{i+1}, b_{i+1}] $ 
		\item 
		% pic1
	\end{enumerate}
	\begin{proof}
		Знаем $ \cap [a_n, b_n] \neq \emptyset $ \\
		Предположим, что это не 1 элем. мн-во \\
		Тогда $ \exists c, d \in  \cap [a_n, b_n], c < d $ \\
		$ \forall n \in \N : c, d \in  [a_n, b_n] \Rightarrow a_n \leq c < d \leq b_n  \Rightarrow b_n - a_n \geq $
		% pic2,3
	\end{proof}
\end{theorem}
\begin{definition}
	Говорят, что $ b_n $ подпоследовательность $ a_n $ если \\
	возр. посл-ть натуральных чисел $ m_i $ \\
	$ \forall n \in \N$
	
\end{definition}
\begin{theorem}
	Теорема Больцано-Вейерштрасса \\

	Пусть $ x_n$ - ограниченная последовательность. Тогда в $ x_n $ можно выбрать сходящуюся подпоследовательность
	\begin{proof}
	$ (x_n) \Rightarrow \exists a, b \in \R : a \leq b, \forall n : x_n \in [a, b] $ \\
	Из отрезков $ [a, a+b / 2] $ и $ [a+b/2, b] $ выберем $ [a_1, b_1] $ такой, что $ \{i \mid x_i \in [a_1, b_1]\} $ бесконечно. \\
 	Из отрезков $ [a_1, a_1+b_1 / 2] $ и $ [a_1+b_1/2, b_1] $ выберем $ [a_2, b_2] $ такой, что $ \{i \mid x_i \in [a_2, b_2]\} $ бесконечно. \\
    $ b_i - a_i = \frac{1}{2^n} (b-a) $ \\
    По теореме о стягивающихся отрезках 
    $ \cap [a_n, b_n] = \{c\}$ \\
    $ m_1 $ такое число, что $ x_{m_1} \in [a_1, b_1] $ \\
    $ m_2 $ такое число, что $ x_{m_2} \in [a_2, b_2] $ \\
    ...\\
    $ x_{m_i} $ - подпоследовательность \\
    $ a_i \leq x_{m_i} \leq b_i  \Rightarrow x_m \rightarrow c$ 
    \end{proof}
	Зам. Легко видеть, что  если $ x_n \rightarrow c \in \R $ \\
	То любая подпоследовательность $ x_{m_i} \rightarrow c $ \\
\end{theorem}
Дополнение \\
Пусть $x_n$ - неограниченная последовательность. Тогда в ней есть подпоследовательность $ x_{n_i}  \rightarrow +\infty $ или $ \rightarrow -\infty $ \\
\begin{proof}
	Пусть $ x_n $ не ограничена сверху \\
	Тогда легко видеть, что $ \forall n \in \N $ существует бесконечно много $ i \in \N : x_i > n $ \\
	Иначе существуеть лишь конечн. i \\
	Пусть это так, тогда $ max(x_1, x_2, ..., n) $ - верхняя граница $x_n$\\
	$ m_1 $ - любое нат. число, $ x_{m_1} > 1 $ \\
	$ m_2, x_{m_2} > 2, m_2 > m_1 $ \\
	$ (x_{m_i})  -$ посл-ть \\
	$ \{i \mid x_{m_i} \notin (n, +\infty) \} \subset \{ 1,2,..., n-1 \} $ \\
	То $ x_{m_i} \rightarrow +\infty $\\
	Аналогично для $ -\infty $ 
\end{proof} 
\begin{consequence}
	%pic5
\end{consequence}

\begin{definition}
	Пусть $x_n$ - последовательность. Она называется фундаментальной, если выполненое след. св-во \\
	$ \forall \eps,  \exists N \in \N, \forall m,n \geq N : | x_m - x_n | < \eps $\\
	Очевидно, сходящаяся последовательность фундаментальна. Можно взять предел и окресность $ \frac{\eps}{2} $ \\
	Фундаментальная $=$ сходящаяся в себе $=$ последовательность Коши.
\end{definition}
\begin{theorem}
	Теорема Больцано-Коши(Критерий Коши) \\
	Последовательность $x_n$ фундаментальна $ \Leftrightarrow $ сходится
	\begin{proof}
		Пусть $ (x_n) $ - фундаментальна \\
		1. Докажем, что $(x_n)$ - ограничена \\
		По определению, $  \exists N \in \N \forall n \geq N : x_n \in (x_{N} - 1, x_N +1)$\\
		$ a = min(x_N - 1, x_n, ..., )$ %pic6
		2. По т. Б-В, $ (x_n)$ содержит сх. подпоследовательность $(x_{m_i}) $ \\
		Пусть $ x_{m_i} \rightarrow c $ \\
		Д-м, что $ x_n \rightarrow c $ \\
		По опр фунд. посл $ \exists N \in \N  : \forall m, n \geq N : |x_m-x_n| < \frac{\eps}{2}$ \\
		$\exists N' \in \N : \forall i \geq N' : |x_{m_i} -c | < \frac{\eps}{2} $ \\
		Пусть i таково, что %pic7,8,9,10
		
	\end{proof}
\end{theorem}
I R как множество дедекиндовых сечений\\
2 R как множество классов посл-тей коши\\
M - мн-во последовательностей Коши рац. чисел %pic11


	
%7-адический показатель \\
%$ v_7\left( 7^b \dfrac{p}{l} \right) $ \\
%$ v_7 \left( \dfrac{3}{14} \right)  = -1 $ \\
%7-адическое расстояние \\
%$ \rho (a, b) = 7  \ \ \ -v_7(a - b) $ \\
%$ \rho (a, a) = 0 $ \\
%$ a_i - $ последовательность Коши \\
%$ \forall \eps > 0, \exists N \in \N, \forall i,j > N, v_7(a, b) < \eps$ \\
%$ M \setminus \sim = \mathbb{Q}_2 $ \\
\Section{Верхние и нижние пределы. Частичные пределы}

\begin{definition}
$ x_k - $ ограничено \\
$ y_n = \sup\limits_{k \geq n} x_k \in \R $ \\
$ y_1 \geq y_2 \geq y_3 ...$\\
$ y_n - $ ограничено \\
$ x_k \geq C \Rightarrow \forall n : y_n \geq C $ \\
$ \lim y_n = \varlimsup\limits_{n \rightarrow \infty} x_n = \lim\sup\limits_{n \rightarrow \infty} x_n  - $ верхний предел посл. $ x_n $ \\
$ x_n $ не огр. сверху $ \Rightarrow \varlimsup x_n = -\infty $\\
$ x_n $ огр сверху, но не огр. снизу $ \Rightarrow \varlimsup \in \N \cup -\infty $ \\
$ z_n = \inf\limits_{k \geq n} x_k $ \\
$ z_1 \leq z_2 ... $\\
$ \lim z_n = \varliminf x_n = \lim\inf x_n $ - нижний предел 
\end{definition}
Прежлож . Пусть $x_n$ - произв. послед \\
$ \varliminf \leq \varlimsup $\\
$ y_n \leq z_n \Rightarrow \lim y_n \leq \lim z_n $ \\

\begin{definition}
	Пусть $ x_n $ посл-ть, $ a \in \R $ \\
	$ a $ - частичный предел $ x_n $, если в $ x_n $ есть подпосл-ть, стремящаяся к $ a $ \\
	\begin{theorem}
		$ x_n $ - посл-ть \\
		1. Её верхний предел - наибольший частичный предел \\
		2. Её нижний предел - наименьший частичный предел \\
		3. $ \exists \lim x_n \Leftrightarrow \varlimsup x_n = \varliminf x_n $ \\
		$  \lim x_n =  \varlimsup x_n = \varliminf x_n $ 
		\begin{proof}
			Пусть $ A = \varlimsup x_n $ Предположим $ A \in \R $\\
			$ u_n = \left( A - \dfrac{1}{n}, A + \dfrac{1}{n} \right) $ \\
			$ x_{m_1} \in u_1, x_{m_2} \in u_2 $ \\
			$ y_n = \sup\limits_{k \geq n} x_k \rightarrow A \Rightarrow \exists N : \left\{ \begin{array}{ll} y_n < A + \eps \Rightarrow \forall n \geq N x_n < A + \eps \\ y_n > A - \eps \end{array} \right. $ \\
			Предпол. $ x_i \notin u_1 \ \forall i \left[ \begin{array}{ll} x_i < A - \eps \\ x_i > A + \eps \end{array} \right.$\\ % --pic3,4 
			$\hspace*{2cm} \Rightarrow \forall n \geq N : x_i < A-\eps  $ \\
			$ \hspace*{2cm} \Rightarrow y_n \leq A - \eps $\\ 
			Т.о. $ \exists m_1 \in \N : x_{m_1} \in U_1 $\\
			Аналогично $ \exists m_2 > m_1, x_{m_2} \in U_2 $\\
			
			Предп $ \exists A' > A : A' - $ частичный предел $ x_n $, т.е. $ \exists x_{m_k} \rightarrow A' \ \  \eps = (A' - A) / 2 $ \\
			$ \exists N : \forall k \geq N \ x_{m_k} > A' - \eps $ \\
			$ y_{m_k} \geq x_{m_k} > A - \eps $ \\
			$ \lim y_n = \lim y_{m_k} \geq A' - \eps = \dfrac{A'+A}{2} > A $
		\end{proof}
		\begin{proof}
			Пусть $ \exists \lim x_n = A \Rightarrow $ любой частичный предел равен $ A $  \\
			$ \varliminf x_n = \varlimsup x_n = A $\\
			Пусть $ \varliminf x_n = \varlimsup x_n = A $ \\
			$ A \in \R $ \\
			$ \varliminf x_n = A \Rightarrow \exists N_1 \forall n \geq N_1 : z_n \in  $ \\% pic6]
		\end{proof}
	\end{theorem}
\end{definition}
Предл. Пусть $ \forall n \in \N : a_n \leq b_n $ Тогда $\varlimsup a_n \leq \varlimsup b_n, \varliminf a_n \leq \varliminf b_n $ \\
\begin{theorem}
	1. $ A = \varliminf x_n  \Leftrightarrow \left\{ \begin{matrix}
	\forall \eps > 0 \  \exists N, \forall n \geq N, x_n > a - \eps \\
	\forall \eps > 0 \  \forall N, \exists n \geq N , x_n  < a + \eps 
	\end{matrix}  \right. $\\
	2. $ A = \varlimsup x_n  \Leftrightarrow \left\{ \begin{matrix}
	\forall \eps > 0 \ \forall N, \exists n \geq N, x_n > a - \eps \\
	\forall \eps > 0 \  \exists N, \forall n \geq N , x_n  < a + \eps 
	\end{matrix} \right.  $ \\
	\begin{proof}
		$ \Rightarrow x_n \geq z_n \rightarrow a (z_n  \inf\limits_{k \geq n} x_k) $\\
		$ \exists N_0 \forall n \geq N_0 : z_n < a + \eps  $ \\
		т. е. $ \exists k \geq n : x_k < a + \eps $ \\
		$ \Leftarrow \eps > 0, \exists n \forall n \geq N : z_n \in u_{\eps}(a) $\\
		$ z_n > a - \eps $ \\
		По (1) $ \exists N_1 \forall n \geq N_1 : x_n > a - \frac{\eps}{2} \Rightarrow \forall n \geq N_1 : z_n \geq a - \frac{\eps }{2} > a - \eps $ \\
		$ \forall n \in \N : z_n = \inf\limits_{k \geq n} x_k < a + \eps $ (при нек. k) \\
		Т.е. $ z_n \rightarrow k $
		% pic7
	\end{proof}
\end{theorem} 
Предложение. Пусть $ a_n b_n $ последовательности
\begin{enumerate}
	\item $ \varliminf(a_n + b_n) \geq \varliminf a_n + \varliminf b_n $ 
	\begin{proof}
		$ \inf\limits_{k \geq n} (a_k + b_k) \geq  \inf\limits_{k \geq n} a_k +  \inf\limits_{k \geq n} b_k $ \\
		$ \forall k \geq n : a_k + b_k \geq  \inf\limits_{k \geq n} a_k +  \inf\limits_{k \geq n} b_k \Rightarrow \lim\inf\limits_{k \geq n} (a_k + b_k ) \geq \lim( \inf\limits_{k \geq n} a_k +  \inf\limits_{k \geq n} b_k) $\\
	\end{proof}
	\item $ \varlimsup(a_n+b_n) \leq \varlimsup a_n + \varlimsup b_n $ 
\end{enumerate}

\Section{Ряды}

\begin{definition}
	$ \sum_{k=1}^{\infty} a_k $\\
	Говорят, что сумма ряда $  \sum_{k=1}^{\infty} a_k $ равна  $ b \in \R $ если $ S_n \rightarrow b, $ где $ S_n =  \sum_{k=1}^{n} a_k$
	Посл-ть частичных сумм ряда $  \sum_{k=1}^{\infty} a_k$ \\
	$ (c_n) a_n = \left\{ \begin{matrix}
		c_1, n = 1 \\
		c_n - c_1, n > 1 
	\end{matrix} \right. $ 
	Если $  \sum_{k=1}^{\infty} a_k = b \in \R, $ говорят, что ряд сходится.
\end{definition}
Предл. Пусть $  \sum_{k=1}^{\infty} a_k $ сходится, тогда $ a_k \rightarrow 0 $ 
Д-во $ a_k = S_k - S_k-1 = b - b = 0$\\
\begin{example}
	 $ \sum_{k=1}^{\infty} q^n $\\
	 $ s_k = \dfrac{q - q^n}{1 - q} $ \\
	 $ |q | > 1\ \ S_k \rightarrow \infty $ \\
	 $ q = 1 \ \ S_k = k \rightarrow \infty $ \\
	 $ | q | < 1, S_k \rightarrow \dfrac{q}{1- q} $ \\
\end{example}
\begin{example}
	$  \sum_{k=1}^{\infty} \dfrac{1}{k} = +\infty $ \\
\end{example}
\begin{properties}
	\begin{enumerate}
		\item $ \sum(a_k + b_k) = \sum a_k + \sum b_k $ если $a_k $ $ b_k $ сходятся.
		\item $ \sum (c \cdot a_k ) = c \cdot \sum a_k $ 
		\item Если ряд сходится, то сходится и имеет ту же самую сумму ряд, полученный расстановкой скобок.
	\end{enumerate}
\end{properties}
\begin{theorem} Теорема Штольца \\
	Пусть $ y_1 < y_2, \lim y_n =  +\infty $ \\
	$ x_n, \lim \dfrac{x_n - x_{n-1}}{y_n - y_{n-1}} = l \in \R \Rightarrow \dfrac{x_n}{y_n} = l $
	\begin{proof}
		1. $ l = 0, \eps_n = \dfrac{x_n - x_{n-1}}{y_n - y_{n-1}} $ \\
		$ \eps > 0, \exists m \forall n \geq m \ \ |\eps_n | < \eps $ и $ y_m > 0 $\\
		$ x_n - x_m = \sum_{k=m+1}^{n} (x_k - x_{k-1}) = \sum_{k=m+1}^{n} \eps_k (y_k - y_{k-1}) $ \\
		$ |x_n - x_m| = \sum_{k=m+1}^{n} |\eps_k|_{\leq \eps} | y_k - y_{k-1}| < \eps \sum_{k=m+1}^n (y_k - y_{k-1}) = \eps (y_n - y_m) \leq  $ \\
		$ \left| \dfrac{x_n}{y_n} \right| \leq \dfrac{|x_m - x_n |}{|y_n|}_{<\eps} + \dfrac{x_m}{y_n}_{\rightarrow 0} <  2\eps $ при достаточно больших $ \eps $ \\
		2. $ l \in \R \ \ \tilde{x}_n = x_n - ly_n $ \\
		$ \dfrac{\tilde{x} - \tilde{x}_{n-1}}{y_n - y_{n-1}} = \dfrac{x_n - x_{n-1}}{y_n - y_{n-1}}_{\rightarrow l} - \dfrac{ly_n - ly_{n-1}}{y_n - y_{n-1}}_{\rightarrow l} \rightarrow 0 $ \\
		$ \dfrac{\tilde{x}_n}{y_n} \rightarrow 0$ \\
		$ \dfrac{x_n}{y_n} = \dfrac{\tilde{x}_n + ly_n}{y_n} = \dfrac{\tilde{x}_n}{y_n} + l \rightarrow l $ \\
		3. $ l = +\infty $ \\
		$ \dfrac{x_n - x_{n+1}}{y_n - y_{n+1}} > 1 $ н.с.н.м \\
		$ x_n - x_{n-1} > y_n - y_{n-1} $ при $ n \geq m$ в частн $ x_n - x_{n-1} > 0 $ \\
		$ x_n - x_m > y_n - y_m $ \\
		$ x_n > y_n + (x_m - y_m) \rightarrow +\infty \Rightarrow x_n \rightarrow +\infty $ \\
		$ \lim \dfrac{y_n - y_{n-1}}{x_n - x_{n-1}} = \left( \lim \dfrac{x_n - x_{n-1}}{y_n - y_{n-1}} \right)^{-1} = 0 $ \\
		$ \Rightarrow \lim \dfrac{y_n}{x_n} = 0 \Rightarrow \dfrac{x_n}{y_n} $ - бесконечно большая $ \Rightarrow \dfrac{x_n}{y_n} \rightarrow +\infty $ \\
		4. $ l = -\infty $ \\
		$ \tilde{x}_n = -x_n $ \\
		$ \dfrac{\tilde{x}_n}{y_n} \rightarrow +\infty \Rightarrow \dfrac{x_n}{y_n} \rightarrow -\infty $
	\end{proof}
\end{theorem}
$ \lim\limits_{n \rightarrow \infty} \dfrac{1}{n^{m+1}} \sum_{k=1}^{n} k^m $ \\
$ y_n = n^{m+1} $ \\
$ x_n = \sum_{k=1}^{n} k^m $ \\
$ \dfrac{x_n - x_{n-1}}{y_n - y_{n-1}} = \dfrac{n^m}{n^{m+1}- (n-1)^{m+1}} = \dfrac{n^m}{n^{m+1} - (n^{m+1} - (m+1)n^m) + \frac{(m+1)m}{2} n^{m-1}} = \dfrac{n^m}{(m+1) n^m - \frac{(m+1)m}{2} n^{m-1} } = \dfrac{1}{m+1} $ \\
	
	
\begin{theorem}
	Теорема Штольца для отношения бесконечно малых\\
	$ y_1 > y_2 > ... > 0 $ \\
	$ \lim x_n = \lim y_n = 0 $ \\
	$ \lim \dfrac{x_n - x_{n-1}}{y_n - y_{n-1}} = l \in \overline{\R} $ \\
	$ \lim \dfrac{x_n}{y_n} = l $ \\
	\begin{proof}
		1. $l = 0 \ \ \ | \eps_k | < \eps $ при $ k \geq m \geq N $ \\
		$ | x_n - x_m |_{\rightarrow |x_m|} \leq \eps (y_m - y_n)_{\rightarrow \eps y_m} $ \\
		$ |x_m| \leq \eps y_m $ \\
		$ \dfrac{|x_m|}{y_m} \leq \eps $ \\
		Т.о. $ \dfrac{x_m}{y_m} \rightarrow 0 $ \\
		2. $ l \in \R $ сводится к п.1
		3. $ l = + \infty $ \\
		$ \dfrac{ x_k - x_{k-1}}{y_k - y_{k-1}} > 1 \Rightarrow x_k \leq x_{k-1} $\\
		$ x_k \searrow, x_k \rightarrow 0 \Rightarrow x_k > 0$ начиная с нек. $k$ \\
		$ \dfrac{y_k - y_{k-1}}{x_k - x_{k-1}} \rightarrow 0 \Rightarrow \dfrac{y_k}{x_k} \Rightarrow 0 \Rightarrow \dfrac{x_k}{y_k} \rightarrow +\infty $\\
		4. 
 	\end{proof}
\end{theorem}

\Section{Пределы функций}

\Subsection{Предельные точки множеств}

$ E \subset \R $ \\
$ a \in \R - $ предельная точка мн-ва E \\
$ \forall \eps > 0 \exists x \in E : \left\{ \begin{matrix}
|x-a| < \eps \\
x \neq a 
\end{matrix}\right. $
\begin{enumerate}
	\item $ (a, b') = [a, b] $ 
	\item $ \mathbb{Q}' =  \R $
	\item $\dfrac{1}{\R}' = {0} $ 
\end{enumerate}
Предл. След 3 условия эквивалентны
\begin{enumerate}
	\item а предельная точка Е 
	\item В любой окресности а есть бесконечно много точек Е 
	\item Сущ посл-ть $(x_n), \forall n : x_n$
\end{enumerate}
\begin{proof}
	$ 1 \Rightarrow 2 $ \\
	Пусть $ \dot{u}_{\eps} (a) \cap = {x_1 .. x_n} $ \\
	Пусть $ \eps' = \min (\eps, |x_i - a | ) $\\
	$ \dot{u}_{\eps'} (a) \cap E = \emptyset $ \\
\end{proof}
\Subsection{Предел функции}
Пусть $ E \subset \R $ \\
$ f : E \rightarrow \R $ \\
a - пред. точка E \\
Говорят, предел $ f $ в а равен $y$  $ ( \lim\limits_{x \rightarrow a} f(x) = y $ или $ f(x) \xrightarrow{x \rightarrow a} y )$ \\
1. (Предел по Коши) Если $ \forall $ окресн. $ U_{\eps} (y) $ сущ $ \delta > 0 $ \\
$ f (\dot{U_{\delta}}(a) \cap E) \subset U_{\delta} (y) $ \\
$ \forall \eps > 0 \exists \delta > 0 \ \forall x \in E \setminus \{a\} \ \ | x - a | < \delta \Rightarrow | f(x) - y | < \eps $ \\
2. В любой окрестности a есть бесконечно много точек E \\
3. Существует посл-ть $ (x_n) $ т.ч. $ \forall n: E \setminus \{a\}$ и $ x_n \rightarrow a $ \\
$ x_n $ любая т. $ \dot{U_{\frac{1}{n}}} (a) \cap E  \ \ |x_n -a| < \dfrac{1}{n} $ \\
$ x_n \rightarrow a $ 


	\Section{Матрицы и операции над ними}

R - кольцо\\
$ \begin{pmatrix}
	a_{11} & ... & a_{1m} \\
	...& ...& ... \\
	a_{n1} & ... & a_{nm}
\end{pmatrix}\\
Mat(R, n, m) \\
A = (a_{ij})$\\
$ R \times Mat(R, n, m) \rightarrow Mat(R, n, m) $\\
$ a \times A =  \begin{pmatrix}
a\cdot a_{11} & ... & a\cdot  a_{1m} \\
...& ...& ... \\
a\cdot  a_{n1} & ... & a\cdot a_{nm}
\end{pmatrix}$ \\
$ A + B = (a_{ij} + b_{ij})  \forall i < n, j < m$ \\
$ + : Mat(R, n, m) \times Mat(R, n, m) \rightarrow Mat(R, n, m) $\\
$ \cdot : Mat(R, n, m) \times Mat(R, m, l) \rightarrow Mat(R, n, l) $ \\
$ A (i, j), B(j, k) $ \\
$ AB = C = (c_{i k}) $ \\
$ c_{ik} = \sum_{j=1}^{m} a_{ij} b_{jk} $ \\
$ \begin{pmatrix}
	a_{i1} & ... & a_{im} \\
\end{pmatrix}  \begin{pmatrix}
	b_{1k}  \\
	...  \\
	b_{mk}
\end{pmatrix} = (c_{ik})\\$
$A - n \times m, B - m \times l $\\
$ x_1, ..., x_e $ \\
$y_1, ..., y_m $ \\
$z_1, ..., z_m $ \\
Линейное преобразование $ f(ax) = a \cdot f(x) $\\ 
y линейно зависит от х \\
z линейно зависит от y\\
$ y_1 = b_{11} x_1 + ... + b_{1l}x_l $ \\
$ \vdots ---- $\\
$ y_m = b_{m1}x_1 + ...+ b_{ml} x_l$

$ z_i = \sum_{r=1}^m a_{ir} y_r = \sum_{r=1}^{m} a_{ir}(\sum_{j=1}^{l} b_r x_j )  = \sum_{r=1}^{m} \sum_{j=1}^{l} a_{ir}b{rj}x_j =\\=  \sum_{i=1}^{l} \sum_{j=1}^{m} a_{ir}b{rj}x_j = \sum_{i=1}^{l}  (\sum_{r=1}^{m}a_{ir}b_{rj})x_j $ \\
Матрицы с одним столбцом называются векторами-столбцами, с 1 строкой - векторами-строками.
 
$ X =\begin{pmatrix}
X_1 \\
\vdots \\
X_l
\end{pmatrix}  Y=\begin{pmatrix}
Y_1 \\
\vdots \\
Y_m
\end{pmatrix}X =\begin{pmatrix}
Z_1 \\
\vdots \\
Z_n
\end{pmatrix}$\\
$ Y = B \cdot X, Z = A \cdot Y, Z = (A\cdot B) \cdot X $\\
$ A \in M(n,m,R)\\
B \in M(m,l,R)\\
C \in M(l,k,R)\\
(AB)C = A(BC) $\\
$ ((AB)C)_{ij} = \sum_{s=1}^{l} (AB)_{ij} C_{sj} $\\
$ \sum_{s=1}^{l} (\sum_{r=1}^{m} (A_{ir} B{rs} )) = \sum_{r=1}^{m} \sum_{s=1}^{l} A_{ir} (B_{rs} C_{sj}) = \sum_{r=1}^{m} A_{ir} (\sum_{s=1}^{l} B_{rs} C_{sj}) = \sum_{r=1}^{m} A_{ir} (BC)_{rj}  = (A(BC))_{ij}$\\
$ A \in M(n,m,R)\\
B,C \in M(m,l,R)$\\
$ D \in M(l,k,R) $ \\
Тогда $ A(B+C) = AB+AC$\\
$ (B+C)D = BC+CD $ - доказать упражнение \\

$ a\in R, B,C  - $ матрицы одного размера \\
1. $ a(B+C) = aB + aC $\\
2. $ a(BC) = (aB)C $\\
3. Если R -коммутативно \\
$ a(BC) = B(aC) $\\

Матричное умножение некоммутативно \\
$ A - n \times m$\\
$ B m \times l $ \\
$ AB, l \neq m \Rightarrow BA $ не определено \\
$ l = m $\\
$ AB - n \times n, BA - m \times m $\\
Если $ n \neq n $ то размеры получающихся матриц не совпадают
$ l = m = n $ \\
$ n = 1, R $- коммутативно, $ M(1,1,R) $ - коммутативно \\
$ n \geq 2 - $ не коммутативно.\\
$ A = \begin{pmatrix}
0 & 1\\
0 & 0 
\end{pmatrix}
B = \begin{pmatrix}
0 & 0 \\
1 & 0
\end{pmatrix} \\
AB = \begin{pmatrix}
1 & 0 \\
0 & 0 
\end{pmatrix}
BA = \begin{pmatrix}
0 & 0\\
0 & 1
\end{pmatrix} AB \neq BA$
\begin{theorem}
	R - кольцо с 1\\
	$ M_n (R) = M(n,n,R) - $ кольцо с 1, если $ n \geq 2 $ то не коммутативн\\
	$ \overline{M_n} (R)  $ - кольцо матриц размера n под кольцом R \\
	Для сложения свойства очевидны. 
	$ \mathds{O} = \begin{pmatrix}
		0 & ... & 0 \\
		\vdots & ... & \vdots \\
		0 & ... & 0
	\end{pmatrix} $\\
	 $ A = (a_{ij}) \\
	 -A = (-a_{ij}) $
	 $ \mathds{1} = \begin{pmatrix}
	 1 & ... & 0 \\
	 \vdots & 1 & \vdots \\
	 0 & ... & 1
	 \end{pmatrix} $\\
	 Если $ n \geq 2, $  в $ M_n(R) $ есть нетривиальные делители нуля - не всякий элемент обратим \\
	 $ A = \begin{pmatrix}
	 	0 & 1\\
	 	0 & 0 
	 \end{pmatrix} \neq \mathds{0} $ \\
	 $ A^2 = \begin{pmatrix}
	 0 & 1\\
	 0 & 0 
	 \end{pmatrix} \begin{pmatrix}
	 0 & 1\\
	 0 & 0 
	 \end{pmatrix} = \begin{pmatrix}
	 0 & 0\\
	 0 & 0 
	 \end{pmatrix} 
	 A^2 = 0 \Rightarrow A - $ необратим
 \end{theorem}

\subsection{Ещё раз о комплексных числах}

\begin{definition}
	R - кольцо \\
	$ \emptyset \neq R_1 \subseteq R $\\
	$ R_1 - $ подкольцо в R если оно является кольцом отностительно тех же операций, что и в R \\
	$ + : R \times R \rightarrow R $ \\
	$ + : R_1 \times R_1 \rightarrow R_1 $ \\
	Говорят, что $ R_1 $ замкнуто относительно сложения (умножения)
\end{definition}
Предложение $ \emptyset \neq R_1 \subseteq R $ является подкольцом если оно замкнуто по сложению, взятию обратного по сложению и умножения. 
$ a, b \in R_1, a+b \in R_1, -a \in R_1, ab \in R_1 $ \\
Рассмотрим $ M_2(\R) $ \\
$\left\{  \begin{pmatrix}
a & -b\\
b & a 
\end{pmatrix} \top a,b \in \R \right\} = C$ \\
$  \begin{pmatrix}
a & -b\\
b & a 
\end{pmatrix} +  \begin{pmatrix}
c & -d\\
d & c 
\end{pmatrix} =  \begin{pmatrix}
a+c & -(b+d)\\
b+d & a+c 
\end{pmatrix} \in C $ \\
$  \begin{pmatrix}
a &- b\\
b & a 
\end{pmatrix}  \begin{pmatrix}
c & -d\\
d & c 
\end{pmatrix} =  \begin{pmatrix}
ac-bd & -(ad+bc)\\
bc+ad & -bd+ac 
\end{pmatrix} \in C $ \\
$ \begin{pmatrix}
c & -d\\
d & c 
\end{pmatrix}  \begin{pmatrix}
a &- b\\
b & a 
\end{pmatrix} = 
 \begin{pmatrix}
ca-db & -(cb+da)\\
da+cb & -db+ca
\end{pmatrix} =  \begin{pmatrix}
a &- b\\
b & a 
\end{pmatrix}  \begin{pmatrix}
c & -d\\
d & c 
\end{pmatrix} $
C - коммутативное кольцо\\
$ (a,b) \neq (0,0) $ \\
$  \begin{pmatrix}
a &- b\\
b & a 
\end{pmatrix}  \cdot \dfrac{1}{a^2+b^2}  \begin{pmatrix}
a & b\\
-b & a 
\end{pmatrix} = \dfrac{1}{a^2+b^2} \begin{pmatrix}
a^2+b^2 & ab-ba\\
ba-ab & b^2+a^2
\end{pmatrix}   = \begin{pmatrix}
1 & 0\\
0 & 1 
\end{pmatrix}   \Rightarrow C - $ поле \\
Изоморфизм $ \mathbb{C} $ и C \\
$ (a, b) \rightarrow 
 \begin{pmatrix}
a & -b\\
b & a 
\end{pmatrix} $\\
$ \phi : \mathbb{C} \rightarrow C $ \\
$ z_1 = a+bi = (a,b) \\
z_2 = c + di = (c,d) $\\
$ z_1 + z_2 = a + c + (b+d)i $ \
$ \phi(z_1+z_2) = \phi(z_1) + \phi (z_2) $\\
$ z_1 z_2 = ac - bd + (ad + bc)i $ \\
$ \phi(z_1 z_2) = \phi(z_1) \phi(z_2) $\\
$ \mathbb{C} \cong C $

\Section{Тело кватернионов}

$ M_2 (\mathbb{C}) $ \\
$ \mathcal{H} =\left\{ \begin{pmatrix}
z_1 & - \overline{z_2} \\
z_2 &  \overline{z_1}
\end{pmatrix}, z_1,z_2 \in \mathbb{C} \right\}$\\
$  -\begin{pmatrix}
z_1 & - \overline{z_2} \\
z_2 &  \overline{z_1}
\end{pmatrix} =  \begin{pmatrix}
-z_1 & -\overline{z_2} \\
-z_2 &  -\overline{z_1}
\end{pmatrix} \in \mathcal{H} $\\
$ -\overline{(z_2)} = \overline{z_2}$ \\
%pic2,3
$ \mathcal{H} $ - тело, но не поле \\






	\subsection{title}

$ 1 = \begin{pmatrix}
1 & 0 \\
0 & 1
\end{pmatrix}
I = \begin{pmatrix}
i & 0 \\
0 & -i
\end{pmatrix}
J = \begin{pmatrix}
0 & i \\
i & 0
\end{pmatrix}
K = \begin{pmatrix}
0 & -i \\
1 & 0
\end{pmatrix} $ \\
$ z = c_0 + c_1 I + c_2 J + c_3 K $ \\
$ I^2 = J^2 = K^2 = -1 $ \\
$ IJ = -JI = K $ \\
$ JK = -KJ = I $\\
$ KI = -IK = K $ \\
Вторая конструкция \\
$ \{ ( c_0, c_1, c_2, c_3 ) | c_i \in \R \}$ \\
$ 1 = (1, 0, 0, 0) $ \\
$ i = (0, 1, 0, 0) \\
j = (0, 0, 1, 0) \\
k = (0, 0, 0, 1) $ \\
$ (c_0, c_1, c_ 2, c_3) \cdot (d_0, d_1, d_2, d_3) = \begin{pmatrix}
c_0d_0 - c_1d_1 - c_2d_2 -c_3d_3\\
c_0d_1 + c_1d_0 + c_2d_3 - c_3d_2\\
c_0d_2 - c_1d_3 + c_2d_0 +c_3d_1 \\
c_0d_3 + c_1d_2 - c_2d_1 + c_3d_0 
\end{pmatrix}
$\\
$ \mathcal{H} \subseteq M_2(\mathbb{C}) $ \\
$ \mapsto \mathbb{H} $ \\
$ c_0\cdot 1 + c_1 I + c_2 J + c_3 K \mapsto (c_0, c_1,c_2, c_3) $\\
$ \{ (a_0, a_1, 0, 0) | a_i \in \R \} = \{ a_0 + a_1I \} $ - Изоморфно комплексным числам  $ \backsimeq $\\
$ \{ (a_0, 0, a_2, 0) | a_i \in \R \} = \{ a_0 + a_2О \}  \backsimeq $ \\
$ \{ (a_0, 0, 0, a_3) | a_i \in \R \} = \{ a_0 + a_3Л \} $ \\

$ \det \left( \begin{pmatrix} a & b \\ c & d \end{pmatrix} \cdot \begin{pmatrix} a' & b' \\ c' & d' \end{pmatrix} \right) = \det \begin{pmatrix} a & b \\ c & d \end{pmatrix} \cdot \det \begin{pmatrix} a' & b' \\ c' & d' \end{pmatrix} $ \\
$ \begin{pmatrix} a & b \\ c & d \end{pmatrix} \cdot \begin{pmatrix} a' & b' \\ c' & d' \end{pmatrix} = \begin{pmatrix} aa' + bc' & ab' + bd' \\ ca' + dc' & cb' + dd' \end{pmatrix} $ \\
$ (aa' + bc')(cb' + dd') - (ab' + bd')(ca'+dc') ?=(ad-bc)(a'd' -b'c') $\\ % pic1
$ \mathcal{H} =  \left\{\begin{pmatrix} z_1 & - \overline{z}_2 \\ z^2 &  \overline{z_1} \end{pmatrix} | z_i \in \mathbb{C}  \right\} $ \\
$ z_1 = a_0 + a_1 i, z_2 = a_3 + a_2 i $ \\
$ \det \begin{pmatrix}  z_1 & - \overline{z}_2 \\ z^2 &  \overline{z_1}  \end{pmatrix} = z_1 \overline{z_1} + z_2 \overline{z_2} = a_0^2 +a_1^2+a_2^2+a_3^2  $\\
$\sqrt{ a_0^2 +a_1^2+a_2^2+a_3^2 } - $ модуль кватерниона $ a_0 + a_1I + a_2J + a_3K $\\
$ (c_0 + c_1I + c_2J + c_3K) \cdot (d_0 + d_1I + d_2J + d_3 K)  $ \\
$ (c_0^2 + c_1^2 + c_2^2 + c_3^2) \cdot (d_0^2 + d_1^2 + d_2^2 + d_3^2) = \\
(c_0d_0 - c_1d_1 - c_2d_2 -c_3d_3 )^2\\
(c_0d_1 + c_1d_0 + c_2d_3 - c_3d_2)^2\\
(c_0d_2 - c_1d_3 + c_2d_0 +c_3d_1)^2 \\
(c_0d_3 + c_1d_2 - c_2d_1 + c_3d_0 )^2 $ \\
$ z = (c_0, c_1, c_2, c_3) = c_0 + c_1 i + c_2 j + c_3 k $ \\
$ Re(z) = c_0 $ - вещественная часть\\
$ \vec{V} = c_1i + c_2j + c_3k $ - векторная часть \\
$ z = c_0 + \vec{V} $ \\
$ \overline{z} = c_0 - \vec{V} = c_0 - c_1i -c_2j - c_3k $ \\
$ z \cdot \overline{z} = (c_0 + c_1i + c_2j + c_3k) (c_0 - c_1 -c_2j -c_3k) = c_0^2 + c_1^2 + c_2^2 + c_3^2 = | z | $ \\
2-й способ д-ва тождества Эйлера \\
$ \overline{\overline{z}} = z $ \\
$ \overline{z_1 + z_2 } = z_1 + \overline{z_2} $ \\
$ \overline{z_1z_2} = \overline{z_2} \cdot \overline{z_1} $ \\
$ a \in \R \ \ \ \overline{az} = a \cdot \overline{z} $\\
$ |z_1z_2|^2 = z_1z_2\cdot \overline{ z_1z_2} = z_1 z_2 \overline{ z_2} \ \overline{z_1} =  = z_1 (z_2 \overline{z_2}) \overline{ z_1} = |z_1|^2 \cdot |z_2|^2$ \\ 

$ (c_0 + \vec{V}) (d_0) + \vec{U} = c_0d_0 - \vec{v} \cdot  \vec{u} + c_0 \cdot \vec{u} + \vec{v} \cdot d_0  $ \\%pic2,3

\Section{Теория делимости в коммутативных кольцах}

$ R $ - комм кольцо \\
$ a | b $ a делит b $ \exists c : b = ac $ \\
$ b \divby a  $ b делится на a \\

\begin{properties}
	\begin{enumerate}
		\item  $ a | 0 $
		\item $ a | b , b | c \Rightarrow a | c $ 
		\item $ R > 1 \Rightarrow a | a, a = a \cdot 1 $ \\
		Замечание $ a | b \& b | a \centernot\Rightarrow a = b $
		\item $ a| b, a|c \Rightarrow a | (b \pm c ) $
	\end{enumerate}
\end{properties}
\begin{definition}
	Если $ a \cdot b = 0 $, но $ a, b \neq 0 $ то a и b называются нетривиальными делителями нуля \\
	R - область целостности если нет нетривиальных делителей нуля 
\end{definition}
Замечание: для некомм R : \\
Если $ b = ac $ ,то a - левый, b - правый делитель \\

$ R > 1 - $ комм \\
$ R^* = \{ a \in R : a | 1 \} = \{ a \in R : \exists b, ab=1 \} $ - мн-во обратимых элементов \\
$ R^*m \cdot $ - группа \\
$ R^* - $ мультипликативная группа кольца \\
$ R^* \times R^* \rightarrow R^* $\\
$ a \in R^* \exists b ab=1 \Rightarrow ab = ba = 1 \Rightarrow b \in R^* $\\
$ c \in R^* \exists d \in R cd = 1 $ \\
$ (ac)(db) = a (cd) b = ab = 1 \Rightarrow ac \in R^* $ 
Примеры: $ \Z^* = \{\pm 1\} $ \\
K - поле $ K^* = K \setminus \{0\} $\\
К - поле $ (k[x])^* = K^* $ \\
Упр. 1 $ \Z[i] = \{ a + bi | a,b \in \Z \} \subseteq \mathbb{C} $ - Гауссовы числа\\
Д-те что $ \Z[i] $ - кольцо и найдите $(\Z[i])^* $\\
Упр. 2 $ \omega = \dfrac{-1 + \sqrt{3} i}{2}, \omega^3 = 1 $ \\
$ \Z[\omega] = \{ a + b\omega | a,b \in \Z \} $ - Кольцо чисел Эйзенштейна \\
$ d \in \Z $ не явл целым квадратом \\
$ \Z[\sqrt{d}] \subseteq \mathbb{C} $ если $ d < 0 $ \\
$ \hspace*{12mm} \subseteq \R $ если $ d > 0 $ \\
$ \Z[\sqrt{d}] = \{ a + b\sqrt{d} | a, b \in \Z  \} $ \\
Каждый эл-т из $ \Z[\sqrt{d}] $ единственным образом представляется в виде $ a + b\sqrt{d}, a,b \in \Z $ \\
5. Д-те, что $ (\Z[\sqrt{2}])^*, (Z[\sqrt{3}])^*, (\Z[\sqrt{5}])^*, (\Z[\sqrt{6}])^* $ - бесконечны \\ 
В общем случае связано с тем, что $ x^2 - dy^2 = \pm 1$ имеет бесконечно много решений в целых, если $ d \geq 0 $ полный квадрат.

\Subsection{Ассоцированность} 

R - кольцо коммут, $ 1 \in R $ \\
a ассоцировано с b если $ a | b \& b | a $ \\
$ a \sim b $\\
$ \sim $ - отнош эквивалентности \\
\begin{theorem}
	1. Если $ \varepsilon \in R^* $, то $ a \sim a\eps $\\
	2. Если R - область целостности, $ a \sim b \Rightarrow \exists \eps \in R^*, b = a\eps $
	\begin{proof}
		1. $ \eps \in R^*, \exists \mu \in R : \eps \mu = 1 $ \\
		$ a \eps = a \cdot \eps  \ \ \ \ a | a\eps $ \\
		$ a = a \cdot 1 = a \cdot \eps \cdot \mu  \ \  \ \ a\eps | a \Rightarrow a \sim a \eps $\\
		2. $ a | b \& b | a $ \\
		$ a = 0 \Rightarrow b = 0, 0 = 0 \cdot 1 \Rightarrow \eps = 1	$ \\
		$ a \neq 0, a | b \ \exists \eps \in R, b = a\eps $ \\
		$ b | a \ \exists r \in R \Rightarrow a = r \cdot b $ \\
		$ a = r \eps a  $ \\
		$ 0 = a (1 - re) \Rightarrow 1 - r \eps = 0 $ \\
		$ \eps | 1, \eps \in R^* $
	\end{proof}
\end{theorem}
Примеры \\
$ \Z, Z^* = \{\pm 1 \}$ \\
$ \{0\}, \{\pm 1\}, \{ \pm 2 \} $ - классы ассоцированности \\
$ K[X], K - $ поле $ \{0\} $ \\
Любой другой класс ассоцированности содержит ровно 1 многочлен со старш коэфф 1(приведённый, унитарный) \\

\Subsection{Идеал в кольце}

R - произвольное кольцо (коммут)\\ %pic@
$ \emptyset \neq I \subseteq R$ \\
$ I $ - идеал в R, если \\
1. $ \forall a,b \in I \ \ \ a \pm b \in I $ \\
2. $ \forall a \in I, \forall r \in R  \ \ \ ra \in I $ 
Пример. 1) $ \Z, I $ - мн-во чётных чисел \\
2) $ \Z, n \in \N, I = n\Z = \{ na | a \in \Z  \} $ \\
$I = R$ \\
$ I = \{0\} $ - идеалы \\ 
Для некомм колец различают левые и правые идеалы \\
1. остаётся \\
2(левый). $ \forall a \in I, \forall r \in R, ra \in I, R\cdot I \subseteq I $ \\ 2(правый). $ \forall a \in I, \forall r \in R, ar \in I,  I \cdot R \subseteq  (\sum r_is_i + \sum m_js_j)I $ \\

Замеч. в усл 1 $ \forall a,b \in I, a \pm b \in I $\\
$ 1' : \forall a,b \in I, a-b \in I , \\
1' \Rightarrow 1 $ I - непусто \\
$ a \in I, I \neq 0 $ \\
$ a - a \in I, 0 \in I $\\
$ a \in I, 0 \in  I, 0 - a \in I, -a \in I $\\
$ \forall a,b, -b \in I, a - (-b) \in I $\\

$ R, S \subseteq R $ \\
Какой наименьш идеал содержит S \\
$ s \in S \subseteq I $ \\
$ \forall r \in R, r \cdot s \in I $\\
$ \forall m \in \Z, m \cdot s \in I $ \\
$ \{ \sum_{i=1}^{n} r_is_i + \sum_{j=1}^{k} m_js_j | n, k \in \N \cup \{0\}, r_i \in R, m_j \in \Z \} $ - это мн-во образует идеал \\
а значит это и есть минимальный идеал, содержащий S \\
$ (\sum r_is_i + \sum m_js_j) \pm  (\sum r'_is'_i + \sum m'_js'_j) - $ вновь сумма такого же вида \\
Идеал, порождённый мн-вом S $ (S) $ \\














	$ S = \{a\} $\\
$ (a) - $ идеал, порождённый a, главный идеал\\
$ (a) = \{ ra + ma | r \in R, m \in \Z \} $\\
Если R кольцо с 1, то $ (a) = \{ra\} = Ra $\\
\begin{definition}
	Коммутативная область целостности R - область главных идеалов, если каждый идеал в ней главный. (ОГИ, PID (principal ideal domain))  
\end{definition}
Пример: $ R = \Z $ \\
$ (n) = \Z \cdot n = n\Z $\\
$ n = 0 \ \ (0) $\\
$ n = 1 \ \ \Z $ \\
$ \Z $ - ОГИ \\
$ K[x], K $ - поле - ОГИ \\
Пример: K - поле \\
$ K[x, y] $ \\
$ (x, y) = \{x\cdot f(x, y) + y\cdot g(x, y)  | f, g \in K[x, y] \}  = \{ h \in K[x, y] | h(0, 0) = 0 \}$ \\
$ h = \sum c_{i,j} x^iy^j = \sum_i \sum_j c_{i, j} x^i y^j + \sum_i c_{0, j} y_j $ \\
$ h(0, 0) = 0, h(x, y) = xf(x, y) + yg(x , y) $\\
$ (x, y) $ не главный \\
$ (x, y) = (h) = \{ h \cdot r | r \in K[x, y] \} $\\
$ x \in (h) $\\
$ y \in (h) $ \\
$ h | x \Rightarrow h \in \{ c, cx | c \in K^* \} $\\
$ h | y \Rightarrow h \in \{ c, cy | c \in K^* \} $ \\
$ h = const \neq 0 $\\
$ h(x, y) = xf(x, y) + yg(x, y) $\\
\Subsection{Операции над идеалами} 

R, $ \{ I_{\alpha} \}_{a \in A} \ \ I_{\alpha} -  $ идеалы в R \\
$ \bigcap_{a \in R} I_{\alpha} - $ идеал в R \\
$ \forall \alpha, 0 \in I_{\alpha} \Rightarrow 0 \in  \bigcap_{a \in R} I_{\alpha} $\\
$ a, b \in  \bigcap_{a \in R} I_{\alpha} $ \\
$ \forall \alpha \in A, a, b \in I_{\alpha} $\\
$ \forall a \in A, a \pm b \in I_{\alpha} $\\
$ r \in R, a \in  \bigcap_{a \in R} I_{\alpha} , a\pm b in  \bigcap_{a \in R} I_{\alpha} $\\
Объединение идеалов идеалом быть не обязано \\
$ 2 \Z \cup 3\Z | 3 + (-2) \notin  2 \Z \cup 3\Z$ \\
$ I_1, ..., I_n $ - идеалы в R \\
$ I_1 + ... + I_n = \{ a_1 + a_2 + .. + a_n | a_j in I_j \}$ \\
$ I_1 I_2 = \{a_1b_1 + ... + a_nb_n | a_j \in I_j, b_j \in I_j \} $\\

$ n\Z \cap m\Z  = LCM(n,m)\Z $\\
$ n\Z \cdot m\Z = nm\Z $\\
$ n\Z + m\Z = GCD(n, m) \Z $ \\

\Subsection{Сравнение по модулю идеала} 

R - кольцо с 1\\
I - двусторонний идеал в R \\
$ a \equiv b(I) $ если $ a - b \in I $ \\
$ \equiv $ - область эквивалентности \\
$ a \equiv a(I), a - a = 0 \in I $\\
$ a \equiv b (I) , a - b \in I, b - a \in I, \b \equiv a(I) $ \\
$ a \equiv b(I) , b \equiv c(I) $ \\
$ a - b \in I, b - c \in I, a - c = (a-b) + (b - c) \in I $ \\
$ [a] $ - класс эквивалентности (сравнимости) \\
$ [a] = a + I = \{a_i | i \in I \} $\\
Если $ a \equiv b(I), c \equiv d(I) $ \\
$ a + c \equiv (b + d)(I) $ \\
$ ac \equiv bd(I) $ \\

$ a - b  \in I, c - d \in I, (a+c) - (b+d) \in I \Rightarrow a+c \equiv b+d(i) $\\

$ a - b \in I, c - d \in I, (a-b)c \in I, b(c - d) \in I $\\
$ ac - bc + bc -bd \in I $\\
$ ac \equiv bd (I) $\\

\Subsection{Гомоморфизмы колец} 
$ R_1, R_2 - $ кольца \\
\begin{definition}
	c$ \phi : R_1 \rightarrow R_2 $ называется гомоморфизмом, если \\
	$ \phi(a+b) = \phi(a) + \phi(b)$ \\
	$ \phi(ab) = \phi(a) \phi(b) $
\end{definition}

$ \phi(0) = \phi(0+0) = \phi(0) + \phi(0) \Rightarrow \phi(0_{R_1}) = 0_{R_2} $\\
$ \ker \phi = \{ a \in R | \phi(a) = 0 \} $\\
Ядро гомоморфизма $ \phi $ \\
$  \phi : R_1 \rightarrow R_2 $ - гомоморфизм \\
Тогда $ \ker \phi $ - двусторонний идеал в R \\
\begin{proof}
	$ 0 \in \ker \phi $
	$ a, b \in \ker \phi $ \\
	$ \phi (a \pm b) = \phi(a) \pm \phi(b) \ \ 0+0=0$
	$ a \pm b \in  \ker \phi $\\
	$ a \in \ker \phi $ \\
	$ r \in R$ \\
	$ \phi(ra) : \phi(r) \cdot \phi(a) = \phi(r) \cdot 0 = 0 \Rightarrow ra \in \ker \phi $\\
	$ \phi (ar) = \phi(a) \phi(r) = 0 \cdot \phi(r) = 0 \Rightarrow ar \in \ker \phi $\\ 
\end{proof}

\Subsection{Факторкольцо по двустороннему идеалу}

$ R, I $ - двусторонний идеал в R \\
$ a \equiv b (I) $\\
$ R / \equiv $ - мн-во всех классов сравнимости \\
Введём на $ R/\equiv $ структуру кольца \\
$ [a] + [b] = [a+b] $\\
$ [a][b] = [ab] $\\
Проверка корректности \\
$ [a] = [c], [b] = [d] $ \\
$ [a+b] = [c+d], [ab] = [cd] $\\
$ a+b \equiv c+d(I) $\\
$ ab = cd (I) $\\
%pic1@
$ [0] + [a] = [0+a] = [a] $ \\
$ -[a] = [-a] $ \\
$ [a] + [-a] = [a + (-a)] = [0] $ \\
Если $ R \ni 1 $ \\
$ [1] \cdot [a] = [1 \cdot a ] = [a] $ \\
$ [a] \cdot [1] = [a \cdot 1] = [a] $  
$ R / \equiv, +, \cdot  $ - кольцо \\
Факторкольцо R по идеалу I \\
$ R / I $ \\
$ R = \Z , I = 2\Z $ \\
$ \Z / 2\Z = \{ [0], [1] \} $ \\

$\begin{array}{c|cc}
	+ & [0] & [1] \\
	\hline
	[0] & [0] & [1] \\ \relax
	[1] & [1] & [0] 
\end{array} \ \ \ \ \ 
\begin{array}{c|cc}
* & [0] & [1] \\
\hline
[0] & [0] & [0] \\ \relax
[1] & [0] & [1] 
\end{array}  $\\
Упражнение $ \Z / 3\Z $ \\
Пример $ \Z / 4/Z = \{ [0], [1], [2], [3] \} $ - кольцо с делителями 0 \\
$ [2] \cdot [2] = 0 $ 

$ \phi : R \rightarrow R / I $ \\
$ a \mapsto [a] $ \\
$ \phi(a + b) = [a+b] = [a] + [b] $\\
$ \phi(ab) = [ab] = [a][b] $\\
$ \phi $ - гомоморфизм сюръективный \\
$ \phi(a) = [0]  \Leftrightarrow [a] = [0] \Leftrightarrow a \equiv 0(I) \Leftrightarrow a \in I $\\
Всякий двусторонний идеал в R есть ядро некоторого гомоморфизма \\
\begin{definition}
	$ R = \Z, I = n\Z $ \\
	$ \Z / n\Z $ - кольцо вычетов по модулю n\\
	$ \Z / (n) $
\end{definition}

\Subsection{Максимальные идеалы}

R - коммутативное кольцо \\
$ I \neq R $ называется максимальным идеалом если для любого J т.ч. $ I \subseteq J \subseteq R $ либо $ I = J $ либо $ R = J $ \\
\begin{theorem}
	R - коммутативное кольцо, I - идеал \\
	$ \Leftarrow R / I $ поле $ \Leftrightarrow $ I - максимальный идеал \\
	$ R / I \ni x \neq 0_{R /I} $ \\
	$ x = [a], a \notin I $ \\
	$ I \subsetneqq Ra + I \subseteq R $\\
	$ Ra + I = R $ \\
	$ 1 \in Ra + I $ \
	$ 1 = ra + i, r \in R, i \in I $\\
	$ [1] = [ra + i] = [r] \cdot [a] + [i] = [r][a] = [r]\cdot x $\\
	$ x^{-1}  = [r] \Rightarrow R / I $ - поле \\
	$ \Rightarrow R / I $ - поле \\
	$ I \subsetneqq Y \subseteq R $ \\
	$ a \in Y \setminus I $ \\
	$ [a] \neq [0] $ \\
	$ \Rightarrow \exists r \in R, [r] [a] = [1] $ \\
	$ r - ra \in I $ \\
	$ 1 = ra + i, r \in Y, i \in I \subseteq J \Rightarrow 1 \in Y  \Rightarrow J = R $ 
\end{theorem}
 
 $ 1 \in R - $ комм \\
 $ a | b \Leftrightarrow (a) \supseteq (b) $ 
 $ b \in (b) \subseteq (a) $ \\
 $ b \in (a) $ \\
 $ b = a \cdot c $ 








   
	\Section{Наибольший обший делитель}

R комм кольцо с 1 
\begin{definition}
	$ a_1, ..., a_n \in R $\\
	$d $ - наиб общ делитель если \\
	1. $ d | a_1, ..., d | a_n $ \\
	2. $ \forall d' (d' | a_1, ..., d' | a_n) \Rightarrow d' | d $ \\
\end{definition}
Замечание Наибольшие общие делители определены с точностью до ассоцированности \\
$ d' | d, d | d' \Rightarrow d \sim d' $ \\
Если R - обл цел и $ d \sim d' $ то $ d' = d \cdot \eps, \eps in R^*, d = d' \cdot \eps^{-1} d' | a_1, ..., d' | a_n, d'' | a_1, ..., d'' | a_n, d'' | d | d' $ \\
В $ \Z $ НОД можно выбрать для неотрицательных \\
НОД определён не во всяком кольце

\begin{theorem}
	R - О.Г.И. $ a_1, ..., a_n \in  R $ \\
	1. d - НОД $ (a_1, ..., a_n) $ \\
	2. $ (d) =  (a_1, ..., a_n) $ \\
	3. d - общий делитель, допускающий линейное представление \\
	т.е. $ \exists x_1, ..., x_n \in R, d = a_1x_1 + ... + a_nx_n $ \\
	\begin{proof}
		$ 2 \Rightarrow 3 \ \ $ $ (d) =  (a_1, ..., a_n)  $\\
		$ a_i \in  (a_1, ..., a_n)  = (d) = \{dr | r \in R \} $ \\
		$ a_i = dr_i, d|a_i $ \\
		$ d $ - общ делитель \\
		$ d \in (d) =  (a_1, ..., a_n) = \{ a_1x_1, + ... + a_nx_n \} $\\
		$ \Rightarrow \exists x_1, ..., x_n \in R, d = a_1x_1 + ... + a_nx_n $ \\
		$ 3 \Rightarrow 1 \ \ d - $ общий делитель \\
		$ d' - $ другой общий делитель \\
		$   d = a_1x_1 + ... + a_nx_n, a_1 | d', a_n | d' \Rightarrow d' | d \Rightarrow d - $ НОД \\
		$ 1 \Rightarrow 2  \ \  d - $ НОД $  (a_1, ..., a_n) $ \\
		$ I =  (a_1, ..., a_n) $ \\
		$ \exists m \in R : I = (m) $ \\
		$ ? m \sim d $ \\
		$ m = a_1x_1 + ... + a_nx_n $ \\
		$ d | m $ \\
		$ a_i \in m, m | a_i, m - $ общ делитель $ \Rightarrow m | d $ \\
		$ m \sim d, R - $ О.Ц.  \\
		$ \exists \eps \in R^*, m = d\eps, d = m\eps^{-1}, (m) = (d) $ \\
		$ (d) =  (a_1, ..., a_n) $ \\
	\end{proof}
\end{theorem}

Следствие 1 $ R - $ ОГИ \\
$ \forall a_1, ..., a_n \in R \exists $ НОД $  (a_1, ..., a_n) $\\
\begin{proof}
	$  (a_1, ..., a_n)  = (d), d -$ НОД $  (a_1, ..., a_n) $ \\
\end{proof}
Следствие 2 $ R $- ОГИ \\
$ d =$ НОД $  (a_1, ..., a_n) $\\
$ \Rightarrow d $ допускает представление $ d = a_1x_1, ..., a_nx_n $ \\

\Subsection{Взаимо простые элементы}

$ R - $ комм кольцо с 1, Обл.цел. \\
$ a_1, ..., a_n $ - вз. просты, если \\
НОД $  (a_1, ..., a_n) = 1$\\
Следует различать взаимно простые и попарно взаимно простые \\
\begin{theorem}
	$ R $ - ОГИ, $ a_1, ..., a_n \in R $ \\
	$  (a_1, ..., a_n) $ вз просты $ \Leftrightarrow $ \\
	$ \exists x_1, ..., x_n $ \\
	$ \Rightarrow 1 = a_1x_1 + ... + a_nx_n $ \\
	Д-во $ 1 = $ НОД $  (a_1, ..., a_n) $ \\
	По сл.2 к пред теореме 1 допускает лин. представление \\
	$ \Leftarrow 1 =  a_1x_1, ..., a_nx_n $ \\
	$ 1 | a_1, ..., 1 | a_n -$ общ делител, по пред теореме 1 = НОД 
\end{theorem}
\begin{theorem}
	R - ОГИ, a, b - взаимно просты \\
	$ a | bc \Rightarrow a | c $ \\
	\begin{proof}
		$ \exists x, y, ax + by = 1, acx_{\divby a} + bcy_{\divby a} = c_{\divby a} $
	\end{proof}
\end{theorem}

\Section{Евклидовы кольца}
R - область целостности \\
\begin{definition}
	R - евклидово кольцо, если \\
	$ \exists $ функция $ \lambda : R \setminus \{0\} \rightarrow N \cup \{0\} $ (евклидова норма), имеющая св-ва \\
	$ \exists a \in R, \forall b \in R \setminus \{0\} $ \\
	$ \exists a, r \in R, a = bq + r, $ и либо $ r = 0$ либо $ \lambda(r) , \lambda(b) $
\end{definition}
Пример $ \Z $ - евклидово \\
$ \lambda(n) = |n| $ \\
\begin{theorem}
	Теорема о делении с остатком в кольце многочленов \\
	R - коммут кольцо с 1 \\
	$ f, g \in R[x] $ \\
	$ n = \deg g, g = a_nx^n + ... + a_0, a_n \neq 0 $ \\
	Пусть $ \exists q, r \in R[x] $ \\
	$ f = g \cdot q = r $ и $ \deg r < \deg g $ \\
	\begin{proof}
		Индукция по степени многочлена $ m = \deg f $ \\
		Тогда $ m < n $ \\
		$ g = g \cdot 0 + f, \deg f = m < n = \deg g  $\\
		$ q = 0, r = f $ \\
		Переход $ m \geq n $ Предположим, что для всех многочленов степени < m теорема уже доказана  \\
		$ f = b_mx^m+ ... + b_0 $ \\
		$ f_1 = f - b_ma_n^{-1}x^{m-n}g $ \\
		% pic1
		$ \deg f_1 < \deg f $ \\
		По инд. предположению $ f_1 g \cdot q_1 + r_1 $ \\
		$ f = f_1 + b_ma_n^{-1} x ^{m-1}g $ \\
		$ = g(q_1 + b _ma_n^{-1} x^{m-n}) + r_1 $\\
		$ \deg r = \deg r_1 < \deg g $ 
	\end{proof}
\end{theorem}
Сл. 1 К - поле $ \Rightarrow $ в $ K(x) $ выполнена теорема о делении с остатком \\
$ \forall f,g \in K[x], g \neq 0, \exists q, r : f = gq+r, \deg r < \deg g $\\
Сл.2 - К - поле, К[x] - евклидово \\
$ \lambda(g) = \deg g $ \\
Пример $ \Z[i] = \{a = bi| a, b \in \Z \} $\\
$ \lambda(a + bi) = a^2 + b^2 $ \\
\begin{proof}
	$ z_1 = a + b_i, z_2 = c + d_i, z_2 \neq 0 $\\
	$ z_1 = z_2q + r \ \ |r|^2 < |z_2|^2 $ \\
	$ \dfrac{z_1}{z_2} = \dfrac{z_1 \overline{ z_2}}{|z_2|^2} \in \Q(i) $\\
	$ \dfrac{z_1}{z_2} = a + b_i, a, b \in \Q $ \\
	$ x \in \R, < x > $ - ближайшее целое число к x, $ | x - <x> | \leq \dfrac{1}{2} $ \\
	$ q = <a> + < b> i \in \Z[i] $ \\
	$ r = z_1 - z_2 q \in \Z[i] $ \\
	$ \lambda(r) = |r|^2 = |z_1 - z_2 \cdot q |^2 = |z_2|^2 \left| \dfrac{z_1}{z_2} - q \right|^2 $ \\
	%pic2
\end{proof}
\begin{theorem}
	R - евклидово $ \Rightarrow R - $ ОГИ \\
	\begin{proof}
		I - идеал в R \\
		$ I = \{0\} = (0) $ \\
		$ I \neq 0 $ \\
		$ < = \{ \lambda(a) a \in I \setminus \{0\} \} \neq 0 $ \\
		$ < \subseteq \N \cup \{0\} $ \\
		Выберем m - наименьший в L\\
		$ \exists b \in I \setminus \{0\} \lambda(b) = m $ \\
		$ I = (b)  \ \ b \in I, (b) \subseteq I $ \\
		$ a \in I, a = bq+r, r = 0, $ или $ \lambda (r) < \lambda (b) $ \\
		$ r = a - bq \in I $ \\
		Если $ r= 0, a = bq \in (b) $ \\
		Если $ r \neq 0, r \in I \setminus \{0\}$ \\
		$ \lambda(r) \subset \lambda(b) = m$ \\
		Невозможно в силу минимальности m 
	\end{proof}
\end{theorem}
\Subsection{Алгоритм Евклида}
$ a, b \in R, R - $ Евклидово \\
$ b \neq 0 $ \\
$ r_0 = a, r_1 = b $ \\
$ r_0 = r_1q_1 + r_2, \lambda(r_2) < \lambda(r_1) $ \\
$ r_{i-1} = r_i q_i + r_{i+1} \ \ \lambda (r_{i-1}) <  \lambda(r_i) $\\
...\\
$ r_{n-2} = r_{n-1} q_{n-1} + r_n $ - последний ненулевой остаток \\
$ r_{n-1} = r_nq_n $\\
\begin{lemma}
	НОД $ (r_{i-1}, r_i) = $НОД $( r_i, r_i+1) $ \\
	Достаточно д-ть, что $ (r_{i-1}, r_i) = ( r_i, r_i+1) $ \\
	$ r_i \in (r_{i-1}, r_i) $\\
	$ r_{i+1} = $ % pic4
\end{lemma}
$ (r_0, r_1) = (r_1, r_2) = ... = (r_{n-1}, r_n) = (r_nq_n, r_n) = (r_n) $\\
Алгоритм завершает свою работу если $ \lambda (r_1) \subset \lambda (r_2) \subset ... \subset \lambda (r_n) \geq 0 $ \\
Обратный ход алгоритма Евклида и линейное представление  \\
$ d = r_n $ \\
$ d = r_ix_i+r_{i+1}y_i $ \\
$ i = n-2, ..., 0 $ \\
%pic5















	\Section{Производная}

\begin{definition}
	Пусть $ f : <a, b> \rightarrow \R $ \\
	f называется дифференцируемой в $ x_0 \in <a, b> $ \\
	если $ \exists k \in \R $ \\
	$ f(x) = f(x_0) + k(x - x_0) + o(x - x_0) $ при $ x \rightarrow x_0 $\\
	k - производная f в точке $ x_0 $
	\begin{lemma}
		Если f дифф в $ x_0 $ то $ f'(x_0) $ корректно определена 
		\begin{proof}
			$ f(x) = f(x_0) + k(x - x_0) + o(x - x_0) $ \\
			$ = f(x_0) + \tilde{k}(x - x_0) + o(x - x_0) $ \\
			$ \Rightarrow k(x-x_0) + o(x-x_0) = \tilde{k} (x -x_0) + o(x-x_0) $\\
			$ \Rightarrow k+\dfrac{o(x-x_0)}{x - x_0} = \tilde{k} + \dfrac{o(x-x_0)}{x-x_0} $ 
		\end{proof}$ \lim\lim\limits_{y \rightarrow y_0} g(y) = A $\\
	\end{lemma}
\end{definition}
\begin{definition}
	Произв $ f<a,b> \rightarrow \R $ в $ x_o \in <a,b> $ \\
	наз $ \lim\limits_{x \rightarrow x_0} \dfrac{f(x) - f(x_0)}{x - x_0} = f'(x_0) $ \\
	Если $ f'(x_0) \in \R$, то дифф 
\end{definition}

Предложение. Пусть $ f : <a, b> \rightarrow \R, x_0 \in <a, b> $ Тогда f дифф-ма в $x_0$ если и только если существует ф-ция $ \phi : <a,b> \rightarrow \R $ непрерывная в $x_0$ т.ч. $ f(x) - f(x_0) = \phi(x) (x-x_0), x \in <a,b> $ \\
$ f'(x_0) = \phi(x_0) $\\
\begin{proof} 
	$\Rightarrow$ предположим $ \phi(x) = \left\{ \begin{array}{ll}
		 \dfrac{f(x) - f(x_0)}{x - x_0}, x \neq x_0 \\
		 f'(x_0)
	\end{array} \right. $ \\
	Опр. 2 $ \phi(x) \rightarrow_{x - x_0} f'(x_0) = \phi(x_0) $\\
	То $\phi$ непрерывна в $x_0$ \\
	$\Leftarrow \phi$ непрерывна в $ x_0 \Leftrightarrow \phi(x) \rightarrow_{x - x_0} \phi(x_0) \Leftrightarrow \phi(x)  = \phi(x_0) + o(1)_{\text{при} \ x \rightarrow x_0} $ \\
	$ f(x) - f(x_0) = \phi(x_0) (x - x_0) + o(x-x_0) \Rightarrow f $ дифференци в $ x_0 $ и $phi(x_0) = f'(x_0) $    
\end{proof} 

\begin{definition}
	Говорят, что $ f'(x_0) = +\infty (-\infty) $, если \\
	$ \lim\limits_{x \rightarrow x_0} \dfrac{f(x) - f(x_0)}{x - x_0} =  +\infty (-\infty) $
\end{definition}

\begin{example}
	$ f(x) = \sqrt[3]{x}, x \in \R $ \\
	$ \lim\limits_{x \rightarrow 0} \dfrac{\sqrt[3]{x}}{x} =  \lim\limits_{x \rightarrow 0} \dfrac{1}{x^{2/3}} = +\infty $ \\
	Т.е. $ f'(x) = +\infty $ 
\end{example}

\begin{definition}
	$ f_+' (x_0) = \lim\limits_{x \rightarrow x_0+} \dfrac{f(x) - f(x_0)}{x - x_0} $ - правая производная \\
	$ f_-' (x_0) = \lim\limits_{x \rightarrow x_0-} \dfrac{f(x) - f(x_0)}{x - x_0} $ - левая производная \\
\end{definition}

Очевидно, пусть $ x_0 \in (a, b) $ \\ %pic1
Тогда  $ \exists f'(x_0) \Leftrightarrow \left\{ \begin{array}{ll}
	
\end{array} \right. $
\begin{example}
	1. $ f(x) = x^2 $ \\
	$ f(x) - f(x_0) = x^2 - x_0^2 = (x-x_0)(x+x_0) $ \\
	$  \dfrac{f(x) - f(x_0)}{x - x_0} =_{x \neq x_0} x + x_0 \rightarrow_{x \rightarrow x_0} 2x_0  $\\
	$ f'(x_0) = 2x_0 $ \\
	2. $ f(x) = \sin(x) $ \\
	$ \sin(x) = \sin(x_0 + (x - x_0))  = \sin x_0 + \cos(x - x_0) + \cos(x_0) \sin(x-x_0) $ \\
	$ f(x) - f(x_0) = \underbrace{\sin(x_0) \cdot (\cos(x-x_0) - 1)}_{o(x-x_0)} + cos(x_0) (x- x_0) + o(x-x_0)  = \cos x_0 (x - x_0) + o(x - x_0) $ \\
	$ f'(x) = cos(x_0) $ 
\end{example}

\begin{properties}
	f - дифф-ма в $x_0 \Rightarrow f \ $ непрерывна в $x_0 $ \\
	$ f(x) = f(x_0) + \phi(x_0) (x - x_0), \phi $ непрерывна в $x_0 \Rightarrow $ f непрерывна в $x_0$ \\
	\begin{example}
		1. $ f(x) = |x| $ \\
		$ \dfrac{f(x) - 0}{x - 0} = \left\{ \right. $ \\
		$ f_+' (0) = 1, f_-' (0) = -1 $ \\
		2. $ f(x \neq 0) = x + \sin \dfrac{1}{x}, f(0) = 0, \lim\limits_{x \rightarrow 0} f(x) = 0 \Rightarrow f  $ непрерывна в 0 \\
		$ \dfrac{f(x) - f(0)}{x-0} = \dfrac{x \sin \dfrac{1}{x}}{x} =_{x>0} \sin \dfrac{1}{x} (x_{\pi} = \dfrac{1}{\pi n}, f(x_{\pi}) = 0, \tilde{x_{\pi}} =  \dfrac{1}{\pi n + \frac{\pi}{2}}, f(  \tilde{x_{\pi}} ) = 1  ) $ не имеет предела
	\end{example}   
\end{properties}


\Subsection{Физический и геометрический смысл производной}

$ x = f(t) $ \\
$ \dfrac{f(t) - f(t_0)}{t - t_0} $ - средняя скорость \\
$ \lim\limits_{x \rightarrow x_0} \dfrac{f(x) - f(x_0)}{x - x_0} = v(t_0) = f'(t_0) $ \\
%pic2

\Subsection{Вычисление производных}

Предложение: пусть $ f, g < a, b> \rightarrow \R, x_0 \in <a,b>, f,g \text{дифф-мы в} x_0  $ \\
1. $ (\alpha f + \beta g)'(x_0) = \alpha f'(x_0) + \beta g'(x_0), \alpha, \beta \in \R  $ \\
2.$ (fg)'(x_0) = f'(x_0) g(x_0) + f(x_0) g'(x_0) $ \\
3. $ g(x_0) \neq 0, \text{тогда} \left( \dfrac{f}{g} \right)' (x_0) = \dfrac{ f'(x_0) g(x_0) - f(x_0) g'(x_0)}{g(x_0)^2} $ \\
\begin{proof}
	1. $ \lim \dfrac{  \alpha (f(x) - f(x_0)) + \beta (g(x) - g(x_0))}{x - x_0} = \alpha f'(x_0) + \beta g'(x_0) $ \\
	2. $ f(x) g(x) - f(x_0) g(x_0)  = f(x) g(x) - f(x_0) g(x) + f(x_0) g(x) - f(x_0) g(x_0) = (f(x) - f(x_0)) g(x) + f(x_0) (g(x) - g(x_0))$\\
	$ \lim\limits_{x \rightarrow 0} \dfrac{f(x)g(x) - f(x_0) g(x_0) }{x-x_0} = \lim\limits_{x \rightarrow 0} ( \dfrac{f(x) - f(x_0)}{x-x_0} \cdot g(x) ) + f(x_0) \lim\limits_{x \rightarrow 0} \dfrac{g(x) - g(x_0)}{x - x_0} = f'(x_0) g(x_0) + f(x_0)g'(x_0) $ \\
	3. $ (\dfrac{1}{g})'(x_0) = \lim\limits_{x \rightarrow 0} \dfrac{\frac{1}{g(x)} - \frac{1}{g(x_0)}}{x - x_0} = \lim\limits_{x \rightarrow 0} \dfrac{g(x_0) - g(x)}{(x-x_0) \underbrace{g(x_0) g(x)}_{g(x_0)^2} } = -\dfrac{g'(x_0)}{g(x_0)^2} $ \\
	$ (\dfrac{f}{g})' (x_0) = (f \cdot \dfrac{1}{g})' (x_0) = \dfrac{ f'(x_0) g(x_0) - f(x_0) g'(x_0)}{g(x_0)^2} $ 
\end{proof}

\begin{consequence}
	Пусть $ f(x) = a_n x^n + ... + a_1 x + a_0 $ \\
	Тогда $ f'(x_0) = na_n x_0^{n-1} + ... + 2 a_2 x_0 + a_1 $ \\
	Проверим $ (x^n)' = nx^{n-1} $ \\
	Очевидно верно при n = 1 \\
	$  (x^n)' = (x^{n-1} x )' = (x^{n-1})' x + x^{n-1} x' $ \\
	$ = (n-1) x^{n-1} x x^{n-1} \cdot 1 = nx^{n-1} $   
\end{consequence}

Предложение \\
Пусть $ f : <a,b> \rightarrow \R, g : <c, d> \rightarrow \R $ \\
$ f $ дифф-ма в $ x_0 \in <a,b> $ \\
$ f(<a,b>) \in <c,d> $ \\
$ g \text{дифф-ма в} f(x_0) $ \\
Тогда $ g \circ f $ дифф-ма в $ x_0 $ \\
$ (g \circ f)' (x_0) = g'(f(x_0)) \cdot f'(x_0) $ \\
$ f(x) - f(x_0) =(x-x_0) \cdot \phi(x), \phi $ непрерывн в $x_0$ \\
$ g(y) - g(y_0) = (y - y_0) \cdot \psi (y), \psi $ непрерывн в $ y_0 $ \\
$ g(f(x)) - g(f(x_0)) = ( f(x) - f(x_0) ) \cdot \psi(f(x)) = (x-x_0) \cdot \phi(x) \cdot \psi(f(x)), x \in <a,b> $ \\
$ f(x) $ непр в x \\
$ \psi(f(x)) $ непр в x \\
$ g \circ f $ дифф в $ x_0$ \\
$ (g \circ f)'(x_0) = \phi(x_0) \cdot \psi(f(x_0)) = f'(x_0) \cdot g'(f(x_0)) $ \\

$ (\sin(x^2)) = \cos(x^2) \cdot 2x $

% pic3 

$ f^{-1} = g $ \\
$ g \circ f = id $ \\
$ g'(f(x_0)) \cdot f'(x_0) = 1 $ \\
$ (f^{-1})' (f(x_0)) = \dfrac{1}{f'(x_0)} $\\

 






 
	\begin{theorem}
	Пусть $ f : <a, b> \rightarrow \R $, строго монотонна и непрерывна \\
	$ x_0 \in <a,b> $ \\
	$ f $ дифф-ма в $ x_0,  f'(x_0) \neq 0 $ \\
	Тогда обратная к f ф-я дифф-ма в $ f(x_0) $ \\
	$ (f^{-1})'(f(x_0)) = \dfrac{1}{f'(x_0)} $ \\
	\begin{proof}
		$ f(x) - f(x_0) = \phi(x) \cdot (x - x_0) \ \ \phi $ непр в $ x_0 $ \\
		Из монотонности $ \phi(x) \neq 0 $ при $ x \neq x_0 $ \\
		$ \phi(x_0) = f'(x_0) \neq 0 $
		$ \dfrac{1}{\phi(x)} $ опр на $ < a, b> $ и непрерывн в $ x_0 $ \\
		$ g = f^{-1}, y = f(x), y_0 = f(x_0) $ \\
		$ y - y_0 = f(x) - f(x_0) = \phi(x) (x - x_0) = \phi(x)( g(y) - g(y_0) ) \Rightarrow g(y) - g(y_0) = \dfrac{1}{\phi(x)} \cdot (y - y_0) $\\
		$ \phi \circ g $ непрерывн в $ y_0 $ т.к. g непр в $y_0$, $ \phi $ непрерывн в $ g(y_0) = x_0 $ \\
		$ \Rightarrow g $ дифф в $y_0, g'(y_0) = \dfrac{1}{\phi(g(y_0))} = \dfrac{1}{f'(x_0)}$ \\ 
  	\end{proof}
\end{theorem}

\subsection{Производные элементарных ф-ций}

Предложение: Пусть $ a > 0, f(x) = a^x $ \\
Тогда $ f'(x) = \ln a \cdot a^x $ \\
\begin{proof}
	$ x_0 \in \R $ \\
	$ \dfrac{f(x) - f(x_0)}{x - x_0} = \dfrac{a^x - a^{x_0}}{x - x_0} = a^{x_0} \cdot \dfrac{a^{x-x_0} - 1}{x - 1} \rightarrow_{x \rightarrow x_0} a^{x_0} \cdot \ln a $
\end{proof}


$\begin{array}{|c|c|}
\hline
f(x)	& f'(x)  \\ 
\hline
x	& 1 \\ 
\sin x	& \cos x \\ 
a^x	& \ln a \cdot a^x \\ 
	\hline 
x^n	& nx^{n-1}  \\ 
\log_a x & \dfrac{1}{x \cdot \ln a}  \\
\hline
\end{array} $ $ a > 0, a \neq 1, \log_a : (0, +\infty) \rightarrow \R, \log_a' (a^{x_0}) = \dfrac{1}{\ln a \cdot a^{x_0}}, \log_a (x) = \dfrac{1}{x \ln a}, x > 0$ \\
$ f(x) = x^p = e^{\ln(x^p)} = e^{p \ln x} $\\
$ f'(x) = e^{p \ln x} \cdot \dfrac{p}{x} = x^p \cdot \dfrac{p}{x} = p \cdot x^{p-1} $\\
$ (\cos x)' = \cos (x + \dfrac{\pi}{2} \cdot (x + \dfrac{\pi}{2}))' = -\sin x $ \\
$ (\tan x)' = (\dfrac{\sin x}{\cos x}) = \dfrac{1}{(\cos x)^2} $ \\
$ (\cot x)' = (\dfrac{\cos x}{\sin x}) = \dfrac{-\sin^2 x - \cos^2 x}{\sin^2 x} = \dfrac{-1}{\sin^2 x}  $\\
$ (\arcsin(x))' = \dfrac{1}{\cos(\arcsin(x))} = \dfrac{1}{\sqrt{1 - x^2}} $ \\
$ \sin \theta = x $ \\
$ (\cos \theta )^2 = 1 - x^2 q	$\\
$ (\arctan x)' = \cos^2 (\arctan x) $ \\
$ \tan \theta = x $ \\
$ (\cos \theta)^2 = \dfrac{1}{1 + x^2} $ \\

\begin{theorem}
	Теорема Ферма \\
	$ f : <a, b> \rightarrow \R $ \\
	Дифф-ма в $ x_0 \in (a, b) $ \\
	Если $ f(x_0) = \max f(x) $ или $ f(x_0) = \min f(x), $ то $ f'(x_0) = 0 $ \\
	\begin{proof}
		Можно считать $ f(x_0) = \max f(x) $ \\
		$ x > x_0, \dfrac{f(x) - f(x_0)}{x- x_0} \leq 0 \Rightarrow \lim\limits_{x \rightarrow x_0+} \dfrac{f(x) - f(x_0)}{x- x_0} \leq 0 $ \\
		$ x < x_0, \dfrac{f(x) - f(x_0)}{x- x_0} \geq 0 \Rightarrow \lim\lim\limits_{x \rightarrow x_0 -} \dfrac{f(x) - f(x_0)}{x- x_0} \geq 0 = f_-'(x_0) $ \\
		$ f_-'(x_0) = f_+'(x_0) \Rightarrow f'(x_0) = 0 $
	\end{proof}
\end{theorem}

\begin{theorem}
	Теорема Ролля \\
	$ f : [a, b] \rightarrow \R $ непр на $ [a, b] $, дифф на $ (a, b) $ \\
	Если $ f(a) = f(b),$ то $ \exists c \in (a, b) : f'(c) = 0 $ \\
	\begin{proof}
		По т. Вейерштрасса $ \exists p \in [a, b] : f(p) = \max_{x \in [a,b]} f(x) $ \\
		$ \exists q, f(q) = \min_{x \in [a,b]} f(x) $ \\
		1. $ p, q \in \{a, b\} \Rightarrow f(p) = f(q) \Rightarrow \forall x \in [a. b], f(x) = f(p) \Rightarrow \forall c \in [a, b] : f'(c) = 0$ \\
		2. $ p \in (a, b) $ или $ q \in (a, b) $ \\
		Тогда для $ c = p ( c = q ) $ вып условие т. Ферма $ \Rightarrow f'(c) = 0 $
	\end{proof}
	Замечание: нельзя отменить условие дифф-ти на $ (a,b) $ \\
	$ f(x) = |x| $ на $ [-1, 1] $
\end{theorem}

\begin{theorem}
	Теорема Лагранжа \\
	$ f : [a, b] \rightarrow \R $ непр на $ [a, b] $, дифф на $ (a, b) $ \\
	Тогда $ \exists c \in (a, b) $ т.ч. $ f(b) - f(a) = f'(c) (b - a) $ \\
	\begin{proof}
		$ g(x) = f(x) - \dfrac{f(b) - f(a)}{b -a } \cdot (x - a) $ \\
		$ g(a) = f(a) $ \\
		$ g(b) = f(b) -  \dfrac{f(b) - f(a)}{b -a } \cdot (b - a) = f(a) $\\
		По т. Ролля $ \exists c \in (a, b) : g'(c) = 0 $ \\
		$ g'(c) = f'(c) - \dfrac{f(b) - f(a)}{b -a } $\\
		$ \Rightarrow f'(c) = \dfrac{f(b) - f(a)}{b -a }  $ 
	\end{proof}
	Следствие Пусть $ f : [a, b) \rightarrow \R $ непр, дифф на $ (a, b) $ \\
	$ \forall x \in (a, b) m \leq f'(x) \leq M $\\
	Тогда $ m \leq \dfrac{f(b) - f(a)}{b -a} \leq M $
\end{theorem}

\begin{theorem}
	Теорема Коши \\
	Пусть $  f,g : [a, b] \rightarrow \R $ непр на $ [a, b] $, дифф на $ (a, b), f'(g) \neq 0 \forall x \in (a, b) $ \\ 
	Тогда $ \exists c \in (a, b) : \dfrac{f(b) - f(a)}{g(b) - g(a)} = \dfrac{f'(c)}{g'(c)} $ \\
	\begin{proof}
		$ g(b) \neq g(a) $ ввиду теоремы Ролля \\
		$ h(x) = f(x) - \dfrac{f(b) - f(a)}{g(b) - g(a)} \cdot (g(x) - g(a)) $ \\
		$ h(a) = f(a) = h(b) $\\
		$ h(x) $ непр на $ [a, b] $, дифф на $ (a, b) $ \\ 
		По т. Ролля $ \exists c \in (a, b) : h'(c) = 0 $ \\
		$ h'(c) = f'(c) - \dfrac{f(b) - f(a)}{g(b) - g(a)} \cdot g'(c)  $ \\
		$ f'(c) = \dfrac{f(b) - f(a)}{g(b) - g(a)} \cdot g'(c) $ \\
		$ \dfrac{f(b) - f(a)}{g(b) - g(a)} = \dfrac{f'(c)}{g'(c)} $
	\end{proof}
\end{theorem}
Предл:  $ f : <a, b> \rightarrow \R $ непр на $ <a, b> $, дифф на $ (a, b) $ \\
Тогда \\
1. $ f $ возрастает на $ <a, b> \Leftrightarrow f'(x) \geq 0 \forall x \in (a, b) $ \\
2. $ f'(x) > 0 $ на $ (a, b) \Rightarrow f $ строго возрастает на $ <a, b> $ \\
\begin{proof}
	1. $ \Rightarrow $ Пусть $ x_0 \in (a, b) $ \\
	$ x > x_0 \ \dfrac{f(x) - f(x_0)}{x - x_0} \geq 0 $\\
	$ \Rightarrow $ по т. о предельном переходе в нер-ве $ f'(x_0) \geq 0 $ \\
	%$ x < x_0 \ \dfrac{f(x) - f(x_0)}{x - x_0} \geq 0 
	1. $ \Leftarrow, 2 \ x_0, x_1 \in <a,b> , x_0 < x_1 $ \\
	$ [x_0, x_1] \subset <a,b> $ \\
	f непр на $ [x_0, x_1], $ дифф на $ (x_0, x_1) $ \\
	По т. Лагранжа $ \exists c \in (x_0, x_1) : f(x_1) - f(x_0) = f'(c) (x_1 - x_0) $ \\
	сл. 1 $ f(x_1) - f(x_0) \geq 0 $ \\ 
	сл. 2 $ f(x_1) - f(x_0) > 0 $ \\ 
\end{proof}

Сл. 1 $ f : <a, b> \rightarrow \R $, непр, дифф на $(a,b) $\\
Тогда $ f $ пост на $ <a,b> \Leftrightarrow f'(c) = 0 \forall c \in (a,b) $ \\
Сл. 2 Пусть $ f, g : [a, b> \rightarrow \R, b \in \R \cup \{+\infty \} $\\
непр на $ [a, b> $ дифф на $ (a,b)$\\
Предпол, $ f(a) = g(a) $ и $ \forall x \in (a,b) : f'(x) < g(x) $ \\
Тогда $ f(x) < g(x) \forall x \in (a,b) $ \\
\begin{proof}
	$ h(x) = g(x) - f(x) $ \\
	$ h'(x) = g'(x) - f'(x) > 0, \forall x \in (a,b) $\\
	h строго возр на $ [a, b) $ (при $b = +\infty $) $ b_0 > a, b_0 \in \R \Rightarrow $ на $ [a, +\infty] $ \\
	$ h(a) = 0 $ \\
	$ \Rightarrow \forall x \in (a,b) ; h(x) > 0 $ \\
	т.е. $ g(h) > f(x) $ 
\end{proof}  

\begin{theorem}
	Теорема Дарбу
	Пусть $ f : [a,b] \rightarrow \R, $ дифф-ма на $ [a, b] $ \\
	Тогда $ \forall t: f'(a) < t < f'(b) $ \\
	Найдётся $ c \in (a, b) : f'(c) = t $ \\
	\begin{proof}
		Пусть сперва $ t = 0 $ \\
		f - непрерывна $ \Rightarrow \exists c \in [a,b] : f(c) = \min_{x \in [a,b]} f(x)  $ \\
		Если $ c \in (a,b) $ то по т. Ферма $f'(c) = 0$\\
		При $ c = a \ \ 0 = t > f'(a) = \lim\lim\limits_{x \rightarrow a_+} \dfrac{f(x) - f(a)}{x - a} \geq 0  $ \\
		Случай $ c = a $ невозможен, аналогично невозможен $с = b$ \\
		Общий сл. $ g(x) = f(x) - tx $ \\
		$ g'(x) = f'(x) - t $ \\
		$ f'(a) - t (= g'(a)) < 0 < f'(b) - t = g'(b) $ \\
		По рассм случаю $ \exists c \in (a, b) : g'(c) = 0 \Rightarrow f'(c) = t $ \\		
	\end{proof}
\end{theorem}
Следств 1.  Пусть $ f : <a,b> \rightarrow \R,$ дифф, $ \forall x \in <a,b> : f'(x) \neq 0 $ \\
Тогда f - строго монотонна 
\begin{proof}
	Предположим не строго монотонная \\
	$ \exists c_1 \in (a, b) : f'(c_1) \leq 0 $ \\
	и $ \exists c_2 \in (a, b) : f'(c_2) \geq 0 $ \\
	$ \Rightarrow f'(c_1) < 0, f'(c_2) > 0 $ \\
	Применим т. Дарбу к f на $ [c_1, c_2] $ или $ [c_2, c_1] $ \\
	$ \Rightarrow c \in [c_1, c_2] : f'(c) = 0 $ - против условия
\end{proof} 

Сл. 2 Пусть $ f : <a,b> \rightarrow \R $ - дифф-ма \\
Тогда $ f'(<a,b>) $ - промежуток \\

\begin{proof}
	$ m = \inf_{c \in <a,b>} f'(c) , M = \sup_{c \in <a,b>} f'(c) $ \\
	Д.м. $ \forall t \in (m, M) \exists c \in <a,b>  f'(c) = t,  t < m  \Rightarrow \exists c_2 \in <a,b>  : f'(c_2) > t $ \\
	$ t > m \Rightarrow \exists c_1 \in <a,b> : f'(c_1) < t $ \\
	По т. Дарбу $ \exists c \in (c_1, c_2), f'(c) = t $ \\
	То $ f'(<a,b>) \supset (m, M) \Rightarrow f'(<a,b>) = (m, M] $ или $ [m, M] $ 
\end{proof}

Пример: нер-во Бернулли \\
$ (1+x)^p \geq 1 + px, p > 1, x > -1 $ \\
$ f(x) = (1+x)^p, g(x) = 1 + px $ \\
$ x \geq 0 $ \\
$ x = 0, f(0) = g(0)  $ \\
$ f'(x) = p(1+x)^{p-1}, g'(x) = p $\\
$ f'(x) > g'(x) \Rightarrow \forall x > 0: f(x) > g(x) $ \\
$ x < 0 $ \\
$ -1 < x < 0, f'(x) < g'(x) \Rightarrow f(x) > g(x) $ при $ x < 0 $ \\

Пример 2 $ \sin x < x, \forall x > 0 $ \\
$ \cos x = \cos^2 \dfrac{x}{2} - \sin^2 \dfrac{x}{2} = 1 - 2\sin^2 \dfrac{x}{2} > 1 - 2 \left(\dfrac{x}{2}\right)^2 = 1 - \dfrac{x^2}{2} $ \\
$ \sin x > x - \dfrac{x^3}{6}, x > 0 $ \\
$ f(0) = g(0) = 0 $ \\
$ f'(x) = \cos x $ \\
$ g'(x) = 1 - \dfrac{x^2}{2} $ \\
$ f'(x) > g'(x) $ \\
$ \forall x > 0 \Rightarrow \forall x > 0\sin x > x - \dfrac{x^3}{6} $ \\



\Section{Правило Лопиталя}

\begin{theorem} (Правило Лопиталя для $ \dfrac{\infty}{\infty} $ )\\
	Пусть $ -\infty \leq a < b \leq +\infty $ \\
	$ f, g (a, b) \rightarrow \R $ дифф-мы \\
	$ g'(x) \neq 0 $ при всех $ x \in (a, b) $\\
	$ \lim\limits_{x \rightarrow a} g(x) = \infty $ \\
	Если  $ \lim\limits_{x \rightarrow a} \dfrac{f'(x)}{g'(x)} = l \in \overline{R}, $ то $ \lim \dfrac{f(x)}{g(x)} = l $ 
	\begin{proof}
		$ x_n \rightarrow a, x_1 > x_2 > ... $ \\
		$ \forall x \in (a,b) : g'(x) \neq 0 \Rightarrow g $ строго монотонная \\
		$ \Rightarrow g(x_n) $ - строго монотонная посл \\
		$ \Rightarrow g(x_n) $ одного знака при дост. больших n \\
		$ \Rightarrow g(x_n) \rightarrow +\infty $ или $  g(x_n) \rightarrow -\infty $ \\
		$ \dfrac{f(x_{n+1}) - f(x_n)}{g(x_{n+1}) - g(x_n)} = \dfrac{f'(c_n)}{g'(c_n)} $ для нек $ c_n \in (x_n+1, x_n) $ по т. Коши, $ c_n \rightarrow a \Rightarrow \dfrac{f'(c_n)}{g'(c_n)} \rightarrow l $ \\
		По т. Штольца $ \dfrac{f(x_n)}{g(x_n)} \rightarrow l $ \\
		По Гейне $ \lim\limits_{x \rightarrow b} \dfrac{f(x)}{g(x)} = l$
	\end{proof}
\end{theorem}

\begin{theorem}(Правило Лопиталя для $ \dfrac{0}{0} $ )\\
	Пусть $ -\infty \leq a < b \leq +\infty $ \\
	$ f, g (a, b) \rightarrow \R $ дифф-мы \\
	$ g'(x) \neq 0 $ при всех $ x \in (a, b) $\\
	$ \lim\limits_{x \rightarrow a} f(x) = \lim\limits_{x \rightarrow a} g(x) = 0 $ \\
	Если  $ \lim\limits_{x \rightarrow a} \dfrac{f'(x)}{g'(x)} = l \in \overline{R}, $ то $ \lim \dfrac{f(x)}{g(x)} = l $ 
\end{theorem}

Пример: Докажем, что $ \dfrac{\ln x}{x^p} \rightarrow_{x \rightarrow \infty}  0 \forall p > 0 $ \\
$ f'(x) = \dfrac{1}{x}, g'(x) = px^{p-1} $ \\
$ \dfrac{f'(x)}{g'(x)} = \dfrac{1}{px^p} \rightarrow 0 \Rightarrow  \dfrac{\ln x}{x^p} \rightarrow_{x \rightarrow \infty} 0 $\\


















	\Subsection{Характеристика поля}
K - поле \\
$ 1+1+..+1, n \in \N$ \\
1 случай - $ \forall n \in N, 1+1+...+1 \neq 0 $ \\
В этом случае характеристика поля равна 0 \\
2 cлучай - $ \exists n \in N, 1+...+1 = 0 $ \\
Тогда хар-ка поля равна наименьшему такому n\\
\begin{lemma}
	Характеристика поля 0 или простое число \\
	$ 1+...+1 = (1+...+1)(1+...+1) $ \\
	$ n = ab $ \\
	В силу минимальности характеристики n не может быть составным.
\end{lemma}
char K = 0 : $ \Q, \R, \CC $ \\
char K = p : $ \Z / p\Z $ \\
char K = 2 : $ \F_2 = \Z /2\Z $ \\
K - поле, какое наименьшее подполе L содержит K \\
char K = 0 \\
$ 1 \in K \Rightarrow 1 \in L, 1+1+...+1 \in L $ \\
$ - (1+...+1) \in L $ \\
$ x \in K, n \in \N $ \\
$ n \cdot x = x+x+x...+x $ \\
$ \{ n\cdot 1, 0, -(n \cdot 1) \} \cong \Z  $ \\
$ \forall n \in \Z, \dfrac{1}{n\cdot 1} \in \Z $ \\
$ \forall m \in \Z, \forall n \in \Z \setminus 0, \dfrac{n\cdot 1}{n\cdot 1} \in L $ \\
$ \dfrac{m\cdot 1}{n \cdot 1} \cdot \dfrac{m'\cdot 1}{n'\cdot 1} = \dfrac{(mm') \cdot 1}{(nn')\cdot 1} $ \\
%pic1
char k = p > 0 \\
$ 0, 1, ..., (p-1)\cdot 1 $ \\
$ a \neq 0, a \cdot 1, (a,p) = 1 $ \\
$ (ab) \equiv 1 (p)$ \\
$ \exists b,c, ab $ и $ pc  = 1$ \\
$ b' $  -остаток от деления b на p \\
$ ab' \equiv 1(p) $ \\
$ (1+1+...+1)(1+1+...1)  =1+...+1 (1+pk) $ раз $ = 1 $ \\
$ \Rightarrow $ любой элемент обратим \\
char k = p \\
Найдём подполе в k изоморфно $ \Z / p\Z $ \\
$ n \cdot 1 \mapsto [n] $ \\
Вывод - всякое поле содержит подполе, изоморфное $ \Q $ либо $ \Z /p\Z $ и это поле однозначно определено характеристикой \\
\begin{definition}
	Простые поля: $ \Q, \Z/p\Z$ 
\end{definition}
char k = p, $ a \in k $ \\
$ pa = a+a+...+a = 0 $ \\
$ a+...+a = a(1+...+1) = a \cdot 0 = 0$ 
 
\Subsection{Производная многочлена}

K - поле, K[x]  \\
$ f = a_0+a_1x+...+a_nx^n $ \\
$ f' = a_1 + 2a_2x+... + ka_kx^{k-1} +... + na_nx^{n-1} $\\
$ f' = \sum_{k=1}^{n}k a_kx^{k-1} = \sum_{k=0}^{n} k a_k x^{k-1} $ \\
Св-ва производной \\
1. $ (f+g)' = f' + g' $ \\
2. $ c \in K, (c \cdot f)' = c \cdot f' $ \\
3. $ (fg)' = f'g + fg' $ \\
4. char k = 0 $ f' = 0 \Leftrightarrow f = c \in K $\\
char k = p >0 $ f'=0 \Leftrightarrow f \in K[x^p] $ \\
\begin{proof}
	1. $ f = \sum_{k=0}^{n} a_k x^{k} $ \\
	$ g=  \sum_{k=0}^{n}b_k x^{k} $ \\ 
	%pic2
	3. Верно для мономов \\
	Верно для монома и многочлена \\
	Верно для 2-х многочленовт\\
	%pic3-7
\end{proof}

\Subsection{Корни многочлена}
K - поле, $ f \in K[x] $ \\
$ c \in K$ \\
$ f = \sum_{k=1}^{n} a_kx^k $ \\
$ f(c) = \sum_{k=1}^{n} a_kc^k \in K $ \\
Значение f в точке c \\
\begin{definition}
	c - корень f если $f(c) = 0$ \\
\end{definition}
\begin{theorem}
	$ f(x) = (x-c)\cdot g(x) + f(c)$ \\
	\begin{proof}
		$ f(x) = (x-c) g(x) + r, r \in K $\\
		$ f(c) = (c - c) \cdot g(c) + r $ \\
		$ r= f(c)$
	\end{proof}
\end{theorem}
Сл. (теорема Безу) \\
$ c - $ корень из f $ \Leftrightarrow (x - c) | f $ \\
$ \Leftarrow f(x) = (x-c' \cdot g(x)) \Rightarrow f(c) = 0 \Rightarrow c $ корень \\
$ \Rightarrow f(x) = (x-c)\cdot g(x) + f(c)_{=0} = (x-c) \cdot g(x) $ \\
\begin{definition}
	c - корень из f кратности k \\
	Если $ (x-c)^k|f, a (x-c)^{k+1} \centernot| f $ \\
	%pic8
\end{definition}
Кратные множители \\
Осн теорема арифметики для $K[x]$\\
$ f \neq 0, f(x) = c \cdot \prod_{i} q_i $ \\
$ f(x) = c \cdot \prod_{i=1}^{n} q_i^{d_i} $ \\
$ q_i $ неприводимы, $ d_i $ - кратность многочлена $ q_i $\\
$ q_i $ - множ кратн d \\
$ q_i^{d_i} | f, q_i^{d_i+1} \centernot| f $ \\
\begin{theorem}
	q - непрерывн множитель f кратный d \\
	Пусть выполнено одно из условий \\
	1. char $K = 0$ \\
	2. char $K = p > 0, p \centernot| d, q' \neq 0$  \\
	Тогда q - множитель f' кратный $d-1$ \\
	\begin{proof}
		%pic9
	\end{proof}
\end{theorem}
Сл. с - корень f кратн d т либо char K = 0 либо char $K = p > 0, p \centernot| d $ \\
Тогда с - корень f' кратн d-1 \\
Пример $ \Z / 2\Z  $ \\
$ f(x) = x^2 + x^5 $ 0 - корень кратности 2 \\
$ f' = x^4 $  0 - корень кратности 4 \\
char $ K = 2 | d $ \\
Замечание Существуют поля K, char K = p \\
т.ч. $ \exists g \in K[x^n], $ по g неприводим \\
(конечное не годится для этого примера) \\
char K = 0 \\
$ f \in K[x] $\\
Найти g, мн-во неприводимых делителей f и g совпадают по g не имеют кратных множителей \\
%pic10
\Subsection{Число корней многочлена. Формальное функциональное равенство мночленов}

\begin{theorem}
	$ 0 \neq f \in K[x] $\\
	Число корней f  с учётом их кратности $ \leq $ deg f \\
	\begin{proof}
		$ f = c \prod_{i=1}^{n} (x - c_i)^{d_i} \prod_{q \text{неприводим}, deg q_i \geq 2} q_j^{c_j} $ \\
		$ d_1 + ... +d_m \leq deg f $ 
	\end{proof} 
\end{theorem}
Сл. 1 $ f \in K[x] $ \\
deg f = n \\
$ \exists c_1 ... c_m, m > n $ \\
$ f(c_1) = ... = f(c_m) = 0 $ \\
$ \Rightarrow f = 0 $ \\

$ f \in K[x]  $ \\
$ \phi_f : K \rightarrow K $ \\
$ c \mapsto f(c) $ \\
\begin{definition}
	f и g фукционально равны если $ \phi_f = \phi_g $ \\
	$ f =o g $ \\
\end{definition}
Пример $ \F_2 $ \\
$ f = 0, g = x^2 +x $ \\
$ \forall x \in \{1,0\} $ \\
$ \phi_f (x) = 0, \phi_g(0)= 0, \phi_g(1) = 0 $ \\
\begin{theorem}
	$ |K| > \max \{ \deg f, \deg g \} $ \\
	Если $ f =o g$ то $ f = g $
	\begin{proof}
		$ f =o g, f - g =o 0 $ \\
		$ \Rightarrow \forall c \in K, c $ - корень f - g \\
		$ \Rightarrow \deg (f-g) \leq \max ( \deg f, \deg g ) $ \\
		По сл. к теореме 1,$ f - g = 0, f = g $
	\end{proof}
\end{theorem}



















	\Subsection{Алгебраически замкнутые поля}
K - поле, $ x-c \in K[x] $ - неприводим \\
\begin{theorem}
	K - поле, $K[x]$ \\
	1. Всякий неприводимый $ f \in K[x] $ линейный \\
	2. Всякий $ f \in K[x], \deg f \geq 1$ , делится на линейный \\
	3. Всякий $ f \in K[x], \deg f \geq 1$ полностью раскладывается в произведение сомножителей \\
	4. Всякий $ f \in K[x], \deg f \geq 1 $ имеет в K хотя бы 1 корень \\
	5. Всякий $ f \in K[x], \deg f = n \geq 1 $ имеет в K ровно n корней с учётом кратных \\
	\begin{definition}
		Пусть для K выполняется любое из равносильных условий теоремы. Такое поле K называется алгебраически замкнутым.
	\end{definition}
	\begin{proof}
		$ 1 \Rightarrow 3 $ - т-ма о разложении на множители \\
		$ 3 \Rightarrow 1 $ - $ \deg f \geq 2 \Rightarrow f - $ составной \\
		$ 3 \Rightarrow 2 $ очевидно \\
		$ 5 \Rightarrow 4 $ очевидно \\
		$ 3 \Rightarrow 5 - f(x) = c \prod_i (x-a)^{d_i}, \sum d_i = n = \deg f$ \\
		$ 4 \Rightarrow 2 $ - Теорема Безу \\
		$  2 \Rightarrow 3 $ Индукция по $ n = \deg f $ \\
		База: $n = 1 \ \ f = a(x-c) $ \\
		Переход: $ n, \deg f = n, n > 1 $ \\
		$ f(x) = (x-c) g(x)  \ \deg g = n - 1 \geq 1 $ \\
		$ g(x) a \prod (x - c_i)^{d_i} $ \\
		$ f(x) = a (x-c) \prod_i (x - c_i)^{d_i} $ 
	\end{proof}
\end{theorem}
\begin{theorem} (без док-ва) \\
	$ \CC $  - алгебраически замкнутое
\end{theorem}

\Subsection{Неприводимые многочлены над полем вещественных чисел} 

\begin{lemma}
	$ f \in \R [x] \subset \CC[x] $ \\
	$ z \in \CC \setminus \R $, z - корень f \\
	$ \Rightarrow \bar{z} $ - корень из z \\
	\begin{proof}
		$f(x) =  a_0 + a_1 x + ... + a_n x^n , a_i \in \R $ \\
		$ 0 =  a_0 + a_1 z + ... + a_n z^n , = f(x) $ \\
		$ 0 = \bar{0} = \overline{ a_0 + a_1 z + ... + a_n z^n } =  \bar{a}_0 + \bar{a}_1 \bar{z} + ... + \bar{a}_n (\bar{z})^n =  a_0 + a_1 \bar{z} + ... + a_n (\bar{z})^n  = f(\bar{z}) $ \\
		$ \bar{z} $ - корень f 
	\end{proof}
\end{lemma}
\begin{lemma}
	$ f \in \R [x], \deg f \geq 1 $ \\
	$ \Rightarrow f $ делится либо на линейный $ \in \R[x]$ либо на квадратичный $ \in \R[x] $, с $ D < 0 $ \\
	\begin{proof}
		$ f \in \CC[x] $ \\
		$ \exists $ комплексный корень z, $ f(z) = 0 $ \\
		1 сл. $ z \in \R $ \\
		По т-ме Безу $ f(x) = (x-z) \cdot g(x) $\\
		2 сл. $ z \in \CC \setminus \R $ \\
		%pic1,2
		Осталось показать, что $ g(x) \in \R[x] $ \\
		$ f(x) = h(x) \cdot g(x) $ \\
		Поделим f с остатком в $ R[x] $\\
		$ f(x) = h(x) \cdot \tilde{g}(x) + r(x) , \tilde{g}, r \in \R[x], \deg r \leq  1 $ \\
		$ f(x) = h(x) g(X) + r(x) $ в $ \CC[x] $\\
		По теореме о делении с остатком в $ \CC[x], g = \tilde{g}, r = 0, g \in \R[x] $
		%pic3 - единственность g 
	\end{proof}
\end{lemma}
\begin{theorem}
	Неприводимые в $ \R[x] $ это в точности  (до ассоцир) \\
	$ (x - c) $ и $ x^2 + ax + b, a^2 - 4b < 0 $ \\
	\begin{proof}
		1) Лин неприводим \\
		$ x^2 + ax + b, a^2 - 4b < 0 \Rightarrow $ нет веществ корней $ \Rightarrow $ нет веществ. линейных делителей $ \Rightarrow $ неприводим \\
		2) Пусть f не линейный и не отриц дискрим. \\
		По лемме 2 делится на линейный, либо на квадратичн с отриц. дискриминантом 
	\end{proof}
\end{theorem}
%pic 4,5,6
\Subsection{Ещё одна конструкция поля комплексных чисел}

$ \R[x]$ \\
$ x^2 + 1 $ неприводим в $ \R[x] $ \\
$ (x^2 + 1 )$ - максимальный идеал в $ \R[x] $\\
$ (x^2 + 1) \subseteq I \subseteq \R[x] $ \\
$ g | (x^2 + 1) \Rightarrow g = const \Rightarrow (g) = \R[x], g \sim x^2 + 1, \Rightarrow (g) = (x^2 + 1 ) $ \\
$ \R[x] / (x^2 + 1) $ - поле \\
$ [f] $ В каждом классе сравнимости есть единственн f, $ \deg r \leq 1 $ \\
$ f(x) = (x^2 + 1)g(x) + r(x), \deg r \leq 1  $ \\
$ [r] = [r_1], \deg r, \deg r_1 \leq 1 $ \
$ [r - r_1] = [0], (x^2 + 1) | (r - r_1) \Rightarrow r = r_1 $ \\
$  \R[x] / (x^2 + 1)  = \{ [bx + a] | a,b \in \R \} $ \\
$ [bx + a] + [dx + c] = [(b+d)x + (a+c) ] $ \\
$ [bx + a][dx+c] = [bdx^2 + (bc+ad)x+ac] = [bdx^2 + bd - bd + (bc+ad)x + ac]  = [bd(x^2+1) + (bc-ad)x + ac - bd] = [(ad - bc) x + ac - bd] $ \\
$ \CC \ \ \ \ \ \R[x] / (x^2 + 1) $ \\
$ x = a + bi \mapsto [a+bx] $ \\
$ \CC \cong \R[x] / (x^2 + 1) $ \\
$ i \mapsto [x] $ \\

Упр. $ x^2 + Ax + B, A^2 - 4B < 0 $ \\
$ \R[x] / (x^2 + Ax+B) \cong \CC $ \\

\Subsection{Интерполяционная задача}
$ K, x_1, ..., x_n \in K $ \\
$ y_1, ..., y_n \in K $ \\
Интерполяционная задача \\
Найти многочлен степени $ f \in K[x] $ \\
$ f(x_i) = y_i, i = 1, ..., n $ \\
Найти многочлен степени $ \leq n - 1 $ \\
Обобщённая интерполяционная задача \\
$ x_1, ..., x_n \in K $ \\
$ d_1, ..., d_n \in \N $ \\
$ d_i $ - количество условий(производных)\\
%pic7, 8
Зам. $ f \in K[x], char K = p $ \\
$ k \geq p, f^{(k)} = \sum_l \underbrace{l(l-1)...(l-k+1) }_{p \text{если } k \geq p}a_l x^{l-k} $ \\
Если char K = p,$ k \geq p $\\
$ f^{(k)}s = 0 $ \\
%pic&

%-------

1. Метод Лагранжа \\
$ \begin{array}{c|ccccc}
x & x_1 & x_{i-1} & x_i & x_{i+1} & x_n \\
\hline
0 & 0 & 0 & 1 & 0 & 0 
\end{array} $\\
$ L_i (x) = \dfrac{(x - x_i) (x - x_{i-1} ) (x - x_{i+1}) ... (x - x_n)}{ (x_i - x_1)...(x_i - x_{i-1}) (x_i - x_{i+1}) (x_i - x_n)} $ \\
$ f(x) = \sum_{i=1}^n y_i L_i = \sum_{i=1}^n f(x_i) L_i (x) $ \\
2. Метод Ньютона \\
Предполагаем, что знаем решение с k точками и добавляем ещё одну точку \\
$ \begin{array}{c|c}
x & x_1 \\
\hline 
f(x) & y_1
\end{array} \ \ \ f_1(x) = y_1 $ \\
$ \begin{array}{c|c}
x & x_1...x_k \\
\hline 
f(x) & y_1...y_k
\end{array} \ \ \ f_k (x) $ \\
$ \begin{array}{c|c}
x & x_1...x_{k+1} \\
\hline 
f(x) & y_1...y_{k+1}
\end{array} $ \\
$ f_{k+1} (x) = f_k(x) + a_k(x - x_1) ... (x - x_k) $ \\
$ y_{k+1} = f_{k+1} (x_{k+1}) = f_k(x_{k+1}) +  a_k(x_{k+1} - x_1) ... (x_{k+1} - x_k) $ \\
$ a_k = \dfrac{y_{k+1} - f_k(x_{k+1})}{(x_{k+1} - x_1) ... (x_{k+1} - x_k)} $ \\
$ \deg f_{k+1} \leq \max(\deg f_{k}, k) \leq \max (k-1, k) = k $ \\
















	$ f(x) \geq f(x_0) + f'(x) (x-x_0) $ \\
\begin{definition}
	Пусть f определена  в окр-ти $x_0$, дифф-ма в $x_0$ Говорят, что $x_0$  точка перегиба, если $ f(x) - f(x_0) - f'(x_0) (x-x_0) $ меняет знак при переходе через $x_0$ \\
\end{definition}
Предл. Необходимое усл-е точки перегиба \\
Пусть f дважды дифф-ма в $x_0$. Тогда $f''(x_0) = 0 $ \\
\begin{proof}
	В окр-ти $x_0$: \\
	$ f(x) = f(x_0) + f'(x_0)(x-x_0) + \dfrac{f''(x_0)}{2} (x - x_0)^2 + o((x-x_0)^2) $ \\
	Пусть $f''(x_0) \neq 0 $ напр $ f''(x_0) >0 $ \\
	$ f(x) - f(x_0) -f'(x_0)(x-x_0) = (x - x_0)^2 (\dfrac{f''(x_0)}{2} + o(1)) \Rightarrow f(x) - f(x_0) - f'(x_0) (x- x_0) > 0 $ в нек проколотой окр. $ x_0 $ 
\end{proof}
Предл. (достаточное усл-е точки перегиба) \\
Пусть f трижды дифф-ма в $ x_0 $, $f''(x_0) = 0, f'''(x_0) \neq 0 $ \\
Тогда $ x_0 $ - тчк перегиба для f \\
\begin{proof}
	$ f(x) - f(x_0) - f'(x_0) (x - x_0) = \dfrac{f''(x_0)}{6}(x- x_0)^3 + o((x - x_0)^3) = (x - x_0)^3 (\dfrac{f''(x_0)}{6} + o(1)) $ 
\end{proof}

\Subsection{Классические неравенства}

\begin{theorem}
	Неравенство Йенсена \\
	Пусть $ f <a,b> \rightarrow \R $ выпукл \\
	$ \lambda_1 ... \lambda_n  \geq 0, \lambda_1 + ... + \lambda_n = 1 $ \\
	Пусть $ x_1, ... x_n \in <a,b> $ \\
	Тогда $ f(\lambda_1 x_1 + ... + x_nx_n) \leq \lambda_1f(x_1) + ... + \lambda_n f(x_n) $ \\
	\begin{proof}
		Индукция по n \\
		$ n = 2 $ опр-е выпукл \\
		$ \lambda_{n+1} \neq 1 $ \\
		$ f(\lambda_1x_1 + ... + \lambda_n x_n + \lambda_{n+1} x_{n+1}) \leq (1 - \lambda_{n+1}) f(y) + \lambda_{n+1} f(x_{n+1}) $ \\
		$ y \in <a,b> $ т.к. $ y = \dfrac{\lambda_1 x_1 + ... + \lambda_n x_n}{\lambda_1 + ... + \lambda_n} $ % pic2
	\end{proof}
\end{theorem}

Сл. 1 Неравенство о средних \\
Пусть $ x_1 ... x_n \geq 0 $ \\
Тогда $ \dfrac{x_1 + ... + x_n}{n} \geq \sqrt[n]{x_1...x_n} $ \\
\begin{proof}
	$ x_i = 0 $ - тривиальный случай \\
	Пусть $ x_1, ..., x_n > 0 $ \\
	$ f(x) = \ln x $ - вогн\\
	По нерав Йенсена для  $ \lambda_1 = .. = \lambda_n = \dfrac{1}{n} $ \\
	$ \ln(\dfrac{x_1 + ... + x_n}{n}) \geq \dfrac{1}{n} \ln x_1 + ... + \dfrac{1}{n} \ln x_n \Rightarrow \dfrac{x_1 + ... + x_n}{n} \geq e $ 
	%pic3
\end{proof}
Cл.2 (Нер-во между средними ?) \\
Пусть $ a_1, ..., a_n > 0, p \leq q, p, q \in \R \setminus \{0\} $ \\
Тогда $ (\dfrac{a_1^p + ... + a_n^p}{n})^{\frac{1}{p}} \leq  (\dfrac{a_1^q + ... + a_n^q}{n})^{\frac{1}{q}} $ \\
\begin{proof}
	%pic4
\end{proof}
Сл.3 Пусть $ a_1, ... a_n \geq 0 $ \\
$ M_p \left\{ \begin{array}{cc}
\left( \dfrac{q_1^p + ... + q_n^p}{n} \right)^{\frac{1}{p}} & p \neq 0  \\
\sqrt[n]{a_1 ... a_n} & p = 0 
\end{array} \right. $  \\
Тогда $ M_p \leq M_q \forall p \leq q, (p, q \in \R) $ \\
\begin{proof}
	% pic5
\end{proof}

Предл. Неравенство Гёльдера \\
$ (a_1, .. a_n), (b_1, b_n) \geq 0 $ \\
$ p, q > 1, \dfrac{1}{p} + \dfrac{1}{q} = 1 $ \\
Тогда $ a_1b_1 + ... + a_nb_n \leq (a_1^p + ... + a_n^p)^{\frac{1}{p}} (b_1^q + ... + b_n^q)^{\frac{1}{q}}  $ \\
\begin{proof}
	$ f(x)  = x^p $ \\$ x_k = \dfrac{a_k}{b^{\frac{q}{p}}_k} $ \\
	$ \lambda_k = \dfrac{b_k^q}{b_1^q + ... + b_n^q} $  \\
	$ \lambda_1 + .. + \lambda_n = 1 $ \\
	$ \lambda_k x_k = \dfrac{a_kb_k}{b_1^q + ... + b_n^q}  $ \\
	%pic6
\end{proof}
Сл. Пусть $ p, q > 1, \dfrac{1}{p} + \dfrac{1}{q} = 1, a_i, b_i \in \R, i = 1...n $ \\
Тогда $ |a_1b_1 + ... + a_nb_n | \leq  (|a|^p) + ... + |a|^p)^{\frac{1}{p}}  (|b|^q) + ... + |b|^q)^{\frac{1}{q}} $ \\
\begin{proof}
	$ |a_1b_1| + ... + |a_nb_n| \leq |a_1| |b_1| + ... + |a_n||b_n|$
\end{proof}
Сл. Неравенство Коши-Буньковского (Шварца) \\ 
Пусть $ a_i, b_i \in \R, i = 1, ..., n $ \\
$ (a_1b_1 + ... + a_nb_n)^2 \leq (a_1^2 + ... + a_n^2)(b_1^2 + ... + b_n^2) $ \\
$ p = q = 2 $ \\
Предл. Неравенство Минковского \\
Пусть $ a_1...a_n, b_1...b_n \geq 0, p \geq 1 $ \\
$ \Rightarrow (a_1^p + ... a_n^p)^{\frac{1}{p}} +  (b_1^p + ... b_n^p)^{\frac{1}{p}} \geq  ((a_1+b_1)^p + ... (a_n+b_n)^p)^{\frac{1}{p}} $ \\
\begin{proof}
	$ p = 1 $ - тривиальный случай \\
	$ p > 1 $ \\
	$ q = \dfrac{p}{p-1} \Rightarrow  \dfrac{1}{p} + \dfrac{1}{q} = 1$ \\
	$ (a_1 + b_1)^p + ... + (a_n + b_n)^p = (a_1 + b_1)(a_1 + b_1)^{p-1} + ... + (a_n + b_n)(a_n + b_n)^{p-1} = a_1 (a_1 + b_1)^{p-1} + ... + a_n ( a_n + b_n)^{p-1} + b_1 (a_1 + b_1)^{p - 1} + ... + b_n (a_n + ... + b_n)^{p-1}  $ \\
	$ \leq (a_1^p + ... + a_n^p)^{\frac{1}{p}} ((a_1 + b_1)^p  + ... + (a_n + b_n)^{p})^{\frac{1}{q}} +  (b_1^p + ... + b_n^p)^{\frac{1}{p}} ((a_1 + b_1)^p  + ... + (a_n + b_n)^{p})^{\frac{1}{q}}  = ((a_1^p + ... + a_n^p)^{\frac{1}{p}} +  (b_1^p + ... + b_n^p)^{\frac{1}{p}}) ((a_1 + b_1)^p  + ... + (a_n + b_n)^{p})^{\frac{1}{q}} = 1 $
\end{proof}
Зам.1 При $ 0 < p < 1 $ получим нер-во с фикс знаком \\
Зам.2 При $ p = 2 $ получим $ ||a|| + ||b|| \geq ||a + b || $ \\






\end{document}
